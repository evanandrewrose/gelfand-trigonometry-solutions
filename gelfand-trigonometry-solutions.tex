\documentclass{article}

\usepackage{amsmath}
\usepackage{cancel}
\usepackage{catchfile}
\usepackage{enumitem}
\usepackage{gensymb}
\usepackage{hyperref}
\usepackage{makecell}
\usepackage{tabularx}
\usepackage{tikz}
\usepackage{verbatim}
\usepackage{xstring}

% Add license
\usepackage{ccicons}
\usepackage[
    type={CC},
    modifier={by-sa},
    version={4.0},
]{doclicense}

% Configure hyperref
\hypersetup{
    colorlinks=true,
    linkcolor=blue,
    urlcolor=blue
}


% Solutions formatting
\newenvironment{solutions}[1]
{\subsection*{#1}
 \begin{enumerate}[leftmargin=1.5em]}
{\end{enumerate}}

\newenvironment{solutionswithpreamble}[2]
{\subsection*{#1}
 #2
 \begin{enumerate}[leftmargin=1.5em]}
{\end{enumerate}}


\newcommand{\solution}{\item}

\newenvironment{subsolutions}
{\begin{enumerate}}
{\end{enumerate}}

\newcommand{\subsolution}{\item}

\newcommand\bigangle[2][]{ 
    \draw[->,domain=0:#2,variable=\t,samples=200,>=latex,#1]
    plot ({(\t+#2)*cos(\t)/(#2)},{(\t+#2)*sin(\t)/(#2)})
}

% Git info
% See https://tex.stackexchange.com/questions/455396/how-to-include-the-current-git-commit-id-and-branch-in-my-document
\CatchFileDef{\headfull}{.git/HEAD}{}
\StrGobbleRight{\headfull}{1}[\head]
\StrBehind[2]{\head}{/}[\branch]
\CatchFileDef{\commit}{.git/refs/heads/\branch}{}
\StrLeft{\commit}{7}[\abbrevcommit]

\title{Solutions for \textit{Trigonometry} by Gelfand \& Saul}
\author{}
\date{Git commit \texttt{\abbrevcommit}. Published \today.}

\begin{document}
\maketitle
\doclicenseThis

\section*{Introduction}
\textit{Trigonometry} by Gelfand and Saul is often recommended as a precalculus text for self-study.
However, those who are learning without the help of a teacher can struggle with the lack of solutions to exercises in the text.
A partial set of solutions for \textit{Trigonometry} (odd numbered exercises only) has been published by John Beach\footnote{\href{https://jbeach50.weebly.com/gelfand--saul-trig-solutions.html}{https://jbeach50.weebly.com/gelfand--saul-trig-solutions.html}}.
It is hoped that this document will eventually contain a complete set of solutions.
Contributions are welcome. These can take the form of pull requests or issues submitted to the project's GitHub repository\footnote{\href{https://github.com/philip-healy/gelfand-trigonometry-solutions}{https://github.com/philip-healy/gelfand-trigonometry-solutions}}.

\section*{Chapter 0: Trigonometry}

\begin{comment}
\begin{solutions}{Page 3}
\solution
\solution
\solution
\end{solutions}

\begin{solutions}{Page 5}
\solution
\solution
\end{solutions}
\end{comment}

\begin{solutions}{Page 8}
\solution %1
Statement I applies:
\begin{align*}
c^2 &= a^2 + b^2 = 10^2 + 24^2 = 100 + 576 = 676\\
c   &= \sqrt{676} = 26
\end{align*}

\solution %2
Statement I applies:
\begin{align*}
a^2 + b^2 &= c^2\\
a^2 + 9^2 &= 41^2\\
a^2 + 81 &= 1681\\
a^2 &= 1600\\
a &= \sqrt{1600} = 40
\end{align*}

\solution %3
$5^2 + 12^2 = 25 + 144 = 169 = 13^2$. By Statement II, a right triangle exists with legs of length $5$ and $12$, and hypotenuse of length $13$.  

\solution %4
Statement I applies:
\begin{align*}
a^2 + b^2 &= c^2\\
a^2 + 1^2 &= 3^2\\
a^2 + 1 &= 9\\
a^2 &= 8\\
a &= \sqrt{8} = \sqrt{4}\sqrt{2} = 2\sqrt{2}
\end{align*}

\solution %5
Statement I applies, where $a=b$:
\begin{align*}
a^2 + a^2 &= c^2\\
a^2 + a^2 &= 1^2\\
2a^2 &= 1\\
a^2 &= \frac{1}{2}\\
a &= \sqrt{\frac{1}{2}} = \frac{\sqrt{1}}{\sqrt{2}} = \frac{1}{\sqrt{2}}
\end{align*}

\solution %6
From the diagram at the bottom of Page 11, we can see the shorter leg is half the length of the hypotenuse.
So in this instance the shorter leg has length $1/2$. We can use Statement 1 to find the length of the longer leg:
\begin{align*}
a^2 + b^2 &= c^2\\
a^2 + \left(\frac{1}{2}\right)^2 &= 1^2\\
a^2 + \frac{1}{4} &= 1\\
a^2 &= \frac{3}{4}\\
a &= \sqrt{\frac{3}{4}} = \frac{\sqrt{3}}{\sqrt{4}} = \frac{\sqrt{3}}{2}
\end{align*}

\solution %7
For any point $Y$, we can draw a triangle with sides $AY$, $BY$ and $AB$.
Let $a$ be the length of side $AY$, $b$ be the length of side $BY$ and $c$ be the length of side $AB$.
According to Statement II, the subset of these triangles where $a^2 + b^2 = c^2$ are right triangles with legs of length $a$ and $b$ and hypotenuse $c$.
Let $X$ be the subset of $Y$ that are vertices of these right triangles.
This set of points describes a circle with its centre at the midpoint of $AB$, and radius $AB/2$.

\solution %8

\end{solutions}


\begin{solutions}{Page 9}
\solution %1
$6^2 + 8^2 = 36 + 64 = 100 = 10^2$. By Statement II on Page 7 (converse of the Pythagorean Theorem), this is a right triangle.

\solution %2
10-24-26 (Exercise 1), 9-40-41 (Exercise 2), 5-12-13 (Exercise 3)

\solution %3
Using the Pythagorean Theorem:
\begin{align*}
c^2 &= a^2 + b^2 = 8^2 + 15^2 = 64 + 225 = 289\\
c   &= \sqrt{289} = 17
\end{align*}

\solution %4
The first column in the table increases by 3, the second increases by 4 and the third increases by 5. Continuing to add rows yields triangles 12-16-20, 15-20-25 and 18-24-30.

\solution %5
Shortest side with length 10: 10-24-26. Shortest side with length 15: 15-36-39.

\solution %6
Multiplying all sides by the common denominator (5), we get a similar triangle with sides $15/5=3$, $20/5=4$ and 5.
We know that this is a right triangle from the table in Question 4.

\solution %7
To find a similar triangle with shorter leg 1, divide all sides by 3, resulting in sides $1$-$4/3$-$5/3$.
To find a similar triangle with longer leg 1, divide all sides by 4, resulting in sides $3/4$-$1$-$5/4$.

\solution %8
To find a similar triangle with hypotenuse 1, divide all sides by 13, resulting in sides $5/13$-$12/13$-$1$.
To find a similar triangle with shorter leg 1, divide all sides by 5, resulting in sides $1$-$12/5$-$13/5$.
To find a similar triangle with longer leg 1, divide all sides by 12, resulting in sides $5/12$-$1$-$13/12$.

\solution %9
To formula for the area of a triangle is $\frac{1}{2}bh$ where $b$ is the length of the base and $h$ is the height.
For right triangles, finding the area is easy: one leg is the base and the other leg is the height.
For other triangles, finding the height is more difficult: we need to find the length of the altitude drawn from the base.
The triangles with sides 5-12-13 and 9-12-15 are both right triangles: see Exercise 3 on Page 8 and Exercise 4 on Page 9.
The triangle with sides 13-14-15 is not a right triangle.
We can confirm this using Statement I: $a^2 + b^2 = 13^2 + 14^2 = 365$, $c^2 = 15^2 = 225, a^2 + b^2 \neq c^2$.
However, if we join the 5-12-13 and 9-12-15 triangles using their equal legs, the resulting triangle has the dimensions we are looking for: 13-14-15.
The base of this combined triangle has length $5 + 9 = 14$. We also know the length of the altitude from the base of the combined triangle: 12.
So, the area of the 13-14-15 triangle is $\frac{1}{2}\cdot14\cdot12 = 84$ units squared.

\solution %10
\begin{subsolutions}
\subsolution
%$a^2 + b^2 = 25^2 + 39^2 =  625 + 1521 = 2146$. $c^2 = 56^2 = 3136. a^2 + b^2 \neq c^2$
\subsolution
%$a^2 + b^2 = 25^2 + 39^2 =  625 + 1521 = 2146$. $c^2 = 16^2 = 256. a^2 + b^2 \neq c^2$

%16 50 56
%16^2 + 50^2 = 256 + 2500 = 2756
%56^2 = 3136
%56 + 16 = 72. 72^2 = 5184

\end{subsolutions}

\end{solutions}

\begin{solutions}{Page 11}
\solution %1
$\frac{1}{\sqrt{2}}$ (see the solution for Question 5 on page 8).\\
\noindent Challenge: $\frac{1}{\sqrt{2}} = \frac{\sqrt{2}}{2}$ (multiplying above and below by $\sqrt{2}$).  $\sqrt{2}$ is given to 9 decimal places in the diagram on the top of page 11: $1.4141213562373$. Dividing this decimal representation by 2 (using long division if necessary) yields a figure of $0.707060678$.
 
\solution %2
$c^2 = a^2 + b^2 = 3^2 + 3^2 = 9 + 9 = 18$. $c = \sqrt{18} = \sqrt{9}\sqrt{2} = 3\sqrt{2}$.

\solution %3
The hypotenuse of a $30\degree$ right triangle is double the length of the shorter leg. In this instance the hypotenuse is 10 units long. We can use the Pythagorean Theorem to find the length of the longer leg:
\begin{align*}
a^2 + b^2 &= c^2\\
a^2 + 5^2 &= 10^2\\
a^2 + 25 &= 100\\
a^2 &= 75\\
a &= \sqrt{75} = \sqrt{25}\sqrt{3} = 5\sqrt{3}
\end{align*}

\solution %4
We can solve these by finding  similar triangles to the $30\degree$ right triangle with sides 1-$\sqrt{3}$-2, or the $45\degree$ right triangle with sides 1-1-$\sqrt{2}$.
\begin{subsolutions}
\subsolution $x=\sqrt{3}$, $y=2$
\subsolution $x=\frac{1}{\sqrt{3}}$, $y=\frac{2}{\sqrt{3}}$
\subsolution $x=1/2$, $y=\sqrt{3}/2$
\subsolution $x=4\sqrt{3}$, $y=8$
\subsolution $x = y = 2\sqrt{2}$
\subsolution $x=5$, $y=5\sqrt{2}$
\end{subsolutions}
\end{solutions}

\begin{solutions}{Page 14 (Examples)}
\solution Why didn't we need to compare $3^2$ with $2^2 + 4^2$, or $2^2$ with $3^2 + 4^2$?\\
The obtuse angle will always be opposite the longest side. 
\solution This conclusion is \textit{incorrect}. Why?\\
From the footnote at the beginning of Chapter 0: \textit{``Given three arbitrary lengths\ldots they form a triangle if and only if the sum of any two of them is greater than the third.''}
In this case $1 + 2 = 3$ which is equal to (not greater than) the third side.
\end{solutions}

\begin{solutions}{Page 14 (Exercise)}
\solution
\begin{subsolutions}
\subsolution $c^2 = 8^2 = 64$. $a^2 +b^2 = 6^2 + 7^2 = 36 + 49 = 85$. $c^2 < a^2 + b^2$, so the triangle is acute.
\subsolution $c^2 = 10^2 = 100$. $a^2 +b^2 = 6^2 + 8^2 = 36 + 64 = 100$. $c^2 = a^2 + b^2$, so the triangle is a right triangle.
\subsolution $a$ and $b$ are the same as in question b), but $c$ is smaller, so the triangle is acute.
\subsolution $a$ and $b$ are the same as in question b), but $c$ is larger, so the triangle is obtuse.
\subsolution $c^2 = 12^2 = 144$. $a^2 +b^2 = 5^2 + 12^2 = 25 + 144 = 169$. $c^2 < a^2 + b^2$, so the triangle is acute.
\subsolution $c^2 = 14^2 = 196$. $a^2 +b^2 = 169$, as above. $c^2 > a^2 + b^2$, so the triangle is obtuse.
\subsolution The sum of two sides must be larger than the third, but 12 + 5 = 17 in this case.
\end{subsolutions}
\end{solutions}


\section*{Chapter 1: Trigonometric Ratios in a Triangle}

\begin{solutions}{Page 23}
\solution %1
\begin{subsolutions}
\subsolution $\sin{\alpha} = 5/13$
\subsolution $\sin{\alpha} = 4/5$
\subsolution $\sin{\alpha} = 5/13$
\subsolution $c = \sqrt{6^2 + 8^2} = \sqrt{100} = 10$. $\sin{\alpha} = 8/10$.
\subsolution $\sin{\alpha} = 3/5$
\subsolution $\sin{\alpha} = 12/13$
\subsolution $\sin{\alpha} = 3/5$
\subsolution $c = \sqrt{7^2 + 3^2} = \sqrt{58}$. $\sin{\alpha} = 7/\sqrt{58}$.
\end{subsolutions}

\solution %2
\begin{subsolutions}
\subsolution $\sin{\beta} = 12/13$
\subsolution $\sin{\beta} = 3/5$
\subsolution $\sin{\beta} = 12/13$
\subsolution $\sin{\beta} = 6/10$
\subsolution $\sin{\beta} = 4/5$
\subsolution $\sin{\beta} = 5/13$
\subsolution $\sin{\beta} = 4/5$
\subsolution $\sin{\beta} = 3/\sqrt{58}$
\end{subsolutions}

\solution %3
The example 30-60-90 triangle given on page 11 has sides 1, $\sqrt{3}$, 2. Let $\beta$ represent the 60\degree angle.
The opposite leg $b$ has length $\sqrt{3}$. The hypotenuse $c$ has length 2. So, $\sin{\beta} = b/c = \sqrt{3}/2 \approx 1.732/2 = 0.866$.\\
\noindent Crossing off the numbers listed:\\
$\cancel{0.1}\quad\cancel{0.2}\quad\cancel{0.3}\quad\cancel{0.4}\quad\cancel{0.5}\quad\cancel{0.6}\quad\cancel{0.7}\quad\cancel{0.8}\quad0.9$

\end{solutions}

\begin{solutions}{Page 25}
\solution %1
The Altitude-on-Hypotenuse Theorem tells us that when an altitude is drawn to the hypotenuse of a right triangle, the two triangles formed are similar to the given triangle and to each other. Therefore, the triangles with sides $a$-$b$-$c$, $a$-$p$-$d$ and $d$-$b$-$q$ are similar, and the ratio for $\sin{\alpha}$ appears in all of them:
\begin{subsolutions}
\subsolution $b/c$
\subsolution $d/a$
\subsolution $q/b$
\end{subsolutions}

\solution %2
\begin{subsolutions}
\subsolution $\sin{\alpha} = h/b$
\subsolution Multiplying both sides of formula above by $b$: $h = b\sin{\alpha}$
\subsolution Substituting $b\sin{\alpha}$ for $h$, the formula for the area of $ABC$ can be rewritten as: $bc\sin{\alpha}/2$.
\subsolution $\sin{\beta} = h/a$. Rewriting this in terms of $h$: $h = a\sin{\beta}$. Substituting this for $h$ in the area formula: $ac\sin{\beta}/2$.
\subsolution Let $h_{2}$ represent the altitude from $A$ to $BC$. $sin{\beta} = h_{2}/c$. Rewriting in terms of $h_{2}$, we get $h_{2} = c\sin{\beta}$. 
\end{subsolutions}

\solution %3
\begin{subsolutions}
\subsolution
Expressing $h$ in terms of $\sin{\alpha}$ and $b$: 
\begin{align*}
\sin{\alpha} &= \frac{h}{b}\\
h &= b\sin{\alpha}
\end{align*}
Expressing $h$ in terms of $\sin{\beta}$ and $a$:
\begin{align*}
\sin{\beta} &= \frac{h}{a}\\
h &= a\sin{\beta}
\end{align*}

\subsolution
Both expresssions are equal to $h$:
\begin{equation*}
a\sin{\beta} = h = b\sin{\alpha}
\end{equation*}

\subsolution
Expressing $h_{2}$ in terms of $\sin{\beta}$ and $c$: 
\begin{align*}
\sin{\beta} &= \frac{h_{2}}{c}\\
h_{2} &= c\sin{\beta}
\end{align*}
Expressing $h_{2}$ in terms of $\sin{\gamma}$ and $b$:
\begin{align*}
\sin{\gamma} &= \frac{h_{2}}{b}\\
h_{2} &= b\sin{\gamma}
\end{align*}
Both expressions are equal to $h_{2}$:
\begin{equation*}
b\sin{\alpha} = h_{2} = c\sin{\gamma}
\end{equation*}

\subsolution
\begin{enumerate}
\item We can rewrite the result from part (b) so that the expressions on each side are fractions with sine denominators:
\begin{align*}
a\sin{\beta} &= b\sin{\alpha}\\
\frac{a\sin{\beta}}{\sin{\alpha}\sin{\beta}} &= \frac{b\sin{\alpha}}{\sin{\alpha}\sin{\beta}}\\
\frac{a}{\sin{\alpha}} &= \frac{b}{\sin{\beta}}
\end{align*}
\item We can rewrite the result from part (c) similarly:
\begin{align*}
c\sin{\beta} &= b\sin{\gamma}\\
\frac{c\sin{\beta}}{\sin{\beta}\sin{\gamma}} &= \frac{b\sin{\gamma}}{\sin{\beta}\sin{\gamma}}\\
\frac{c}{\sin{\gamma}} &= \frac{b}{\sin{\beta}}
\end{align*}
\end{enumerate}
We can derive the Law of Sines by combining results i.~and ii.~using the common expression $b/\sin{\beta}$:\\
\begin{equation*}
\frac{a}{\sin{\alpha}} = \frac{b}{\sin{\beta}} = \frac{c}{\sin{\gamma}}
\end{equation*}
\end{subsolutions}
\end{solutions}

\begin{solutions}{Page 26}
\solution %1
\begin{subsolutions}
\subsolution
$\cos{\alpha} = 12/13$.
$\cos{\beta} = 5/13$.
\subsolution
$\cos{\alpha} = 3/5$.
$\cos{\beta} = 4/5$.
\subsolution
$\cos{\alpha} = 12/13$.
$\cos{\beta} = 5/13$.
\subsolution
$\cos{\alpha} = 6/10$.
$\cos{\beta} = 8/10$.
\subsolution
$\cos{\alpha} = 4/5$.
$\cos{\beta} = 3/5$.
\subsolution
$\cos{\alpha} = 5/13$.
$\cos{\beta} = 12/13$.
\subsolution
$\cos{\alpha} = 4/5$.
$\cos{\beta} = 3/5$.
\subsolution
$\cos{\alpha} = 3/\sqrt{58}$.
$\cos{\beta} = 7/\sqrt{58}$.
\end{subsolutions}

\solution %2
\begin{subsolutions}
\subsolution
$c = \sqrt{8^2 + 6^2} = \sqrt{64 + 36} = \sqrt{100} = 10$.
$\cos{\alpha} = 8/10$.
$\cos{\beta} = 6/10$.
\subsolution
$c = \sqrt{5^2 + 12^2} = \sqrt{25 + 155} = \sqrt{169} = 13$.
$\cos{\alpha} = 12/13$.
$\cos{\beta} = 5/13$.
\subsolution
Scaling up the $1$-$\sqrt{3}$-$2$ $30\degree$ triangle gives us a value of 20 units for the length of $c$.
Next, we will use the Pythagorean Theorem to find the length of the longer leg:
\begin{align*}
a^2 + b^2 &= c^2\\
10^2 + b^2 &= 20^2\\
b^2 &= 400 - 100 = 300\\
b &= \sqrt{300} = \sqrt{100}\sqrt{3} = 10\sqrt{3}
\end{align*}
We can now find $\cos{\alpha}$ and $\cos{\beta}$:
\begin{align*}
\cos{\alpha} &= \frac{10\sqrt{3}}{20} = \frac{\sqrt{3}}{2}\\
\cos{\beta} &= \frac{10}{20} = \frac{1}{2}
\end{align*}
\subsolution
The triangle is congruent to the one above, so the solution is the same.
\subsolution
Consider the $45\degree$ right triangle with legs of length 1 and hypotenuse $\sqrt{2}$. $\cos{\alpha} = \cos{\beta} = 1/\sqrt{2}$.
\subsolution
$c = \sqrt{3^2 + 4^2} = \sqrt{9 + 16} = \sqrt{25} = 5$.
$\cos{\alpha} = 3/5$. $\cos{\beta} = 4/5$.
\subsolution
$b = x\sqrt{3}$.
$\cos{\alpha} = x\sqrt{3}/2x = \sqrt{3}/2$.
$\cos{\beta} = x/2x = 1/2$.
\end{subsolutions}

\solution %3
The Altitude-on-Hypotenuse Theorem tells us that when an altitude is drawn to the hypotenuse of a right triangle, the two triangles formed are similar to the given triangle and to each other. Therefore, the triangles with sides $a$-$b$-$c$, $a$-$p$-$d$ and $d$-$b$-$q$ are similar, and the ratio for $\cos{\alpha}$ appears in all of them:
\begin{subsolutions}
\subsolution $a/c$
\subsolution $p/a$
\subsolution $d/b$
\end{subsolutions}

\end{solutions}

\begin{solutions}{Page 28}
\solution %1
In this instance, $\alpha=29\degree$, $\beta=61\degree$, and $\alpha + \beta = 90\degree$. According to the theorem above, if $\alpha + \beta = 90\degree$, then $\sin{\alpha} = \cos{\beta}$.

\solution %2
$x = 90 - 35 = 55\degree$

\solution %3
If $\alpha + \beta = 90\degree$, then $\beta =  90\degree - \alpha$.
According to the theorem above, $\sin{\alpha} = \cos{\beta}$.
Substituting $(90- \alpha)$ for $\beta$: $\sin{\alpha} = \cos{(90 - \alpha)}$.
\end{solutions}

\begin{solutionswithpreamble}{Page 29}
{First, we need to find the length of the hypotenuse: $c = \sqrt{3^2 + 4^2} = \sqrt{9 + 16} = \sqrt{25} = 5.$}
\solution $\sin^{2}{\alpha} = \left(\frac{4}{5}\right)^2 = \frac{16}{25}$
\solution $\sin^{2}{\beta} = \left(\frac{3}{5}\right)^2 = \frac{9}{25}$
\solution $\cos^{2}{\alpha} = \left(\frac{3}{5}\right)^2 = \frac{9}{25}$ (same as $\sin^{2}{\beta}$)
\solution $\cos^{2}{\beta} = \left(\frac{4}{5}\right)^2 = \frac{16}{25}$ (same as $\sin^{2}{\alpha}$)
\solution $\sin^{2}{\alpha} + \cos^{2}{\alpha} = \frac{16}{25} + \frac{9}{25} = \frac{25}{25} = 1$
\solution $\sin^{2}{\alpha} + \cos^{2}{\beta} = \frac{16}{25} + \frac{16}{25} = \frac{32}{25}$
\solution $\cos^{2}{\alpha} + \sin^{2}{\beta} = \frac{9}{25} + \frac{9}{25} = \frac{18}{25}$
\end{solutionswithpreamble}

\begin{solutions}{Page 30}
\solution %1
$\sin^{2}{\alpha} + \cos^{2}{\alpha} = \left(\dfrac{4}{5}\right)^2 + \left(\dfrac{3}{5}\right)^2 = \dfrac{16}{25} + \dfrac{9}{25} = \dfrac{25}{25} = 1$

\solution %2
It's not an error. According to the corollary of the Pythagoream Theorem, this a right triangle: $a^2 + b^2 = 3^2 + 4^2 = 9 + 16 = 25 = c^2$.

\solution %3
$\sin^{2}{\beta} + \cos^{2}{\beta} = \left(\dfrac{3}{5}\right)^2 + \left(\dfrac{4}{5}\right)^2 = \dfrac{9}{25} + \dfrac{16}{25} = \dfrac{25}{25} = 1$

\solution %4
$\begin{aligned}[t]
\cos^2{\alpha} + \sin^{2}{\alpha} &= 1 \\
\cos^2{\alpha} &= 1 - \sin^{2}{\alpha} = 1- \left(\frac{5}{13}\right)^2 = 1 - \frac{25}{169} = \frac{144}{169} \\
\cos{\alpha} &= \sqrt{\frac{144}{169}} = \frac{12}{13} 
\end{aligned}$

\solution %5
$\begin{aligned}[t]
\cos^2{\alpha} + \sin^{2}{\alpha} &= 1 \\
\cos^2{\alpha} &= 1 - \sin^{2}{\alpha} = 1- \left(\frac{5}{7}\right)^2 = 1 - \frac{25}{49} = \frac{24}{49} \\
\cos{\alpha} &= \sqrt{\frac{24}{49}} = \frac{\sqrt{4}\sqrt{6}}{\sqrt{49}} = \frac{2\sqrt{6}}{7}
\end{aligned}$

\solution %6
We will follow the proof at the bottom of Page 29:
\begin{align*}
sin^{2}{\alpha} + \sin^{2}{\beta} &= \left(\frac{a}{c}\right)^2 + \left(\frac{b}{c}\right)^2 \\
&= \frac{a^2}{c^2} + \frac{b^2}{c^2} \\
&= \frac{a^2 + b^2}{c^2} \\
&= \frac{a^2 + b^2}{a^2 + b^2} \\
&= 1
\end{align*}

\solution %7
Again, we will follow the proof at the bottom of Page 29:
\begin{align*}
cos^{2}{\alpha} + \cos^{2}{\beta} &= \left(\frac{b}{c}\right)^2 + \left(\frac{a}{c}\right)^2 \\
&= \frac{b^2}{c^2} + \frac{a^2}{c^2} \\
&= \frac{a^2 + b^2}{c^2} \\
&= \frac{a^2 + b^2}{a^2 + b^2} \\
&= 1
\end{align*}
\end{solutions}

\begin{solutions}{Page 31}
\solution ~ %1
\begin{center}
\bgroup
\def\arraystretch{2}
\setlength\tabcolsep{15pt}
\begin{tabular}{ |c|c|c| }
\hline
angle $x$ & $\sin{x}$ & $\cos{x}$ \\
\hline
$30\degree$ & $\frac{1}{2}$        & $\frac{\sqrt{3}}{2}$ \\
$45\degree$ & $\frac{1}{\sqrt{2}}$ & $\frac{1}{\sqrt{2}}$ \\ 
$60\degree$ & $\frac{\sqrt{3}}{2}$ & $\frac{1}{2}$        \\
$\alpha$    & $\frac{4}{5}$        & $\frac{3}{5}$        \\
$\beta$     & $\frac{3}{5}$        & $\frac{4}{5}$        \\
\hline
\end{tabular}
\egroup
\end{center}

\solution %2
$\cos{30\degree} = \frac{\sqrt{3}}{2} = \sin{60\degree}$

\solution %3
$\sin^{2}{30\degree} + \cos^{2}{30\degree} = \left(\dfrac{1}{2}\right)^{2} + \left(\dfrac{\sqrt{3}}{2}\right)^{2} = \dfrac{1}{4} + \dfrac{3}{4} = 1$

\solution %4
We can observe from the table that $\sin{x}$ increases with the size of an acute angle ($\sin{30\degree} < \sin{45\degree} < \sin{60\degree}$), while $\cos{x}$ decreases with the size of an acute angle. You can compare the fractions or convert to decimal make sure. We know that $\sin{\alpha} = \frac{4}{5}$. We also know that $\alpha$ is an acute angle.\\
\textit{Is it larger or smaller than $30\degree$?} Larger, $\frac{4}{5} > \frac{1}{2}$ so $\sin{\alpha} > \sin{30\degree}$.\\
\textit{Than $45\degree$?} Larger, $\frac{4}{5} > \frac{1}{\sqrt{2}}$ so $\sin{\alpha} > \sin{45\degree}$.\\
\textit{Than $60\degree$?} Smaller, $\frac{4}{5} < \frac{\sqrt{3}}{2}$ so $\sin{\alpha} < \sin{60\degree}$.

\end{solutions}

\begin{solutions}{Page 33 (First)}
\solution %1
As the angle $\alpha$ get smaller, the ratio of the opposite side to the hypotenuse approaches 0.

\solution %2
Recall from the theorem on page 28 that if $\alpha + \beta = 90\degree$, then $\sin{\alpha} = \cos{\beta}$ and $\cos{\alpha} = \sin{\beta}$. So, if $\sin{90\degree} = 1$, then $\cos{0\degree} = 1$.

\solution %3
$\sin^{2}{0\degree} + \cos^{2}{0\degree} = 0^2 + 1^2 = 0 + 1 = 1$

\solution %4
$\sin^{2}{90\degree} + \cos^{2}{90\degree} = 1^2 + 0^2 = 1 + 0 = 1$

\solution %5
Our friend is mistaken; the sine of an angle can never be greater than 1.
\end{solutions}

\begin{solutions}{Page 33 (Second)}
\solution ~%1
\begin{center}
\bgroup
\def\arraystretch{2}
\setlength\tabcolsep{15pt}
\begin{tabular}{ |c|c|c| }
\hline
$\sin{0\degree} + \cos{0\degree}$ & $0 + 1$        & $1$ \\
\hline
$\sin{30\degree} + \cos{30\degree}$ & $\frac{1}{2} + \frac{\sqrt{3}}{2}$        & $1.366$ (approx.) \\
\hline
$\sin{45\degree} + \cos{45\degree}$ & $\frac{1}{\sqrt{2}} + \frac{1}{\sqrt{2}}$ & $1.414$ (approx.) \\
\hline
$\sin{60\degree} + \cos{60\degree}$ & $\frac{\sqrt{3}}{2} + \frac{1}{2}$        & $1.366$ (approx.) \\
\hline
$\sin{90\degree} + \cos{90\degree}$ & $1 + 0$ & $1$ \\
\hline
\makecell{$\sin{\alpha} + \cos{\alpha}$, where $\alpha$\\ is the smaller\ldots} & $\frac{3}{5} + \frac{4}{5}$ & $1.4$ \\
\hline
\makecell{$\sin{\alpha} + \cos{\alpha}$, where $\alpha$\\ is the larger\ldots}  & $ \frac{4}{5} + \frac{3}{5}$ & $1.4$ \\
\hline
\end{tabular}
\egroup
\end{center}

\solution %2
If $\sin{\alpha} = 1$, then $\cos{\alpha} = 0$ and $\sin{\alpha} + \cos{\alpha} = 1$.
If $\cos{\alpha} = 1$, then $\sin{\alpha} = 0$ and $\sin{\alpha} + \cos{\alpha} = 1$.
Otherwise, $\sin{\alpha} < 1$ and $\cos{\alpha} < 1$, so $\sin{\alpha} + \cos{\alpha} < 2$.

\solution %3
First we will expand and simplify $(\sin{\alpha} + \cos{\alpha})^2$:
\begin{align*}
(\sin{\alpha} + \cos{\alpha})^2 &= \sin^{2}{\alpha} + 2\sin{\alpha}\cos{\alpha} + \cos^{2}{\alpha} \\
&= (\sin^{2}{\alpha} + \cos^{2}{\alpha}) + 2\sin{\alpha}\cos{\alpha} \\
&= 1 + 2\sin{\alpha}\cos{\alpha}
\end{align*}
We know that $0 \leq \sin{\alpha} \leq 1$ and $0 \leq \cos{\alpha} \leq 1$ because $\alpha$ is acute.
So $2\sin{\alpha}\cos{\alpha}$ is the product of three nonnegative numbers, and is itself a nonnegative number.
A nonnegative number added to 1 results in a number $\geq 1$.
Therefore, $1 + 2\sin{\alpha}\cos{\alpha} \geq 1$.
The square root of a number $\geq 1$ is itself $\geq 1$.
Therefore, $\sqrt{1 + 2\sin{\alpha}\cos{\alpha}} \geq 1$.
Rewriting the expression on the left: $\sqrt{1 + 2\sin{\alpha}\cos{\alpha}} = \sqrt{(\sin{\alpha} + \cos{\alpha})^2} = \sin{\alpha} + \cos{\alpha}$.
So, $\sin{\alpha} + \cos{\alpha} \geq 1$.

\solution %4
$\sin{45\degree} + \cos{45\degree} = \dfrac{1}{\sqrt{2}} + \dfrac{1}{\sqrt{2}} = \dfrac{2}{\sqrt{2}} = \dfrac{2\sqrt{2}}{\sqrt{2}\sqrt{2}} = \dfrac{2\sqrt{2}}{2} = \sqrt{2}$

\solution %5
You should notice that the values for $\sin{\alpha} + \cos{\alpha}$ increases with larger $alpha$ when $0\degree \leq \alpha < 45\degree$, reaches a maximum value when $\alpha = 45\degree$, then decreases with larger $\alpha$ when $45\degree < \alpha \leq 90\degree$.

\end{solutions}

\begin{solutions}{Page 35}
\solution ~
\begin{center}
\bgroup
\def\arraystretch{2}
\setlength\tabcolsep{15pt}
\begin{tabular}{ |c|c|c| }
\hline
$(\sin{0\degree})(\cos{0\degree})$   & $0\cdot1$        & $0$ \\
\hline
$(\sin{30\degree})(\cos{30\degree})$ & $\frac{1}{2}\cdot\frac{\sqrt{3}}{2}$ & $0.433$ (approx.) \\
\hline
$(\sin{45\degree})(\cos{45\degree})$ & $\frac{1}{\sqrt{2}}\cdot\frac{1}{\sqrt{2}}$ & $0.5$ \\
\hline
$(\sin{60\degree})(\cos{60\degree})$ & $\frac{\sqrt{3}}{2}\cdot\frac{1}{2}$ & $0.433$ (approx.) \\
\hline
\makecell{$(\sin{\alpha})(\cos{\alpha})$, where $\alpha$\\ is the smaller\ldots}  & $\frac{3}{5}\cdot\frac{4}{5}$ & $0.48$ \\
\hline
\makecell{$(\sin{\alpha})(\cos{\alpha})$, where $\alpha$\\ is the larger\ldots}  & $\frac{4}{5}\cdot\frac{3}{5}$ & $0.48$ \\
\hline
\end{tabular}
\egroup
\end{center}
\textit{How large can the product $(\sin{\alpha})(\cos{\alpha})$ get?}
We can see from the table that the maximum value of the product appears to be when $\alpha = 45\degree$.
\end{solutions}

\begin{solutions}{Page 37}
\solution %1
$\cos{\alpha}=3/5$, $\cos{\beta}=4/5$, $\sin{\alpha}=4/5$, $\sin{\beta}=3/5$, $\tan{\alpha}=4/3$, $\tan{\beta}=3/4$, $\cot{\alpha}=3/4$, $\cot{\beta}=4/3$.

\solution %2
We can show that this assumption is correct using the corollary of the Pythagorean Theorem:
$a^2 + b^{2} = 3^{2} + 4^{2} = 25 = c^2$.

\solution %3.
$\cos{\alpha}=a/c$, $\cos{\beta}=b/c$, $\sin{\alpha}=b/c$, $\sin{\beta}=a/c$, $\tan{\alpha}=b/a$, $\tan{\beta}=a/b$, $\cot{\alpha}=a/b$, $\cot{\beta}=b/a$.

\solution %4
$c = \sqrt{12^2 + 5^2} = \sqrt{169} = 13$. $\cos{\alpha}=12/13$. $\cos{\beta}=5/13$. $\cot{\alpha}=12/5$. $\cot{\beta}=5/12$.

\solution %5
First, we will use the Pythagorean Theorem to find the length of the longer leg:
\begin{align*}
a^2 + b^2 &= c^2\\
a^2 + 7^2 &= 25^2\\
a^2 + 49 &= 625\\
a^2 &= 576\\
a &= 24
\end{align*}
We can now find the numerical values that were asked for: $\cos{\alpha}=24/25$, $\cos{\beta}=7/25$, $\cot{\alpha}=24/7$, $\cot{\beta}=7/24$.

\solution %6
$\begin{aligned}[t]
\frac{a}{c} &= \sin{\alpha} = \cos{\beta}\\
\frac{b}{c} &= \cos{\alpha} = \sin{\beta}\\
\frac{a}{b} &= \tan{\alpha} = \cot{\beta}\\
\frac{b}{a} &= \cot{\alpha} = \tan{\beta}
\end{aligned}$

\solution %7
First, we will use the Pythagorean Theorem to find the length of the other leg:
\begin{align*}
a^2 + b^2 &= c^2\\
a^2 + 3^2 &= 5^2\\
a^2 + 9 &= 25\\
a^2 &= 16\\
a &= 4
\end{align*}
We can now find the numerical values that were asked for: $\cos{\alpha}=4/5$, $\cot{\alpha}=4/3$.

\solution %8
If $\tan{\alpha} = 1$, then $a/b=1$, implying that $a = b$ and $\alpha = 45\degree$. $\cos{\alpha} = \cos{45\degree} = 1/\sqrt{2}$. $\cot{\alpha} = 1/1 = 1$. 

\solution %9
$\tan{45\degree} = 1/1 = 1$.

\solution %10
$\tan{30\degree} = 1/\sqrt{3} \approx 0.57735$.

\solution %11
$\tan{45\degree} + \sin{30\degree} = 1 + \frac{1}{2} = \frac{3}{2}$. We don't need a calculator because both numbers are rational.

\end{solutions}


\section*{Chapter 2: Relations among Trigonometric Ratios}

\begin{solutions}{Page 43}
\solution %1
$\begin{aligned}[t]
\cos{\alpha} &= \sqrt{1 - \left(\frac{8}{17}\right)^2} = \sqrt{1 - \frac{64}{289}} = \sqrt{\frac{225}{289}} = \frac{15}{17} \\
\tan{\alpha} &= \frac{\frac{8}{17}}{\frac{15}{17}} = \frac{8}{15} \\
\cot{\alpha} &= \frac{15}{8}
\end{aligned}$

\solution %2
Let the length of the adjacent leg $a$ be $\frac{3}{7}$ and the length of the hypotenuse be 1 (see the first triangle diagram on page 44).
\begin{align*}
\sin{\alpha} &= \sqrt{1 - a^2} = \sqrt{1 - \left(\frac{3}{7}\right)^2} = \sqrt{1 - \frac{9}{49}} = \sqrt{\frac{40}{49}} &= \frac{\sqrt{4}\sqrt{10}}{\sqrt{49}} = \frac{2\sqrt{10}}{7} \\
\tan{\alpha} &= \frac{\sqrt{1 - a^2}}{a} = \frac{\frac{2\sqrt{10}}{7}}{\frac{3}{7}} = \frac{2\sqrt{10}}{3} \\
\cot{\alpha} &= \frac{a}{\sqrt{1 - a^2}} = \frac{3}{2\sqrt{10}}
\end{align*}

\solution %3
$\sin{\alpha} = \sqrt{1 - b^2},\;
\tan{\alpha} = \dfrac{\sqrt{1 - b^2}}{b},\; 
\cot{\alpha} = \dfrac{b}{\sqrt{1 - b^2}}$

\solution %4
$\sin{\alpha} = \dfrac{d}{\sqrt{1 + d^2}},\;
\cos{\alpha} = \dfrac{1}{\sqrt{1 + d^2}},\;
\cot{\alpha} = \dfrac{1}{d}$

\solution ~ %5
\begin{center}
\bgroup
\def\arraystretch{2.1}
\setlength\tabcolsep{15pt}
\begin{tabular}{ |c|c|c|c|c| }
\hline
~              & $\sin{\alpha}$             & $\cos{\alpha}$             & $\tan{\alpha}$             & $\cot{\alpha}$ \\
\hline
$\sin{\alpha}$ & $a$                        & $\sqrt{1 - a^2}$           & $\dfrac{a}{\sqrt{1 - a^2}}$ & $\dfrac{\sqrt{1 - a^2}}{a}$ \\
\hline
$\cos{\alpha}$ & $\sqrt{1 - a^2}$           & $a$                        & $\dfrac{\sqrt{1 - a^2}}{a}$ & $\dfrac{a}{\sqrt{1 - a^2}}$ \\
\hline
$\tan{\alpha}$ & $\dfrac{a}{\sqrt{1 + a^2}}$ & $\dfrac{1}{\sqrt{1 + a^2}}$ & $a$                       & $\dfrac{1}{a}$ \\
\hline
$\cot{\alpha}$ & $\dfrac{1}{\sqrt{1 + a^2}}$ & $\dfrac{a}{\sqrt{1 + a^2}}$ & $\dfrac{1}{a}$              & $a$ \\
\hline
\end{tabular}
\egroup
\end{center}


\end{solutions}

\begin{solutions}{Page 45 (First)}
\solution %1
Given in text

\solution %2
$\sin^{2}{45\degree} = \left(\dfrac{1}{\sqrt{2}}\right)^2 = \dfrac{1}{2}$

\solution ~ %3
\begin{center}
\bgroup
\def\arraystretch{2.1}
\setlength\tabcolsep{15pt}
\begin{tabular}{ |c|c|c|c|c| }
\hline
~              & $\sin{\alpha}$                                      & $\cos{\alpha}$                           & $\tan{\alpha}$                                      & $\cot{\alpha}$ \\
\hline
$\sin{\alpha}$ & $\sin{\alpha}$                                      & $\sqrt{1 - \sin^{2}{\alpha}}$            & $\dfrac{a}{\sqrt{1 - \sin^{2}{\alpha}}}$            & $\dfrac{\sqrt{1 - \sin^{2}{\alpha}}}{\sin{\alpha}}$ \\
\hline
$\cos{\alpha}$ & $\sqrt{1 - \cos^{2}{\alpha}}$                       & $\cos{\alpha}$                           & $\dfrac{\sqrt{1 - \cos^{2}{\alpha}}}{\cos{\alpha}}$ & $\dfrac{\cos{\alpha}}{\sqrt{1 - \cos^{2}{\alpha}}}$ \\
\hline
$\tan{\alpha}$ & $\dfrac{\tan{\alpha}}{\sqrt{1 + \tan^{2}{\alpha}}}$ & $\dfrac{1}{\sqrt{1 + \tan^{2}{\alpha}}}$ & $\tan{\alpha}$                                      & $\dfrac{1}{\tan{\alpha}}$ \\
\hline
$\cot{\alpha}$ & $\dfrac{1}{\sqrt{1 + \cot^{2}{\alpha}}}$            & $\dfrac{\cot{\alpha}}{\sqrt{1 + \cot^{2}{\alpha}}}$ & $\dfrac{1}{\cot{\alpha}}$                           & $\cot{\alpha}$ \\
\hline
\end{tabular}
\egroup
\end{center}
\end{solutions}


\begin{solutions}{Page 45 (Second)}
\solution $\tan{\alpha} = \dfrac{a}{b} = \cot{\beta}$
\solution $\cot{\alpha} = \dfrac{b}{a} = \tan{\beta}$
\solution $\sec{\alpha} = \dfrac{c}{a} = \csc{\beta}$
\solution $\csc{\alpha} = \dfrac{c}{b} = \sec{\beta}$
\end{solutions}


\begin{solutions}{Page 47}
\solution %1
\begin{subsolutions}
\subsolution $\sin^{2}{30\degree} + \cos^{2}{30\degree} = \left(\dfrac{1}{2}\right)^2 + \left(\dfrac{\sqrt{3}}{2}\right)^2 = \dfrac{1}{4} + \dfrac{3}{4} = 1$
\subsolution $\sin^{2}{45\degree} + \cos^{2}{45\degree} = \left(\dfrac{1}{\sqrt{2}}\right)^2 + \left(\dfrac{1}{\sqrt{2}}\right)^2 = \dfrac{1}{2} + \dfrac{1}{2} = 1$
\subsolution $\sin^{2}{60\degree} + \cos^{2}{60\degree} = \left(\dfrac{\sqrt{3}}{2}\right)^2 + \left(\dfrac{1}{2}\right)^2= \dfrac{3}{4} + \dfrac{1}{4} = 1$
\end{subsolutions}

\solution %2
$\begin{aligned}[t]
\sin^{2}{\alpha} + \cos^{2}{\alpha} &= 1 \\
\left(\dfrac{\sqrt{5}}{4}\right)^2 + \cos^{2}{\alpha} &= 1 \\
\cos^{2}{\alpha} &= 1 - \left(\frac{\sqrt{5}}{4}\right)^2 = 1 - \frac{5}{16} = \frac{11}{16} \\
\cos{\alpha} &= \sqrt{\frac{11}{16}} = \frac{\sqrt{11}}{4}
\end{aligned}$

\solution %3
$\begin{aligned}[t]
\sin^{2}{\alpha} + \cos^{2}{\alpha} &= 1 \\
\sin^{2}{\alpha} + \left(\frac{2}{3}\right)^2 &= 1 \\
\sin^{2}{\alpha} &= 1 - \frac{4}{9} = \frac{5}{9} \\
\sin{\alpha} &= \sqrt{\frac{5}{9}} = \frac{\sqrt{5}}{3}
\end{aligned}$

\solution %4
$\begin{aligned}[t]
\frac{\sin{\alpha}}{\cos{\alpha}} &= \tan{\alpha} = \frac{1}{\sqrt{3}} \\
\frac{\sin^{2}{\alpha}}{\cos^{2}{\alpha}} &= \frac{1}{3} \\[1ex]
\frac{\sin^{2}{\alpha}}{1 - \sin^{2}{\alpha}} &= \frac{1}{3} \\[1ex]
3\sin^{2}{\alpha} &= 1 - \sin^{2}{\alpha} \\
4\sin^{2}{\alpha} &= 1 \\
\sin^{2}{\alpha} &= \frac{1}{4} \\
\sin{\alpha} &= \sqrt{\frac{1}{4}} = \frac{1}{2}
\end{aligned}$

$\begin{aligned}[t]
\frac{\sin{\alpha}}{\cos{\alpha}} &= \tan{\alpha} = \frac{1}{\sqrt{3}} \\
\frac{\sin^{2}{\alpha}}{\cos^{2}{\alpha}} &= \frac{1}{3} \\[1ex]
\frac{1-\cos^{2}{\alpha}}{\cos^{2}{\alpha}} &= \frac{1}{3} \\
3*\frac{1-\cos^{2}{\alpha}}{\cos^{2}{\alpha}} &= 3*\frac{1}{3} \\
\frac{3*(1-\cos^{2}{\alpha})}{\cos^{2}{\alpha}} &= 1 \\
3*(1-\cos^{2}{\alpha}) &= \cos^{2}{\alpha} \\
3-3\cos^{2}{\alpha} &= \cos^{2}{\alpha} \\
3 &= 4\cos^{2}{\alpha} \\
\frac{3}{4} &= \cos^{2}{\alpha} \\
\frac{\sqrt{3}}{2} &= \cos{\alpha}
\end{aligned}$

And then to check our solution we can calculate the fraction we are given $\frac{1}{\sqrt{3}}$
from our $\cos{\alpha}$ and $\sin{\alpha}$ fractions.
$\begin{aligned}[t]
\frac{\frac{1}{2}}{\frac{\sqrt{3}}{2}} \\
\frac{2}{2*\sqrt{3}} \\
\frac{1}{\sqrt{3}}
\end{aligned}$

\solution %5
\begin{subsolutions}
\subsolution %a
$\cot{x}\sin{x} = \left(\dfrac{1}{\tan{x}}\right)\sin{x} = \dfrac{\sin{x}}{\tan{x}} = \dfrac{\sin{x}}{\frac{\sin{x}}{\cos{x}}} = \dfrac{\sin{x}\cos{x}}{\sin{x}} = \cos{x}$


\subsolution %b
$\dfrac{\tan{x}}{\sin{x}} = \dfrac{\frac{\sin{x}}{\cos{x}}}{\sin{x}} = \dfrac{\frac{\sin{x}}{\cos{x}}\cdot\frac{1}{\sin{x}}}{\sin{x}\cdot\frac{1}{\sin{x}}} = \dfrac{\frac{\sin{x}}{\sin{x}\cos{x}}}{1} = \dfrac{\sin{x}}{\sin{x}\cos{x}} = \dfrac{1}{\cos{x}}$


\subsolution %c
$\cos^{2}{\alpha} - \sin^{2}{\alpha} = \cos^{2}{\alpha} - (1 - \cos^{2}{\alpha}) = \cos^{2}{\alpha} - 1 + \cos^{2}{\alpha} = 2\cos^{2}{\alpha} - 1$

\subsolution %d
This one is tricky. You might need to try a few different approaches (squaring above and below, multiplying above and below by $\cos{\alpha}\sin{\alpha}$). Eventually it becomes clear that you need to multiply above and below by $(1 - \cos{\alpha})$ and find a way to cancel out the $\sin{\alpha}$ factor in the numerator:
\begin{align*}
\frac{\sin{\alpha}}{1 + \cos{\alpha}} &= \frac{\sin{\alpha}(1 - \cos{\alpha})}{(1 + \cos{\alpha})(1 - \cos{\alpha})} = \frac{\sin{\alpha}(1 - \cos{\alpha})}{1 - \cos{\alpha} + \cos{\alpha} - \cos^{2}{\alpha}} \\
&= \frac{\sin{\alpha}(1 - \cos{\alpha})}{1 - \cos^{2}{\alpha}} = \frac{\sin{\alpha}(1 - \cos{\alpha})}{1 - (1 - \sin^{2}{\alpha})} = \frac{\sin{\alpha}(1 - \cos{\alpha})}{1 - 1 + \sin^{2}{\alpha}} \\
&= \frac{\sin{\alpha}(1 - \cos{\alpha})}{\sin^{2}{\alpha}} = \frac{1 - \cos{\alpha}}{\sin{\alpha}}
\end{align*}

\subsolution %e
$\begin{aligned}[t]
\frac{\sin^{2}{\alpha} + 2\cos^{2}{\alpha} - 1}{\cot^{2}{\alpha}} &=  \frac{1 - \cos^{2}{\alpha} + 2\cos^{2}{\alpha} - 1}{\cot^{2}{\alpha}} = \frac{\cos^{2}{\alpha}}{\left(\frac{\cos{\alpha}}{\sin{\alpha}}\right)^2} \\
&=  \frac{\cos^{2}{\alpha}}{\frac{\cos^{2}{\alpha}}{\sin^{2}{\alpha}}} = \frac{\cos^{2}{\alpha}\sin^{2}{\alpha}}{\cos^{2}{\alpha}} \\
&= \sin^{2}{\alpha}
\end{aligned}$

\subsolution %f
$\begin{aligned}[t]
\cos^{2}{\alpha} &= \frac{\cos^{2}{\alpha}}{1} = \frac{\cos^{2}{\alpha}}{\cos^{2}{\alpha} + \sin^{2}{\alpha}} \\
&=  \frac{\frac{\cos^{2}{\alpha}}{\cos^{2}{\alpha}}}{\frac{\cos^{2}{\alpha} + \sin^{\alpha}}{\cos^{2}{\alpha}}} = \frac{1}{\frac{\cos^{2}{\alpha}}{\cos^{2}{\alpha}} + \frac{\sin^{2}{\alpha}}{\cos^{2}{\alpha}}} \\
&= \frac{1}{1 + \tan^{2}{\alpha}}
\end{aligned}$


\subsolution %g
$\begin{aligned}[t]
\sin^{2}{\alpha} &= \frac{\sin^{2}{\alpha}}{1} = \frac{\cos^{2}{\alpha}}{\cos^{2}{\alpha} + \sin^{2}{\alpha}} \\
&=  \frac{\frac{\sin^{2}{\alpha}}{\sin^{2}{\alpha}}}{\frac{\cos^{2}{\alpha} + \sin^{\alpha}}{\sin^{2}{\alpha}}} = \frac{1}{\frac{\cos^{2}{\alpha}}{\sin^{2}{\alpha}} + \frac{\sin^{2}{\alpha}}{\sin^{2}{\alpha}}} \\
&= \frac{1}{\cot^{2}{\alpha} + 1}
\end{aligned}$

\subsolution %h
$\begin{aligned}[t]
\frac{1 - \cos{\alpha}}{1 + \cos{\alpha}} &= \frac{(1 - \cos{\alpha})(1 + \cos{\alpha})}{(1 + \cos{\alpha})(1 + \cos{\alpha})} = \frac{1 + \cos{\alpha} - \cos{\alpha} - \cos^{2}{\alpha}}{(1 + \cos{\alpha})^2} \\
&= \frac{1 - \cos^{2}{\alpha}}{(1 + \cos{\alpha})^2} = \frac{\sin^{2}{\alpha}}{(1 + \cos{\alpha})^2} \\
&= \left(\frac{\sin{\alpha}}{1 + \cos{\alpha}}\right)^2
\end{aligned}$

\subsolution %i
The key to solving this one is the formula for factoring a difference of cubes: $a^3 - b^3 = (a - b)(a^2 + ab + b^2)$.
\begin{align*}
\frac{\sin^{3}{\alpha} - \cos^{3}{\alpha}}{\sin{\alpha} - \cos{\alpha}} &= \frac{(\sin{\alpha} - \cos{\alpha})(\sin^{2}{\alpha} + \sin{\alpha}\cos{\alpha} + \cos^{2}{\alpha})}{\sin{\alpha} - \cos{\alpha}} \\
&= \sin^{2}{\alpha} + \sin{\alpha}\cos{\alpha} + \cos^{2}{\alpha} \\
&= 1 + \sin{\alpha}\cos{\alpha}
\end{align*}
\end{subsolutions}

\solution %6
\begin{subsolutions}

\subsolution %a
We can rewrite the LHS to show that $\sin^{4}{\alpha} - \cos^{4}{\alpha} = \cos^{2}{\alpha} - \sin^{2}{\alpha}$:
\begin{align*}
\sin^{4}{\alpha} - \cos^{4}{\alpha} &= (\sin^{2}{\alpha} + \cos^{2}{\alpha})(\sin^{2}{\alpha} - \cos^{2}{\alpha}) = 1(\sin^{2}{\alpha} - \cos^{2}{\alpha}) \\
&= \sin^{2}{\alpha} - \cos^{2}{\alpha}
\end{align*}
Answer: There are no angles $\alpha$ for which $\sin^{4}{\alpha} - \cos^{4}{\alpha} > \cos^{2}{\alpha} - \sin^{2}{\alpha}$ because the expressions on either side of the inequality are equivalent.

\subsolution %b
$\sin^{4}{\alpha} - \cos^{4}{\alpha} >= \cos^{2}{\alpha} - \sin^{2}{\alpha}$ for all angles $\alpha$ because the expressions on either side of the inequality are equivalent.

\end{subsolutions}

\solution %7
If we rewrite $2\sin{\alpha}\cos{\alpha}$ as a fraction, we can divide above and below by $\cos{\alpha}$ to convert the numerator and denominator into expressions in terms of $\tan{\alpha}$:
\begin{align*}
2\sin{\alpha}\cos{\alpha} &= \frac{2\sin{\alpha}\cos{\alpha}}{1} = \frac{2\sin{\alpha}\cos{\alpha}}{\sin^{2}{\alpha} + \cos^{2}{\alpha}} \\
&= \frac{\frac{2\sin{\alpha}\cos{\alpha}}{\cos^{2}{\alpha}}}{\frac{\sin^{2}{\alpha} + \cos^{2}{\alpha}}{\cos^{2}{\alpha}}} = \frac{\frac{2\sin{\alpha}}{\cos{\alpha}}}{\frac{\sin^{2}{\alpha}}{\cos^{2}{\alpha}} + \frac{\cos^{2}{\alpha}}{\cos^{2}{\alpha}}} \\
&= \frac{2\tan{\alpha}}{\tan^{2}{\alpha} + 1}
\end{align*}
Now we can plug in the given value for $\tan{\alpha}$ to find the value of $2\sin{\alpha}\cos{\alpha}$ in this instance:
\begin{equation*}
2\sin{\alpha}\cos{\alpha} = \frac{2\tan{\alpha}}{\tan^{2}{\alpha} + 1} = \frac{2(\frac{2}{5})}{(\frac{2}{5})^2 + 1} = \frac{\frac{4}{5}}{\frac{4}{25} + 1} = \frac{\frac{4}{5}}{\frac{4}{25} + \frac{25}{25}} = \frac{\frac{20}{25}}{\frac{29}{25}} = \frac{20}{29}
\end{equation*}


\solution %8
First, we will rewrite the expresssion $\cos^{2}{\alpha} - \sin^{2}{\alpha}$ in terms of $\tan{\alpha}$:
\begin{align*}
\cos^{2}{\alpha} - \sin^{2}{\alpha} &= \frac{\cos^{2}{\alpha} - \sin^{2}{\alpha}}{1} = \frac{\cos^{2}{\alpha} - \sin^{2}{\alpha}}{\cos^{2}{\alpha} + \sin^{2}{\alpha}} = \frac{\frac{\cos^{2}{\alpha} - \sin^{2}{\alpha}}{\cos^{2}{\alpha}}}{\frac{\cos^{2}{\alpha} + \sin^{2}{\alpha}}{\cos^{2}{\alpha}}} \\
&= \frac{1 - \tan^{2}{\alpha}}{1 + \tan^{2}{\alpha}}
\end{align*}

\begin{subsolutions}

\subsolution
To find the numerical value of $\cos^{2}{\alpha} - \sin^{2}{\alpha}$ when $\tan{\alpha} = \frac{2}{5}$ we can substitute $\frac{2}{5}$ for $\tan{\alpha}$ in the formula above:
\begin{equation*}
\cos^{2}{\alpha} - \sin^{2}{\alpha} = \frac{1 - \tan^{2}{\alpha}}{1 + \tan^{2}{\alpha}} = \frac{1 - \left(\frac{2}{5}\right)^2}{1 + \left(\frac{2}{5}\right)^2} = \frac{1 - \frac{4}{25}}{1 + \frac{4}{25}} = \frac{\frac{21}{25}}{\frac{29}{25}} = \frac{21}{29}
\end{equation*}

\subsolution
Substituting $r$ for $\tan{\alpha}$ in the formula above:
\begin{equation*}
\cos^{2}{\alpha} - \sin^{2}{\alpha} = \frac{1 - r^2}{1 + r^2}
\end{equation*}

\end{subsolutions}

\solution %9
First, we will rewrite the expresssion in terms of $\tan{\alpha}$:

\begin{equation*}
\frac{\sin{\alpha} - 2\cos{\alpha}}{\cos{\alpha} - 3\sin{\alpha}} = \frac{\frac{\sin{\alpha} - 2\cos{\alpha}}{\cos{\alpha}}}{\frac{\cos{\alpha} - 3\sin{\alpha}}{\cos{\alpha}}} = \frac{\frac{\sin{\alpha}}{\cos{\alpha}} - \frac{2\cos{\alpha}}{\cos{\alpha}}}{\frac{\cos{\alpha}}{\cos{\alpha}} - \frac{3\sin{\alpha}}{\cos{\alpha}}} = \frac{\tan{\alpha} - 2}{1 - 3\tan{\alpha}}
\end{equation*}

Next, we substitute  $\frac{2}{5}$ for $\tan{\alpha}$:
\begin{equation*}
\frac{\tan{\alpha} - 2}{1 - 3\tan{\alpha}} = \frac{\frac{2}{5} - 2}{1 - 3\left(\frac{2}{5}\right)} = \frac{\frac{2}{5} - \frac{10}{5}}{\frac{5}{5} - \frac{6}{5}} = \frac{-\frac{8}{5}}{-\frac{1}{5}} = 8
\end{equation*}


\solution %10
First, we will rewrite the expresssion in terms of $\tan{\alpha}$:
\begin{equation*}
\frac{a\sin{\alpha} + b\cos{\alpha}}{c\cos{\alpha} + d\sin{\alpha}} = \frac{\frac{a\sin{\alpha}}{\cos{\alpha}} + \frac{b\cos{\alpha}}{\cos{\alpha}}}{\frac{c\cos{\alpha}}{\cos{\alpha}} + \frac{d\cos{\alpha}}{\cos{\alpha}}} = \frac{a\tan{\alpha} + b}{c + d\tan{\alpha}}
\end{equation*}

Next, we substitute  $\frac{2}{5}$ for $\tan{\alpha}$ and simplify:
\begin{equation*}
\frac{a\tan{\alpha} + b}{c + d\tan{\alpha}} = \frac{a\left(\frac{2}{5}\right) + b\left(\frac{5}{5}\right)}{c\left(\frac{5}{5}\right) + d\left(\frac{2}{5}\right)} = \frac{\frac{2a + 5b}{5}}{\frac{5c+2d}{5}} = \frac{2a + 5b}{5c + 2d}
\end{equation*}

Now we can see why the problem included the restriction that $5c + 2d \neq 0$; the value of the expression is undefined if the denominator is zero. The sum of two rational numbers is a rational number. Therefore the numerator and denominator in the expression are both rational numbers. The quotient of two rational numbers is a rational number. Therefore, the entire expression evaluates to a rational number for arbitratrary rational values of $a$, $b$, $c$ and $d$.

\solution %11
We can expand and simplify the expression:
\begin{align*}
& (\sin{\alpha} + \cos{\alpha})^2 + (\sin{\alpha} - \cos{\alpha})^2 \\
&= \sin^{2}{\alpha} + 2\sin{\alpha}\cos{\alpha} + \cos^{2}{\alpha} + \sin^{2}{\alpha} - 2\sin{\alpha}\cos{\alpha} + \cos^{2}{\alpha} \\
&= 2\sin^{2}{\alpha} + 2\cos^{2}{\alpha} \\
&= 2(\sin^{2}{\alpha} + \cos^{2}{\alpha}) \\
&= 2(1) \\
&= 2
\end{align*}
As the expression evaluates to a constant, it is as large as possible for all values of $\alpha$.

\end{solutions}

\begin{solutions}{Page 49}

\solution %1

Rewriting any instances of $\sec{\alpha}$ or $\csc{\alpha}$ on either side of the identities:

\begin{subsolutions}

\subsolution %a
$\begin{aligned}[t]
\tan{\alpha}\csc{\alpha} &= \sec{\alpha} \\
\tan{\alpha}\dfrac{1}{\sin{\alpha}} &= \dfrac{1}{\cos{\alpha}} \\[1ex]
\dfrac{\tan{\alpha}}{\sin{\alpha}} &= \dfrac{1}{\cos{\alpha}}
\end{aligned}$

\subsolution %b
$\begin{aligned}[t]
\cot{\alpha}\csc{\alpha} &= \sec{\alpha} \\
\cot{\alpha}\dfrac{1}{\cos{\alpha}} &= \dfrac{1}{\sin{\alpha}} \\[1ex]
\dfrac{\cot{\alpha}}{\cos{\alpha}} &= \dfrac{1}{\sin{\alpha}}
\end{aligned}$

\subsolution %c
$\begin{aligned}[t]
\dfrac{1}{\sec{\alpha}}\csc{\alpha} &= \cot{\alpha} \\
\dfrac{1}{\frac{1}{\cos{\alpha}}}\cdot\dfrac{1}{\sin{\alpha}} &= \cot{\alpha} \\
\cos{\alpha}\dfrac{1}{\sin{\alpha}} &= \cot{\alpha} \\[1ex]
\dfrac{\cos{\alpha}}{\sin{\alpha}} &= \cot{\alpha}
\end{aligned}$

\subsolution %d
$\begin{aligned}[t]
\tan^{2}{\alpha} &= (\sec{\alpha} + 1)(\sec{\alpha} - 1) \\
\tan^{2}{\alpha} &= \sec^{2}{\alpha} - 1 \\
\tan^{2}{\alpha} &= \dfrac{1}{\cos^{2}{\alpha}} - 1
\end{aligned}$

\subsolution %e
$\begin{aligned}[t]
\csc^{2}{\alpha} &= 1 + \cot^{2}{\alpha} \\
\dfrac{1}{\sin^{2}{\alpha}} &= 1 + \cot^{2}{\alpha}
\end{aligned}$

\end{subsolutions}


\solution %2
Rewriting any instances of $\sin{\alpha}$ or $\cos{\alpha}$ on either side of the identities, and eliminating fractions:
% TODO are these correct? Some of the solutions look simplistic
\begin{subsolutions}

\subsolution %a
$\begin{aligned}[t]
\frac{\tan{\alpha}}{\sin{\alpha}} = \frac{1}{\cos{\alpha}} \\
\tan{\alpha}\frac{1}{\sin{\alpha}} = \sec{\alpha} \\
\tan{\alpha}\csc{\alpha} = \sec{\alpha}
\end{aligned}$

\subsolution %b
$\begin{aligned}[t]
\frac{1}{\sin{\alpha}}\cos{\alpha} = \cot{\alpha} \\
\frac{\cos{\alpha}}{\sin{\alpha}} = \cot{\alpha} \\
\cot{\alpha} = \cot{\alpha}
\end{aligned}$

\subsolution %c
$\begin{aligned}[t]
\tan^{2}{\alpha} + 1 = \frac{1}{\cos^{2}{\alpha}} \\
\tan^{2}{\alpha} + 1 = \sec^{2}{\alpha}
\end{aligned}$

\subsolution %d
$\begin{aligned}[t]
\frac{1}{\sin^{2}{\alpha}} = 1 + \cot^{2}{\alpha} \\
\csc^{2}{\alpha} = 1 + \cot^{2}{\alpha}
\end{aligned}$

\end{subsolutions}

\end{solutions}



\begin{solutions}{Page 50}

\solution % 1
First, we find the value of $a^2 + b^2$:
\begin{align*}
a^2 + b^2 &= (\cos^{2}{\alpha} - \sin^{2}{\alpha})^2 + (2\sin{\alpha}\cos{\alpha})^2 \\
&= \cos^{4}{\alpha} - 2\cos^{2}{\alpha}\sin^{2}{\alpha} + \sin^{4}{\alpha} + 4\sin^{2}{\alpha}\cos^{2}{\alpha} \\
&= \cos^{4}{\alpha} + 2\cos^{2}{\alpha}\sin^{2}{\alpha} + \sin^{4}{\alpha} \\
&= (\cos^{2}{\alpha} + \sin^{2}{\alpha})^2 \\
&= (1)^2 \\
&= 1
\end{align*}
According to the lemma on Page 50, as $a^2 + b^2 = 1$, an angle $\theta$ exists such that $a = \cos{\theta}$ and $b = \sin{\theta}$.

\solution %2
First, we find the value of $a^2 + b^2$:
\begin{align*}
a^2 + b^2 &= \left(\sqrt{\frac{1 + \cos{\alpha}}{2}}\right)^2 + \left(\sqrt{\frac{1 - \cos{\alpha}}{2}}\right)^2 \\
&= \frac{1 + \cos{\alpha}}{2} + \frac{1 - \cos{\alpha}}{2} \\
&= \frac{1 + \cos{\alpha} + 1 - \cos{\alpha}}{2} \\
&= \frac{2}{2} \\
&= 1
\end{align*}

\solution %3
First, we will rewrite $a$ and $b$ to eliminate the cube exponents:
\begin{align*}
a &= 4\cos^{3}{\alpha} - 3\cos{\alpha} \\
&= 4\cos{\alpha}\cos^{2}{\alpha} - 3\cos{\alpha} \\
&= 4\cos{\alpha}(1 - \sin^{2}{\alpha}) - 3\cos{\alpha} \\
&= 4\cos{\alpha} - 4\sin^{2}{\alpha}\cos{\alpha} - 3\cos{\alpha} \\
&= \cos{\alpha} - 4\sin^{2}{\alpha}\cos{\alpha}
\end{align*}
\begin{align*}
b &= 3\sin{\alpha} - 4\sin^{3}{\alpha} \\
&= 3\sin{\alpha} - 4\sin{\alpha}\sin^{2}{\alpha} \\
&= 3\sin{\alpha} - 4\sin{\alpha}(1 - \cos^{2}{\alpha}) \\
&= -\sin{\alpha} + 4\sin{\alpha}\cos^{2}{\alpha}
\end{align*}
Next, we will expand $a^2$ and $b^2$:
\begin{align*}
a^2 &= (\cos{\alpha} - 4\sin^{2}{\alpha}\cos{\alpha})^2 \\
&= \cos^{2}{\alpha} - 8\sin^{2}{\alpha}\cos^{2}{\alpha} + 16\sin^{4}{\alpha}\cos^{2}{\alpha}
\end{align*}
\begin{align*}
b^2 &= (-\sin{\alpha} + 4\sin{\alpha}\cos^{2}{\alpha})^2 \\
&= \sin^{2}{\alpha} - 8\sin^{2}{\alpha}\cos^{2}{\alpha} + 16\sin^{2}{\alpha}\cos^{4}{\alpha}
\end{align*}
Next, we add the expressions for $a^2$ and $b^2$ and simplify to 1:
\begin{align*}
a^2 + b^2 &= \cos^{2}{\alpha} - 8\sin^{2}{\alpha}\cos^{2}{\alpha} + 16\sin^{4}{\alpha}\cos^{2}{\alpha} +\sin^{2}{\alpha} - 8\sin^{2}{\alpha}\cos^{2}{\alpha} + \\
&\quad\quad16\sin^{2}{\alpha}\cos^{4}{\alpha} \\
&= \cos^{2}{\alpha} + \sin^{2}{\alpha} - 16\sin^{2}{\alpha}\cos^{2}{\alpha} + 16\sin^{4}{\alpha}\cos^{2}{\alpha} + 16\sin^{2}{\alpha}\cos^{4}{\alpha} \\
&= \cos^{2}{\alpha} + \sin^{2}{\alpha} + 16\sin^{2}{\alpha}\cos^{2}{\alpha}(-1 + \sin^{2}{\alpha} + \cos^{2}{\alpha}) \\
&= 1 + 16\sin^{2}{\alpha}\cos^{2}{\alpha}(0) \\
&= 1
\end{align*}
According to the lemma on Page 50, as $a^2 + b^2 = 1$, an angle $\theta$ exists such that $a = \cos{\theta}$ and $b = \sin{\theta}$.

\solution %4
First, we find the value of $a^2 + b^2$:
\begin{align*}
a^2 + b^2 &= \left(\frac{1 - t^2}{1 + t^2}\right)^2 + \left(\frac{2t}{1 + t^2}\right)^2 \\[1ex]
&= \frac{(1 - t^2)^2}{(1 + t^2)^2} + \frac{(2t)^2}{(1 + t^2)^2} \\[1ex]
&= \frac{(1 - t^2)^2 + (2t)^2}{(1 + t^2)^2} \\[1ex]
&= \frac{1 - 2t^2 + t^4 + 4t^2}{(1 + t^2)(1 + t^2)} \\[1ex]
&= \frac{(1 + t^2)(1 + t^2)}{(1 + t^2)(1 + t^2)} \\
&= 1
\end{align*}
According to the lemma on Page 50, as $a^2 + b^2 = 1$, an angle $\theta$ exists such that $a = \cos{\theta}$ and $b = \sin{\theta}$.

\solution %5
We expand $(p^2 - q^2)^2 + (2pq)^2$ and use the fact that $p^2 + q^2 = 1$ to simplify to 1:
\begin{align*}
(p^2 - q^2)^2 + (2pq)^2 &= p^{4} - 2p^2q^2 + q^4 + 4p^2q^2 \\
&= p^{4} + 2p^2q^2 + q^4 \\
&= (p^2 + q^2)^2 \\
&= (1)^2 \\
&= 1
\end{align*}
This is similar to Exercise 1 above.

\end{solutions}


\begin{solutions}{Page 51}
\solution %1
$\sin{\alpha} < 1$ when $\alpha$ is acute, therefore $1 - \sin{\alpha} > 0$ when $\alpha$ is acute. $1 - \sin{\alpha} = 0$ when $\sin{\alpha} = 1$, i.e., $\alpha = 90\degree$.

\solution %2
$\cos{\alpha} < 1$ when $\alpha$ is acute, therefore $1 - \cos{\alpha} > 0$ when $\alpha$ is acute. $1 - \cos{\alpha} = 0$ when $\cos{\alpha} = 1$, i.e., $\alpha = 0\degree$.

\solution %3
Statement a) is always true. Statements b) and c) both include the case that $\sin^{2}{\alpha} + \cos^{2}{\alpha} = 1$, which is always true.

\solution %4
Let $x$ be the maximum cost of the items in a supermarket. In Supermarket A, $x \leq \$1$. In Supermarket B, $x < \$1$. In Supermarket C, $x \leq \$1$. In Supermarket D, $x > \$1$. We can see that Supermarkets A and C are offering the same terms.

\solution %5
Inequality a) is correct. For b) to be correct, an angle $\alpha$ would have to exist such that $\sin{\alpha} + \cos{\alpha} = 2$.
We know that this is not the case. When $\alpha = 90\degree$, $\sin{\alpha} = 1$ and $\cos{\alpha} = 0$. When $\alpha = 0\degree$, $\sin{\alpha} = 0$ and $\cos{\alpha} = 1$. When $0\degree < \alpha < 90\degree$, $\sin{\alpha} < 1$ and $\cos{\alpha} < 1$. In all cases, $\sin{\alpha} + \cos{\alpha} < 2$.

\solution %6
The largest possible value of $\sin{\alpha}$ is 1, and occurs when $\alpha = 90\degree$. The largest possible value of $\cos{\alpha}$ is 1, and occurs when $\alpha = 0\degree$.
See Page 32.
\end{solutions}

\begin{solutions}{Page 52}

\solution %1
$\sin{30\degree} = 0.5$, $\sin{45\degree} = 0.707$, $\sin{60\degree} = 0.866$.

\solution %2
By using the \texttt{tan} button to calculate $\tan{60\degree}$, and the \texttt{sqrt} button to calculate $\sqrt{3}$, Betty can compare the results: both are $1.732$.

\solution %3
Press \texttt{tan}, then enter the angle degree measure, then press \texttt{1/$x$}

\solution ~ %4
\begin{center}
\bgroup
\def\arraystretch{2.2}
\setlength\tabcolsep{15pt}
\begin{tabular}{ |c|c|c|c|c| }
\hline
\multicolumn{5}{ |c| }{in radical or rational form} \\
\hline
$\alpha$    & $\sin{\alpha}$        & $\cos{\alpha}$        & $\tan{\alpha}$        & $\cot{\alpha}$ \\
\hline
$30\degree$ & $\dfrac{1}{2}$        & $\dfrac{\sqrt{3}}{2}$ & $\dfrac{1}{\sqrt{3}}$ & $\sqrt{3}$ \\
\hline
$45\degree$ & $\dfrac{1}{\sqrt{2}}$ & $\dfrac{1}{\sqrt{2}}$ & $1$                   & $1$ \\
\hline
$60\degree$ & $\dfrac{\sqrt{3}}{2}$ & $\dfrac{1}{2}$        & $\sqrt{3}$            & $\dfrac{1}{\sqrt{3}}$ \\
\hline
\end{tabular}
\egroup
\end{center}
\begin{center}
\bgroup
\def\arraystretch{2.1}
\setlength\tabcolsep{15pt}
\begin{tabular}{ |c|c|c|c|c| }
\hline
\multicolumn{5}{ |c| }{in decimal form, from calculator} \\
\hline
$\alpha$    & $\sin{\alpha}$ & $\cos{\alpha}$ & $\tan{\alpha}$  & $\cot{\alpha}$ \\
\hline
$30\degree$ & $0.5$          & $0.866$        & $0.577$         & $1.732$ \\
\hline
$45\degree$ & $0.707$        & $0.707$        & $1$             & $1$ \\
\hline
$60\degree$ & $0.866$        & $0.5$          & $1.732$         & $0.577$ \\
\hline
\end{tabular}
\egroup
\end{center}

\end{solutions}


\begin{solutions}{Page 53}
\solution %1
The sine of the larger angle is $4/5=.8$. We can use the inverse sine function to find the angle: $\arcsin{.8} = 53.1301\degree$. The sum of the three angles in the triangles is: $\arcsin{.6} + \arcsin{.8} + 90\degree = 36.8699\degree + 53.1301\degree + 90\degree = 180\degree$.

\solution %2
\begin{subsolutions}
\item $\arcsin{1} = 90\degree$
\item $\arccos{0.7071067811865} = 45\degree$
\end{subsolutions}

\solution %3
$\arccos{0.8} = 36.8699\degree$

\solution %4
$\arcsin{0.6} = 36.8699\degree$

\solution %5
Half of $\sin{30\degree}$ (0.25) seems like a reasonable estimate. The actual value is $0.2588$.

\solution %6

\solution %7

\solution %8

\solution %9

\solution %10

\solution %11

\solution %12

\solution %13

\solution %14

\end{solutions}

\begin{solutions}{Page 55}
\solution
\solution
\solution
\solution
\end{solutions}

\begin{solutions}{Page 56}
\solution
\solution
\solution
\solution
\end{solutions}

\begin{solutions}{Page 59}
\solution
\solution
\solution
\solution
\solution
\solution
\solution
\solution
\solution
\solution
\solution
\end{solutions}

\begin{solutions}{Page 62}
\solution %1
The degree measure of a semicircle is $180\degree$. The degree measure of a quarter circle is $90\degree$.
\solution %2
The measure of arc cut off by one side of regular pentagon inscribed in a circle is $360\degree/5 = 72\degree$.
For a regular hexagon: $360\degree/6 = 60\degree$.
For a regular octagon: $360\degree/8 = 45\degree$.
\end{solutions}

\begin{solutions}{Page 64}
\solution
\solution
\solution
\end{solutions}

\begin{solutions}{Page 65}
\solution
\solution
\solution
\solution
\solution
\end{solutions}


\section*{Chapter 3: Relationships in a Triangle}

\begin{solutions}{Page 68}
\solution
\end{solutions}

\begin{solutions}{Page 70}
\solution
\solution
\solution
\solution
\end{solutions}

\begin{solutions}{Page 71}
\solution
\end{solutions}

\begin{solutions}{Page 73}
\solution
\solution
\solution
\solution
\solution
\solution
\solution
\solution
\end{solutions}

\begin{solutions}{Page 75 (First)}
\solution
\solution
\end{solutions}

\begin{solutions}{Page 75 (Second)}
\solution
\solution
\solution
\solution
\solution
\solution
\solution
\solution
\solution
\solution
\solution
\solution
\solution
\end{solutions}

\begin{solutions}{Page 79}
\solution
\end{solutions}

\begin{solutions}{Page 80}
\solution
\solution
\solution
\solution
\solution
\solution
\solution
\solution
\solution
\solution
\solution
\solution
\solution
\solution
\end{solutions}

\begin{solutions}{Page 83}
\solution
\solution
\solution
\end{solutions}

\begin{solutions}{Page 84}
\solution
\solution
\end{solutions}

\begin{solutions}{Page 85}
\solution
\solution
\end{solutions}

\begin{solutions}{Page 86}
\solution
\end{solutions}

\begin{solutions}{Page 88}
\solution
\solution
\solution
\solution
\solution
\end{solutions}

\section*{Chapter 4: Angles and Rotations}

\begin{solutions}{Page 93}
\solution %1
\begin{subsolutions}
\subsolution %a
\begin{tikzpicture}[scale=0.5]
\def\rad{4};
\def\ang{160};
\draw (0,0) circle [radius=\rad];
\draw (0,0) -- (\ang:\rad);
\draw[fill=black] (\ang:\rad) circle [radius=0.1];
\draw (0,0) -- (0:\rad);
\draw[fill=black] (0:\rad) circle [radius=0.1] node[above right] {$A$};
\draw [-latex,domain=0:\ang] plot ({1.5*cos(\x)}, {1.5*sin(\x)});
\node[below=.05cm] at (\ang:1.5) {$\ang\degree$};
\end{tikzpicture}

\subsolution %b
\begin{tikzpicture}[scale=0.5]
\def\rad{4};
\def\ang{190};
\draw (0,0) circle [radius=\rad];
\draw (0,0) -- (\ang:\rad);
\draw[fill=black] (\ang:\rad) circle [radius=0.1];
\draw (0,0) -- (0:\rad);
\draw[fill=black] (0:\rad) circle [radius=0.1] node[above right] {$A$};
\draw [-latex,domain=0:\ang] plot ({1.5*cos(\x)}, {1.5*sin(\x)});
\node[below=.05cm] at (\ang:1.5) {$\ang\degree$};
\end{tikzpicture}

\subsolution %c
\begin{tikzpicture}[scale=0.5]
\def\rad{4};
\def\ang{400};
\draw (0,0) circle [radius=\rad];
\draw (0,0) -- (\ang:\rad);
\draw[fill=black] (\ang:\rad) circle [radius=0.1];
\draw (0,0) -- (0:\rad);
\draw[fill=black] (0:\rad) circle [radius=0.1] node[above right] {$A$};
\bigangle{\ang} node[right=.25cm] {$\ang\degree$};
\end{tikzpicture}

\subsolution %d
\begin{tikzpicture}[scale=0.5]
\def\rad{4};
\def\ang{600};
\draw (0,0) circle [radius=\rad];
\draw (0,0) -- (\ang:\rad);
\draw[fill=black] (\ang:\rad) circle [radius=0.1];
\draw (0,0) -- (0:\rad);
\draw[fill=black] (0:\rad) circle [radius=0.1] node[above right] {$A$};
\bigangle{\ang} node[below right=.1cm] {$\ang\degree$};
\end{tikzpicture}

\subsolution %e
\begin{tikzpicture}[scale=0.5]
\def\rad{4};
\def\ang{1200};
\draw (0,0) circle [radius=\rad];
\draw (0,0) -- (\ang:\rad);
\draw[fill=black] (\ang:\rad) circle [radius=0.1];
\draw (0,0) -- (0:\rad);
\draw[fill=black] (0:\rad) circle [radius=0.1] node[above right] {$A$};
\bigangle{\ang} node[above right=.3cm] {$\ang\degree$};
\end{tikzpicture}

\subsolution %f
\begin{tikzpicture}[scale=0.5]
\def\rad{4};
\def\ang{-70};
\draw (0,0) circle [radius=\rad];
\draw (0,0) -- (\ang:\rad);
\draw[fill=black] (\ang:\rad) circle [radius=0.1];
\draw (0,0) -- (0:\rad);
\draw[fill=black] (0:\rad) circle [radius=0.1] node[above right] {$A$};
\draw [-latex,domain=0:\ang] plot ({1.5*cos(\x)}, {1.5*sin(\x)});
\node[left=.05cm] at (\ang:1.5) {$\ang\degree$};
\end{tikzpicture}

\subsolution %g
\begin{tikzpicture}[scale=0.5]
\def\rad{4};
\def\ang{-400};
\draw (0,0) circle [radius=\rad];
\draw (0,0) -- (\ang:\rad);
\draw[fill=black] (\ang:\rad) circle [radius=0.1];
\draw (0,0) -- (0:\rad);
\draw[fill=black] (0:\rad) circle [radius=0.1] node[above right] {$A$};
\bigangle{\ang} node[below left=.05cm] {$\ang\degree$};
\end{tikzpicture}

\subsolution %h
\begin{tikzpicture}[scale=0.5]
\def\rad{4};
\def\ang{360};
\draw (0,0) circle [radius=\rad];
\draw (0,0) -- (\ang:\rad);
\draw[fill=black] (\ang:\rad) circle [radius=0.1];
\draw (0,0) -- (0:\rad);
\draw[fill=black] (0:\rad) circle [radius=0.1] node[above right] {$A$};
\bigangle{\ang} node[above=.05cm] {$\ang\degree$};
\end{tikzpicture}

\subsolution %i
\begin{tikzpicture}[scale=0.5]
\def\rad{4};
\def\ang{-270};
\draw (0,0) circle [radius=\rad];
\draw (0,0) -- (\ang:\rad);
\draw[fill=black] (\ang:\rad) circle [radius=0.1];
\draw (0,0) -- (0:\rad);
\draw[fill=black] (0:\rad) circle [radius=0.1] node[above right] {$A$};
\draw [-latex,domain=0:\ang] plot ({1.5*cos(\x)}, {1.5*sin(\x)});
\node[right=.05cm] at (\ang:1.5) {$\ang\degree$};
\end{tikzpicture}

\end{subsolutions}

\end{solutions}

\begin{solutions}{Page 98}
\solution %1
\begin{subsolutions}
\subsolution %a
\begin{tikzpicture}
\def\rad{1.5};
\def\fac{1.5};
\def\ang{30};

\draw (0,0) circle [radius=\rad];
\draw[-latex] (-\fac*\rad,0) -- (\fac*\rad,0);
\draw[-latex] (0,-\fac*\rad) -- (0,\fac*\rad);

\draw (0,0) -- (\ang:\rad);
\draw[fill=black] (\ang:\rad) circle [radius=0.05];
\draw [-latex,domain=0:\ang] plot ({0.5*cos(\x)}, {0.5*sin(\x)});
\node[above=.05cm] at (\ang:0.5) {$\ang\degree$};
\end{tikzpicture}

Because $390\degree = 360\degree + 30\degree$, a point rotated through an angle of $390\degree$ would end up at the same location as a point rotated through an angle of $30\degree$. Therefore,
\[
\sin{390\degree} = \sin{30\degree} = \dfrac{1}{2}.
\]

\subsolution %b
\begin{tikzpicture}
\def\rad{1.5};
\def\fac{1.5};
\def\ang{120};

\draw (0,0) circle [radius=\rad];
\draw[-latex] (-\fac*\rad,0) -- (\fac*\rad,0);
\draw[-latex] (0,-\fac*\rad) -- (0,\fac*\rad);

\draw (0,0) -- (\ang:\rad);
\draw[fill=black] (\ang:\rad) circle [radius=0.05];
\draw [-latex,domain=0:\ang] plot ({0.5*cos(\x)}, {0.5*sin(\x)});
\node[above right=.15cm] at (\ang:0.5) {$\ang\degree$};

\draw [-latex,domain=180:\ang] plot ({0.5*cos(\x)}, {0.5*sin(\x)});
\node[left=.05cm] at (\ang:0.5) {$60\degree$};
\end{tikzpicture}

Because $3720\degree = 10 \cdot 360\degree + 120\degree$, a point rotated through an angle of $3720\degree$ would end up at the same location as a point rotated through an angle of $120\degree$. Therefore,
\[
\cos{3720\degree} = \cos{120\degree} = -\cos{60\degree} = -\dfrac{1}{2}.
\]

\subsolution %c
\begin{tikzpicture}
\def\rad{1.5};
\def\fac{1.5};
\def\ang{45};

\draw (0,0) circle [radius=\rad];
\draw[-latex] (-\fac*\rad,0) -- (\fac*\rad,0);
\draw[-latex] (0,-\fac*\rad) -- (0,\fac*\rad);

\draw (0,0) -- (\ang:\rad);
\draw[fill=black] (\ang:\rad) circle [radius=0.05];
\draw [-latex,domain=0:\ang] plot ({0.5*cos(\x)}, {0.5*sin(\x)});
\node[above=.05cm] at (\ang:0.5) {$\ang\degree$};
\end{tikzpicture}

Because $1845\degree = 5 \cdot 360\degree + 45\degree$, a point rotated through an angle of $1845\degree$ would end up at the same location as a point rotated through an angle of $45\degree$. Therefore,
\[
\tan{1845\degree} = \tan{45\degree} = \dfrac{\sin{45\degree}}{\cos{45\degree}} = \dfrac{\sqrt{2}/2}{\sqrt{2}/2} = 1.
\]
\subsolution %d
\begin{tikzpicture}
\def\rad{1.5};
\def\fac{1.5};
\def\ang{315};

\draw (0,0) circle [radius=\rad];
\draw[-latex] (-\fac*\rad,0) -- (\fac*\rad,0);
\draw[-latex] (0,-\fac*\rad) -- (0,\fac*\rad);

\draw (0,0) -- (\ang:\rad);
\draw[fill=black] (\ang:\rad) circle [radius=0.05];
\draw [-latex,domain=0:\ang] plot ({0.5*cos(\x)}, {0.5*sin(\x)});
\node[below=.1cm] at (\ang:0.5) {$\ang\degree$};
\draw [-latex,domain=360:\ang] plot ({0.5*cos(\x)}, {0.5*sin(\x)});
\node[right=.05cm] at (\ang:0.5) {$45\degree$};
\end{tikzpicture}

Because $315\degree = 360\degree - 45\degree$, a point rotated through an angle of $315\degree$ would end up at the same location as a point rotated through an angle of $-45\degree$. Therefore,
\[
\sin{315\degree} = \sin\left(-45\degree\right) = -\sin{45\degree} = -\dfrac{\sqrt{2}}{2}
\]

\subsolution %e
\begin{tikzpicture}
\def\rad{1.5};
\def\fac{1.5};
\def\ang{60};

\draw (0,0) circle [radius=\rad];
\draw[-latex] (-\fac*\rad,0) -- (\fac*\rad,0);
\draw[-latex] (0,-\fac*\rad) -- (0,\fac*\rad);

\draw (0,0) -- (\ang:\rad);
\draw[fill=black] (\ang:\rad) circle [radius=0.05];
\draw [-latex,domain=0:\ang] plot ({0.5*cos(\x)}, {0.5*sin(\x)});
\node[right=.05cm] at (\ang:0.5) {$\ang\degree$};
\end{tikzpicture}

Because $420\degree = 360\degree + 60\degree$, a point rotated through an angle of $420\degree$ would end up at the same location as a point rotated through an angle of $60\degree$. Therefore,
\[
\cot{420\degree} = \cot{60\degree} = \dfrac{\cos{60\degree}}{\sin{60\degree}} = \dfrac{1/2}{\sqrt{3}/2} = \dfrac{1}{\sqrt{3}}.
\]

\subsolution %f
\begin{tikzpicture}
\def\rad{1.5};
\def\fac{1.5};
\def\ang{-30};

\draw (0,0) circle [radius=\rad];
\draw[-latex] (-\fac*\rad,0) -- (\fac*\rad,0);
\draw[-latex] (0,-\fac*\rad) -- (0,\fac*\rad);

\draw (0,0) -- (\ang:\rad);
\draw[fill=black] (\ang:\rad) circle [radius=0.05];
\draw [-latex,domain=0:\ang] plot ({0.5*cos(\x)}, {0.5*sin(\x)});
\node[right=.05cm] at (\ang:0.5) {$\ang\degree$};
\end{tikzpicture}

\[
\tan\left(-30\degree\right) = \dfrac{\sin\left(-30\degree\right)}{\cos\left(-30\degree\right)} = \dfrac{-\sin{30\degree}}{\cos{30\degree}} = -\dfrac{1/2}{\sqrt{3}/2} = -\dfrac{1}{\sqrt{3}}
\]

\end{subsolutions}

\solution %2
\begin{subsolutions}
\subsolution %a
\[
\tan{360\degree} = \tan{0\degree} = \dfrac{\sin{0\degree}}{\cos{0\degree}} = \dfrac{0}{1} = 0
\]
\subsolution %b
\[
\sin{180\degree} = \sin{0\degree} = 0
\]
\subsolution %c
\[
\cos{180\degree} = -\cos{0\degree} = -1
\]
\subsolution %d
\[
\cot{90\degree} = \dfrac{\cos{90\degree}}{\sin{90\degree}} = \dfrac{0}{1} = 0
\]
\subsolution %e
\[
\cot{360\degree} = \cot{0\degree} = \dfrac{\cos{0\degree}}{\sin{0\degree}} = \dfrac{1}{0} \implies \cot{360\degree} \text{ is undefined.}
\]
\subsolution %f
\[
\tan\left(-270\degree\right) = \tan{90\degree} = \dfrac{\sin{90\degree}}{\cos{90\degree}} = \dfrac{1}{0} \implies \tan\left(-270\degree\right) \text{ is undefined.}
\]
\end{subsolutions}
\end{solutions}

\begin{solutions}{Page 100}
\solution %1
$400\degree = 360\degree + 40\degree$, so $P$ will lie in the first quadrant.

$3600\degree = 10 \cdot 360\degree + 0\degree$, so $P$ will lie in the first quadrant (on the positive x-axis).

$1845\degree = 5 \cdot 360\degree + 45\degree$, so $P$ will lie in the first quadrant.

$-30\degree = -360\degree + 330\degree$, so $P$ will lie in the fourth quadrant.

$-359\degree = -360\degree + 1\degree$, so $P$ will lie in the first quadrant.

\solution ~ %2
\begin{center}
\bgroup
\def\arraystretch{2.1}
\setlength\tabcolsep{15pt}
\begin{tabular}{ |c|c|c|c| }
\hline
$\sin{30\degree}$
& $1/2$
& $\sin\left(-30\degree\right)$
& $-1/2$ \\
\hline
$\sin{135\degree}$
& $\sqrt{2}/2$
& $\sin\left(-135\degree\right)$
& $-\sqrt{2}/2$ \\
\hline
$\sin{210\degree}$
& $-1/2$
& $\sin\left(-210\degree\right)$
& $1/2$ \\
\hline
$\sin{300\degree}$
& $-\sqrt{3}/2$
& $\sin\left(-300\degree\right)$
& $\sqrt{3}/2$ \\
\hline
$\sin{390\degree}$
& $1/2$
& $\sin\left(-390\degree\right)$
& $-1/2$ \\
\hline
$\sin{480\degree}$
& $\sqrt{3}/2$
& $\sin\left(-480\degree\right)$
& $-\sqrt{3}/2$ \\
\hline
\end{tabular}
\egroup
\end{center}

From the table, we can see that $\sin\left(-\alpha\right)=-\sin{\alpha}$.

\solution %3
\begin{subsolutions}
\subsolution %a
\[
\sin{\alpha}=0 \implies \alpha=180\degree
\]
\subsolution %b
\[
\cos{\alpha}=0 \implies \alpha=90\degree, 270\degree
\]
\subsolution %c
\[
\sin{\alpha}=1 \implies \alpha=90\degree
\]
\subsolution %d
$\cos{\alpha}=1$ is true for $\alpha=0\degree$ and $\alpha=360\degree$, but these are not in the interval $0 < \alpha < 360\degree$.
\subsolution %e
\[
\sin{\alpha}=-1 \implies \alpha=270\degree
\]
\subsolution %f
\[
\cos{\alpha}=\dfrac{1}{2} \implies \alpha=60\degree,300\degree
\]
\subsolution %g
\[
\sin{\alpha}=-\dfrac{1}{2} \implies \alpha=210\degree,330\degree
\]
\subsolution %h
\[
\sin^{2}{\alpha} = \dfrac{1}{2} \implies \sin{\alpha} = \pm \dfrac{\sqrt{2}}{2} \implies \alpha=45\degree,135\degree,225\degree,315\degree
\]
\subsolution %i
This equation has no solutions because the square of a real number cannot be negative.

\end{subsolutions}
\solution %4
\begin{subsolutions}
\subsolution %a
If $\sin{\alpha}=5/13$, then $\alpha$ can lie in either the first or second quadrant. $\cos{\alpha}=12/13$ when $\alpha$ is in the first quadrant, and $\cos{\alpha}=-12/13$ when $\alpha$ is in the second quadrant.

\subsolution %b
If $\sin{\alpha}=-5/13$, then $\alpha$ can lie in either the third or fourth quadrant. $\cos{\alpha}=12/13$ when $\alpha$ is in the fourth quadrant, and $\cos{\alpha}=-12/13$ when $\alpha$ is in the third quadrant.

\end{subsolutions}

\solution ~ %5
\begin{center}
\begin{tikzpicture}
\def\rad{1.5};
\def\fac{1.5};
\def\ang{130};

\draw (0,0) circle [radius=\rad];
\draw[-latex] (-\fac*\rad,0) -- (\fac*\rad,0);
\draw[-latex] (0,-\fac*\rad) -- (0,\fac*\rad);

\draw (0,0) -- (\ang:\rad);
\draw[fill=black] (\ang:\rad) circle [radius=0.05] node[above left] {$P=(a,b)$};
\draw [-latex,domain=0:\ang] plot ({0.5*cos(\x)}, {0.5*sin(\x)});
\node[above right] at (\ang/2:0.5) {$\theta$};
\end{tikzpicture}
\end{center}

Let $P = (a,b)$ be a point on the coordinate plane. Since $a^2+b^2=1$, $P$ lies on a circle of radius 1 centered at the origin (\textit{the unit circle}). Consider an angle $\theta$ between the positive x-axis and the line segment connecting $P$ to the origin. By the extended definitions of the sine and cosine functions, $\sin{\theta} = b$ and $\cos{\theta} = a$.
\end{solutions}

\begin{solutions}{Page 102}
\solution %1
\begin{subsolutions}
\subsolution %a
Even
\[
f\left(-x\right) = \left(-x\right)^6-\left(-x\right)^2+7 = x^6-x^2+7 = f\left(x\right)
\]
\subsolution %b
Odd
\[
f\left(-x\right) = \left(-x\right)^3-\sin\left(-x\right)= -x^3+\sin{x}= -f\left(x\right)
\]
\subsolution %c
Neither
\[
f\left(-x\right) = \dfrac{1}{-x+1}
\]
\subsolution %d
Even
\[
f\left(-x\right) = \sec\left(-x\right) = \dfrac{1}{\cos\left(-x\right)} = \dfrac{1}{\cos{x}} = \sec{x} = f\left(x\right)
\]
\subsolution %e
Odd
\[
f\left(-x\right) = \csc\left(-x\right) = \dfrac{1}{\sin\left(-x\right)} = -\dfrac{1}{\sin\left(x\right)} =-\csc{x} = -f\left(x\right)
\]
\subsolution %f
Odd
\[
f\left(-x\right) = 2\sin\left(-x\right)\cos\left(-x\right) = -2\sin\left(x\right)\cos\left(x\right) = -f\left(x\right)
\]
\subsolution %g
Even
\[
f\left(-x\right) = \sin^{2}\left(-x\right)  = \left(-\sin{x}\right)^{2} = \sin^{2}{x} = f\left(x\right)
\]

\subsolution %h
Even
\[
f\left(-x\right) = \cos^{2}\left(-x\right) = \left(\cos{x}\right)^{2} = \cos^{2}{x} = f\left(x\right)
\]

\subsolution %i
Even
\[
f\left(-x\right) = \sin^{2}\left(-x\right) + \cos^{2}\left(-x\right) =  \sin^{2}{x} + \cos^{2}{x} = f\left(x\right)
\]
\end{subsolutions}

\solution %2
\[
g\left(-x\right) = \dfrac{1}{2}\left[f\left(-x\right)+f\left(-\left(-x\right)\right)\right] = \dfrac{1}{2}\left[f\left(-x\right)+f\left(x\right)\right] = g\left(x\right),
\]
so $g\left(x\right)$ is even.

\[
h\left(-x\right) = \dfrac{1}{2}\left[f\left(-x\right)-f\left(-\left(-x\right)\right)\right] = \dfrac{1}{2}\left[f\left(-x\right)-f\left(x\right)\right] = -h\left(x\right),
\]
so $h\left(x\right)$ is odd.

For any function $f\left(x\right)$,
\[
f\left(x\right) = \dfrac{1}{2}\left[f\left(x\right) + f\left(-x\right)\right] + \dfrac{1}{2}\left[f\left(x\right) - f\left(-x\right)\right] = g\left(x\right) + h\left(x\right),
\]
so $f\left(x\right)$ can be written as the sum of an even function and an odd function.

\solution %3
Following the notation of the previous question, we denote the even part of $f\left(x\right)$ as $g\left(x\right)$ and the odd part of $f\left(x\right)$ as $h\left(x\right)$.
\begin{subsolutions}
\subsolution %a
Since we know $\cos{x}$ is an even function and $\sin{x}$ is an odd function,
\[
g\left(x\right)=\cos{x}, h\left(x\right)=\sin{x}
\]

\subsolution %b
Since a polynomial is an even function when all variables are raised to even exponents and an odd function when all variables are raised to odd exponents,
\[
g\left(x\right)=x^{2}+1, h\left(x\right)=x^{3}+x
\]

\subsolution %c
Applying the result from the previous exercise,
\[
g\left(x\right)=\dfrac{1}{2}\left(2^{x}+2^{-x}\right), h\left(x\right)=\dfrac{1}{2}\left(2^{x}-2^{-x}\right)
\]

\subsolution %d
Applying the result from the previous exercise,
\[
g\left(x\right)=\dfrac{1}{2}\left(\dfrac{1-\sin{x}}{1+\sin{x}} + \dfrac{1+\sin{x}}{1-\sin{x}}\right) = \dfrac{1}{2}\left(\dfrac{2+2\sin^{2}{x}}{1-\sin^{2}{x}}\right) = \dfrac{1+\sin^{2}{x}}{1-\sin^{2}{x}}
\]
\[
h\left(x\right)=\dfrac{1}{2}\left(\dfrac{1-\sin{x}}{1+\sin{x}} - \dfrac{1+\sin{x}}{1-\sin{x}}\right) = \dfrac{1}{2}\left(\dfrac{-4\sin{x}}{1-\sin^{2}{x}}\right) = \dfrac{-2\sin{x}}{1-\sin^{2}{x}} 
\]

\subsolution %e
Applying the result from the previous exercise,
\[
g\left(x\right)=\dfrac{1}{2}\left(\dfrac{1}{x+2} + \dfrac{1}{-x+2}\right)=\dfrac{1}{2}\left(\dfrac{4}{4-x^2}\right) = \dfrac{2}{4-x^2}
\]

\[
h\left(x\right)=\dfrac{1}{2}\left(\dfrac{1}{x+2} - \dfrac{1}{-x+2}\right)=\dfrac{1}{2}\left(\dfrac{-2x}{4-x^2}\right) = -\dfrac{x}{4-x^2}
\]

\end{subsolutions}
\end{solutions}


\section*{Chapter 5: Radian Measure}

\begin{solutions}{Page 107}
\solution %1
180 degrees is equal to $\pi$ radians. 90 degrees is equal to $\pi/2$ radians.

\solution %2
\[
\pi^r = 180\degree \implies 1^r = (180/\pi)\degree \implies 2^r = (360/\pi)\degree \approx 114.6\degree
\]

\solution %3
Since a full rotation is $2\pi$ radians, 1/4 of a full rotation will be $\pi/2$ radians.

\solution %4
Since 45 degrees is 1/8 of a full rotation, 1/8 of a full rotation will be $\pi/4$ radians.

\solution %5
Filled table below

\begin{center}
\bgroup
\def\arraystretch{1.1}
\setlength\tabcolsep{10pt}
\begin{tabular}{ |c|c| }
\hline
Degree Measure
& Radian Measure\\
\hline
$90$
& $\mathbf{\pi / 2}$\\
\hline
$180$
& $\mathbf{\pi}$\\
\hline
$270$
& $\mathbf{3\pi / 2}$\\
\hline
$360$
& $\mathbf{2\pi}$\\
\hline
$\mathbf{90}$
& $\pi / 2$\\
\hline
$\mathbf{180}$
& $\pi$\\
\hline
$\mathbf{270}$
& $3\pi / 2$\\
\hline
$\mathbf{360}$
& $2\pi$\\
\hline
\end{tabular}
\egroup
\end{center}

\solution %6
Filled tables below

\begin{center}
\bgroup
\def\arraystretch{1.1}
\setlength\tabcolsep{10pt}
\begin{tabular}{ |c|c| }
\hline
Degree Measure
& Radian Measure\\
\hline
$0$
& $\mathbf{0}$\\
\hline
$30$
& $\mathbf{\pi/6}$\\
\hline
$72$
& $\mathbf{2\pi / 5}$\\
\hline
$120$
& $\mathbf{2\pi/3}$\\
\hline
$135$
& $\mathbf{3\pi / 4}$\\
\hline
$\mathbf{30}$
& $\pi/6$\\
\hline
$\mathbf{36}$
& $\pi / 5$\\
\hline
$\mathbf{45}$
& $\pi/4$\\
\hline
$\mathbf{60}$
& $\pi/3$\\
\hline
$\mathbf{120}$
& $2\pi/3$\\
\hline
$\mathbf{126}$
& $7\pi/10$\\
\hline
\end{tabular}
\egroup
\end{center}

\begin{center}
\bgroup
\def\arraystretch{1.1}
\setlength\tabcolsep{10pt}
\begin{tabular}{ |c|c| }
\hline
Degree Measure
& Radian Measure\\
\hline
$198$
& $\mathbf{11\pi/10}$\\
\hline
$210$
& $\mathbf{7\pi/6}$\\
\hline
$216$
& $\mathbf{6\pi / 5}$\\
\hline
$225$
& $\mathbf{5\pi/4}$\\
\hline
$240$
& $\mathbf{4\pi / 3}$\\
\hline
$\mathbf{198}$
& $11\pi/10$\\
\hline
$\mathbf{200}$
& $10\pi / 9$\\
\hline
$\mathbf{210}$
& $7\pi/6$\\
\hline
$\mathbf{216}$
& $6\pi/5$\\
\hline
$\mathbf{225}$
& $5\pi/4$\\
\hline
$\mathbf{240}$
& $4\pi/3$\\
\hline
\end{tabular}
\egroup
\end{center}

\solution %7
Since 360 degrees is equal to $2\pi$ radians, then 1 degree is equal to $2\pi/360$, or $\pi/180$, radians.

\solution %8
\begin{subsolutions}
    \subsolution $\sin (1^r) \approx  0.8415$
    \subsolution $\sin (1^{\circ}) \approx 0.0175$
\end{subsolutions}

\solution %9
Filled table below

\begin{center}
\bgroup
\def\arraystretch{1.3}
\setlength\tabcolsep{10pt}
\begin{tabular}{ |c|c|c| }
\hline
$\alpha$ (in radian)
& $\sin{\alpha}$
& $\cos{\alpha}$\\
\hline
$\pi / 6$
& $1/2$
& $\sqrt{3}/2$\\
\hline
$\pi / 3$
& $\sqrt{3}/2$
& $1/2$\\
\hline
$\pi / 2$
& $1$
& $0$\\
\hline
$2\pi / 3$
& $\sqrt{3}/2$
& $-1/2$\\
\hline
$7\pi / 6$
& $-1/2$
& $-\sqrt{3}/2$\\
\hline
$5\pi / 4$
& $-\sqrt{2}/2$
& $-\sqrt{2}/2$\\
\hline
$3\pi / 2$
& $-1$
& $0$\\
\hline
$11\pi / 6$
& $-1/2$
& $\sqrt{3}/2$\\
\hline
\end{tabular}
\egroup
\end{center}

\solution %10
With an angle of 2 radians and a radius of 1, the length of the arc is $2 \times 1 = 2$.

With an angle of 3 radians and a radius of 1, the length of the arc is $3 \times 1 = 3$.

With an angle of $\pi$ radians and a radius of 1, the length of the arc is $\pi \times 1 = \pi$.

\solution %11
With an angle of 2 radians and a radius of 3, the length of the arc is $2 \times 3 = 6$.

With an angle of 3 radians and a radius of 3, the length of the arc is $3 \times 3 = 9$.

With an angle of $\pi$ radians and a radius of 3, the length of the arc is $\pi \times 3 = 3\pi$.

\solution %12
Using the fact that sine and cosine are cofunctions (see Section 4 of Chapter 1),
\[
\sin{\dfrac{\pi}{9}} = \cos\left(\dfrac{\pi}{2}-\dfrac{\pi}{9}\right) = \cos{\dfrac{7\pi}{18}},
\]
so $\alpha = 7\pi / 18$.

\solution %13
Again, applying the properties of cofunctions, we know that $\sin{\alpha} = \cos\left(\pi/2 - \alpha\right)$.

\solution %14
A complete rotation around a circle corresponds to an angle of $2\pi$ radians, so each of the six sectors in the diagram is an angle of $2\pi/6$, which is a bit more than 1 radian (since $2\pi > 6$). 

However, $2\pi/6$ radians is equal to 60 degrees, so if $2\pi/6$ is a bit more than 1, then 1 radian is less than 60 degrees

Geometric solution:
\begin{center}
\begin{tikzpicture}
\def\rad{3}
\draw (0,0) circle [radius=\rad];
\draw (-\rad,0) -- (\rad,0);
\draw (120:\rad) -- (300:\rad);
\draw (60:\rad) -- (240:\rad);
\draw (0:\rad) -- (60:\rad) -- (120:\rad) -- (180:\rad) -- (240:\rad) -- (300:\rad) -- (0:\rad);
\node[below left=0.15cm] at (0,0) {$O$};
\node[right] at (\rad,0) {$A$};
\node[above right] at (60:\rad) {$B$};
\node[above left] at (120:\rad) {$C$};
\node[left] at (180:\rad) {$D$};
\node[below left] at (240:\rad) {$E$};
\node[below right] at (300:\rad) {$F$};
\end{tikzpicture}
\end{center}

In the diagram above, we place the six points $A$ through $F$ so that they are equally spaced across the circle $O$. This means that adjacent points on the circle are separated by $360\degree / 6 = 60\degree$. Focusing on $\triangle{AOB}$, we notice that this triangle is isosceles because $AO$ and $BO$ are both equal to the radius of the circle. This implies that $\angle{OAB}$ is congruent to $\angle{OBA}$. Because the three angles in a triangle total to $180\degree$, we have that
\[
m\angle{AOB} = m\angle{OAB} =  m\angle{OBA} = 60\degree.
\]
In other words, $\triangle{AOB}$ is an equilateral triangle. This means that the side $AB$ is equal to the radius of $O$. Therefore, the arc that is intercepted by the chord $AB$ must be longer than the radius of $O$ because the shortest path between two points is a straight line. This implies that the central angle $\angle{AOB}$ is greater than 1 radian. Since $\angle{AOB}$ is a $60\degree$ angle, we have shown that $60\degree$ is greater than $1$ radian.

\end{solutions}

\begin{solutions}{Page 111}
\solution %1
Since a circle of radius 1 traveling 1 foot corresponds to a rotation of 1 radian, a circle of radius 1 traveling 5 feet down a road corresponds to a rotation of 5 radians.

\solution %2
From the previous question, the circle has rotated 5 radians, which is equal to $5 \cdot 180\degree / \pi \approx 286.48\degree$ (two decimal places).

\solution %3
Since a circle of radius 1 rotates 1 radian to travel 1 foot, if it rotates 4 radians, then it has traveled 4 feet.

\solution %4
Since $120\degree = 2\pi/3$ radians, the wheel has traveled $2\pi/3$ feet down the road.

\solution %4
A sector of radius 1 with angle $\alpha$ radians has an arc length of $1 \times \alpha$, so:

An angle of $1/2$ radian has an arc length of $1/2$.

An angle of $\pi/2$ radian has an arc length of $\pi/2$.

An angle of $\alpha$ radian has an arc length of $\alpha$.

\solution %6
Since $2\pi$ radians equal to $360\degree$ then

$720\degree = 4\pi$ radians

$1440\degree = 8\pi$ radians

$3600\degree = 20\pi$ radians

$15120\degree = 84\pi$ radians

$12\pi$ radians = $2160\degree$

$12\pi$ radians = $2160\degree$

$15\pi$ radians = $2700\degree$

$100\pi$ radians = $18000\degree$

\solution %7
A sector of radius 3 with angle $\alpha$ radians has an arc length of $3\alpha$, so:

An angle of $1/2$ radian has an arc length of $3/2$.

An angle of $\pi/2$ radian has an arc length of $3\pi/2$.

An angle of $\alpha$ radian has an arc length of $3\alpha$.

\solution %8
A sector of radius 3 and angle $\alpha$ has arc length of $3\alpha$.

An angle of 1.5 radians has an arc length of 4.5.

\solution %9
A sector of radius 5 and angle $\alpha$ has arc length of $5\alpha$.

An angle of 80 degrees is equal to $4\pi/9$ radians. Thereforem its arc length is $5 \times 4\pi/9 = 20\pi/9$.

\solution %10
A sector of radius 2 and angle $\alpha$ has an arc length of $2\alpha$. This means that for an arc length of $\alpha$, its angle would be $\alpha/2$. If the arc length is 3, then its central angle is 3/2 radians.

\solution %11
A sector of radius 6 and angle $\alpha$ has an arc length of $6\alpha$. This means that for arc length of $\alpha$, its angle would be $\alpha/6$. If the arc length is 2, then its central angle is 1/3 radians, which is equal to $\left(60/\pi\right)\degree \approx 19.1\degree$.

\solution %12
A circle of radius 7 units will travel 7 units with a rotation of 1 radian. That is, each unit of travel requires a rotation of 1/7 radians. Therefore, 20 units of travel require 20/7 radians of rotation.

\solution %13
A circle of radius 8 units will travel 8 units with a rotation of 1 radian. Since 150 degrees is equal to $5\pi/6$ radians, the circle has rolled $8 \times 5\pi/6 = 20\pi/3$ units.

\solution %14
In twelve hours, the hour hand makes a complete rotation around the watch face (e.g., 12:00 AM to 12:00 PM), so in one hour, the hour hand makes $1/12$ of a full rotation. Because the hands of a clock rotate in a clockwise sense, the angle of rotation will be negative. Thus, the angle through which the hour hand rotates in one hour is
\[
-\dfrac{1}{12} \cdot 2\pi = -\dfrac{\pi}{6}.
\]

\solution %15
In one hour, the minute hand makes one complete rotation around the watch face, so it rotates through an angle of $-2\pi$. In the same time, the second hand makes 60 complete rotations around the watch face since it completes one rotation in one minute. Therefore, the second hand rotates through an angle of $-60 \cdot 2\pi = -120\pi$.

\solution %16
Because the hands of a watch travel clockwise, Joe's angle should be negative since counterclockwise rotations are positive by convention.

\solution %17
\[
\dfrac{1000 \text{ in.}}{2\pi \text{ in.}} \approx 159.2
\]
This trip consists of about 159 full rotations of the hour hand. Multiplying by 12 gives the number of hours the trip takes since a complete rotation of the hour hand takes 12 hours.
\[
\dfrac{1000}{2\pi} \cdot 12\text{ hours} \approx 1910\text{ hours}
\]

\solution %18
The angular speed of the hour hand on a pocket watch and the hour hand on Big Ben should be the same because in one hour, the hour hand travels the same angular distance on both clocks. Since in both this exercise and the previous exercise the clocks are traveling through an angle of 1000 radians, both trips should take the same length of time: 1910 hours.

\solution %19
The wheel turns $2\pi$ radians before the spoke returns to the same position. After the wheel turns $\pi$ radians, the spoke will go from pointing vertically downwards to pointing vertically upwards. 

\solution %20
\begin{subsolutions}
\subsolution %a
A wheel of radius one meter will roll $2\pi$ meters each revolution, so the blue marks will be $2\pi$ meters apart.
\subsolution %b
The wheel will make a full revolution in this time, so it has rolled through an angle of $2\pi$ radians.
\subsolution %c
Not more than once. Suppose we have a red mark and a blue mark coinciding. When they next coincide, the distance between the two coincidences will be equal to $3n$ (measuring using the gaps between red dots) or $2\pi m$ (measuring using the gaps between blue dots), where $n$ and $m$ are positive integers. Because these are two equivalent ways to measure the same distance, we can say that $3n = 2\pi m$. Solving for $\pi$, we find that $\pi = \tfrac{3n}{2m}$. However, this is not possible as $\pi$ is irrational, so it cannot be expressed as the ratio of integers. This contradiction means that if a red dot and blue dot do coincide, then they cannot coincide again.

\subsolution ~%d
\begin{center}
\begin{tikzpicture}
\def\rad{2}
\def\ang{16.2}
\def\t{0.05}
\draw (0,0) circle [radius=\rad];
\path [fill=blue] (88:\rad) circle [radius=\t];
\foreach \x in {0,...,22}
    \path [fill=red] (20+\x*\ang:\rad) circle [radius=\t];
\end{tikzpicture}
\end{center}

Let's consider how the red marks hit the wheel as it rolls. Hitting three red marks in a row corresponds to traveling 6 meters (count the space in between the marks). Because the radius of the wheel is 1 meter, when the wheel hits a triple of red marks, the last mark will hit the wheel in a position that is 6 radians ahead of the position where the first mark was hit. Alternatively, we can say that the last mark hits the wheel in a position that is $2\pi - 6$ radians \textit{behind} the position of the first mark. In the above diagram, the red marks are separated by $2\pi-6$ radians, so they represent the positions of every other red mark that hits the wheel in some period of time. Notice that the position of the red marks on the wheel implies that the blue mark cannot be more than $2\pi-6$ radians away from a red mark. This means that no matter how far the wheel travels, the blue mark will eventually be placed within $2\pi-6$ meters of a red mark.

Another perspective:

Compare the location of marks by plotting the marks' locations

\begin{tikzpicture}[xscale=0.25]
    \draw[help lines,xstep=5,color=gray!50,dashed](0,0)grid(40,2);
    \draw[thick] (0,0)--(40,0);
    \draw[thick] (0,0)--(0,2.5);
%    %Draw marks on x axis
    \foreach \xticks in {5,10,...,40}
    \node[below] at (\xticks,0) {\xticks};
%    %Draw the marks 
    \foreach \reds in {0,1,...,13}
    \filldraw[red] (\reds*3,2) circle (0.2);
    \foreach \blues in {0,1,...,5}
    \filldraw[blue] (\blues*2*pi,1) circle (0.2);
\end{tikzpicture}

Note the distance between blue and red markers will be different but the closest they come together are

%\begin{table}[]
\begin{tabular}{|c|c|c|}
\hline
Red & Blue & Difference (approx)\\
\hline
6 & $2\pi$ & 0.283 \\
\hline
12 & $4\pi$ & -0.566 \\
\hline
18 & $6\pi$ & -0.850 \\
\hline
24 & $8\pi$ & -1.133 \\
\hline
30 & $10\pi$ & -1.416 \\
\hline
\end{tabular}
%\end{table}

\subsolution %e
We assume that the blue dot and red dot coincide to begin with. 100 rotations corresponds to the wheel travelling $200\pi$ meters. Notice that $200\pi \approx 209 \cdot 3 + 1.3$. Because $\tfrac{1}{4} \cdot 3 < 1.3 < \tfrac{1}{2} \cdot 3$, the blue dot will be between two pink dots: one of the pink dots will be at the midpoint of two red dots, and the second pink dot will be directly behind the first.

\end{subsolutions}

\end{solutions}

\begin{solutions}{Page 114}
\solution %1
The calculator was in radian mode, since $\sin (1^r) \approx  0.8415$, while $\sin (1^{\circ}) \approx 0.0175$.

\solution ~ %2

\begin{center}
\bgroup
\def\arraystretch{1.3}
\setlength\tabcolsep{10pt}
\begin{tabular}{ |c|c|c| }
\hline
$x$ (in radians)
& $\sin{x}$
& Difference (approx.)\\
\hline
$0.2$
& $0.19867$
& $1.33 \times 10^{-3}$\\
\hline
$0.15$
& $0.14944$
& $5.62 \times 10^{-4}$\\
\hline
$0.05$
& $0.04998$
& $2.08 \times 10^{-5}$\\
\hline
\end{tabular}
\egroup
\end{center}

In all cases, $x > \sin x$.

\solution %3
In the below table, the difference being calculated is $x-x^3/6 - \sin{x}$.

\begin{center}
\bgroup
\def\arraystretch{1.3}
\setlength\tabcolsep{10pt}
\begin{tabular}{|c|c|c|c|}
\hline
$x$ (in radians) & $\sin x$ & $x-x^3/6$ & Difference (approx.)\\
\hline
0.2 & 0.1986693 & 0.1986667 & $-2.66 \times 10^{-6}$\\
\hline
0.15 & 0.1494381 & 0.1494375 & $ -6.32 \times 10^{-7}$\\
\hline
0.05 & 0.0499792 & 0.0499792 & $-2.60 \times 10^{-9}$\\
\hline
\end{tabular}
\egroup
\end{center}

\solution %4
$\sin 10\degree \approx 0.174$, and $10/60 \approx 0.167$, so the error in the approximation is $0.007$.

\solution %5
\begin{subsolutions}
\subsolution %a
\[
\sin{0.1} \approx 0.1 - \dfrac{0.1^{3}}{6} = \dfrac{1}{10} - \dfrac{1}{6000} = \dfrac{599}{6000}
\]
Error: $\sin{0.1}$ is greater than $599/6000$ by $8.33 \times 10^{-8}$.

\subsolution %b
\[
\sin{0.1\degree} = \sin{\dfrac{\pi}{1800}} \approx  \dfrac{\pi}{1800} \approx \dfrac{3.14}{1800} = \dfrac{157}{90000}
\]
Error: $\sin{0.1\degree}$ is greater than $157/90000$ by $8.84 \times 10^{-7}$.
\end{subsolutions}

\solution %6
\begin{subsolutions}
    \subsolution %a
    $\sin 1000\degree \approx -0.9848$
    \subsolution %b
    $\sin 1000^r \approx 0.8269$
\end{subsolutions}

\solution
\begin{subsolutions}
    \subsolution $\sin(\sin 1000^r) \approx 0.7358$
    \subsolution $\sin 3.14^r \approx 0.00159$
\end{subsolutions}

\solution %8
Let $\varepsilon = \pi/2 - 1.5707 \approx 9.6 \times 10^{-5}$.
\[
\cos{1.5707} = \cos\left(\dfrac{\pi}{2}-\varepsilon\right) = \sin{\varepsilon} < \varepsilon < 0.0001
\]
\end{solutions}

\begin{solutions}{Page 116}
\solution
Using the graph below

\begin{tikzpicture}[xscale=1,yscale=1]
    \draw[help lines,xstep=pi/2,ystep=0.5,color=gray!50,dashed](-pi,-1.4) grid (2.1*pi,1.4);
    \draw[->,thick] (-pi,0)--(2.1*pi,0) node[right]{$x$};
    \draw[->,thick] (0,0)--(0,1.5) node[right]{$y$};
    \draw[->,thick] (0,0)--(0,-1.5);
    %Draw marks on x axis
    \foreach \i in {-1,1,2}
    \draw [very thin,gray](\i*pi,-0.1)--(\i*pi,0.1) node[below] at (\i*pi,-0.1) {$\i\pi$};
    %Draw marks on y axis
    \foreach \yticks in {-1,0,1}
    \draw [very thin,gray](-0.1,\yticks)--(0.1,\yticks) node[left] at (0,\yticks) {$\yticks$};
    %Draw the curve
    \draw[red]plot[domain=-pi:2*pi,samples=90] (\x,{sin(\x r)});
    \draw[blue, dashed](7*pi/5,-0.98)--(7*pi/5,0)node[below]{$7\pi/5$};
    \draw[blue, dashed](-3*pi/7,-0.98)--(-3*pi/7,0)node[below]{$-3\pi/7$};
\end{tikzpicture}

\begin{subsolutions}
\subsolution %a
$\sin{7\pi / 5}$ is negative because $\pi < 7\pi / 5 < 2\pi$.

Estimated  $x\approx-1$ (actual is $-0.9511$)

\subsolution %b
$\sin{-3\pi / 7}$ is negative because $-\pi < -3\pi / 7 < 0$.

Estimated $x\approx-1$ (actual is $-0.9749$)
\subsolution %c
By symmetry, $\sin\left(\pi - \pi/6\right) = \sin{5\pi/6} = 1/2$. By periodicity, $\sin\left(2\pi+\pi/6\right) = \sin{13\pi/6} = 1/2$. In general, $\sin{x}=1/2$ when $x=\pi/6 + 2\pi n$ or $x=5\pi/6 + 2\pi n$, where $n$ is an integer.

\subsolution %d
Draw a line at $y=\sin(\pi/12) \approx 0.26$ and then estimate the location of the  intersections with the sine wave.
\begin{align*}
    x &= \pi/12, 11\pi/12, 25\pi/12, \ldots\\
      &\approx 0.26, 2.88, 6.54, \ldots
\end{align*}

\begin{tikzpicture}[xscale=1,yscale=1]
    \draw[help lines,xstep=pi/2,ystep=0.5,color=gray!50,dashed](-pi,-1.4) grid (2.1*pi,1.4);
    \draw[->,thick] (-pi,0)--(2.1*pi,0) node[right]{$x$};
    \draw[->,thick] (0,0)--(0,1.5) node[right]{$y$};
    \draw[->,thick] (0,0)--(0,-1.5);
    %Draw marks on x axis
    \foreach \i in {-1,1,2}
    \draw [very thin,gray](\i*pi,-0.1)--(\i*pi,0.1) node[below] at (\i*pi,-0.1) {$\i\pi$};
    %Draw marks on y axis
    \foreach \yticks in {-1,0,1}
    \draw [very thin,gray](-0.1,\yticks)--(0.1,\yticks) node[left] at (0,\yticks) {$\yticks$};
    %Draw the curve
    \draw[red]plot[domain=-pi:2*pi,samples=90] (\x,{sin(\x r)});
    \draw[blue, dashed] (0,{sin((pi/12) r)}) -- (2*pi,{sin((pi/12) r)});
    \draw[blue, dashed](0.26,0.25)--(0.25, 0) node[below]{0.26};
    \draw[blue, dashed](2.88,0.25)--(2.89, 0) node[below]{2.88};
\end{tikzpicture}

\subsolution %e
Draw a line at $y=0.8$ and then estimate the location of the intersections with the sine wave.
$x \approx 0.93, 2.21, 7.21, \ldots$

\begin{tikzpicture}[xscale=1,yscale=1]
    \draw[help lines,xstep=pi/2,ystep=0.5,color=gray!50,dashed](-pi,-1.4) grid (2.1*pi,1.4);
    \draw[->,thick] (-pi,0)--(2.1*pi,0) node[right]{$x$};
    \draw[->,thick] (0,0)--(0,1.5) node[right]{$y$};
    \draw[->,thick] (0,0)--(0,-1.5);
    %Draw marks on x axis
    \foreach \i in {-1,1,2}
    \draw [very thin,gray](\i*pi,-0.1)--(\i*pi,0.1) node[below] at (\i*pi,-0.1) {$\i\pi$};
    %Draw marks on y axis
    \foreach \yticks in {-1,0,1}
    \draw [very thin,gray](-0.1,\yticks)--(0.1,\yticks) node[left] at (0,\yticks) {$\yticks$};
    %Draw the curve
    \draw[red]plot[domain=-pi:2*pi,samples=90] (\x,{sin(\x r)});
    \draw[blue, dashed] (0, 0.8) -- (2*pi,0.8);
    \draw[blue, dashed](0.93,0.8)--(0.93, 0) node[below]{0.93};
    \draw[blue, dashed](2.21,0.8)--(2.21, 0) node[below]{2.21};
\end{tikzpicture}

Alternatively, notice that $\sin{\pi/4} = \sqrt{2}/2 \approx 0.70$ and $\sin{\pi/3} = \sqrt{3}/2 \approx 0.85$. Taking a weighted average, we can estimate that the sine of
\[
\dfrac{1}{3} \cdot \dfrac{\pi}{4} + \dfrac{2}{3} \cdot
\dfrac{\pi}{3} = \dfrac{11\pi}{36}
\]
is approximately $0.8$. Then, based on the properties of the sine function, $25\pi/36$ and $83\pi/36$ should also have sines of approximately 0.8.
\end{subsolutions}

\end{solutions}

Note: For parts (a) and (b) of the above exercise, we can use the approximation $\cos{x} \approx 1 - \tfrac{1}{2}x^{2}$ to get better estimates. This approximation can be derived from the approximation $\sin{x} \approx x$ using the identity $\sin^{2}{x} + \cos^{2}{x} = 1$. Let us suppose that $x$ is a positive angle close to zero. Replacing $\sin{x}$ with $x$ in the aforementioned identity, we get $\cos{x} \approx \sqrt{1 - x^{2}}$. Using the binomial approximation, $\left(1 + z\right)^{\alpha} \approx 1 + \alpha z$, we arrive at the desired approximation for $\cos{x}$:
\[
\cos{x} \approx \sqrt{1-x^{2}} = \left[1 + \left(-x^{2}\right) \right]^{1/2} \approx 1-\dfrac{1}{2}x^{2}.
\]

\noindent For part (a), we get the estimate
\[
\sin{\dfrac{7\pi}{5}} = -\sin{\dfrac{2\pi}{5}} = -\sin\left(\dfrac{\pi}{2}-\dfrac{\pi}{10}\right) = -\cos{\dfrac{\pi}{10}} \approx \dfrac{1}{2}\left(\dfrac{\pi}{10}\right)^{2} - 1 \approx \dfrac{1}{20} - 1 = -0.95.
\]
For part (b), we get the estimate
\[
\sin{\dfrac{-3\pi}{7}} = -\sin{\dfrac{3\pi}{7}} = -\sin\left(\dfrac{\pi}{2}-\dfrac{\pi}{14}\right) = -\cos{\dfrac{\pi}{14}} \approx \dfrac{1}{2}\left(\dfrac{\pi}{14}\right)^{2} - 1 \approx \dfrac{1}{40} - 1 = -0.975.
\]
Both of these estimates improve upon the first-order approximation of $-1$.

\begin{solutions}{Page 120}
\solution %1
Results are written to 5 decimal places.

\begin{center}
\bgroup
\def\arraystretch{1.3}
\setlength\tabcolsep{10pt}
\begin{tabular}{|c|c|c|}
\hline
$\alpha$ (radians) & $\alpha$ (degrees)  & $\sin{\alpha}$\\ \hline
1       & 57.29578 & 0.84147   \\ \hline
0.5     & 28.64789 & 0.47943   \\ \hline
0.2     & 11.45916 & 0.19867   \\ \hline
0.1     & 5.72958  & 0.09983   \\ \hline
0.01    & 0.57296  & 0.01000   \\ \hline
0.02    & 1.14592  & 0.02000   \\ \hline
0.001   & 0.05730  & 0.00100   \\ \hline
0.002   & 0.11459  & 0.00200   \\ \hline
0.005   & 0.28648  & 0.00500   \\ \hline
\end{tabular}
\egroup
\end{center}

\solution %2
Since $\sin x \approx x$ for angles close to zero (when $x$ is in radians), $\sin 0.00123456 \approx 0.00123456$. Since $x > \sin{x}$ for positive angles, we know this is an overestimate. The calculator result is $0.0012345597$, so our estimate is accurate to 7 decimal places.

\solution %3
\begin{subsolutions}

\subsolution %a
Results are written to 5 decimal places.

\begin{center}
\bgroup
\def\arraystretch{1.3}
\setlength\tabcolsep{10pt}
\begin{tabular}{|c|c|c|}
\hline
$\alpha$
& $\alpha - \alpha^{3}/6$
& $\sin{\alpha}$\\
\hline
$1$
& $0.83333$
& $0.84147$\\
\hline
$0.5$
& $0.47917$
& $0.47943$\\
\hline
$0.2$
& $0.19867$
& $0.19867$\\
\hline
$0.1$
& $0.09983$
& $0.09983$\\
\hline
$0.05$
& $0.04998$
& $0.04998$\\
\hline
$0.01$
& $0.00999$
& $0.00999$\\
\hline
$0.001$
& $0.00099$
& $0.00099$\\
\hline
\end{tabular}
\egroup
\end{center}

\subsolution %b
Recall that we can multiply the degree measure of an angle by $\pi/180$ to get the radian measure, so we simply replace $\alpha$ by $\pi D / 180$ in our approximation. This gives
\[
\sin{D} \approx \dfrac{\pi D}{180} - \dfrac{1}{6}\left(\dfrac{\pi D}{180}\right)^{3} = \dfrac{\pi D}{180} - \dfrac{\pi^{3}D^{3}}{34 992 000}
\]

For $D=1\degree$, the above approximation gives $\sin{1\degree} \approx 0.017421$, while the calculator result is $0.017452$. This is an underestimate by $3.1 \times 10^{-5}$.
\end{subsolutions}

\solution %4
Using the fact that $\alpha = \pi D/180$ from the previous part, the largest possible error for an angle measured in degrees is given by
\[
\dfrac{1}{120} \left(\dfrac{\pi D}{180}\right)^{5} = \dfrac{\pi^{5}D^{5}}{22 674 816 000 000}.
\]

\solution %5
\begin{align*}
    x^5/120 &< 0.001\\
    x^5 &< 0.12\\
x &< \sqrt[5]{0.12}\\
x &< 0.65438
\end{align*}

\solution %6
The signs are alternating between terms, so the third term should be positive.

The exponent of $x$ increases by two in each successive term, so the third term should have an exponent of 5.

The denominator of first term is 1! and the denominator of second term is 3!. Therefore the denominator of the fifth term should be $5! = 120$.

Therefore, the third term is \[+\frac{x^5}{120}. \]

Hint: This is the Taylor Series for $\sin{x}$.

Note: \textbf{!} is used to denote the factorial function, where the factorial of a positive integer $n$ is the product of all positive integers less than or equal to $n$.

\solution %7
The aliens appear to take clockwise angles to be positive, which is the opposite of the convention we typically use. Furthermore, it is not clear that the aliens even use counterclockwise angles since all of the examples only feature clockwise angles. This could mean that the aliens only work with non-negative (or in our system, non-positive) angles.

The aliens do consider angles corresponding to rotations greater than a full rotation, as we do. $\varphi$ seems to be used to denote one complete rotation. Comparing to our system, $\varphi = -2\pi$ radians.

\solution %8
It is easier to rewrite each of the radian angle measurements as multiples of $\pi$ since the first quadrant contains angles between $0$ and $0.5\pi$, the second quadrant contains angles between $0.5\pi$ and $\pi$, etc.

\begin{center}
\bgroup
\def\arraystretch{1.3}
\setlength\tabcolsep{10pt}
\begin{tabular}{|c|c|c|}
\hline
Angle in radians      & Angle in terms of $\pi$    & Quadrant \\ \hline
1     & $0.318\pi$ & 1        \\ \hline
2     & $0.636\pi$ & 2        \\ \hline
3     & $0.954\pi$ & 2        \\ \hline
4     & $1.273\pi$ & 3        \\ \hline
5     & $1.591\pi$ & 4        \\ \hline
6     & $1.909\pi$ & 4           \\ \hline
\end{tabular}
\egroup
\end{center}

1000 radians is approximately equal to $318.31\pi$, which represents 159 full rotations with a remainder of $0.31\pi$ radians. Thus an angle of 1000 radians lies in the first quadrant.

1000 degrees corresponds to $1000/360 \approx 2.78$ rotations, which represents 2 whole rotations and a further 0.78 of a rotation, which is slightly more than a $3/4$ rotation. Thus, an angle of 1000 degrees lies in the fourth quadrant.

\solution %9
Approximately 1/4 of the angles lie in each quadrant. See the solution below the question in the textbook.

\end{solutions}

\section*{Chapter 6: The Addition Formulas}

\begin{solutions}{Page 123}
\solution ~ %1
\begin{center}
\bgroup
\def\arraystretch{2.1}
\setlength\tabcolsep{15pt}
\begin{tabular}{ |c|c|c|c|c|c| }
\hline
$\alpha$
& $\beta$
& $\sin{\alpha}$
& $\sin{\beta}$
& $\sin{\alpha} + \sin{\beta}$
& $\sin\left(\alpha+\beta\right)$ \\
\hline
$60\degree$
& $30\degree$
& $\sqrt{3}/2$
& $1/2$
& $\left(\sqrt{3}+1\right)/2$
& $\sin{90\degree} = 1$\\
\hline
$\pi / 4$
& $\pi / 4$
& $\sqrt{2}/2$
& $\sqrt{2}/2$
& $\left(\sqrt{2}+\sqrt{2}\right)/2 = \sqrt{2}$
& $\sin{\pi/2} = 1$\\
\hline
$\pi / 6$
& $\pi / 3$
& $1/2$
& $\sqrt{3}/2$
& $\left(1+\sqrt{3}\right)/2$
& $\sin{\pi/2} = 1$\\
\hline
\end{tabular}
\egroup
\end{center}

\solution %2
For these values of $\alpha$ and $\beta$, $\sin{\alpha}$ and $\sin{\beta}$ are both at least $1/2$. Furthermore, at least one of $\sin{\alpha}$ and $\sin{\beta}$ is strictly greater than $1/2$. Therefore,
\[
\sin{\alpha} + \sin{\beta} > \dfrac{1}{2} + \dfrac{1}{2} = 1 = \sin\left(\alpha+\beta\right).
\]
\solution %3
\begin{subsolutions}

\subsolution %a
\[
\sin{60\degree} + \sin{30\degree} = \dfrac{\sqrt{3}}{2} + \dfrac{1}{2}
\]
\[
\sin\left(60\degree + 30\degree\right) = \sin{90\degree} = 1
\]
This identity is not correct.
\subsolution %b
\[
\sin\left(60\degree - 30\degree\right) = \sin{30\degree} = \dfrac{1}{2}
\]
\[
\sin{60\degree} - \sin{30\degree} = \dfrac{\sqrt{3}}{2} - \dfrac{1}{2}
\]
This identity is not correct.
\subsolution %c
\[
\sin^{2}{60\degree} - \sin^2{30\degree} = \left(\dfrac{\sqrt{3}}{2}\right)^2 - \left(\dfrac{1}{2}\right)^2 = \dfrac{3}{4} - \dfrac{1}{4} = \dfrac{1}{2}
\]
\[
\sin\left(60\degree + 30\degree\right)\sin\left(60\degree - 30\degree\right) = \sin{90\degree} \cdot \sin{30\degree} = 1 \cdot \dfrac{1}{2} = \dfrac{1}{2}
\]
This identity is correct for the given angles.

\end{subsolutions}

\solution %4
See Chapter 2, Section 12 and the Appendix of Chapter 2 to review some of the geometry used in this solution.
\begin{subsolutions}
\subsolution %a
Because $\angle{ABC}$ is subtended by the diameter $\overline{AC}$, $\angle{ABC}$ is a right angle and $\triangle{ABC}$ is a right triangle (this fact is known as \textit{Thales's Theorem}). Therefore, $\sin{\alpha}$ is equal to the length of the opposite side ($BC$) divided by the length of the hypotenuse ($AC$). $\overline{AC}$ is a diameter of the circle, so it has length 1. Thus, we have that $\sin{\alpha}$ is simply equal to $BC$.

A similar argument shows that $\triangle{ADC}$ is a right triangle with hypotenuse $\overline{AC}$ of length 1, which implies that $\sin{\beta} = DC$

\subsolution %b
Recall that chords of congruent circles which subtend equal angles are themselves equal. This implies that $BC$ in the diagram of part (a) is equal to $BC$ in the diagram of part (b) because in both diagrams, the chord $\overline{BC}$ subtends an angle of measure $\alpha$. Similarly, $DC$ is the same in both diagrams because in both diagrams, the chord $\overline{DC}$ subtends an angle of measure $\beta$. Therefore, $BC$ is still equal to $\sin{\alpha}$, and $DC$ is still equal to $\sin{\beta}$.

\subsolution %c
From part (b) above, we can conclude that a chord which subtends an inscribed angle with measure $\alpha$ in a circle with diameter 1 has length $\sin{\alpha}$. Thus, we draw $\overline{BD}$, the chord which subtends $\angle{BAD}$ in both figures and which consequently has length $\sin\left(\alpha+\beta\right)$.
\end{subsolutions}
Note that the above reasoning implies that the sine of an angle with measure less than $180\degree$ cannot exceed 1 since the diameter is the longest chord in a circle.

\solution %5
Recall that the sine of any angle is at most 1. Therefore,
\[
\sin{105 \degree} \leq 1 = \frac{1}{2} + \frac{1}{2} < \sin{45\degree} + \sin{60\degree},
\]
which shows that $\sin{105 \degree}$ cannot equal $\sin{45\degree} + \sin{60\degree}$.

\end{solutions}

\begin{solutions}{Page 125}
\solution %1
Addition formula for sine:
\begin{align*}
\sin\left(60\degree + 30\degree\right)
&= \sin{60\degree}\cos{30\degree} + \cos{60\degree}\sin{30\degree} \\
&= \dfrac{\sqrt{3}}{2} \cdot \dfrac{\sqrt{3}}{2} + \dfrac{1}{2} \cdot \dfrac{1}{2}\\
&= \dfrac{3}{4} + \dfrac{1}{4} \\
&= 1 \\
&= \sin{90\degree}
\end{align*}

Addition formula for cosine:
\begin{align*}
\cos\left(60\degree + 30\degree\right)
&= \cos{60\degree}\cos{30\degree} - \sin{60\degree}\sin{30\degree} \\
&= \dfrac{1}{2} \cdot \dfrac{\sqrt{3}}{2} - \dfrac{\sqrt{3}}{2} \cdot \dfrac{1}{2}\\
&= \dfrac{\sqrt{3}}{4} - \dfrac{\sqrt{3}}{4} \\
&= 0 \\
&= \cos{90\degree}
\end{align*}

Difference formula for sine:
\begin{align*}
\sin\left(60\degree - 30\degree\right)
&= \sin{60\degree}\cos{30\degree} - \cos{60\degree}\sin{30\degree} \\
&= \dfrac{\sqrt{3}}{2} \cdot \dfrac{\sqrt{3}}{2} - \dfrac{1}{2} \cdot \dfrac{1}{2}\\
&= \dfrac{3}{4} - \dfrac{1}{4} \\
&= \dfrac{1}{2} \\
&= \sin{30\degree}
\end{align*}

Difference formula for cosine:
\begin{align*}
\cos\left(60\degree - 30\degree\right)
&= \cos{60\degree}\cos{30\degree} + \sin{60\degree}\sin{30\degree} \\
&= \dfrac{1}{2} \cdot \dfrac{\sqrt{3}}{2} + \dfrac{\sqrt{3}}{2} \cdot \dfrac{1}{2}\\
&= \dfrac{\sqrt{3}}{4} + \dfrac{\sqrt{3}}{4} \\
&= \dfrac{\sqrt{3}}{2} \\
&= \cos{30\degree}
\end{align*}

\solution %2
Addition formula for sine ($\alpha = 0$):
\begin{align*}
\sin\left(0 + \beta\right) 
&= \sin{0}\cos{\beta} + \cos{0}\sin{\beta} \\
&= 0 \cdot \cos{\beta} + 1 \cdot \sin{\beta} \\
&= \sin{\beta}
\end{align*}

Addition formula for cosine ($\alpha = 0$):
\begin{align*}
\cos\left(0 + \beta\right) 
&= \cos{0}\cos{\beta} - \sin{0}\sin{\beta} \\
&= 1 \cdot \cos{\beta} - 0 \cdot \sin{\beta} \\
&= \cos{\beta}
\end{align*}

Difference formula for sine ($\alpha = 0$):
\begin{align*}
\sin\left(0 - \beta\right) 
&= \sin{0}\cos{\beta} - \cos{0}\sin{\beta} \\
&= 0 \cdot \cos{\beta} - 1 \cdot \sin{\beta} \\
&= -\sin{\beta}
\end{align*}
\quad Notice that this demonstrates that the sine function is \textit{odd}.

Difference formula for cosine ($\alpha = 0$):
\begin{align*}
\cos\left(0 - \beta\right) 
&= \cos{0}\cos{\beta} + \sin{0}\sin{\beta} \\
&= 1 \cdot \cos{\beta} + 0 \cdot \sin{\beta} \\
&= \cos{\beta}
\end{align*}
\quad Notice that this demonstrates that the cosine function is \textit{even}.

Addition formula for sine ($\beta = 0$):
\begin{align*}
\sin\left(\alpha + 0\right) 
&= \sin{\alpha}\cos{0} + \cos{\alpha}\sin{0} \\
&= \sin{\alpha} \cdot 1 + \cos{\alpha} \cdot 0 \\
&= \sin{\alpha}
\end{align*}

Addition formula for cosine ($\beta = 0$):
\begin{align*}
\cos\left(\alpha + 0\right) 
&= \cos{\alpha}\cos{0} - \sin{\alpha}\sin{0} \\
&= \cos{\alpha} \cdot 1 - \sin{\alpha} \cdot 0 \\
&= \cos{\alpha}
\end{align*}

Difference formula for sine ($\beta = 0$):
\begin{align*}
\sin\left(\alpha - 0\right) 
&= \sin{\alpha}\cos{0} - \cos{\alpha}\sin{0} \\
&= \sin{\alpha} \cdot 1 - \cos{\alpha} \cdot 0 \\
&= \sin{\alpha}
\end{align*}

Difference formula for cosine ($\beta = 0$):
\begin{align*}
\cos\left(\alpha + 0\right) 
&= \cos{\alpha}\cos{0} + \sin{\alpha}\sin{0} \\
&= \cos{\alpha} \cdot 1 + \sin{\alpha} \cdot 0 \\
&= \cos{\alpha}
\end{align*}

\solution %3
Following the hint, we notice that in a right triangle, the side opposite one of the acute angles is the side adjacent to the other acute angle. Thus, if $\alpha + \beta = \pi/2$, then $\sin{\alpha} = \cos{\beta}$ and $\sin{\beta} = \cos{\alpha}$ (see also Chapter 1, Section 4).
\begin{align*}
\sin\left(\alpha + \beta\right)
&= \sin{\alpha} \cos{\beta}  + \cos{\alpha}\sin{\beta} \\
&= \sin{\alpha}\sin{\alpha} + \cos{\alpha}\cos{\alpha} \\
&= \sin^{2}{\alpha} + \cos^{2}{\alpha} \\
&= 1
\end{align*}

\solution %4
Addition formula for sine:
\begin{align*}
\sin\left(\dfrac{\pi}{4} + \dfrac{\pi}{4}\right)
&= \sin{\dfrac{\pi}{4}}\cos{\dfrac{\pi}{4}} + \cos{\dfrac{\pi}{4}}\sin{\dfrac{\pi}{4}} \\
&= \dfrac{\sqrt{2}}{2} \cdot \dfrac{\sqrt{2}}{2} + \dfrac{\sqrt{2}}{2} \cdot \dfrac{\sqrt{2}}{2} \\
&= \dfrac{1}{2} + \dfrac{1}{2} \\
&= 1 \\
&= \sin{\dfrac{\pi}{2}}
\end{align*}

Addition formula for cosine:
\begin{align*}
\cos\left(\dfrac{\pi}{4} + \dfrac{\pi}{4}\right)
&= \cos{\dfrac{\pi}{4}}\cos{\dfrac{\pi}{4}} - \sin{\dfrac{\pi}{4}}\sin{\dfrac{\pi}{4}} \\
&= \dfrac{\sqrt{2}}{2} \cdot \dfrac{\sqrt{2}}{2} - \dfrac{\sqrt{2}}{2} \cdot \dfrac{\sqrt{2}}{2} \\
&= \dfrac{1}{2} - \dfrac{1}{2} \\
&= 0 \\
&= \cos{\dfrac{\pi}{2}}
\end{align*}

\solution %5
Recall that $\left(A \pm B\right)^2 = A^2 \pm 2AB + B^2$.
\begin{align*}
&\left(\sin{\alpha}\cos{\beta} + \cos{\alpha}\sin{\beta}\right)^2 + \left(\cos{\alpha}\cos{\beta} - \sin{\alpha}\sin{\beta}\right)^2 \\
&\qquad =\sin^{2}{\alpha}\cos^{2}{\beta} + 2\sin{\alpha}\cos{\beta}\cos{\alpha}\sin{\beta} + \cos^{2}{\alpha}\sin^{2}{\beta} + \\
&\qquad \phantom{=}\cos^{2}{\alpha}\cos^{2}{\beta} - 2\cos{\alpha}\cos{\beta}\sin{\alpha}\sin{\beta} + \sin^{2}{\alpha}\sin^{2}{\beta} \\
&\qquad =\sin^{2}{\alpha}\cos^{2}{\beta} + \cos^{2}{\alpha}\sin^{2}{\beta} + \cos^{2}{\alpha}\cos^{2}{\beta} + \sin^{2}{\alpha}\sin^{2}{\beta} \\
&\qquad =\sin^{2}{\alpha} \left(\cos^{2}{\beta} + \sin^{2}{\beta}\right) + \cos^{2}{\alpha} \left(\sin^{2}{\beta} + \cos^{2}{\beta}\right) \\
&\qquad = \sin^{2}{\alpha} \cdot 1 + \cos^{2}{\alpha} \cdot 1 \\
&\qquad = 1
\end{align*}

\solution %6
After expanding using the identity $\left(A+B\right)\left(A-B\right) = A^2-B^2$, we cleverly ``add by zero'' to get the desired result.
\begin{align*}
&\left(\sin{\alpha}\cos{\beta} + \cos{\alpha}\sin{\beta}\right)\left(\sin{\alpha}\cos{\beta} - \cos{\alpha}\sin{\beta}\right) \\
&\qquad =\sin^{2}{\alpha}\cos^{2}{\beta} - \cos^{2}{\alpha}\sin^{2}{\beta} \\
&\qquad =\sin^{2}{\alpha}\cos^{2}{\beta} + \sin^{2}{\alpha}\sin^{2}{\beta} - \sin^{2}{\alpha}\sin^{2}{\beta} - \cos^{2}{\alpha}\sin^{2}{\beta} \\
&\qquad =\sin^{2}{\alpha}\left(\cos^{2}{\beta} + \sin^{2}{\beta}\right) - \sin^{2}{\beta}\left(\sin^{2}{\alpha} + \cos^{2}{\alpha}\right) \\
&\qquad =\sin^{2}{\alpha} \cdot 1 - \sin^2{\beta} \cdot 1 \\
&\qquad =\sin^{2}{\alpha} - \sin^2{\beta}
\end{align*}
\end{solutions}

\begin{solutions}{Page 129}
\solution %1

\solution %2
\end{solutions}

\begin{solutions}{Page 131}
\solution %1
\begin{align*}
\sin\left(30\degree + 30\degree\right)
&= \sin{30\degree}\cos{30\degree} + \cos{30\degree}\sin{30\degree} \\
&= \dfrac{1}{2} \cdot \dfrac{\sqrt{3}}{2} + \dfrac{\sqrt{3}}{2} \cdot \dfrac{1}{2}\\
&= \dfrac{\sqrt{3}}{4} + \dfrac{\sqrt{3}}{4} \\
&= \dfrac{\sqrt{3}}{2} \\
&= \sin{60\degree}
\end{align*}

\begin{align*}
\cos\left(30\degree + 30\degree\right)
&= \cos{30\degree}\cos{30\degree} - \sin{30\degree}\sin{30\degree} \\
&= \dfrac{\sqrt{3}}{2} \cdot \dfrac{\sqrt{3}}{2} - \dfrac{1}{2} \cdot \dfrac{1}{2}\\
&= \dfrac{3}{4} - \dfrac{1}{4} \\
&= \dfrac{1}{2} \\
&= \cos{60\degree}
\end{align*}

\solution %2
Assuming $\alpha$ and $\beta$ are acute angles:
\[
\sin{\alpha} = \dfrac{3}{5} \implies \cos{\alpha} = \sqrt{1-\sin^{2}{\alpha}} = \sqrt{1-\left(3/5\right)^2} = \dfrac{4}{5}
\]

\[
\sin{\beta} = \dfrac{5}{13} \implies \cos{\beta} = \sqrt{1-\sin^{2}{\beta}} = \sqrt{1-\left(5/13\right)^2} = \dfrac{12}{13}
\]

\begin{align*}
\sin\left(\alpha+\beta\right) &= \sin{\alpha}\cos{\beta} + \cos{\alpha}\sin{\beta} \\
&= \dfrac{3}{5} \cdot \dfrac{12}{13} + \dfrac{4}{5} \cdot \dfrac{5}{13} \\
&= \dfrac{36}{65} + \dfrac{20}{65} \\
&= \dfrac{56}{65}
\end{align*}

\begin{align*}
\cos\left(\alpha+\beta\right) &= \cos{\alpha}\cos{\beta} - \sin{\alpha}\sin{\beta} \\
&= \dfrac{4}{5} \cdot \dfrac{12}{13} - \dfrac{3}{5} \cdot \dfrac{5}{13} \\
&= \dfrac{48}{65} - \dfrac{15}{65} \\
&= \dfrac{33}{65}
\end{align*}

\solution %3
\begin{align*}
\sin{75\degree} &= \sin\left(45\degree + 30\degree\right) \\
&= \sin{45\degree}\cos{30\degree} + \cos{45\degree}\sin{30\degree} \\
&= \dfrac{\sqrt{2}}{2} \cdot \dfrac{\sqrt{3}}{2} + \dfrac{\sqrt{2}}{2} \cdot \dfrac{1}{2} \\
&= \dfrac{\sqrt{6}+\sqrt{2}}{4}
\end{align*}

\begin{align*}
\cos{75\degree} &= \cos\left(45\degree + 30\degree\right) \\
&= \cos{45\degree}\cos{30\degree} - \sin{45\degree}\sin{30\degree} \\
&= \dfrac{\sqrt{2}}{2} \cdot \dfrac{\sqrt{3}}{2} - \dfrac{\sqrt{2}}{2} \cdot \dfrac{1}{2} \\
&= \dfrac{\sqrt{6}-\sqrt{2}}{4}
\end{align*}

\solution %4
\begin{align*}
\sin{15\degree} &= \sin\left(45\degree - 30\degree\right) \\
&= \sin{45\degree}\cos{30\degree} - \cos{45\degree}\sin{30\degree} \\
&= \dfrac{\sqrt{2}}{2} \cdot \dfrac{\sqrt{3}}{2} - \dfrac{\sqrt{2}}{2} \cdot \dfrac{1}{2} \\
&= \dfrac{\sqrt{6}-\sqrt{2}}{4}
\end{align*}

\begin{align*}
\cos{15\degree} &= \cos\left(45\degree - 30\degree\right) \\
&= \cos{45\degree}\cos{30\degree} + \sin{45\degree}\sin{30\degree} \\
&= \dfrac{\sqrt{2}}{2} \cdot \dfrac{\sqrt{3}}{2} + \dfrac{\sqrt{2}}{2} \cdot \dfrac{1}{2} \\
&= \dfrac{\sqrt{6}+\sqrt{2}}{4}
\end{align*}

Notice that $75\degree$ and $15\degree$ are complementary angles, so we know $\sin{75\degree}=\cos{15\degree}$ and $\sin{15\degree}=\cos{75\degree}$.

\solution %5
\begin{subsolutions}
\subsolution %a
Yes, let $\alpha = \beta = \pi/4$.
\begin{align*}
\cos\left(\dfrac{\pi}{4} + \dfrac{\pi}{4}\right) &= \cos{\dfrac{\pi}{4}}\cos{\dfrac{\pi}{4}} - \sin{\dfrac{\pi}{4}}\sin{\dfrac{\pi}{4}} \\
&= \dfrac{\sqrt{2}}{2} \cdot \dfrac{\sqrt{2}}{2} - \dfrac{\sqrt{2}}{2} \cdot \dfrac{\sqrt{2}}{2} \\
&= \dfrac{1}{2} - \dfrac{1}{2} \\
&= 0
\end{align*}
More generally, we could suppose $\alpha + \beta = \pi/2$ and follow the approach in Exercise 3 of Section 2 earlier in this chapter.

\subsolution %b
If $\alpha$ and $\beta$ are acute angles, then $0 < \alpha + \beta < \pi$. Using the unit circle, we can see that $\sin\left(\alpha+\beta\right)$ must be positive since the angle $\alpha+\beta$ lies in the upper-half of the plane, where the sine function is positive.

\subsolution %c
$\sin\left(\alpha+\beta\right) = \sin{\alpha}\cos{\beta} + \cos{\alpha}\sin{\beta}$ is positive when $\alpha$ and $\beta$ are acute angles since the sum and product of positive real numbers is also positive.

$\cos\left(\alpha+\beta\right)$ need not be positive. As shown in part (a) of this exercise, $\cos\left(\alpha+\beta\right)$ can equal 0. Furthermore, $\cos\left(\alpha+\beta\right)$ can be negative. Let $\alpha = \beta = \pi/3$. Then, assuming that we can extend the cosine addition formula to angles $\alpha$ and $\beta$ such that $\alpha + \beta$ is obtuse,
\begin{align*}
\cos\left(\dfrac{\pi}{3} + \dfrac{\pi}{3}\right) &= \cos{\dfrac{\pi}{3}}\cos{\dfrac{\pi}{3}} - \sin{\dfrac{\pi}{3}}\sin{\dfrac{\pi}{3}} \\
&= \dfrac{1}{2} \cdot \dfrac{1}{2} - \dfrac{\sqrt{3}}{2} \cdot \dfrac{\sqrt{3}}{2} \\
&= \dfrac{1}{4} - \dfrac{3}{4} \\
&= -\dfrac{1}{2}
\end{align*}
\end{subsolutions}

\solution %6
As we saw in Exercise 1 of Section 1 of this chapter, $\sin{\alpha} + \sin{\beta}$ does not equal $\sin\left(\alpha+\beta\right)$ in general. A similar table can be used to show that $\sin{\alpha} - \sin{\beta}$ does not equal $\sin\left(\alpha-\beta\right)$ in general.

\solution %7
This is not a coincidence. The identity holds true even when substituting more ``arbitrary'' values in for $\alpha$ and $\beta$. For example, using $\alpha=37\degree$ and $\beta=19\degree$, we find that both $\sin^{2}{\alpha} - \sin^{2}{\beta}$ and $\sin\left(\alpha+\beta\right)\sin\left(\alpha-\beta\right)$ are equal to approximately $0.2562$.

\solution %8
You may also refer to the proof in Exercise 6 of Section 2 of this chapter.
\begin{align*}
\sin\left(\alpha+\beta\right)\sin\left(\alpha-\beta\right) &= \left(\sin{\alpha}\cos{\beta} + \cos{\alpha}\sin{\beta}\right)\left(\sin{\alpha}\cos{\beta} - \cos{\alpha}\sin{\beta}\right)\\
&= \sin^{2}{\alpha}\cos^{2}{\beta} - \cos^{2}{\alpha}\sin^{2}{\beta} \\
&= \sin^{2}{\alpha}\cos^{2}{\beta} + \sin^{2}{\alpha}\sin^{2}{\beta} - \sin^{2}{\alpha}\sin^{2}{\beta} - \cos^{2}{\alpha}\sin^{2}{\beta} \\
&= \sin^{2}{\alpha}\left(\cos^{2}{\beta} + \sin^{2}{\beta}\right) - \sin^{2}{\beta}\left(\sin^{2}{\alpha} + \cos^{2}{\alpha}\right) \\
&= \sin^{2}{\alpha} \cdot 1 - \sin^2{\beta} \cdot 1 \\
&= \sin^{2}{\alpha} - \sin^2{\beta}
\end{align*}

\solution %9
This proof is nearly identical to the one in the previous part. We just make a small modification in how we ``add by zero'' in order to obtain to the desired result.
\begin{align*}
\sin\left(\alpha+\beta\right)\sin\left(\alpha-\beta\right) &= \left(\sin{\alpha}\cos{\beta} + \cos{\alpha}\sin{\beta}\right)\left(\sin{\alpha}\cos{\beta} - \cos{\alpha}\sin{\beta}\right)\\
&= \sin^{2}{\alpha}\cos^{2}{\beta} - \cos^{2}{\alpha}\sin^{2}{\beta} \\
&= \sin^{2}{\alpha}\cos^{2}{\beta} + \cos^{2}{\alpha}\cos^{2}{\beta} - \cos^{2}{\alpha}\cos^{2}{\beta} - \cos^{2}{\alpha}\sin^{2}{\beta} \\
&= \cos^{2}{\beta}\left(\sin^{2}{\alpha} + \cos^{2}{\alpha}\right) - \cos^{2}{\alpha}\left(\cos^{2}{\beta} + \sin^{2}{\beta}\right) \\
&= \cos^{2}{\beta} \cdot 1 - \cos^2{\alpha} \cdot 1 \\
&= \cos^{2}{\beta} - \cos^2{\alpha}
\end{align*}

\solution %10
We apply the sine addition formula in reverse.
\begin{align*}
\sin{18\degree}\cos{12\degree} + \cos{18\degree}\sin{12\degree} &= \sin\left(18\degree + 12\degree\right) \\
&= \sin{30\degree} \\
&= \dfrac{1}{2}
\end{align*}

\solution %11
\begin{subsolutions}
\subsolution %a
Since we have not proved that the sine addition formula works for all angles $\alpha$ and $\beta$, we use properties of the sine and cosine functions to avoid working with angles larger than $90\degree$.
\begin{align*}
&\sin{113\degree}\cos{307\degree} + \cos{113\degree}\sin{307\degree} \\
&\qquad = \sin\left(180\degree-67\degree\right)\cos\left(360\degree-53\degree\right) + \cos\left(180\degree-67\degree\right)\sin\left(360\degree-53\degree\right) \\
&\qquad = \sin{67\degree}\cos{53\degree} + \left(-\cos{67\degree}\right)\left(-\sin{53\degree}\right) \\
&\qquad = \sin\left(67\degree + 53\degree\right) \\
&\qquad = \sin{120\degree} \\
&\qquad = \sin{60\degree} \\
&\qquad = \dfrac{\sqrt{3}}{2}
\end{align*}

\subsolution %b
Plugging into a calculator,
\[
\sin{113\degree}\cos{307\degree} + \cos{113\degree}\sin{307\degree} \approx 0.866 \approx \dfrac{\sqrt{3}}{2} = \sin{60\degree}
\]
\subsolution %c
Assuming that the sine addition formula does work for non-acute angles, we arrive at the same result.
\begin{align*}
\sin{113\degree}\cos{307\degree} + \cos{113\degree}\sin{307\degree} &= \sin\left(113\degree + 307\degree\right) \\
&= \sin{420\degree} \\
&= \sin{60\degree} \\
&= \dfrac{\sqrt{3}}{2}
\end{align*}

\end{subsolutions}

\solution %12
We can use the addition formulas for sine and cosine by rewriting $2\alpha$ as $\alpha + \alpha$.
\begin{align*}
\sin{2\alpha}\cos{\alpha} - \cos{2\alpha}\sin{\alpha} &= \sin\left(\alpha + \alpha\right)\cos{\alpha} - \cos\left(\alpha + \alpha\right)\sin{\alpha} \\ 
&= \left(\sin{\alpha}\cos{\alpha} + \cos{\alpha}\sin{\alpha}\right)\cos{\alpha} - \left(\cos{\alpha}\cos{\alpha}-\sin{\alpha}\sin{\alpha}\right)\sin{\alpha} \\
&= \sin{\alpha}\cos^{2}{\alpha} + \sin{\alpha}\cos^{2}{\alpha} - \sin{\alpha}\cos^{2}{\alpha} + \sin^{3}{\alpha} \\
&=\sin{\alpha}\cos^{2}{\alpha} + \sin^{3}{\alpha} \\
&= \sin{\alpha}\left(\cos^{2}\alpha+\sin^{2}\alpha\right) \\
&= \sin{\alpha}
\end{align*}

\solution %13
\begin{align*}
\sin\left(\alpha+\beta\right)\sin{\beta} + \cos\left(\alpha+\beta\right)\cos{\beta} &= \left(\sin{\alpha}\cos{\beta} + \cos{\alpha}\sin{\beta}\right)\sin{\beta} + \left(\cos{\alpha}\cos{\beta}-\sin{\alpha}\sin{\beta}\right)\cos{\beta} \\
&= \sin{\alpha}\sin{\beta}\cos{\beta} + \sin^{2}{\beta}\cos{\alpha} + \cos{\alpha}\cos^{2}{\beta} -\sin{\alpha}\sin{\beta}\cos{\beta} \\
&= \sin^{2}{\beta}\cos{\alpha} + \cos{\alpha}\cos^{2}{\beta} \\
&= \cos{\alpha}\left(\sin^{2}{\beta}+\cos^{2}{\beta}\right) \\
&= \cos{\alpha}
\end{align*}

\solution %14
\begin{align*}
\dfrac{\sin\left(\alpha+\beta\right)-\cos{\alpha}\sin{\beta}}{\cos\left(\alpha+\beta\right)+\sin{\alpha}\sin{\beta}} &= \dfrac{\sin{\alpha}\cos{\beta}+\cos{\alpha}\sin{\beta}-\cos{\alpha}\sin{\beta}}{\cos{\alpha}\cos{\beta}-\sin{\alpha}\sin{\beta}+\sin{\alpha}\sin{\beta}} \\
&= \dfrac{\sin{\alpha}\cos{\beta}}{\cos{\alpha}\cos{\beta}} \\
&= \dfrac{\sin{\alpha}}{\cos{\alpha}} \\
&= \tan{\alpha}
\end{align*}

\solution %15
\begin{align*}
\sin\left(\alpha + \dfrac{\pi}{4}\right) &= \sin{\alpha}\cos{\dfrac{\pi}{4}} + \cos{\alpha}\sin{\dfrac{\pi}{4}} \\
&= \sin{\alpha} \cdot \dfrac{\sqrt{2}}{2} + \cos{\alpha} \cdot \dfrac{\sqrt{2}}{2} \\
&= \dfrac{\sqrt{2}}{2} \left(\sin{\alpha} + \cos{\alpha}\right)
\end{align*}

\solution %16
\begin{align*}
\dfrac{\cos\left(\alpha+\beta\right)}{\cos{\alpha}\cos{\beta}} &= \dfrac{\cos{\alpha}\cos{\beta}-\sin{\alpha}\sin{\beta}}{\cos{\alpha}\cos{\beta}} \\
&= 1 - \dfrac{\sin{\alpha}\sin{\beta}}{\cos{\alpha}\cos{\beta}} \\
&= 1 - \dfrac{\sin{\alpha}}{\cos{\alpha}} \cdot \dfrac{\sin{\beta}}{\cos{\beta}} \\
&= 1 - \tan{\alpha}\tan{\beta}
\end{align*}

\solution %17
Applying the law of cosines, we have that $\left(b_{1} + b_{2}\right)^2 = c_{1}^2 + c_{2}^2 - 2c_1c_2\cos\left(\alpha+\beta\right)$. Solving for $\cos\left(\alpha+\beta\right)$, we get
\[
\cos\left(\alpha+\beta\right) = \dfrac{c_{1}^2 + c_{2}^2 - \left(b_{1}+b_{2}\right)^2}{2c_{1}c_{2}}.
\]
Before proceeding further, let's establish some relationships between the variables in the diagram. First, by the Pythagorean theorem, we have that $h^2 = c_{1}^2 - b_{1}^2 = c_{2}^2 - b_{2}^2$. Additionally, we can compute the sines and cosines for the angles $\alpha$ and $\beta$:
\[
\sin{\alpha} = \dfrac{b_{1}}{c_{1}}, \sin{\beta} = \dfrac{b_{2}}{c_{2}}, \cos{\alpha} = \dfrac{h}{c_{1}}, \cos{\beta} = \dfrac{h}{c_{2}}.
\]
We can now simplify our expression for $\cos\left(\alpha+\beta\right)$.
\begin{align*}
\cos\left(\alpha+\beta\right) &= \dfrac{c_{1}^2 + c_{2}^2 - \left(b_{1}+b_{2}\right)^2}{2c_{1}c_{2}} \\
&= \dfrac{c_{1}^2 + c_{2}^2 - b_{1}^2 - 2b_{1}b_{2} - b_{2}^2}{2c_{1}c_{2}} \\
&= \dfrac{2h^2 - 2b_{1}b_{2}}{2c_{1}c_{2}} \\
&= \dfrac{h^2 - b_{1}b_{2}}{c_{1}c_{2}} \\
&= \dfrac{h}{c_{1}} \cdot \dfrac{h}{c_{2}} - \dfrac{b_{1}}{c_{1}} \cdot \dfrac{b_{2}}{c_{2}} \\
&= \cos{\alpha}\cos{\beta} - \sin{\alpha}\sin{\beta}
\end{align*}
\end{solutions}


\section*{Chapter 7: Trigonometric Identities}

\begin{solutions}{Page 141}
\solution %1
\begin{subsolutions}
\subsolution %a
Yes
\subsolution %b
No
\subsolution %c
Yes
\subsolution %d
Yes
\subsolution %e
No
\subsolution %f
No
\end{subsolutions}
\solution %2
\begin{subsolutions}
\subsolution %a
\[
\tan{\alpha} = \dfrac{\sin{\alpha}}{\cos{\alpha}}
\]

\subsolution %b
\begin{align*}
\left(1+\tan{\alpha}\right)\left(1-\tan{\alpha}\right) &= 1-\tan^{2}{\alpha} \\
&= 1- \dfrac{\sin^{2}{\alpha}}{\cos^{2}{\alpha}} \\
&= \dfrac{\cos^{2}{\alpha}-\sin^{2}{\alpha}}{\cos^{2}{\alpha}}
\end{align*}

\subsolution %c
\begin{align*}
\dfrac{\tan{\alpha}+\tan{\beta}}{1-\tan{\alpha}\tan{\beta}} &= \dfrac{\tan{\alpha}+\tan{\beta}}{1-\tan{\alpha}\tan{\beta}} \cdot \dfrac{\cos{\alpha}\cos{\beta}}{\cos{\alpha}\cos{\beta}} \\
&= \dfrac{\sin{\alpha}\cos{\beta}+\cos{\alpha}\sin{\beta}}{\cos{\alpha}\cos{\beta}-\sin{\alpha}\sin{\beta}} %\\
%&=\dfrac{\sin\left(\alpha + \beta\right)}{\cos\left(\alpha + \beta\right)}
\end{align*}

\subsolution %d
\begin{align*}
\tan^{2}{\alpha}+\cot^{2}{\alpha} &= \dfrac{\sin^{2}{\alpha}}{\cos^{2}{\alpha}} + \dfrac{\cos^{2}{\alpha}}{\sin^{2}{\alpha}} \\
&= \dfrac{\sin^{4}{\alpha}+\cos^{4}{\alpha}}{\sin^{2}{\alpha}\cos^{2}{\alpha}}
\end{align*}

\subsolution %e
\begin{align*}
\tan{\alpha}\cot{\alpha} &= \dfrac{\sin{\alpha}}{\cos{\alpha}} \cdot \dfrac{\cos{\alpha}}{\sin{\alpha}} \\
&= 1
\end{align*}

\subsolution %f
\begin{align*}
1+\tan^{2}{\alpha} &= 1 + \dfrac{\sin^{2}{\alpha}}{\cos^{2}{\alpha}} \\
&= \frac{\cos^{2}{\alpha} + \sin^{2}{\alpha}}{\cos^{2}{\alpha}} \\
&= \dfrac{1}{\cos^{2}{\alpha}}
\end{align*}

\end{subsolutions}
\solution %3
The Principle of Analytic Continuation does not apply because $\sqrt{1-\sin^{2}{\alpha}}$ is not a rational trigonometric function. The identity is incorrect for $\alpha=2\pi / 3$ as $\cos\left(2\pi / 3\right) = -1/2$, while $\sqrt{1-\sin^{2}\left(2\pi / 3\right)} = 1/2$.
\solution %4
The Principle of Analytic Continuation does apply because both $\sin^{2}{\alpha}+\cos^{2}{\alpha}$ and $1$ are rational trigonometric functions. The identity is correct for $\alpha=2\pi / 3$.
\[
\sin^{2}{\dfrac{2\pi}{3}}+\cos^{2}{\dfrac{2\pi}{3}} = \left(\dfrac{\sqrt{3}}{2}\right)^2 + \left(\dfrac{1}{2}\right)^2 = \dfrac{3}{4} + \dfrac{1}{4} = 1
\]
\end{solutions}

\begin{solutions}{Page 142}
\solution %1
Since $\alpha$ and $\beta$ are acute angles,
\[
\sin{\alpha} = \dfrac{3}{5} \implies \cos{\alpha} = \sqrt{1-\sin^{2}{\alpha}} = \sqrt{1-\left(\dfrac{3}{5}\right)^2} = \dfrac{4}{5}
\]
\[
\sin{\beta} = \dfrac{5}{13} \implies \cos{\beta} = \sqrt{1-\sin^{2}{\beta}} = \sqrt{1-\left(\dfrac{5}{13}\right)^2} = \dfrac{12}{13}
\]
\begin{align*}
\sin\left(\alpha+\beta\right) &= \sin{\alpha}\cos{\beta} + \cos{\alpha}\sin{\beta} \\
&= \dfrac{3}{5} \cdot \dfrac{12}{13} + \dfrac{4}{5} \cdot \dfrac{5}{13} \\
&= \dfrac{36}{65} + \dfrac{20}{65} \\
&= \dfrac{56}{65}
\end{align*}
\begin{align*}
\cos\left(\alpha+\beta\right) &= \cos{\alpha}\cos{\beta} - \sin{\alpha}\sin{\beta} \\
&= \dfrac{4}{5} \cdot \dfrac{12}{13} - \dfrac{3}{5} \cdot \dfrac{5}{13} \\
&= \dfrac{48}{65} - \dfrac{15}{65} \\
&= \dfrac{33}{65}
\end{align*}
$\alpha+\beta$ lies in the first quadrant because $\sin\left(\alpha+\beta\right)$ and $\cos\left(\alpha+\beta\right)$ are both positive.

\solution %2
Since $\alpha$ and $\beta$ are acute angles,
\[
\sin{\alpha} = \dfrac{4}{5} \implies \cos{\alpha} = \sqrt{1-\sin^{2}{\alpha}} = \sqrt{1-\left(\dfrac{4}{5}\right)^2} = \dfrac{3}{5}
\]
\[
\sin{\beta} = \dfrac{12}{13} \implies \cos{\beta} = \sqrt{1-\sin^{2}{\beta}} = \sqrt{1-\left(\dfrac{12}{13}\right)^2} = \dfrac{5}{13}
\]
\begin{align*}
\sin\left(\alpha+\beta\right) &= \sin{\alpha}\cos{\beta} + \cos{\alpha}\sin{\beta} \\
&= \dfrac{4}{5} \cdot \dfrac{5}{13} + \dfrac{3}{5} \cdot \dfrac{12}{13} \\
&= \dfrac{20}{65} + \dfrac{36}{65} \\
&= \dfrac{56}{65}
\end{align*}
\begin{align*}
\cos\left(\alpha+\beta\right) &= \cos{\alpha}\cos{\beta} - \sin{\alpha}\sin{\beta} \\
&= \dfrac{3}{5} \cdot \dfrac{5}{13} - \dfrac{4}{5} \cdot \dfrac{12}{13} \\
&= \dfrac{15}{65} - \dfrac{48}{65} \\
&= -\dfrac{33}{65}
\end{align*}

$\alpha+\beta$ lies in the second quadrant because $\sin\left(\alpha+\beta\right)$ is positive and $\cos\left(\alpha+\beta\right)$ is negative.

\solution %3
\[
\sin{\alpha} = \dfrac{3}{5} \implies \cos{\alpha} = \pm \sqrt{1-\sin^{2}{\alpha}} = \pm \sqrt{1-\left(\dfrac{3}{5}\right)^2} = \pm \dfrac{4}{5}
\]
\[
\sin{\beta} = \dfrac{5}{13} \implies \cos{\beta} = \pm \sqrt{1-\sin^{2}{\beta}} = \pm \sqrt{1-\left(\dfrac{5}{13}\right)^2} = \pm \dfrac{12}{13}
\]
$\cos{\alpha} > 0$, $\cos{\beta} > 0$:
\begin{align*}
\sin\left(\alpha+\beta\right) &= \sin{\alpha}\cos{\beta} + \cos{\alpha}\sin{\beta} \\
&= \dfrac{3}{5} \cdot \dfrac{12}{13} + \dfrac{4}{5} \cdot \dfrac{5}{13} \\
&= \dfrac{36}{65} + \dfrac{20}{65} \\
&= \dfrac{56}{65}
\end{align*}

$\cos{\alpha} > 0$, $\cos{\beta} < 0$:
\begin{align*}
\sin\left(\alpha+\beta\right) &= \sin{\alpha}\cos{\beta} + \cos{\alpha}\sin{\beta} \\
&= \dfrac{3}{5} \left(-\dfrac{12}{13}\right) + \dfrac{4}{5} \cdot \dfrac{5}{13} \\
&= -\dfrac{36}{65} + \dfrac{20}{65} \\
&= -\dfrac{16}{65}
\end{align*}

$\cos{\alpha} < 0$, $\cos{\beta} > 0$:
\begin{align*}
\sin\left(\alpha+\beta\right) &= \sin{\alpha}\cos{\beta} + \cos{\alpha}\sin{\beta} \\
&= \dfrac{3}{5} \cdot \dfrac{12}{13} + \left(-\dfrac{4}{5}\right) \cdot \dfrac{5}{13} \\
&= \dfrac{36}{65} - \dfrac{20}{65} \\
&= \dfrac{16}{65}
\end{align*}

$\cos{\alpha} < 0$, $\cos{\beta} < 0$:
\begin{align*}
\sin\left(\alpha+\beta\right) &= \sin{\alpha}\cos{\beta} + \cos{\alpha}\sin{\beta} \\
&= \dfrac{3}{5} \left(-\dfrac{12}{13}\right) + \left(-\dfrac{4}{5}\right) \cdot \dfrac{5}{13} \\
&= -\dfrac{36}{65} - \dfrac{20}{65} \\
&= -\dfrac{56}{65}
\end{align*}

There are four possible answers for $\sin\left(\alpha+\beta\right)$.

\solution %4
\begin{subsolutions}
\subsolution %a
\begin{align*}
\sin{\dfrac{2\pi}{3}}\cos{\dfrac{\pi}{3}} - \cos{\dfrac{2\pi}{3}}\sin{\dfrac{\pi}{3}} &= \dfrac{\sqrt{3}}{2} \cdot \dfrac{1}{2} - \left(-\dfrac{1}{2}\right) \dfrac{\sqrt{3}}{2} \\
&= \dfrac{\sqrt{3}}{4} + \dfrac{\sqrt{3}}{4} \\
&= \dfrac{\sqrt{3}}{2} \\
&= \sin\left(\dfrac{\pi}{3}\right) \\
&= \sin\left(\dfrac{2\pi}{3} - \dfrac{\pi}{3}\right)
\end{align*}

\subsolution %b
\begin{align*}
\sin{\dfrac{\pi}{4}}\cos{\dfrac{3\pi}{4}} - \cos{\dfrac{\pi}{4}}\sin{\dfrac{3\pi}{4}} &= \dfrac{\sqrt{2}}{2} \left(-\dfrac{\sqrt{2}}{2}\right) - \dfrac{\sqrt{2}}{2} \cdot \dfrac{\sqrt{2}}{2} \\
&= -\dfrac{1}{2} - \dfrac{1}{2} \\
&= -1 \\
&= \sin\left(-\dfrac{\pi}{2}\right) \\
&= \sin\left(\dfrac{\pi}{4} - \dfrac{3\pi}{4}\right)
\end{align*}

\subsolution %c
\begin{align*}
\sin\left(-\dfrac{\pi}{6}\right)\cos{\dfrac{3\pi}{2}} - \cos\left(-\dfrac{\pi}{6}\right)\sin{\dfrac{3\pi}{2}} &= -\dfrac{1}{2} \cdot 0 - \dfrac{\sqrt{3}}{2} \left(-1\right) \\
&= \dfrac{\sqrt{3}}{2} \\
&= \sin\left(\dfrac{\pi}{3}\right) \\
&= \sin\left(-\dfrac{5\pi}{3}\right) \\
&= \sin\left(-\dfrac{\pi}{6} - \dfrac{3\pi}{2}\right)
\end{align*}

\end{subsolutions}

\solution %5
Applying the identity
\[
(A-B)^{2} + (A+B)^{2} = A^2-2AB+B^2 + A^2+2AB+B^2 = 2A^2 + 2B^2,
\]
we have that,
\begin{align*}
    \cos^{2}\left(\gamma+\delta\right) + \cos^{2}\left(\gamma-\delta\right) &= \left(\cos{\gamma}\cos{\delta} - \sin{\gamma}\sin{\delta}\right)^2 + \left(\cos{\gamma}\cos{\delta} + \sin{\gamma}\sin{\delta}\right)^2 \\
    &= 2\cos^{2}{\gamma}\cos^{2}{\delta} + 2\sin^{2}{\gamma}\sin^{2}{\delta}.
\end{align*}

Therefore,
\begin{align*}
\cos^{2}{\alpha} + \cos^{2}\left(\dfrac{2\pi}{3} + \alpha\right) +  \cos^{2}\left(\dfrac{2\pi}{3} - \alpha\right) &= \cos^{2}{\alpha} + 2\cos^{2}{\dfrac{2\pi}{3}}\cos^{2}{\alpha} + 2\sin^{2}{\dfrac{2\pi}{3}}\sin^{2}{\alpha} \\
&= \cos^{2}{\alpha} + 2\left(\dfrac{1}{4}\right)\cos^{2}{\alpha} + 2\left(\dfrac{3}{4}\right)\sin^{2}{\alpha} \\
&= \dfrac{3}{2} \left(\cos^{2}{\alpha} + \sin^{2}{\alpha}\right) \\
&= \dfrac{3}{2}
\end{align*}

\solution %6
\begin{align*}
\sin\left(x+y\right) + \sin\left(x-y\right) &= \sin{x}\cos{y} + \cos{x}\sin{y} + \sin{x}\cos{y} - \cos{x}\sin{y} \\
&= \sin{x}\cos{y} + \sin{x}\cos{y} \\
&= 2\sin{x}\cos{y}
\end{align*}

\solution %7
\begin{align*}
\cos\left(x+y\right) + \cos\left(x-y\right) &= \cos{x}\cos{y} - \sin{x}\sin{y} + \cos{x}\cos{y} + \sin{x}\sin{y} \\
&= \cos{x}\cos{y} + \cos{x}\cos{y} \\
&= 2\cos{x}\cos{y}
\end{align*}

\solution %8
Since $(A-B)(A+B)=A^2-B^2$,
\begin{align*}
\cos\left(x+y\right)\cos\left(x-y\right) &= \left(\cos{x}\cos{y} - \sin{x}\sin{y}\right) \left(\cos{x}\cos{y} + \sin{x}\sin{y}\right) \\
&= \cos^{2}{x}\cos^{2}{y} - \sin^{2}{x}\sin^{2}{y}
\end{align*}

\solution %9
Since $(A+B)(A-B)=A^2-B^2$,
\begin{align*}
\sin\left(x+y\right)\sin\left(x-y\right) &= \left(\sin{x}\cos{y} + \cos{x}\sin{y}\right) \left(\sin{x}\cos{y} - \cos{x}\sin{y}\right) \\
&= \sin^{2}{x}\cos^{2}{y} - \cos^{2}{x}\sin^{2}{y}
\end{align*}

\solution %10
\begin{align*}
\cos\left(x+y\right)\cos\left(x-y\right) - \sin\left(x+y\right)\sin\left(x-y\right) &= \cos^{2}{x}\cos^{2}{y} - \sin^{2}{x}\sin^{2}{y} - \left(\sin^{2}{x}\cos^{2}{y} - \cos^{2}{x}\sin^{2}{y}\right) \\
&= \cos^{2}{x}\left(\cos^{2}{y} + \sin^{2}{y}\right) - \sin^{2}{x}\left(\sin^2{y}+\cos^{2}{y}\right) \\
&= \cos^{2}{x} - \sin^{2}{x}
\end{align*}

\solution %11
\begin{align*}
\cos{2x} &= \cos\left(x+x\right) \\
&= \cos{x}\cos{x}-\sin{x}\sin{x} \\
&= \cos^{2}{x} - \sin^{2}{x}
\end{align*}
There is no error, because $\cos{2x}=\cos^{2}{x} - \sin^{2}{x}$.

\solution %12
\begin{align*}
\cos\left(\alpha+\beta\right)\cos{\beta} + \sin\left(\alpha+\beta\right)\sin{\beta} &= \left(\cos{\alpha}\cos{\beta}-\sin{\alpha}\sin{\beta}\right)\cos{\beta} + \left(\sin{\alpha}\cos{\beta}+\cos{\alpha}\sin{\beta}\right)\sin{\beta} \\
&= \cos{\alpha}\cos^{2}{\beta} - \sin{\alpha}\sin{\beta}\cos{\beta} + \sin{\alpha}\sin{\beta}\cos{\beta}+\sin^{2}{\beta}\cos{\alpha} \\
&= \cos{\alpha}\cos^{2}{\beta} + \sin^{2}{\beta}\cos{\alpha} \\
&= \cos{\alpha} \left(\cos^{2}{\beta}+\sin^{2}{\beta}\right) \\
&= \cos{\alpha}
\end{align*}
Alternatively, by applying the cosine difference formula in reverse,
\[
\cos\left(\alpha+\beta\right)\cos{\beta} + \sin\left(\alpha+\beta\right)\sin{\beta} = \cos\left(\alpha+\beta - \beta\right) \\
= \cos{\alpha}
\]
\end{solutions}

\begin{solutions}{Page 144}
\solution %1
\begin{align*}
\tan\left(\dfrac{7\pi}{6}+\dfrac{5\pi}{3}\right) &= \dfrac{\tan{\dfrac{7\pi}{6}} + \tan{\dfrac{5\pi}{3}}}{1-\tan{\dfrac{7\pi}{6}}  \tan{\dfrac{5\pi}{3}}} \\
&= \dfrac{1/\sqrt{3} - \sqrt{3}}{1-\left(1/\sqrt{3}\right)\left(-\sqrt{3}\right)} \\
&= \dfrac{1/\sqrt{3} - \sqrt{3}}{2} \\
&= \dfrac{1/\sqrt{3} - \sqrt{3}}{2} \cdot \dfrac{\sqrt{3}}{\sqrt{3}}\\
&= \dfrac{1 - 3}{2\sqrt{3}} \\
&= -\dfrac{1}{\sqrt{3}} \\
&= \tan\left(-\dfrac{\pi}{6}\right) \\
&= \tan\left(\dfrac{17\pi}{6}\right) \\
&= \tan\left(\dfrac{7\pi}{6} + \dfrac{5\pi}{3}\right)
\end{align*}

\solution %2
Because the tangent function is odd, we know $\tan{-\beta}=-\tan{\beta}$.
\begin{align*}
\tan\left(\alpha-\beta\right) &= \tan\left(\alpha + \left(-\beta\right)\right) \\
&= \dfrac{\tan{\alpha} + \tan{-\beta}}{1-\tan{\alpha}  \tan{-\beta}} \\
&= \dfrac{\tan{\alpha} - \tan{\beta}}{1+\tan{\alpha}  \tan{\beta}}
\end{align*}

\solution %3
\begin{align*}
\tan\left(\dfrac{\pi}{4}+\alpha\right) &= \dfrac{\tan{\dfrac{\pi}{4}} + \tan{\alpha}}{1-\tan{\dfrac{\pi}{4}}  \tan{\alpha}} \\
&= \dfrac{1 + \tan{\alpha}}{1 - \tan{\alpha}}
\end{align*}

\solution %4
\begin{align*}
\tan\left(\dfrac{\pi}{4}-\alpha\right) &= \dfrac{\tan{\dfrac{\pi}{4}} - \tan{\alpha}}{1+\tan{\dfrac{\pi}{4}}  \tan{\alpha}} \\
&= \dfrac{1 - \tan{\alpha}}{1 + \tan{\alpha}}
\end{align*}

\solution %5
Since $\beta = \pi/4 - \alpha$,
\begin{align*}
\left(1+\tan{\alpha}\right)\left(1+\tan{\beta}\right) &= \left(1+\tan{\alpha}\right)\left(1+\tan\left(\dfrac{\pi}{4}-\alpha\right)\right) \\
&= \left(1 + \tan{\alpha}\right) \left(1 + \dfrac{1 - \tan{\alpha}}{1 + \tan{\alpha}}\right) \\
&= 1 + \tan{\alpha} + 1 - \tan{\alpha} \\
&= 2
\end{align*}

Alternatively, since $\tan\left(\alpha+\beta\right) = \tan \pi/4 = 1$,
\begin{align*}
\left(1+\tan{\alpha}\right)\left(1+\tan{\beta}\right) &= 1 + \tan{\alpha} + \tan{\beta} + \tan{\alpha}\tan{\beta} \\
&= 1 + \left(\tan{\alpha} + \tan{\beta}\right)\left(\dfrac{1-\tan{\alpha}\tan{\beta}}{1-\tan{\alpha}\tan{\beta}}\right) + \tan{\alpha}\tan{\beta} \\
&= 1 + \left(1-\tan{\alpha}\tan{\beta}\right)\left(\dfrac{\tan{\alpha} + \tan{\beta}}{1-\tan{\alpha}\tan{\beta}}\right) + \tan{\alpha}\tan{\beta} \\
&= 1 + \left(1-\tan{\alpha}\tan{\beta}\right)\tan\left(\alpha+\beta\right) + \tan{\alpha}\tan{\beta} \\
&= 1+ 1-\tan{\alpha}\tan{\beta} + \tan{\alpha}\tan{\beta} \\
&= 2
\end{align*}

\solution %6
\begin{align*}
\tan\left(\alpha+\beta+\gamma\right) &= \dfrac{\tan\left(\alpha+\beta\right) + \tan{\gamma}}{1-\tan\left(\alpha+\beta\right)\tan{\gamma}} \\
&= \dfrac{\dfrac{\tan{\alpha} + \tan{\beta}}{1-\tan{\alpha}\tan{\beta}} + \tan\gamma}{1-\dfrac{\tan{\alpha} + \tan{\beta}}{1-\tan{\alpha}\tan{\beta}}\tan\gamma} \\
&= \dfrac{\dfrac{\tan{\alpha} + \tan{\beta}}{1-\tan{\alpha}\tan{\beta}} + \tan\gamma}{1-\dfrac{\tan{\alpha} + \tan{\beta}}{1-\tan{\alpha}\tan{\beta}}\tan\gamma} \cdot \dfrac{1-\tan{\alpha}\tan{\beta}}{1-\tan{\alpha}\tan{\beta}} \\
&= \dfrac{\tan{\alpha}+\tan{\beta}+\tan{\gamma}\left(1-\tan{\alpha}\tan{\beta}\right)}{1 - \tan{\alpha}\tan{\beta} - \left(\tan{\alpha} + \tan{\beta}\right)\tan{\gamma}} \\
&= \dfrac{\tan{\alpha}+\tan{\beta}+\tan{\gamma} -\tan{\alpha}\tan{\beta}\tan{\gamma}}{1 - \tan{\alpha}\tan{\beta} - \tan{\alpha}\tan{\gamma} - \tan{\beta}\tan{\gamma}}
\end{align*}

\solution %7
Since $\tan\left(\alpha+\beta+\gamma\right) = \tan{\pi} = 0$, from the previous part, we have
\[
\tan{\alpha}+\tan{\beta}+\tan{\gamma} -\tan{\alpha}\tan{\beta}\tan{\gamma} = 0 \implies \tan{\alpha}+\tan{\beta}+\tan{\gamma} = \tan{\alpha}\tan{\beta}\tan{\gamma}.
\]
This is because if a fraction equals zero, its numerator must be zero.

\solution %8
First, applying the tangent addition formula,
\[
\tan{3\alpha} = \tan\left(2\alpha+\alpha\right) = \dfrac{\tan{2\alpha} + \tan{\alpha}}{1-\tan{2\alpha}\tan{\alpha}} \implies \tan{2\alpha} + \tan{\alpha} = \tan{3\alpha}\left(1-\tan{2\alpha}\tan{\alpha}\right).
\]
Therefore,
\[
\tan{3\alpha} - \tan{2\alpha} - \tan{\alpha} = \tan{3\alpha} - \tan{3\alpha}\left(1-\tan{2\alpha}\tan{\alpha}\right) = \tan{3\alpha}\tan{2\alpha}\tan{\alpha}.
\]

$\tan{\alpha}$ is not defined for $\alpha = \tfrac{\pi}{2} + n\pi$. Therefore, $\tan{2\alpha}$ is not defined for $\alpha = \tfrac{\pi}{4} + n\tfrac{\pi}{2}$ and $\tan{3\alpha}$ is not defined for $\alpha = \tfrac{\pi}{6} + n\tfrac{\pi}{3}$.
\end{solutions}

\begin{solutions}{Page 147}
\solution %1
    \begin{subsolutions}
        \solution %a
        Since $\sin^2 \alpha + \cos^2\alpha = 1$, we know $\cos\alpha = \sqrt{1-(\frac{7}{25})^2}$ (and cannot be the negative version because $\cos \alpha$ is given as positive).\\
        Thus $\sin 2\alpha = 2\sin\alpha \cos \alpha = 2\cdot \frac{7}{25}\cdot\sqrt{1-(\frac{7}{25})^2}=\frac{336}{625}$.\\
        And $\cos2\alpha =1-2\sin^2\alpha = 1-2(\frac{7}{25})^2=\frac{527}{625}$.
        \solution %b
        This part is similar except that we use the negative version of cosine, namely $\cos\alpha=-\sqrt{1-(\frac{7}{25})^2}$.\\
        Thus $\sin 2\alpha = 2\sin\alpha \cos \alpha = -\frac{336}{625}$.\\
        However, cosine value remains the same: $\cos2\alpha =1-2\sin^2\alpha = 1-2(\frac{7}{25})^2=\frac{527}{625}$.
    \end{subsolutions}

\solution %2
Firstly, $\sin 2\alpha=2\sin\alpha\cos\alpha$. In other words, the sine value will be a product of rational numbers, so it will also be rational. Similarly, $\cos 2\alpha=2\cos ^2 \alpha - 1$ will be rational because $\cos\alpha$ is rational. Exercise 1 confirms this result.

\solution %3
Let's use the double angle formula $\cos 2\alpha=2\cos^2\alpha -1$.
\[
\cos{2\alpha} = \cos^{2}{\alpha} \implies 2\cos^{2}{\alpha}-1 = \cos^{2}{\alpha}  \implies \cos^{2}{\alpha} = 1 \implies \cos{\alpha} = \pm 1
\]
$\cos{\alpha}$ has a magnitude of 1 precisely when $\alpha$ is an integer multiple of $\pi$, so the student's angle must have also been an integer multiple of $\pi$.

\solution %4
We start with the given equation:
\begin{align*}
    \sin{\alpha}+\cos{\alpha} &= 0.2\\
    \sin^{2}{\alpha}+2\sin{\alpha}\cos{\alpha+}\cos^2\alpha&=0.04 &\text{(Squared both sides.)}\\
    1+2\sin{\alpha}\cos{\alpha}&=0.04\\
    2\sin{\alpha}\cos{\alpha} &=-0.96
\end{align*}
Note that that is simply $\sin 2\alpha$. Hooray!

\solution %5
We can follow a very similar strategy here to find $1-2\sin{\alpha}\cos{\alpha}=0.09$. Then $\sin {2\alpha}=2\sin{\alpha}\cos\alpha=0.91$.

\solution %6
\begin{align*}
\cos{2\alpha}\cos{\alpha} + \sin{2\alpha}\sin{\alpha} &= \left(2\cos^{2}{\alpha}-1\right)\cos{\alpha} + 2\sin{\alpha}\cos{\alpha}\sin{\alpha} \\
&= \cos{\alpha} \left(2\cos^{2}{\alpha - 1 + 2\sin^{2}{\alpha}}\right) \\
&= \cos{\alpha} \left(2-1\right) \\
&= \cos{\alpha}
\end{align*}

Alternatively, we may use the cosine difference formula.
\[
\cos{2\alpha}\cos{\alpha} + \sin{2\alpha}\sin{\alpha} = \cos\left(2\alpha-\alpha\right) \\
= \cos{\alpha}
\]

\solution %7
Applying the sine addition formula,
\[
\sin{2\alpha}\cos{\alpha} + \cos{2\alpha}\sin{\alpha} = \sin\left(2\alpha+\alpha\right) = \sin{3\alpha}
\]
Applying the sine subtraction formula,
\[
\sin{4\alpha}\cos{\alpha} - \cos{4\alpha}\sin{\alpha} = \sin\left(4\alpha-\alpha\right) = \sin{3\alpha}
\]
Since both sides of the identity are equal to $\sin{3\alpha}$, the identity is true.

\solution %8
Yes, the book asked you to prove something incorrect! For counterexample, consider that $\cos {2\alpha}$ can be negative but $\cos^{2}{\alpha}$ is never negative. However, we can prove a relationship.\\
Recall that $\cos{2\alpha}=\cos^{2}{\alpha} - \sin^{2}{\alpha}$. However, $\sin^{2}{\alpha} \geq 0$ since it's a square.
\begin{align*}
    \sin^2\alpha&\geq 0\\
    -\sin^2\alpha&\leq 0\\
    \cos^2\alpha-\sin^2\alpha&\leq \cos^2\alpha\\
    \cos 2\alpha &\leq \cos^2\alpha
\end{align*}

\solution %9
\begin{align*}
\left(\sin{\dfrac{\alpha}{2}} - \cos{\dfrac{\alpha}{2}}\right)^2 &= \sin^{2}{\dfrac{\alpha}{2}} - 2 \sin{\dfrac{\alpha}{2}}\cos{\dfrac{\alpha}{2}} + \cos^{2}{\dfrac{\alpha}{2}} \\
&= 1 - \sin{\alpha}
\end{align*}

\solution %10
Following the hint, we compute the value of $\cos{10\degree}\sin{10\degree}\sin{50\degree}\sin{70\degree}$.
\begin{align*}
M\cos{10\degree} &= \cos{10\degree}\sin{10\degree}\sin{50\degree}\sin{70\degree} \\
&= \dfrac{1}{2}\sin{20\degree}\sin{50\degree}\sin{70\degree} \\
&= \dfrac{1}{2}\cos{70\degree}\sin{50\degree}\sin{70\degree} \\
&= \dfrac{1}{4}\sin{140\degree}\sin{50\degree} \\
&= \dfrac{1}{4}\sin{40\degree}\sin{50\degree} \\
&= \dfrac{1}{4}\cos{50\degree}\sin{50\degree} \\
&= \dfrac{1}{8}\sin{100\degree} \\
&= \dfrac{1}{8}\sin{80\degree} \\
&= \dfrac{1}{8}\cos{10\degree}
\end{align*}
Since $M\cos{10\degree} = \tfrac{1}{8}\cos{10\degree}$, we have that the value of the original expression $M$ is equal to $\tfrac{1}{8}$.

\solution %11
We begin by computing $\sin{20\degree}\cos{20\degree}\cos{40\degree}\cos{80\degree}$.
\begin{align*}
\sin{20\degree}\cos{20\degree}\cos{40\degree}\cos{80\degree} &= \dfrac{1}{2}\sin{40\degree}\cos{40\degree}\cos{80\degree} \\
&= \dfrac{1}{4}\sin{80\degree}\cos{80\degree} \\
&= \dfrac{1}{8}\sin{160\degree} \\
&= \dfrac{1}{8}\sin{20\degree}
\end{align*}
This implies that $\cos{20\degree}\cos{40\degree}\cos{80\degree} = 1/8$.
\solution %12
We begin by computing $\cos{\pi/10}\sin{\pi/10}\sin{\pi/5}$.
\begin{align*}
\cos{\dfrac{\pi}{10}}\sin{\dfrac{\pi}{10}}\cos{\dfrac{\pi}{5}} &= \dfrac{1}{2}\sin{\dfrac{\pi}{5}}\cos{\dfrac{\pi}{5}} \\
&= \dfrac{1}{4}\sin{\dfrac{2\pi}{5}} \\
&= \dfrac{1}{4}\cos{\dfrac{\pi}{10}}
\end{align*}
This implies that $\sin{\pi/10}\cos{\pi/5} = 1/4$.
\end{solutions}

\begin{solutions}{Page 148}
\solution %1
\begin{align*}
\cos{3\alpha} &= \cos\left(2\alpha+\alpha\right) \\
&= \cos{2\alpha}\cos{\alpha} - \sin{2\alpha}\sin{\alpha} \\
&= \left(2\cos^{2}{\alpha}-1\right)\cos{\alpha} - 2\sin{\alpha}\cos{\alpha}\sin{\alpha} \\
&= 2\cos^{3}{\alpha}-\cos{\alpha}-2\sin^{2}{\alpha}\cos{\alpha} \\
&= 2\cos^{3}{\alpha}-\cos{\alpha}-2\left(1-\cos^{2}{\alpha}\right)\cos{\alpha} \\
&= 4\cos^{3}{\alpha} - 3\cos{\alpha}
\end{align*}

\solution %2
\begin{align*}
\sin{3\alpha} &= 3\sin{\alpha}-4\sin^{3}{\alpha} \\
&= 3\left(\dfrac{3}{5}\right)-4\left(\dfrac{3}{5}\right)^3 \\
&= \dfrac{9}{5} - 4 \cdot \dfrac{27}{125} \\
&= \dfrac{117}{125}
\end{align*}

If $\sin{\alpha}=3/5$, then $\cos{\alpha}=\pm 4/5$. Therefore,
\begin{align*}
\cos{3\alpha} &= 4\cos^{3}{\alpha}-3\cos{\alpha} \\
&= 4\left(\pm \dfrac{4}{5}\right)^3 - 3\left(\pm \dfrac{4}{5}\right) \\
&= \pm \dfrac{256}{125} \mp \dfrac{12}{5} \\
&= \mp \dfrac{44}{125}
\end{align*}

\solution %3
If $\cos{\alpha}=4/5$, then $\sin{\alpha}=\pm 3/5$. Therefore,
\begin{align*}
\sin{3\alpha} &= 3\sin{\alpha}-4\sin^{3}{\alpha} \\
&= 3\left(\pm \dfrac{3}{5}\right)-4\left(\pm \dfrac{3}{5}\right)^3 \\
&= \pm \dfrac{9}{5} \mp 4 \cdot \dfrac{27}{125} \\
&= \pm \dfrac{117}{125}
\end{align*}

\begin{align*}
\cos{3\alpha} &= 4\cos^{3}{\alpha}-3\cos{\alpha} \\
&= 4\left(\dfrac{4}{5}\right)^3 - 3\left(\dfrac{4}{5}\right) \\
&= \dfrac{256}{125} - \dfrac{12}{5} \\
&= -\dfrac{44}{125}
\end{align*}

\solution %4
\begin{subsolutions}
\subsolution %a
\begin{align*}
\cos{4\alpha} &= 2\cos^{2}{2\alpha} - 1 \\
&= 2\left(2\cos^{2}{\alpha}-1\right)^2-1 \\
&= 2\left(4\cos^{4}{\alpha}-4\cos^{2}{\alpha}+1\right)-1 \\
&= 8\cos^{4}{\alpha} - 8\cos^{2}{\alpha} + 1
\end{align*}

\subsolution %b
\begin{align*}
\cos{4\alpha} &= 1-2\sin^{2}{2\alpha} \\
&= 1-8\sin^{2}{\alpha}\cos^{2}{\alpha} \\
&= 1-8\sin^{2}{\alpha}\left(1-\sin^{2}{\alpha}\right) \\
&= 1-8\sin^{2}{\alpha}+8\sin^{4}{\alpha}
\end{align*}
\end{subsolutions}

\solution %5
\begin{align*}
\sin{3\alpha}\cos{\alpha}-\cos{3\alpha}\sin{\alpha} &= \left(3\sin{\alpha}-4\sin^{3}{\alpha}\right)\cos{\alpha}-\left(4\cos^{3}{\alpha}-3\cos{\alpha}\right)\sin{\alpha} \\
&= 3\sin{\alpha}\cos{\alpha}-4\sin^{3}{\alpha}\cos{\alpha}-4\cos^{3}{\alpha}\sin{\alpha}+3\sin{\alpha}\cos{\alpha} \\
&= 2\sin{\alpha}\cos{\alpha}\left(3 - 2\sin^{2}{\alpha} - 2\cos^{2}{\alpha}\right) \\
&= 2\sin{\alpha}\cos{\alpha} \left(3-2\right) \\
&= \sin{2\alpha}
\end{align*}

Alternatively, applying the sine difference formula,
\[
\sin{3\alpha}\cos{\alpha}-\cos{3\alpha}\sin{\alpha} = \sin\left(3\alpha-\alpha\right) = \sin{2\alpha}
\]

\solution %6
\begin{align*}
\dfrac{\sin{3\alpha}}{\sin{\alpha}} - \dfrac{\cos{3\alpha}}{\cos{\alpha}} &= \dfrac{3\sin{\alpha}-4\sin^{3}{\alpha}}{\sin{\alpha}} - \dfrac{4\cos^{3}{\alpha}-3\cos{\alpha}}{\cos{\alpha}}\\
&=3-4\sin^{2}{\alpha} - \left(4\cos^{2}{\alpha}-3\right) \\
&= 6-4\sin^{2}{\alpha}-4\cos^{2}{\alpha} \\
&= 2
\end{align*}

Alternatively, using the result from the previous exercise,
\begin{align*}
\dfrac{\sin{3\alpha}}{\sin{\alpha}} - \dfrac{\cos{3\alpha}}{\cos{\alpha}} &= \dfrac{\sin{3\alpha}\cos{\alpha}-\cos{3\alpha}\sin{\alpha}}{\sin{\alpha}\cos{\alpha}} \\
&= \dfrac{\sin{2\alpha}}{\sin{\alpha}\cos{\alpha}} \\
&= \dfrac{2\sin{\alpha}\cos{\alpha}}{\sin{\alpha}\cos{\alpha}} \\
&= 2
\end{align*}

\solution %7
\begin{subsolutions}
\subsolution %a
Applying the result of Exercise 9 from Section 3 of this chapter,
\begin{align*}
4\sin{\alpha}\sin\left(60\degree+\alpha\right)\sin\left(60\degree-\alpha\right) &= 4\sin{\alpha}\left(\sin^{2}{60\degree}\cos^{2}{\alpha} - \cos^{2}{60\degree}\sin^{2}{\alpha}\right) \\
&= 4\sin{\alpha}\left(\dfrac{3}{4}\cos^{2}{\alpha} - \dfrac{1}{4}\sin^{2}{\alpha}\right) \\
&= 3\sin{\alpha}\cos^{2}{\alpha} - \sin^{3}{\alpha} \\
&= 3\sin{\alpha}\left(1-\sin^{2}{\alpha}\right)  - \sin^{3}{\alpha} \\
&= 3\sin{\alpha}-4\sin^{3}{\alpha}  \\
&= \sin{3\alpha}
\end{align*}

\subsolution %b
Applying the result of Exercise 8 from Section 3 of this chapter,
\begin{align*}
4\cos{\alpha}\cos\left(60\degree+\alpha\right)\cos\left(60\degree-\alpha\right) &= 4\cos{\alpha}\left(\cos^{2}{60\degree}\cos^{2}{\alpha} - \sin^{2}{60\degree}\sin^{2}{\alpha}\right) \\
&= 4\cos{\alpha}\left(\dfrac{1}{4}\cos^{2}{\alpha} - \dfrac{3}{4}\sin^{2}{\alpha}\right) \\
&= \cos^{3}{\alpha} - 3\sin^{2}{\alpha}\cos{\alpha} \\
&= \cos^{3}{\alpha} - 3\left(1-\cos^{2}{\alpha}\right)\cos{\alpha} \\
&= 4\cos^{3}{\alpha} - 3\cos{\alpha}
\end{align*}

\end{subsolutions}

\solution %8
\begin{align*}
\sin{4\alpha} &= 2\sin{2\alpha}\cos{2\alpha} \\
&=4\sin{\alpha}\cos{\alpha}\left(2\cos^{2}{\alpha}-1\right) \\
&= 8\sin{\alpha}\cos^{3}{\alpha}-4\sin{\alpha}\cos{\alpha} \\
&\implies \dfrac{\sin{4\alpha}}{\sin{\alpha}} = 8\cos^{3}{\alpha}-4\cos{\alpha} 
\end{align*}

\solution %9
In the penultimate step, we apply the following identity:
\[
\left(A-B\right)^{3} = A^3 - 3A^{2}B + 3AB^{2} - B^3,
\]
taking $A=\cos^{2}{\alpha}$ and $B=\sin^{2}{\alpha}$.

\begin{align*}
\sin{3\alpha}\sin^{3}{\alpha} + \cos{3\alpha}\cos^{3}{\alpha} &= \left(3\sin{\alpha}-4\sin^{3}{\alpha}\right)\sin^{3}{\alpha} + \left(4\cos^{3}{\alpha}-3\cos{\alpha}\right)\cos^{3}{\alpha} \\
&= 3\sin^{4}{\alpha}-4\sin^{6}{\alpha}+4\cos^{6}{\alpha}-3\cos^{4}{\alpha} \\
&=3\sin^{4}{\alpha}-4\sin^{4}{\alpha}\left(1-\cos^{2}{\alpha}\right)+4\cos^{4}{\alpha}\left(1-\sin^{2}{\alpha}\right)-3\cos^{4}{\alpha} \\
&= \sin^{4}{\alpha}\left(4\cos^{2}{\alpha}-1\right) + \cos^{4}{\alpha}\left(1-4\sin^{2}{\alpha}\right) \\
&= \sin^{4}{\alpha}\left(4\cos^{2}{\alpha}-\sin^{2}{\alpha}-\cos^{2}{\alpha}\right) + \cos^{4}{\alpha}\left(\sin^{2}{\alpha}+\cos^{2}{\alpha}-4\sin^{2}{\alpha}\right) \\
&=3\sin^{4}{\alpha}\cos^{2}{\alpha}-\sin^{6}{\alpha}+\cos^{6}{\alpha}-3\cos^{4}{\alpha}\sin^{2}{\alpha} \\
&= \left(\cos^{2}{\alpha}-\sin^{2}{\alpha}\right)^3 \\
&= \cos^{3}{2\alpha}
\end{align*}
\end{solutions}

\begin{solutions}{Page 150}
\solution %1
\[
\cos{\alpha}=1 \implies \cos{\dfrac{\alpha}{2}} = \pm \sqrt{\dfrac{1+1}{2}} = \pm 1
\]
$\cos{\alpha}$ is equal to 1 when $\alpha$ is an integer multiple of $2\pi$. If $\alpha = 4n\pi$ for some integer $n$ (i.e., an even integer multiple of $2\pi$), then $\cos{\alpha/2} = \cos{2n\pi} = 1$. Otherwise, if $\alpha = \left(4n+2\right)\pi$ for some integer $n$ (i.e., an odd integer multiple of $2\pi$), then $\cos{\alpha/2} = \cos\left(2n+1\right)\pi = -1$.

For a particular example of each case, we may set $n=0$.

$\alpha = 4\left(0\right)\pi = 0$:
\[
\cos{0} = 1, \cos{\dfrac{0}{2}} = \cos{0} = 1
\]

$\alpha = \left(4\left(0\right)+2\right)\pi = 2\pi$:
\[
\cos{2\pi} = 1, \cos{\dfrac{2\pi}{2}} = \cos{\pi} = -1
\]

\solution %2
\begin{subsolutions}
\subsolution %a
We take the positive square root because $60\degree / 2 = 30\degree$ lies in the first quadrant.
\begin{align*}
\cos \dfrac{60\degree}{2} &= \sqrt{\dfrac{1+\cos{60\degree}}{2}} \\
&= \sqrt{\dfrac{1+1/2}{2}} \\
&= \sqrt{\dfrac{3}{4}} \\
&= \dfrac{\sqrt{3}}{2}
\end{align*}

\subsolution %b
We take the positive square root because $120\degree / 2 = 60\degree$ lies in the first quadrant.
\begin{align*}
\cos \dfrac{120\degree}{2} &= \sqrt{\dfrac{1+\cos{120\degree}}{2}} \\
&= \sqrt{\dfrac{1-1/2}{2}} \\
&= \sqrt{\dfrac{1}{4}} \\
&= \dfrac{1}{2}
\end{align*}

\subsolution %c
We take the negative square root because $240\degree / 2 = 120\degree$ lies in the second quadrant.
\begin{align*}
\cos \dfrac{240\degree}{2} &= -\sqrt{\dfrac{1+\cos{240\degree}}{2}} \\
&= -\sqrt{\dfrac{1-1/2}{2}} \\
&= -\sqrt{\dfrac{1}{4}} \\
&= -\dfrac{1}{2}
\end{align*}
\end{subsolutions}

\solution ~ %3
\begin{center}
\bgroup
\def\arraystretch{2.1}
\setlength\tabcolsep{15pt}
\begin{tabular}{ |c|c|c|c|c| }
\hline
$\alpha$
& Quadrant $\alpha$?
& $\alpha / 2$
& Quadrant $\alpha / 2$?
& $\cos{\alpha / 2}$\\
\hline
$780\degree$
& I
& $390\degree$
& I
& $\sqrt{3} / 2$\\
\hline
$1020\degree$
& IV
& $510\degree$
& II
& $-\sqrt{3} / 2$\\
\hline
$1140\degree$
& I
& $570\degree$
& III
& $-\sqrt{3} / 2$\\
\hline
$1380\degree$
& IV
& $690\degree$
& IV
& $\sqrt{3} / 2$\\
\hline
$-60\degree$
& IV
& $-30\degree$
& IV
& $\sqrt{3} / 2$\\
\hline
$-300\degree$
& I
& $-150\degree$
& III
& $-\sqrt{3} / 2$\\
\hline
$-420\degree$
& IV
& $-210\degree$
& II
& $-\sqrt{3} / 2$\\
\hline
$-660\degree$
& I
& $-330\degree$
& I
& $\sqrt{3} / 2$\\
\hline
$-780\degree$
& IV
& $-390\degree$
& IV
& $\sqrt{3} / 2$\\
\hline
\end{tabular}
\egroup
\end{center}

\solution %4
\begin{align*}
\sin{15\degree} &= \sqrt{\dfrac{1-\cos{30\degree}}{2}} \\
&= \sqrt{\dfrac{1-\sqrt{3}/2}{2}} \\
&= \sqrt{\dfrac{2-\sqrt{3}}{4}} \\
&= \dfrac{\sqrt{2-\sqrt{3}}}{2}
\end{align*}

\begin{align*}
\cos{15\degree} &= \sqrt{\dfrac{1+\cos{30\degree}}{2}} \\
&= \sqrt{\dfrac{1+\sqrt{3}/2}{2}} \\
&= \sqrt{\dfrac{2+\sqrt{3}}{4}} \\
&= \dfrac{\sqrt{2+\sqrt{3}}}{2}
\end{align*}

\solution %5
Because $\left\lvert \cos{\alpha} \right\rvert \leq 1$, $1 \pm \cos{\alpha} \geq 0$, so the expressions under the square roots in the sine and cosine half-angle formulas will not be negative.

\solution %6
The square root sign in the half-angle formula prevents it from being a rational trigonometric function, so the Principle of Analytic Continuation does not apply.

\solution %7
\begin{subsolutions}
\subsolution %a
\begin{align*}
\tan{\dfrac{\alpha}{2}}\tan{\dfrac{\beta}{2}} + \tan{\dfrac{\alpha}{2}}\tan{\dfrac{\gamma}{2}} + \tan{\dfrac{\beta}{2}}\tan{\dfrac{\gamma}{2}} &= \tan{\dfrac{\alpha}{2}}\tan{\dfrac{\beta}{2}} + \tan{\dfrac{\gamma}{2}} \left(\tan{\dfrac{\alpha}{2}} + \tan{\dfrac{\beta}{2}}\right) \\
&= \tan{\dfrac{\alpha}{2}}\tan{\dfrac{\beta}{2}} + \tan{\dfrac{\gamma}{2}}\tan{\dfrac{\alpha+\beta}{2}}\left(1-\tan{\dfrac{\alpha}{2}}\tan{\dfrac{\beta}{2}}\right) \\
&= \tan{\dfrac{\alpha}{2}}\tan{\dfrac{\beta}{2}} + 1-\tan{\dfrac{\alpha}{2}}\tan{\dfrac{\beta}{2}} \\
&= 1
\end{align*}
Alternatively, we can recall the extended tangent addition formula derived in Exercise 6 of Section 4 of this chapter:
\[
\tan\left(\alpha+\beta+\gamma\right) = \dfrac{\tan{\alpha}+\tan{\beta}+\tan{\gamma} -\tan{\alpha}\tan{\beta}\tan{\gamma}}{1 - \tan{\alpha}\tan{\beta} - \tan{\alpha}\tan{\gamma} - \tan{\beta}\tan{\gamma}}
\]
Since $\alpha/2 + \beta/2 + \gamma/2 = \pi / 2$, $\tan\left(\alpha/2 + \beta/2 + \gamma/2\right)$ is undefined. This implies that the denominator of the tangent addition formula is 0 (assuming that none of $\tan{\alpha/2}$, $\tan{\beta/2}$, or $\tan{\gamma/2}$ are undefined). Therefore, we can conclude
\[
\tan{\dfrac{\alpha}{2}}\tan{\dfrac{\beta}{2}} + \tan{\dfrac{\alpha}{2}}\tan{\dfrac{\gamma}{2}} + \tan{\dfrac{\beta}{2}}\tan{\dfrac{\gamma}{2}} = 1.
\]
\subsolution %b
Following a similar approach as to the previous part, we begin by noting that $\sin{\tfrac{\alpha+\beta}{2}} = \cos{\tfrac{\gamma}{2}}$.
\begin{align*}
4\cos{\dfrac{\alpha}{2}}\cos{\dfrac{\beta}{2}}\cos{\dfrac{\gamma}{2}} &= 4\cos{\dfrac{\alpha}{2}}\cos{\dfrac{\beta}{2}}\sin{\dfrac{\alpha+\beta}{2}} \\
&=4\cos{\dfrac{\alpha}{2}}\cos{\dfrac{\beta}{2}}\left(\sin{\dfrac{\alpha}{2}}\cos{\dfrac{\beta}{2}} + \cos{\dfrac{\alpha}{2}}\sin{\dfrac{\beta}{2}}\right) \\
&= 4\sin{\frac{\alpha}{2}}\cos{\dfrac{\alpha}{2}}\cos^{2}{\dfrac{\beta}{2}} + 4\sin{\dfrac{\beta}{2}}\cos{\dfrac{\beta}{2}}\cos^{2}{\dfrac{\alpha}{2}} \\
&= 2\sin{\alpha}\cos^{2}{\dfrac{\beta}{2}}+2\sin{\beta}\cos^{2}{\dfrac{\alpha}{2}} \\
&= 2\sin{\alpha}\left(\dfrac{1+\cos{\beta}}{2}\right)+2\sin{\beta}\left(\dfrac{1+\cos{\alpha}}{2}\right) \\
&= \sin{\alpha} + \sin{\alpha}\cos{\beta} + \sin{\beta} + \sin{\beta}\cos{\alpha} \\
&= \sin{\alpha} + \sin{\beta} + \sin\left(\alpha+\beta\right) \\
&= \sin{\alpha} + \sin{\beta} + \sin\left(\pi - \alpha - \beta\right) \\
&= \sin{\alpha} + \sin{\beta} + \sin{\gamma}
\end{align*}
\end{subsolutions}
\end{solutions}

\begin{solutions}{Page 152}
\solution %1
Because $1+\cos{\alpha}$ is non-negative, division by $1+\cos{\alpha}$ does not change the sign of $\sin{\alpha}$, which means $\tan \left(\alpha / 2\right)$ and $\sin{\alpha} / \left(1+\cos{\alpha}\right)$ have the same sign.

\solution %2
\begin{align*}
\tan{\dfrac{\alpha}{2}} &= \pm \sqrt{\dfrac{1-\cos{\alpha}}{1+\cos{\alpha}}} \\
&= \pm \sqrt{\dfrac{1-\cos{\alpha}}{1+\cos{\alpha}} \cdot \dfrac{1-\cos{\alpha}}{1-\cos{\alpha}}} \\
&= \pm \sqrt{\dfrac{\left(1-\cos{\alpha}\right)^2}{1-\cos^{2}{\alpha}}} \\
&= \pm \sqrt{\dfrac{\left(1-\cos{\alpha}\right)^2}{\sin^{2}{\alpha}}} \\
&= \pm \dfrac{1-\cos{\alpha}}{\sin{\alpha}}
\end{align*}
For acute angles $\alpha$, we need to take the positive branch of the square root so that the signs of both sides of the half-angle formula agree. By the Principle of Analytic Continuation, since the positive branch is correct for all acute angles and both sides of the formula are rational trigonometric expressions, it is correct for all angles in general, so we have
\[
\tan{\dfrac{\alpha}{2}} = \dfrac{1-\cos{\alpha}}{\sin{\alpha}}
\]

\end{solutions}

\begin{solutions}{Page 153}
\solution %1
\begin{align*}
\dfrac{1}{2}\cos\left(\alpha-\beta\right) - \dfrac{1}{2}\cos\left(\alpha+\beta\right) &= \dfrac{1}{2}\left(\cos{\alpha}\cos{\beta}+\sin{\alpha}\sin{\beta}-\cos{\alpha}\cos{\beta} + \sin{\alpha}\sin{\beta}\right) \\
&= \dfrac{1}{2}\left(2\sin{\alpha}\sin{\beta}\right) \\
&= \sin{\alpha}\sin{\beta}
\end{align*}

\begin{align*}
\dfrac{1}{2}\sin\left(\alpha+\beta\right) + \dfrac{1}{2}\sin\left(\alpha-\beta\right) &= \dfrac{1}{2}\left(\sin{\alpha}\cos{\beta}+\cos{\alpha}\sin{\beta}+\sin{\alpha}\cos{\beta} - \cos{\alpha}\sin{\beta}\right) \\
&= \dfrac{1}{2}\left(2\sin{\alpha}\cos{\beta}\right) \\
&= \sin{\alpha}\cos{\beta}
\end{align*}

\solution %2
\begin{align*}
\sin{75\degree}\sin{15\degree} &= \dfrac{1}{2}\cos\left(75\degree-15\degree\right) - \dfrac{1}{2}\cos\left(75\degree+15\degree\right) \\
&= \dfrac{1}{2}\cos{60\degree} - \dfrac{1}{2}\cos{90\degree} \\
&= \dfrac{1}{2} \cdot \dfrac{1}{2} - \dfrac{1}{2} \cdot 0 \\
&= \dfrac{1}{4}
\end{align*}

\solution %3
\begin{align*}
\sin{75\degree}\cos{15\degree} &= \dfrac{1}{2}\sin\left(75\degree+15\degree\right) + \dfrac{1}{2}\sin\left(75\degree-15\degree\right) \\
&= \dfrac{1}{2}\sin{90\degree} + \dfrac{1}{2}\sin{60\degree} \\
&= \dfrac{1}{2} \cdot 1 + \dfrac{1}{2} \cdot \dfrac{\sqrt{3}}{2} \\
&= \dfrac{2+\sqrt{3}}{4}
\end{align*}

\solution %4
\begin{subsolutions}
\subsolution %a
\begin{align*}
\cos{75\degree}\cos{15\degree} &= \dfrac{1}{2}\cos\left(75\degree+15\degree\right) + \dfrac{1}{2}\cos\left(75\degree-15\degree\right) \\
&= \dfrac{1}{2}\cos{90\degree} + \dfrac{1}{2}\cos{60\degree} \\
&= \dfrac{1}{2} \cdot 0 + \dfrac{1}{2} \cdot \dfrac{1}{2} \\
&= \dfrac{1}{4}
\end{align*}

Alternatively, by the cosine subtraction formula, $\cos{60\degree} = \cos{75\degree}\cos{15\degree} + \sin{75\degree}\sin{15\degree}$ so
\[
\cos{75\degree}\cos{15\degree} = \cos{60\degree} - \sin{75\degree}\sin{15\degree} = \dfrac{1}{2} - \dfrac{1}{4} = \dfrac{1}{4}
\]

Alternatively, using the idea of sine and cosine being cofunctions,
\[
\cos{75\degree}\cos{15\degree} = \sin\left(90\degree-75\degree\right)\sin\left(90\degree-15\degree\right) = \sin{15\degree}\sin{75\degree} = \dfrac{1}{4}
\]
For the above two alternative solutions, we apply the result from Exercise 2 above that $\sin{75\degree}\sin{15\degree} = 1/4$.

\subsolution %b
\begin{align*}
\cos{75\degree}\sin{15\degree} &= \dfrac{1}{2}\sin\left(15\degree+75\degree\right) + \dfrac{1}{2}\sin\left(15\degree-75\degree\right) \\
&= \dfrac{1}{2}\sin{90\degree} - \dfrac{1}{2}\sin{60\degree} \\
&= \dfrac{1}{2} \cdot 1 - \dfrac{1}{2} \cdot \dfrac{\sqrt{3}}{2} \\
&= \dfrac{2-\sqrt{3}}{4}
\end{align*}

Alternatively, by the sine addition formula, $\sin{90\degree} = \sin{75\degree}\cos{15\degree} + \cos{75\degree}\sin{15\degree}$ so
\[
\cos{75\degree}\sin{15\degree} = \sin{90\degree} - \sin{75\degree}\cos{15\degree} = 1 - \dfrac{2+\sqrt{3}}{4} = \dfrac{2-\sqrt{3}}{4}
\]
For the above alternative solution, we apply the result from Exercise 3 above that $\sin{75\degree}\cos{15\degree} = \left(2+\sqrt{3}\right)/4$.
\end{subsolutions}

\solution %5
\begin{align*}
2\cos\left(\dfrac{\pi}{4}+\alpha\right)\cos\left(\dfrac{\pi}{4}-\alpha\right) &= \cos\left(\dfrac{\pi}{4} + \alpha + \dfrac{\pi}{4} - \alpha\right) + \cos\left(\dfrac{\pi}{4} + \alpha - \dfrac{\pi}{4} + \alpha\right) \\
&= \cos{\dfrac{\pi}{2}} \cos{2\alpha} \\
&= \cos{2\alpha}
\end{align*}

Alternatively, applying the result of Exercise 8 in Section 3 of this chapter,
\begin{align*}
2\cos\left(\dfrac{\pi}{4}+\alpha\right)\cos\left(\dfrac{\pi}{4}-\alpha\right) &= 2\left(\cos^{2}{\dfrac{\pi}{4}}\cos^{2}{\alpha} - \sin^{2}{\dfrac{\pi}{4}}\sin^{2}{\alpha}\right) \\
&= 2\left(\dfrac{1}{2}\cos^{2}{\alpha} - \dfrac{1}{2}\sin^{2}{\alpha}\right) \\
&= \cos^{2}{\alpha} - \sin^{2}{\alpha} \\
&= \cos{2\alpha}
\end{align*}

\solution %6
\begin{align*}
&\sin\left(\alpha+\beta\right)\sin\left(\alpha-\beta\right) + \sin\left(\beta+\gamma\right)\sin\left(\beta-\gamma\right) + \sin\left(\gamma+\alpha\right)\sin\left(\gamma-\alpha\right) \\
&= \dfrac{1}{2}\cos{2\beta} - \dfrac{1}{2}\cos{2\alpha} + \dfrac{1}{2}\cos{2\gamma} - \dfrac{1}{2}\cos{2\beta} + \dfrac{1}{2}\cos{2\alpha} - \dfrac{1}{2}\cos{2\gamma} \\
&= 0
\end{align*}

\solution %7
\begin{align*}
&\sin{\alpha}\sin\left(\beta-\gamma\right) + \sin{\beta}\sin\left(\gamma-\alpha\right) + \sin{\gamma}\sin\left(\alpha-\beta\right) \\
&= \dfrac{1}{2}\cos\left(\alpha-\beta+\gamma\right) - \dfrac{1}{2}\cos\left(\alpha+\beta-\gamma\right) + \dfrac{1}{2}\cos\left(\beta-\gamma+\alpha\right) - \dfrac{1}{2}\cos\left(\beta+\gamma-\alpha\right) \\
&+ \dfrac{1}{2}\cos\left(\gamma-\alpha+\beta\right) - \dfrac{1}{2}\cos\left(\gamma+\alpha-\beta\right) \\
&= 0
\end{align*}

\end{solutions}

\begin{solutions}{Page 155}
\solution %1
Recall from the previous section that $\cos\left(\gamma+\delta\right) + \cos\left(\gamma-\delta\right) = 2\cos{\gamma}\cos{\delta}$. Following Example 54, we let $\gamma = \left(\alpha+\beta\right) / 2$ and $\delta = \left(\alpha - \beta\right) / 2$. Substituting for $\gamma$ and $\delta$, we arrive at the first formula,
\[
\cos{\alpha} + \cos{\beta} = 2\cos{\dfrac{\alpha+\beta}{2}}\cos{\dfrac{\alpha-\beta}{2}}
\]

Similarly, we recall that $\cos\left(\gamma-\delta\right) - \cos\left(\gamma+\delta\right) = 2\sin{\gamma}\sin{\delta}$. Performing the same substitution as above, we obtain
\[
\cos{\beta}-\cos{\alpha} = 2\sin{\dfrac{\alpha+\beta}{2}}\sin{\dfrac{\alpha-\beta}{2}} \iff \cos{\alpha}-\cos{\beta} = -2\sin{\dfrac{\alpha+\beta}{2}}\sin{\dfrac{\alpha-\beta}{2}}
\]
\solution %2
\begin{align*}
\cos{70\degree}+\sin{40\degree} &= \sin{20\degree} + \sin{40\degree} \\
&= 2\sin{\dfrac{20\degree + 40\degree}{2}}\cos{\dfrac{20\degree-40\degree}{2}} \\
&= 2 \sin{30\degree}\cos\left(-10\degree\right) \\
&= 2 \cdot \dfrac{1}{2} \cos{10\degree} \\
&= \cos{10\degree}
\end{align*}

\solution %3
\begin{align*}
\cos{55\degree} + \cos{65\degree} &= 2\cos{\dfrac{55\degree+65\degree}{2}}\cos{\dfrac{55\degree-65\degree}{2}} \\
&= 2\cos{60\degree}\cos\left(-5\degree\right) \\
&= 2\cdot \dfrac{1}{2}\cos{5\degree} \\
&= \cos{5\degree}
\end{align*}

\solution %4
\begin{align*}
\cos{20\degree}+\cos{100\degree}+\cos{140\degree} &= 2\cos{\dfrac{100\degree+20\degree}{2}}\cos{\dfrac{100\degree-20\degree}{2}} + \cos{140\degree}\\
&= 2\cos{60\degree}\cos{40\degree} + \cos{140\degree} \\
&= 2\cdot \dfrac{1}{2} \cos{40\degree} + \cos{140\degree} \\
&= \cos{40\degree} + \cos{140\degree} \\
&= 2 \cos{\dfrac{140\degree+40\degree}{2}}\cos{\dfrac{140\degree-40\degree}{2}} \\
&= 2\cos{90\degree}\cos{50\degree} \\
&= 0
\end{align*}

\solution %5
\begin{align*}
\sin{78\degree} + \cos{132\degree} &= \sin{78\degree} - \cos{48\degree} \\
&= \sin{78\degree} - \sin{42\degree} \\
&= 2\cos{\dfrac{78\degree+42\degree}{2}}\sin{\dfrac{78\degree-42\degree}{2}} \\
&= 2\cos{60\degree}\sin{18\degree} \\
&= 2\cdot \dfrac{1}{2}\sin{18\degree} \\
&= \sin{18\degree}
\end{align*}

\solution %6
\begin{align*}
\dfrac{\cos{15\degree}+\sin{15\degree}}{\cos{15\degree}-\sin{15\degree}} &= \dfrac{\sin{75\degree}+\sin{15\degree}}{\sin{75\degree}-\sin{15\degree}} \\
&= \dfrac{2\sin{45\degree}\cos{30\degree}}{2\cos{45\degree}\sin{30\degree}} \\
&= \tan{45\degree}\cot{30\degree} \\
&= \sqrt{3}
\end{align*}

\solution %7
\begin{subsolutions}
\subsolution %a
\[
\sin\left(\alpha+\beta\right) = \sin\left(\pi-\alpha-\beta\right) = \sin{\gamma}
\]
\subsolution %b
\[
\cos\left(\alpha+\beta\right) = -\cos\left(\pi-\alpha-\beta\right) = -\cos{\gamma}
\]
\subsolution %c
\begin{align*}
\sin{2\alpha} + \sin{2\beta} + \sin{2\gamma} &= 2\sin\left(\alpha+\beta\right)\cos\left(\alpha-\beta\right) + \sin{2\gamma} \\
&= 2\sin{\gamma}\cos\left(\alpha-\beta\right) + 2\sin{\gamma}\cos{\gamma} \\
&= 2\sin{\gamma}\left(\cos\left(\alpha-\beta\right) + \cos{\gamma}\right) \\
&= 2\sin{\gamma}\left(\cos\left(\alpha-\beta\right) - \cos\left(\alpha+\beta\right)\right) \\
&= 2\sin{\gamma}\left(2\sin{\alpha}\sin{\beta}\right) \\
&= 4\sin{\alpha}\sin{\beta}\sin{\gamma}
\end{align*}
\end{subsolutions}

\solution %8
\begin{align*}
\sin{\alpha} + \sin\left(\alpha+\dfrac{2\pi}{3}\right) + \sin\left(\alpha+\dfrac{4\pi}{3}\right) &= 2\sin\left(\alpha+\dfrac{\pi}{3}\right)\cos\left(-\dfrac{\pi}{3}\right) + \sin\left(\alpha+\dfrac{4\pi}{3}\right) \\
&= \sin\left(\alpha+\dfrac{\pi}{3}\right) + \sin\left(\alpha+\dfrac{4\pi}{3}\right) \\
&= 2\sin\left(\alpha+\dfrac{5\pi}{6}\right)\cos\left(-\dfrac{\pi}{2}\right) \\
&= 0
\end{align*}

\solution %9
We first note that
\[
\sin{k\alpha} + \sin{(k+2)\alpha} = 2\sin{\dfrac{(k+2)\alpha+k\alpha}{2}}\cos{\dfrac{(k+2)\alpha-k\alpha}{2}} = 2\sin{(k+1)\alpha}\cos{\alpha}.
\]

Therefore,
\begin{align*}
\sin{\alpha}+2\sin{3\alpha}+\sin{5\alpha} &= \sin{\alpha}+\sin{3\alpha}+\sin{3\alpha}+\sin{5\alpha} \\
&= 2\sin{2\alpha}\cos{\alpha} + 2\sin{4\alpha}\cos{\alpha} \\
&= 2\cos{\alpha} \left(\sin{2\alpha} + \sin{4\alpha}\right) \\
&= 2\cos{\alpha} \left(2\sin{3\alpha}\cos{\alpha}\right) \\
&= 4\cos^{2}{\alpha}\sin{3\alpha}.
\end{align*}

\solution %10
\begin{align*}
&\dfrac{\sin\left(\beta-\gamma\right)}{\sin{\beta}\sin{\gamma}} + \dfrac{\sin\left(\gamma-\alpha\right)}{\sin{\gamma}\sin{\alpha}} + \dfrac{\sin\left(\alpha-\beta\right)}{\sin{\alpha}\sin{\beta}} \\
&\qquad= \dfrac{\sin{\beta}\cos{\gamma}}{\sin{\beta}\sin{\gamma}} - \dfrac{\cos{\beta}\sin{\gamma}}{\sin{\beta}\sin{\gamma}} + \dfrac{\sin{\gamma}\cos{\alpha}}{\sin{\gamma}\sin{\alpha}} - \dfrac{\cos{\gamma}\sin{\alpha}}{\sin{\gamma}\sin{\alpha}} +\dfrac{\sin{\alpha}\cos{\beta}}{\sin{\alpha}\sin{\beta}} - \dfrac{\cos{\alpha}\cos{\beta}}{\sin{\alpha}\sin{\beta}} \\
&\qquad= \cot{\gamma} - \cot{\beta} + \cot{\alpha} - \cot{\gamma} + \cot{\beta} - \cot{\alpha} \\
&\qquad= 0
\end{align*}

This result also follows from the identity which was proven in Exercise 7 of the previous section. This can be seen by rewriting the left-hand side of the equation as a single fraction with a common denominator and noting that the numerator of this fraction is zero.

\solution %11
\begin{align*}
\sin\left(\alpha-\beta\right) + \sin\left(\alpha-\gamma\right) + \sin\left(\beta-\gamma\right) &= 2\sin{\dfrac{2\alpha-\beta-\gamma}{2}}\cos{\dfrac{\gamma-\beta}{2}} + \sin\left(\beta-\gamma\right) \\
&= 2\sin{\dfrac{2\alpha-\beta-\gamma}{2}}\cos{\dfrac{\beta-\gamma}{2}} + 2\sin{\dfrac{\beta-\gamma}{2}}\cos{\dfrac{\beta-\gamma}{2}} \\
&= 2\cos{\dfrac{\beta-\gamma}{2}} \left(\sin{\dfrac{2\alpha-\beta-\gamma}{2}} + \sin{\dfrac{\beta-\gamma}{2}}\right) \\
&= 2\cos{\dfrac{\beta-\gamma}{2}} \left(2\sin{\dfrac{\alpha-\gamma}{2}}\cos{\dfrac{\alpha-\beta}{2}}\right) \\
&= 4 \cos{\dfrac{\alpha-\beta}{2}} \sin{\dfrac{\alpha-\gamma}{2}} \cos{\dfrac{\beta-\gamma}{2}}
\end{align*}

\solution %12
\begin{align*}
&\sin\left(\alpha+\beta+\gamma\right) + \sin\left(\alpha-\beta-\gamma\right) + \sin\left(\alpha+\beta-\gamma\right) +
\sin\left(\alpha-\beta+\gamma\right) \\
&\qquad= 2\sin{\alpha}\cos\left(\beta+\gamma\right) + \sin\left(\alpha+\beta-\gamma\right) +
\sin\left(\alpha-\beta+\gamma\right) \\
&\qquad= 2\sin{\alpha}\cos\left(\beta+\gamma\right) + 2\sin{\alpha}\cos\left(\beta-\gamma\right) \\
&\qquad= 2\sin{\alpha}\left(\cos\left(\beta+\gamma\right) + \cos\left(\beta-\gamma\right)\right) \\
&\qquad= 2\sin{\alpha}\left(2\cos{\beta}\cos{\gamma}\right) \\
&\qquad= 4\sin{\alpha}\cos{\beta}\cos{\gamma}
\end{align*}
\end{solutions}

\begin{solutions}{Page 158}
\solution %1
Let $\beta=2\gamma$.
\begin{align*}
\sin^{2}{\beta} + \cos^{2}{\beta} &= \sin^{2}{2\gamma} + \cos^{2}{2\gamma} \\
&= \left(\dfrac{2a}{1+a^{2}}\right)^2 + \left(\dfrac{1-a^{2}}{1+a^{2}}\right)^2 \\
&= \dfrac{4a^{2}}{1+2a^{2}+a^{4}} + \dfrac{1-2a^{2}+a^{4}}{1+2a^{2}+a^{4}} \\
&= \dfrac{1+2a^{2}+a^{4}}{1+2a^{2}+a^{4}} \\
&= 1
\end{align*}

\solution %2
\begin{align*}
\tan{2\beta} &= \dfrac{2a}{1-a^2} \\
&= \dfrac{\dfrac{2a}{1+a^{2}}}{\dfrac{1-a^{2}}{1+a^{2}}} \\
&= \dfrac{\sin{2\beta}}{\cos{2\beta}}
\end{align*}
\end{solutions}

\begin{solutions}{Page 160}
\solution %1
\[
\sin{\alpha} = \dfrac{2 \cdot 2 \cdot 3}{2^{2}+3^{2}} = \dfrac{12}{13}
\]
\[
\cos{\alpha} = \dfrac{2^{2} - 3^{2}}{2^{2}+3^{2}} = \dfrac{-5}{13}
\]
These values give the Pythagorean triple 5, 12, 13, provided we take the absolute value of $\cos{\alpha}$. However, because $\cos{\alpha}$ is negative, $\alpha$ is not an acute angle, so it cannot correspond to an angle in a right triangle.

\solution %2
\[
\sin{\alpha} = \dfrac{2 \cdot 8 \cdot 5}{8^{2}+5^{2}} = \dfrac{80}{89}
\]
\[
\cos{\alpha} = \dfrac{8^{2} - 5^{2}}{8^{2}+5^{2}} = \dfrac{39}{89}
\]
This corresponds to the right triangle with legs of 39 and 80 and a hypotenuse of 89.

\solution %3
\[
\left(2pq\right)^{2} + \left(q^{2}-p^{2}\right)^{2} = 4p^{2}q^{2} + q^4 - 2q^{2}p^{2} + p^{4} = q^{4} + 2q^{2}p^{2} + p^{4} = \left(q^{2}+p^{2}\right)^{2}
\]
The above shows that $q^{2}+p^{2}$ is the hypotenuse.

\end{solutions}

\begin{solutions}{Page 161 (First)}
\solution %1
\begin{itemize}
    \item $\sin{20\degree}\cos{20\degree} \approx 0.3214$
    \item $\sin{10\degree}\cos{10\degree} \approx 0.1710$
    \item $\sin{5\degree}\cos{5\degree} \approx 0.0868$
    \item $\sin{1\degree}\cos{1\degree} \approx 0.0174$
    \item $\sin{70\degree}\cos{70\degree} \approx 0.3214$
    \item $\sin{80\degree}\cos{80\degree} \approx 0.1710$
    \item $\sin{85\degree}\cos{85\degree} \approx 0.0868$
    \item $\sin{89\degree}\cos{89\degree} \approx 0.0174$
\end{itemize}

\solution %2
\[
\sin{30\degree}\cos{30\degree} = \dfrac{1}{2} \cdot \dfrac{\sqrt{3}}{2} = \dfrac{\sqrt{3}}{4}
\]
\[
\sin{45\degree}\cos{45\degree} = \dfrac{\sqrt{2}}{2} \cdot \dfrac{\sqrt{2}}{2} = \dfrac{1}{2}
\]
\[
\sin{60\degree}\cos{60\degree} = \dfrac{\sqrt{3}}{2} \cdot \dfrac{1}{2}  = \dfrac{\sqrt{3}}{4}
\]
\end{solutions}

\begin{solutions}{Page 161 (Second)}
\solution %1
In the below answers for this question, $n$ represents an arbitrary integer.
\begin{subsolutions}
\subsolution %a
\begin{align*}
\sin{x}\cos{x} = \dfrac{1}{2} &\implies \dfrac{1}{2}\sin{2x} = \dfrac{1}{2} \\
&\implies \sin{2x} = 1 \\
&\implies 2x = \dfrac{\pi}{2} + 2n\pi \\
&\implies x = \dfrac{\pi}{4} + n\pi
\end{align*}

\subsolution %b
\[
\sin{x}\cos{x} = \dfrac{\sqrt{3}}{2} \implies \dfrac{1}{2}\sin{2x} = \dfrac{\sqrt{3}}{2} \implies \sin{2x} = \sqrt{3}
\]
Because $\sqrt{3} > 1$, there are no values of $x$ which satisfy the given equation.

\subsolution %c
\begin{align*}
\sin{x}\cos{x} = \dfrac{\sqrt{3}}{4} &\implies \dfrac{1}{2}\sin{2x} = \dfrac{\sqrt{3}}{4} \\
&\implies \sin{2x} = \dfrac{\sqrt{3}}{2} \\
&\implies 2x = \left(\dfrac{\pi}{2} \pm \dfrac{\pi}{6}\right) + 2n\pi \\
& \implies x = \left(\dfrac{\pi}{4} \pm \dfrac{\pi}{12}\right) + n\pi
\end{align*}

\end{subsolutions}

\solution %2
(c) has no solution because $\sin{x}\cos{x} \leq 0.5 < 0.6$.

\solution %3
$\sin{x}\cos{x}=N$ has a solution when $\left\lvert N \right\rvert \leq 1/2$.
\[
\sin{x}\cos{x} = N \implies \dfrac{1}{2}\sin{2x} = N
\implies \sin{2x} = 2N
\]

$2x = \arcsin{2N}$ gives one solution to the above equation. Another (not necessarily distinct) solution is $2x = \pi - \arcsin{2N}$. All solutions to the equation can be generated by adding an integer multiple of $2\pi$ to either of the above angles. Therefore, the general solution to $\sin{x}\cos{x}=N$ is
\[
x = \dfrac{1}{2} \arcsin{2N} + n\pi \text{ or } x = \dfrac{1}{2}\left(\pi-\arcsin{2N}\right) + n\pi.
\]
Using the properties of cofunctions, this solution can be written in a slightly more compact manner as follows:
\[
x = \left(\dfrac{\pi}{4} \pm \dfrac{1}{2}\arccos{2N}\right) + n\pi
\]
\end{solutions}

\begin{solutions}{Page 162 (First)}
\solution %1
\[
\sin{30\degree}+\cos{30\degree} = \dfrac{1}{2} + \dfrac{\sqrt{3}}{2} > \dfrac{1}{2} + \dfrac{1}{2} = 1
\]

\solution %2
\[
\sin{0}+\cos{0} = 1
\]

\solution %3
\[
\sin{\dfrac{\pi}{4}} + \cos{\frac{\pi}{4}} = \dfrac{\sqrt{2}}{2} + \dfrac{\sqrt{2}}{2} = \sqrt{2}
\]
\end{solutions}

\begin{solutions}{Page 162 (Second)}
\solution %1
Yes, because $1.414 < \sqrt{2}$. (Note: $\sin{x}+\cos{x}$ must attain the value $1.414$ for some $x$ because of the Intermediate Value Theorem).

\solution %2
No, because $1.415 > \sqrt{2}$.

\solution %3
We square both sides of the equation to solve for $x$.
\[
\sin{x}+\cos{x}=\sqrt{2} \implies 1 + \sin{2x} = 2 \implies \sin{2x} = 1 \implies 2x = \dfrac{\pi}{2} + 2n\pi \implies x = \dfrac{\pi}{4} + n\pi
\]

\solution %4
Since we know the maximum value of $\left(\sin{x}+\cos{x}\right)^2$ is 2, the minimum value of $\sin{x}+\cos{x}$ cannot be less than $-\sqrt{2}$. The value of $-\sqrt{2}$ is attained when $x=\tfrac{5\pi}{4}$.
\[
\sin\left(\dfrac{5\pi}{4}\right) + \cos\left(\dfrac{5\pi}{4}\right) = -\dfrac{\sqrt{2}}{2} - \dfrac{\sqrt{2}}{2} = -\sqrt{2}
\]
\end{solutions}

\begin{solutions}{Page 163 (First)}
\solution %1
$\sin{x}+\cos{x}$ is maximized when $x + \tfrac{\pi}{4} = \tfrac{\pi}{2} + 2n\pi$, where $n$ is an integer. Therefore,
\[
x = \dfrac{\pi}{4} + 2n\pi
\]
\solution %2
The minimum value of $\sin{x}+\cos{x}$ is $-\sqrt{2}$. This is achieved when
\[
x = -\dfrac{3\pi}{4} + 2n\pi.
\]
\end{solutions}

\begin{solutions}{Page 163 (Second)}
\solution %1
Yes. Because $\sin{\alpha}$ and $\cos{\alpha}$ are both positive, $\alpha$ must be in the first quadrant.

\solution %2
The minimum value of $3\sin{x}+4\cos{x}$ is $-5$. This occurs when
\[
x = -\alpha - \dfrac{\pi}{2} + 2n\pi.
\]

\solution %3
Let $\alpha$ be the positive acute angle such that $\sin{\alpha} = 7/\sqrt{53}$.
\[
2\sin{x}+7\cos{x} = \sqrt{53} \left(\dfrac{2}{\sqrt{53}}\sin{x} + \dfrac{7}{\sqrt{53}}\cos{x}\right) = \sqrt{53}\sin\left(\alpha+x\right)
\]
The above shows that the maximum and minimum values of $2\sin{x}+7\cos{x}$ are $\sqrt{53}$ and $-\sqrt{53}$, respectively.

\end{solutions}

\begin{solutions}{Page 164}
\solution %1
\begin{align*}
\cos{\dfrac{\pi}{16}} &= \sqrt{\dfrac{1+\cos \dfrac{\pi}{8}}{2}} \\
&= \sqrt{\dfrac{1+\dfrac{1}{2}\sqrt{2+\sqrt{2}}}{2}} \\
&= \sqrt{\dfrac{2+\sqrt{2+\sqrt{2}}}{4}} \\
&= \dfrac{1}{2} \sqrt{2+\sqrt{2+\sqrt{2}}}
\end{align*}

\begin{align*}
\sin{\dfrac{\pi}{16}} &= \sqrt{\dfrac{1-\cos \dfrac{\pi}{8}}{2}} \\
&= \sqrt{\dfrac{1-\dfrac{1}{2}\sqrt{2+\sqrt{2}}}{2}} \\
&= \sqrt{\dfrac{2-\sqrt{2+\sqrt{2}}}{4}} \\
&= \dfrac{1}{2} \sqrt{2-\sqrt{2+\sqrt{2}}}
\end{align*}

\solution ~ %2
\begin{center}
\bgroup
\def\arraystretch{2.1}
\setlength\tabcolsep{15pt}
\begin{tabular}{ |c|c|c| }
\hline
$\alpha$
& $\cos{\alpha}$
& $\sin{\alpha}$\\
\hline
$\dfrac{\pi}{16}$
& $\dfrac{1}{2} \sqrt{2+\sqrt{2+\sqrt{2}}}$
& $\dfrac{1}{2} \sqrt{2-\sqrt{2+\sqrt{2}}}$\\
\hline
$\dfrac{\pi}{32}$
& $\dfrac{1}{2} \sqrt{2+\sqrt{2+\sqrt{2+\sqrt{2}}}}$
& $\dfrac{1}{2} \sqrt{2-\sqrt{2+\sqrt{2+\sqrt{2}}}}$\\
\hline
$\dfrac{\pi}{64}$
& $\dfrac{1}{2} \sqrt{2+\sqrt{2+\sqrt{2+\sqrt{2+\sqrt{2}}}}}$
& $\dfrac{1}{2} \sqrt{2-\sqrt{2+\sqrt{2+\sqrt{2+\sqrt{2}}}}}$\\
\hline
$\dfrac{\pi}{128}$
& $\dfrac{1}{2} \sqrt{2+\sqrt{2+\sqrt{2+\sqrt{2+\sqrt{2+\sqrt{2}}}}}}$
& $\dfrac{1}{2} \sqrt{2-\sqrt{2+\sqrt{2+\sqrt{2+\sqrt{2+\sqrt{2}}}}}}$\\
\hline
\end{tabular}
\egroup
\end{center}
\end{solutions}

\begin{solutions}{Page 166}
\solution %1
\[
\dfrac{1}{2}\sqrt{2} \approx 0.7071
\]
\[
\dfrac{1}{2}\sqrt{2+\sqrt{2}} \approx 0.9239
\]
\[
\dfrac{1}{2}\sqrt{2+\sqrt{2+\sqrt{2}}} \approx 0.9809
\]
\[
\dfrac{1}{2}\sqrt{2+\sqrt{2+\sqrt{2+\sqrt{2}}}} \approx 0.9952
\]

\solution %2
\[
2^2 \sqrt{2-\sqrt{2}} \approx 3.0615
\]
\[
2^3 \sqrt{2-\sqrt{2+\sqrt{2}}} \approx 3.1214
\]
\[
2^4 \sqrt{2-\sqrt{2+\sqrt{2+\sqrt{2}}}} \approx 3.1365
\]
\[
2^5 \sqrt{2-\sqrt{2+\sqrt{2+\sqrt{2+\sqrt{2}}}}} \approx 3.1403
\]

\solution %3
\begin{subsolutions}
\subsolution %a
\begin{align*}
\cos{\dfrac{\pi}{12}} &= \sqrt{\dfrac{1+\cos\left(\pi/6\right)}{2}} \\
&= \sqrt{\dfrac{1+\sqrt{3}/2}{2}} \\
&= \dfrac{1}{2}\sqrt{2+\sqrt{3}} \\
&\approx 0.9659
\end{align*}

\subsolution %b
\begin{align*}
\cos{\dfrac{\pi}{24}} &= \sqrt{\dfrac{1+\cos\left(\pi/12\right)}{2}} \\
&= \sqrt{\dfrac{1+\sqrt{2+\sqrt{3}}/2}{2}} \\
&= \dfrac{1}{2}\sqrt{2+\sqrt{2+\sqrt{3}}} \\
&\approx 0.9914
\end{align*}

\subsolution %c
\begin{align*}
\cos{\dfrac{\pi}{48}} &= \sqrt{\dfrac{1+\cos\left(\pi/24\right)}{2}} \\
&= \sqrt{\dfrac{1+\sqrt{2+\sqrt{2+\sqrt{3}}}/2}{2}} \\
&= \dfrac{1}{2}\sqrt{2+\sqrt{2+\sqrt{2+\sqrt{3}}}} \\
&\approx 0.9979
\end{align*}

\subsolution %d
\begin{align*}
\cos{\dfrac{\pi}{96}} &= \sqrt{\dfrac{1+\cos\left(\pi/48\right)}{2}} \\
&= \sqrt{\dfrac{1+\sqrt{2+\sqrt{2+\sqrt{2+\sqrt{3}}}}/2}{2}} \\
&= \dfrac{1}{2}\sqrt{2+\sqrt{2+\sqrt{2+\sqrt{2+\sqrt{3}}}}} \\
&\approx 0.9995
\end{align*}
These values approach 1 because for small angles $x$, $\cos{x}$ gets closer to 1. More formally, the limit of $\cos{x}$ as $x$ approaches zero is 1.
\end{subsolutions}

\solution %4
\begin{subsolutions}
\subsolution %a
\[
12\sin{\dfrac{\pi}{12}} = 6\sqrt{2-\sqrt{3}}\approx 3.1058
\]
\subsolution %b
\[
24\sin{\dfrac{\pi}{24}} = 12\sqrt{2-\sqrt{2+\sqrt{3}}}  \approx 3.1326
\]
\subsolution %c
\[
48\sin{\dfrac{\pi}{48}} = 24\sqrt{2-\sqrt{2+\sqrt{2+\sqrt{3}}}} \approx 3.1394
\]
\subsolution %d
\[
96\sin{\dfrac{\pi}{96}} = 48\sqrt{2-\sqrt{2+\sqrt{2+\sqrt{2+\sqrt{3}}}}} \approx 3.1410
\]
\end{subsolutions}

\end{solutions}

\begin{solutions}{Page 168}
\solution %1
\begin{align*}
\dfrac{1}{\sqrt{1}+\sqrt{2}} + \dfrac{1}{\sqrt{2}+\sqrt{3}} + \ldots + \dfrac{1}{\sqrt{99}+\sqrt{100}} &= \sum_{k=1}^{99} {\dfrac{1}{\sqrt{k} + \sqrt{k+1}}} \\
&= \sum_{k=1}^{99} {\dfrac{1}{\sqrt{k} + \sqrt{k+1}} \cdot \dfrac{\sqrt{k+1} - \sqrt{k}}{\sqrt{k+1} - \sqrt{k}}} \\
&= \sum_{k=1}^{99} {\sqrt{k+1}-\sqrt{k}} \\
&= \left(\sqrt{2}-\sqrt{1}\right) + \left(\sqrt{3}-\sqrt{2}\right) + \ldots + \left(\sqrt{100}-\sqrt{99}\right) \\
&= \sqrt{100} - \sqrt{1} \\
&= 10 - 1 \\
&= 9
\end{align*}

\solution %2
\begin{align*}
1 + 3 + \ldots + \left(2n+1\right) &= \sum_{k=0}^{n} {2k+1} \\
&= \sum_{k=0}^{n} {\left(k+1\right)^{2}-k^{2}} \\
&= \left(1^{2} - 0^{2}\right) + \left(2^{2} - 1^{2}\right) + \ldots + \left( (n+1)^{2} - n^{2} \right) \\
&= \left(n+1\right)^{2} - 0^{2} \\
&= \left(n+1\right)^{2} \\
&= n^{2} + 2n + 1
\end{align*}

\solution %3
\begin{align*}
\left(1-x\right)P &= \left(1-x\right)\left(1+x\right)\left(1+x^{2}\right)\left(1+x^{4}\right)\left(1+x^{8}\right)\left(1+x^{16}\right) \\
&= \left(1-x^{2}\right)\left(1+x^{2}\right)\left(1+x^{4}\right)\left(1+x^{8}\right)\left(1+x^{16}\right) \\
&= \left(1-x^{4}\right)\left(1+x^{4}\right)\left(1+x^{8}\right)\left(1+x^{16}\right) \\
&= \left(1-x^{8}\right)\left(1+x^{8}\right)\left(1+x^{16}\right) \\
&= \left(1-x^{16}\right)\left(1+x^{16}\right) \\
&= 1-x^{32}
\end{align*}
Thus,
\[
P = \dfrac{1-x^{32}}{1-x} = 1 + x + \ldots + x^{31}
\]

\solution %4
\begin{align*}
\sin{20\degree} P &= \sin{20\degree}\cos{20\degree}\cos{40\degree}\cos{80\degree} \\
&= \dfrac{1}{2}\sin{40\degree}\cos{40\degree}\cos{80\degree} \\
&= \dfrac{1}{4}\sin{80\degree}\cos{80\degree} \\
&= \dfrac{1}{8}\sin{160\degree} \\
&= \dfrac{1}{8}\sin{20\degree}
\end{align*}
Thus, $P = \tfrac{1}{8}$.
\end{solutions}

\begin{solutions}{Page 170}
\solution %1
\begin{align*}
2\sin{x}\left(\sin{x} + \sin{3x} + \ldots + \sin{99x}\right) &= 2\sin{x}\sum_{k=0}^{49} {\sin{\left(2k+1\right)x}} \\
&= \sum_{k=0}^{49} {2\sin{x}\sin{\left(2k+1\right)x}} \\
&= \sum_{k=0}^{49} {\cos{2kx} - \cos{\left(2k+2\right)}x} \\
&= \left(\cos{0} - \cos{2x}\right) + \left(\cos{2x} - \cos{4x}\right) + \ldots + \left(\cos{98x} - \cos{100x}\right)\\
&= 1-\cos{100x}
\end{align*}

Thus, the original sum, $\sin{x}+\ldots+\sin{99x}$, is equal to $\left(1-\cos{100x}\right) / \left(2\sin{x}\right)$. Notice that this is equivalent to $\sin^{2}{50x}/\sin{x}$, which is what would be found using the formula at the bottom of page 169.

\solution %2
\begin{align*}
2\sin{\dfrac{\pi}{8}}\left[\sin{x} + \ldots + \sin\left(x+\dfrac{99\pi}{4}\right)\right] &= 2\sin{\dfrac{\pi}{8}} \sum_{k=0}^{99} \sin\left(x+k\dfrac{\pi}{4}\right) \\
&= \sum_{k=0}^{99} 2\sin{\dfrac{\pi}{8}}\sin\left(x+k\dfrac{\pi}{4}\right) \\
&= \sum_{k=0}^{99} \cos\left(x+ \left(k-\dfrac{1}{2}\right)\dfrac{\pi}{4}\right) - \cos\left(x+ \left(k+\dfrac{1}{2}\right)\dfrac{\pi}{4}\right) \\
&= \cos\left(x-\dfrac{\pi}{8}\right) - \cos\left(x + \dfrac{199\pi}{8}\right) \\
&= \cos\left(x-\dfrac{\pi}{8}\right) - \cos\left(x+\dfrac{7\pi}{8}\right) \\
&= \cos\left(x-\dfrac{\pi}{8}\right) + \cos\left(\pi - x - \dfrac{7\pi}{8}\right) \\
&= 2\cos\left(x-\dfrac{\pi}{8}\right) \\
&= 2\cos{x}\cos{\dfrac{\pi}{8}} + 2\sin{x}\sin{\dfrac{\pi}{8}}
\end{align*}
Dividing by $2\sin{\tfrac{\pi}{8}}$, we find that the original sum is equal to $\cot{\tfrac{\pi}{8}}\cos{x} + \sin{x}$. Notice that this is equivalent to $\sin\left(x+99\pi/8\right) / \sin\left(\pi/8\right)$, which is what would be found using the formula at the bottom of page 169.

An alternative way to evaluate this sum is by noticing that the terms of this sum repeat in periods of 8 because $\sin\left(x+\tfrac{8\pi}{4}\right)=\sin\left(x\right)$. Furthermore, the first 8 terms of the sum total to zero (try verifying this yourself using the sine addition formulas). Therefore, to evaluate the whole sum, all we need to evaluate is the sum of the last four terms:
\[
\sin\left(x+\dfrac{96\pi}{4}\right) + \sin\left(x+\dfrac{97\pi}{4}\right) + \sin\left(x+\dfrac{98\pi}{4}\right) + \sin\left(x+\dfrac{99\pi}{4}\right).
\]
By the periodicity in the sum, we can equivalently evaluate the first four terms of the sum. Using the sine addition formulas, we find that the sum is equal to
\begin{align*}
&\sin{x}+\sin\left(x+\dfrac{\pi}{4}\right)+\sin\left(x+\dfrac{\pi}{2}\right)+\sin\left(x+\dfrac{3\pi}{4}\right) \\
&\qquad= \sin{x} + \dfrac{\sqrt{2}}{2}\sin{x} + \dfrac{\sqrt{2}}{2}\cos{x} + \cos{x} - \dfrac{\sqrt{2}}{2}\sin{x} + \dfrac{\sqrt{2}}{2}\cos{x} \\
&\qquad= \sin{x} + \cos{x} + \sqrt{2}\cos{x}.
\end{align*}
Combining this result with our initial evaluation of this sum shows that $\cot{\tfrac{\pi}{8}} = 1 + \sqrt{2}$.

\solution %3
We begin by multiplying the sum by $2\sin{x}$.
\begin{align*}
2\sin{x}\left(\cos{2x}+\cos{4x}+\ldots+\cos{2nx}\right) &= \sum_{k=1}^{n} 2\sin{x}\cos{2kx} \\
&= \sum_{k=1}^{n} \sin{(2k+1)x} - \sin{(2k-1)x} \\
&= \left(\sin{3x}-\sin{x}\right) + \ldots + \left(\sin{(2n+1)x} - \sin{(2n+1)x}\right) \\
&= \sin{(2n+1)x} - \sin{x}
\end{align*}
Dividing by $2\sin{x}$ gives us the value of the sum as
\[
\dfrac{\sin(2n+1)x}{2\sin{x}} - \dfrac{1}{2}.
\]

\solution %4
We begin by multiplying the sum by $2\sin{\tfrac{\pi}{2k}}$.
\begin{align*}
2\sin{\dfrac{\pi}{2k}} \left(\cos{\dfrac{\pi}{k}} + \cos{\dfrac{2\pi}{k}} + \ldots + \cos{\dfrac{n\pi}{k}} \right) &= \sum_{j=1}^{n} 2\sin{\dfrac{\pi}{2k}}\cos{\dfrac{j\pi}{k}} \\
&= \sum_{j=1}^{n} \sin\left(\dfrac{\pi}{k}\left(j+\dfrac{1}{2}\right)\right) - \sin\left(\dfrac{\pi}{k}\left(j-\dfrac{1}{2}\right)\right) \\
&= \sin\left(\dfrac{n\pi}{k} + \dfrac{\pi}{2k}\right) - \sin{\dfrac{\pi}{2k}} \\
&= \sin{\dfrac{n\pi}{k}}\cos{\dfrac{\pi}{2k}} + \cos{\dfrac{n\pi}{k}}\sin{\dfrac{\pi}{2k}} - \sin{\dfrac{\pi}{2k}}
\end{align*}
Dividing by $2\sin{\tfrac{\pi}{2k}}$, we find that the value of the sum is
\[
\dfrac{1}{2}\left(\sin{\dfrac{n\pi}{k}}\cot{\dfrac{\pi}{2k}} + \cos{\dfrac{n\pi}{k}} - 1\right).
\]
\solution %5
The heights of the perpendiculars are given by $\sin{\tfrac{k\pi}{12}}$, where $k$ ranges from $1$ to $11$. We consider this sum multiplied by $2\sin{\tfrac{\pi}{24}}$.
\begin{align*}
\sum_{k=1}^{11} 2\sin{\dfrac{\pi}{24}}\sin{\dfrac{k\pi}{12}} &= \sum_{k=1}^{11} \cos\left(\dfrac{k\pi}{12}-\dfrac{\pi}{24}\right) - \cos\left(\dfrac{k\pi}{12}+\dfrac{\pi}{24}\right) \\
&= \cos{\dfrac{\pi}{24}}-\cos{\dfrac{23\pi}{24}} \\
&= 2\cos{\dfrac{\pi}{24}}
\end{align*}
Dividing this result by $2\sin{\tfrac{\pi}{24}}$ gives the value of the sum as $\cot{\tfrac{\pi}{24}}$.

Alternatively, we can use the formula for series of sines with angles in arithmetic progression, setting $x=0$, $n=11$, $\alpha = \pi / 12$ to find that the sum of the altitudes is
\[
\dfrac{\sin{\dfrac{6\pi}{12}}\sin{\dfrac{11\pi}{24}}}{\sin{\dfrac{\pi}{24}}} = \dfrac{\sin{\dfrac{11\pi}{24}}}{\sin{\dfrac{\pi}{24}}} = \dfrac{\cos{\dfrac{\pi}{24}}}{\sin{\dfrac{\pi}{24}}} = \cot{\dfrac{\pi}{24}}.
\]
Using the half-angle formulas for sine and cosine, we can show that $\cos{\pi/12} = \tfrac{1}{2}\sqrt{2+\sqrt{3}}$ and $\sin{\pi/12} = \tfrac{1}{2}\sqrt{2-\sqrt{3}}$. Therefore,
\begin{align*}
\cot{\dfrac{\pi}{24}} &= \dfrac{1+\cos{\dfrac{\pi}{12}}}{\sin{\dfrac{\pi}{12}}} \\
&= \dfrac{1+\dfrac{1}{2}\sqrt{2+\sqrt{3}}}{\dfrac{1}{2}\sqrt{2-\sqrt{3}}} \\
&= \dfrac{2+\sqrt{2+\sqrt{3}}}{\sqrt{2-\sqrt{3}}} \\
&= \dfrac{2}{\sqrt{2-\sqrt{3}}} + \sqrt{\dfrac{2+\sqrt{3}}{2-\sqrt{3}}} \\
&= \dfrac{2\sqrt{2-\sqrt{3}}}{2-\sqrt{3}} + 2  + \sqrt{3} \\
&=2\sqrt{2-\sqrt{3}}\left(2+\sqrt{3}\right) + 2  + \sqrt{3} \\
&= 2\sqrt{2+\sqrt{3}} + 2  + \sqrt{3} \\
&= 2\left(\sqrt{\dfrac{1}{2}} + \sqrt{\dfrac{3}{2}}\right) + 2  + \sqrt{3} \\
&= \sqrt{2} + \sqrt{6} + 2 + \sqrt{3}.
\end{align*}

Let's elaborate on how to evaluate $\sqrt{2+\sqrt{3}}$. Suppose $\sqrt{2+\sqrt{3}} = \sqrt{a} + \sqrt{b}$ for some rational numbers $a$ and $b$. Then, squaring both sides, we get $2+\sqrt{3} = a+2\sqrt{ab}+b$. Matching the rational and irrational parts, we get two equations relating $a$ and $b$: $a+b=2$ and $4ab=3$. Solving this system shows that $a$ and $b$ are $1/2$ and $3/2$ (the order doesn't matter because the equations are symmetric). Thus, we have shown that
\[
\sqrt{2+\sqrt{3}} = \sqrt{\dfrac{1}{2}} + \sqrt{\dfrac{3}{2}}.
\]
\end{solutions}

\section*{Chapter 8: Graphs of Trigonometric Functions}

\begin{solutions}{Page 177}
\solution
Since $k=5$, Period is $5$, and Frequency is $\frac{\pi}{5}$
\solution
Since $k=\frac{1}{4}$, Period is $\frac{1}{4}$, and Frequency is $\frac{\pi}{1/4}=4\pi$
\solution
Since $k=\frac{4}{5}$, Period is $\frac{4}{5}$, and Frequency is $\frac{\pi}{4/5}= \frac{5\pi}{4}$
\solution
Since $k=\frac{5}{4}$, Period is $\frac{5}{4}$, and Frequency is $\frac{\pi}{5/4}= \frac{4\pi}{5}$
\solution
Period is $\frac{2\pi}{3}$

\begin{tikzpicture}
    \draw[help lines,xstep=pi/2,ystep=1,color=gray!50,dashed](0,-1.4) grid (6.9,1.4);
    \draw[->,thick] (0,0)--(7,0) node[right]{$x$};
    \draw[->,thick] (0,0)--(0,1.5) node[right]{$y$};
    \draw[->,thick] (0,0)--(0,-1.5);
    %Draw marks on x axis
    \draw [very thin,gray](pi/2,-0.1)--(pi/2,0.1) node[below right] at (pi/2,-0.1) {$\frac{\pi}{2}$};
    \draw [very thin,gray](pi,-0.1)--(pi,0.1) node[below right] at (pi,-0.1) {$\pi$};
    \draw [very thin,gray](3*pi/2,-0.1)--(3*pi/2,0.1) node[below right] at (3*pi/2,-0.1) {$\frac{3\pi}{2}$};
    \draw [very thin,gray](2*pi,-0.1)--(2*pi,0.1) node[below right] at (2*pi,-0.1) {$2\pi$};
    %Draw marks on y axis
    \foreach \yticks in {-1,0,1}
    \draw [very thin,gray](-0.1,\yticks)--(0.1,\yticks) node[left] at (0,\yticks) {$\yticks$};
    %Draw the curve
    \draw[red]plot[domain=0:2*pi, samples=90]  (\x,{sin(3*\x r)});
\end{tikzpicture}

\solution
Period is $\frac{2\pi}{1/3}= 6\pi$

\begin{tikzpicture}[xscale=0.5,yscale=1]
    \draw[help lines,xstep=pi,ystep=1,color=gray!50,dashed](0,-1.4) grid (6.1*pi,1.4);
    \draw[->,thick] (0,0)--(6.1*pi,0) node[right]{$x$};
    \draw[->,thick] (0,0)--(0,1.5) node[right]{$y$};
    \draw[->,thick] (0,0)--(0,-1.5);
    %Draw marks on x axis
    \foreach \i in {1,2,...,6}
    \draw [very thin,gray](\i*pi,-0.1)--(\i*pi,0.1) node[below right] at (\i*pi,-0.1) {$\i\pi$};
    %Draw marks on y axis
    \foreach \yticks in {-1,0,1}
    \draw [very thin,gray](-0.1,\yticks)--(0.1,\yticks) node[left] at (0,\yticks) {$\yticks$};
    %Draw the curve
    \draw[red]plot[domain=0:6*pi, samples=90] (\x,{sin(\x/3 r)});
\end{tikzpicture}

\solution
Period is $\frac{2\pi}{3/2}= \frac{4\pi}{3}$

\begin{tikzpicture}[xscale=1,yscale=1]
    \draw[help lines,xstep=pi/2,ystep=1,color=gray!50,dashed](0,-1.4) grid (2.1*pi,1.4);
    \draw[->,thick] (0,0)--(2.1*pi,0) node[right]{$x$};
    \draw[->,thick] (0,0)--(0,1.5) node[right]{$y$};
    \draw[->,thick] (0,0)--(0,-1.5);
    %Draw marks on x axis
    \foreach \i in {1,2}
    \draw [very thin,gray](\i*pi,-0.1)--(\i*pi,0.1) node[below right] at (\i*pi,-0.1) {$\i\pi$};
    %Draw marks on y axis
    \foreach \yticks in {-1,0,1}
    \draw [very thin,gray](-0.1,\yticks)--(0.1,\yticks) node[left] at (0,\yticks) {$\yticks$};
    %Draw the curve
    \draw[red]plot[domain=0:2*pi,samples=90] (\x,{sin(3*\x/2 r)});
\end{tikzpicture}

\solution
Period is $\frac{2\pi}{2/3}= 3\pi$ 

\begin{tikzpicture}[xscale=0.5,yscale=1]
    \draw[help lines,xstep=pi/2,ystep=1,color=gray!50,dashed](0,-1.4) grid (4.1*pi,1.4);
    \draw[->,thick] (0,0)--(4.1*pi,0) node[right]{$x$};
    \draw[->,thick] (0,0)--(0,1.5) node[right]{$y$};
    \draw[->,thick] (0,0)--(0,-1.5);
    %Draw marks on x axis
    \foreach \i in {1,2,...,3}
    \draw [very thin,gray](\i*pi,-0.1)--(\i*pi,0.1) node[below right] at (\i*pi,-0.1) {$\i\pi$};
    %Draw marks on y axis
    \foreach \yticks in {-1,0,1}
    \draw [very thin,gray](-0.1,\yticks)--(0.1,\yticks) node[left] at (0,\yticks) {$\yticks$};
    %Draw the curve
    \draw[red]plot[domain=0:4*pi, samples=90] (\x,{sin(2*\x/3 r)});
\end{tikzpicture}
\solution
Period is $\frac{2\pi}{2/3}= 3\pi$

\begin{tikzpicture}[xscale=0.5,yscale=1]
    \draw[help lines,xstep=pi/2,ystep=1,color=gray!50,dashed](0,-1.4) grid (6.1*pi,1.4);
    \draw[->,thick] (0,0)--(6.1*pi,0) node[right]{$x$};
    \draw[->,thick] (0,0)--(0,1.5) node[right]{$y$};
    \draw[->,thick] (0,0)--(0,-1.5);
    %Draw marks on x axis
    \foreach \i in {1,2,...,6}
    \draw [very thin,gray](\i*pi,-0.1)--(\i*pi,0.1) node[below right] at (\i*pi,-0.1) {$\i\pi$};
    %Draw marks on y axis
    \foreach \yticks in {-1,0,1}
    \draw [very thin,gray](-0.1,\yticks)--(0.1,\yticks) node[left] at (0,\yticks) {$\yticks$};
    %Draw the curve
    \draw[red]plot[domain=0:6*pi, samples=90] (\x,{cos(2*\x/3 r)});
\end{tikzpicture}
\solution
Period is $\frac{2\pi}{3/2}= \frac{4\pi}{3}$

\begin{tikzpicture}[xscale=1,yscale=1]
    \draw[help lines,xstep=pi/2,ystep=1,color=gray!50,dashed](0,-1.4) grid (2.1*pi,1.4);
    \draw[->,thick] (0,0)--(2.1*pi,0) node[right]{$x$};
    \draw[->,thick] (0,0)--(0,1.5) node[right]{$y$};
    \draw[->,thick] (0,0)--(0,-1.5);
    %Draw marks on x axis
    \foreach \i in {1,2}
    \draw [very thin,gray](\i*pi,-0.1)--(\i*pi,0.1) node[below right] at (\i*pi,-0.1) {$\i\pi$};
    %Draw marks on y axis
    \foreach \yticks in {-1,0,1}
    \draw [very thin,gray](-0.1,\yticks)--(0.1,\yticks) node[left] at (0,\yticks) {$\yticks$};
    %Draw the curve
    \draw[red]plot[domain=0:2*pi,samples=90] (\x,{cos(3*\x/2 r)});
\end{tikzpicture}

\solution
For $y=f(3x)$

\begin{tikzpicture}[xscale=1,yscale=1]
    \draw[help lines,xstep=1,ystep=1,color=gray!50,dashed](-1.5,-1.4) grid (6.5,1.4);
    \draw[->,thick] (-1.5,0)--(6.5,0) node[right]{$x$};
    \draw[->,thick] (0,0)--(0,1.5) node[right]{$y$};
    \draw[->,thick] (0,0)--(0,-1.5);
    %Draw marks on x axis
    \foreach \i in {-1,0,...,6}
    \draw [very thin,gray](\i,-0.1)--(\i,0.1) node[below right] at (\i,-0.1) {$\i$};
    %Draw marks on y axis
    \foreach \yticks in {-1,0,1}
    \draw [very thin,gray](-0.1,\yticks)--(0.1,\yticks) node[left] at (0,\yticks) {$\yticks$};
    %Draw the curve
    %\draw[red]plot[domain=-1:5,samples=90] (\x,{asin(sin(\x*pi r))});
    \draw[red] plot[domain=-1:5,samples=180] (\x,{asin(sin(3*\x*pi r))/90});
\end{tikzpicture}

For $y=f(\frac{x}{3})$

\begin{tikzpicture}[xscale=1,yscale=1]
    \draw[help lines,xstep=1,ystep=1,color=gray!50,dashed](-1.5,-1.4) grid (6.5,1.4);
    \draw[->,thick] (-1.5,0)--(6.5,0) node[right]{$x$};
    \draw[->,thick] (0,0)--(0,1.5) node[right]{$y$};
    \draw[->,thick] (0,0)--(0,-1.5);
    %Draw marks on x axis
    \foreach \i in {-1,0,...,6}
    \draw [very thin,gray](\i,-0.1)--(\i,0.1) node[below right] at (\i,-0.1) {$\i$};
    %Draw marks on y axis
    \foreach \yticks in {-1,0,1}
    \draw [very thin,gray](-0.1,\yticks)--(0.1,\yticks) node[left] at (0,\yticks) {$\yticks$};
    %Draw the curve
    %\draw[red]plot[domain=-1:5,samples=90] (\x,{asin(sin(\x*pi r))});
    \draw[red] plot[domain=-1:6,samples=180] (\x,{asin(sin(\x*pi/3 r))/90});
\end{tikzpicture}
\end{solutions}

\begin{solutions}{Page 179}
\solution
$y= 2\sin\left(x\right)$

\begin{tikzpicture}[xscale=1,yscale=0.5]
    \draw[help lines,xstep=pi/2,ystep=1,color=gray!50,dashed](0,-2.4) grid (2.1*pi,2.4);
    \draw[->,thick] (0,0)--(2.1*pi,0) node[right]{$x$};
    \draw[->,thick] (0,0)--(0,2.5) node[right]{$y$};
    \draw[->,thick] (0,0)--(0,-2.5);
    %Draw marks on x axis
    \foreach \i in {1,2}
    \draw [very thin,gray](\i*pi,-0.1)--(\i*pi,0.1) node[below right] at (\i*pi,-0.1) {$\i\pi$};
    %Draw marks on y axis
    \foreach \yticks in {-2,-1,0,1,2}
    \draw [very thin,gray](-0.1,\yticks)--(0.1,\yticks) node[left] at (0,\yticks) {$\yticks$};
    %Draw the curve
    \draw[red]plot[domain=0:2*pi,samples=90] (\x,{2*sin(\x r)});
\end{tikzpicture}

\solution
$y= \frac{1}{2}\sin\left(x\right)$

\begin{tikzpicture}[xscale=1,yscale=1]
    \draw[help lines,xstep=pi/2,ystep=1,color=gray!50,dashed](0,-1.4) grid (2.1*pi,1.4);
    \draw[->,thick] (0,0)--(2.1*pi,0) node[right]{$x$};
    \draw[->,thick] (0,0)--(0,1.5) node[right]{$y$};
    \draw[->,thick] (0,0)--(0,-1.5);
    %Draw marks on x axis
    \foreach \i in {1,2}
    \draw [very thin,gray](\i*pi,-0.1)--(\i*pi,0.1) node[below right] at (\i*pi,-0.1) {$\i\pi$};
    %Draw marks on y axis
    \foreach \yticks in {-1,0,1}
    \draw [very thin,gray](-0.1,\yticks)--(0.1,\yticks) node[left] at (0,\yticks) {$\yticks$};
    %Draw the curve
    \draw[red]plot[domain=0:2*pi,samples=90] (\x,{0.5*sin(\x r)});
\end{tikzpicture}

\solution
$y= 3\sin\left(2x\right)$

\begin{tikzpicture}[xscale=1,yscale=0.5]
    \draw[help lines,xstep=pi/2,ystep=1,color=gray!50,dashed](0,-3.4) grid (2.1*pi,3.4);
    \draw[->,thick] (0,0)--(2.1*pi,0) node[right]{$x$};
    \draw[->,thick] (0,0)--(0,3.5) node[right]{$y$};
    \draw[->,thick] (0,0)--(0,-3.5);
    %Draw marks on x axis
    \foreach \i in {1,2}
    \draw [very thin,gray](\i*pi,-0.1)--(\i*pi,0.1) node[below right] at (\i*pi,-0.1) {$\i\pi$};
    %Draw marks on y axis
    \foreach \yticks in {-3,-2,...,3}
    \draw [very thin,gray](-0.1,\yticks)--(0.1,\yticks) node[left] at (0,\yticks) {$\yticks$};
    %Draw the curve
    \draw[red]plot[domain=0:2*pi,samples=90] (\x,{3*sin(\x r)});
\end{tikzpicture}

\solution
$y= \frac{1}{2}\sin\left(3x\right)$

\begin{tikzpicture}[xscale=1,yscale=1]
    \draw[help lines,xstep=pi/2,ystep=1,color=gray!50,dashed](0,-1.4) grid (2.1*pi,1.4);
    \draw[->,thick] (0,0)--(2.1*pi,0) node[right]{$x$};
    \draw[->,thick] (0,0)--(0,1.5) node[right]{$y$};
    \draw[->,thick] (0,0)--(0,-1.5);
    %Draw marks on x axis
    \foreach \i in {1,2}
    \draw [very thin,gray](\i*pi,-0.1)--(\i*pi,0.1) node[below right] at (\i*pi,-0.1) {$\i\pi$};
    %Draw marks on y axis
    \foreach \yticks in {-1,0,1}
    \draw [very thin,gray](-0.1,\yticks)--(0.1,\yticks) node[left] at (0,\yticks) {$\yticks$};
    %Draw the curve
    \draw[red]plot[domain=0:2*pi,samples=90] (\x,{0.5*sin(3*\x r))});
\end{tikzpicture}

\solution
$y= 4\cos\left(x\right)$

\begin{tikzpicture}[xscale=1,yscale=0.5]
    \draw[help lines,xstep=pi/2,ystep=1,color=gray!50,dashed](0,-4.4) grid (2.1*pi,4.4);
    \draw[->,thick] (0,0)--(2.1*pi,0) node[right]{$x$};
    \draw[->,thick] (0,0)--(0,4.5) node[right]{$y$};
    \draw[->,thick] (0,0)--(0,-4.5);
    %Draw marks on x axis
    \foreach \i in {1,2}
    \draw [very thin,gray](\i*pi,-0.1)--(\i*pi,0.1) node[below right] at (\i*pi,-0.1) {$\i\pi$};
    %Draw marks on y axis
    \foreach \yticks in {-4,-3,...,4}
    \draw [very thin,gray](-0.1,\yticks)--(0.1,\yticks) node[left] at (0,\yticks) {$\yticks$};
    %Draw the curve
    \draw[red]plot[domain=0:2*pi,samples=90] (\x,{4*cos(\x r))});
\end{tikzpicture}

\solution
For $y=3f(x)$

\begin{tikzpicture}[xscale=1,yscale=0.5]
    \draw[help lines,xstep=1,ystep=1,color=gray!50,dashed](-1.5,-3.4) grid (6.5,3.4);
    \draw[->,thick] (-1.5,0)--(6.5,0) node[right]{$x$};
    \draw[->,thick] (0,0)--(0,3.5) node[right]{$y$};
    \draw[->,thick] (0,0)--(0,-3.5);
    %Draw marks on x axis
    \foreach \i in {-1,0,...,6}
    \draw [very thin,gray](\i,-0.1)--(\i,0.1) node[below right] at (\i,-0.1) {$\i$};
    %Draw marks on y axis
    \foreach \yticks in {-3,-2,...,3}
    \draw [very thin,gray](-0.1,\yticks)--(0.1,\yticks) node[left] at (0,\yticks) {$\yticks$};
    %Draw the curve
    \draw[red] plot[domain=-1:6,samples=180] (\x,{3*asin(sin(\x*pi r))/90});
\end{tikzpicture}

\solution
For $y=\frac{1}{3}f(x)$

\begin{tikzpicture}[xscale=1,yscale=1]
    \draw[help lines,xstep=1,ystep=1,color=gray!50,dashed](-1.5,-1.4) grid (6.5,1.4);
    \draw[->,thick] (-1.5,0)--(6.5,0) node[right]{$x$};
    \draw[->,thick] (0,0)--(0,1.5) node[right]{$y$};
    \draw[->,thick] (0,0)--(0,-1.5);
    %Draw marks on x axis
    \foreach \i in {-1,0,...,6}
    \draw [very thin,gray](\i,-0.1)--(\i,0.1) node[below right] at (\i,-0.1) {$\i$};
    %Draw marks on y axis
    \foreach \yticks in {-1,0,1}
    \draw [very thin,gray](-0.1,\yticks)--(0.1,\yticks) node[left] at (0,\yticks) {$\yticks$};
    %Draw the curve
    \draw[red] plot[domain=-1:6,samples=180] (\x,{1/3*asin(sin(\x*pi r))/90});
\end{tikzpicture}

\end{solutions}

\begin{solutions}{Page 181}
\solution
$y= \sin\left(x-\frac{\pi}{6}\right)$

\begin{tikzpicture}[xscale=1,yscale=1]
    \draw[help lines,xstep=pi/2,ystep=1,color=gray!50,dashed](0,-1.4) grid (2.1*pi,1.4);
    \draw[->,thick] (0,0)--(2.1*pi,0) node[right]{$x$};
    \draw[->,thick] (0,0)--(0,1.5) node[right]{$y$};
    \draw[->,thick] (0,0)--(0,-1.5);
    %Draw marks on x axis
    \foreach \i in {1,2}
    \draw [very thin,gray](\i*pi,-0.1)--(\i*pi,0.1) node[below right] at (\i*pi,-0.1) {$\i\pi$};
    %Draw marks on y axis
    \foreach \yticks in {-1,0,1}
    \draw [very thin,gray](-0.1,\yticks)--(0.1,\yticks) node[left] at (0,\yticks) {$\yticks$};
    %Draw the curve
    \draw[red]plot[domain=0:2*pi,samples=90] (\x,{sin((\x - pi/6) r)});
\end{tikzpicture}

\solution
$y= \sin\left(x+\frac{\pi}{6}\right) $

\begin{tikzpicture}[xscale=1,yscale=1]
    \draw[help lines,xstep=pi/2,ystep=1,color=gray!50,dashed](0,-1.4) grid (2.1*pi,1.4);
    \draw[->,thick] (0,0)--(2.1*pi,0) node[right]{$x$};
    \draw[->,thick] (0,0)--(0,1.5) node[right]{$y$};
    \draw[->,thick] (0,0)--(0,-1.5);
    %Draw marks on x axis
    \foreach \i in {1,2}
    \draw [very thin,gray](\i*pi,-0.1)--(\i*pi,0.1) node[below right] at (\i*pi,-0.1) {$\i\pi$};
    %Draw marks on y axis
    \foreach \yticks in {-1,0,1}
    \draw [very thin,gray](-0.1,\yticks)--(0.1,\yticks) node[left] at (0,\yticks) {$\yticks$};
    %Draw the curve
    \draw[red]plot[domain=0:2*pi,samples=90] (\x,{sin((\x + pi/6) r)});
\end{tikzpicture}

\solution
$y= 2\sin\left(x-\frac{\pi}{2}\right)$

\begin{tikzpicture}[xscale=1,yscale=0.5]
    \draw[help lines,xstep=pi/2,ystep=1,color=gray!50,dashed](0,-2.4) grid (2.1*pi,2.4);
    \draw[->,thick] (0,0)--(2.1*pi,0) node[right]{$x$};
    \draw[->,thick] (0,0)--(0,2.5) node[right]{$y$};
    \draw[->,thick] (0,0)--(0,-2.5);
    %Draw marks on x axis
    \foreach \i in {1,2}
    \draw [very thin,gray](\i*pi,-0.1)--(\i*pi,0.1) node[below right] at (\i*pi,-0.1) {$\i\pi$};
    %Draw marks on y axis
    \foreach \yticks in {-2,-1,0,1,2}
    \draw [very thin,gray](-0.1,\yticks)--(0.1,\yticks) node[left] at (0,\yticks) {$\yticks$};
    %Draw the curve
    \draw[red]plot[domain=0:2*pi,samples=90] (\x,{2*sin((\x - pi/2) r)});
\end{tikzpicture}

\solution
$y= \frac{1}{2}\sin\left(x+\frac{\pi}{2}\right)$

\begin{tikzpicture}[xscale=1,yscale=1]
    \draw[help lines,xstep=pi/2,ystep=1,color=gray!50,dashed](0,-1.4) grid (2.1*pi,1.4);
    \draw[->,thick] (0,0)--(2.1*pi,0) node[right]{$x$};
    \draw[->,thick] (0,0)--(0,1.5) node[right]{$y$};
    \draw[->,thick] (0,0)--(0,-1.5);
    %Draw marks on x axis
    \foreach \i in {1,2}
    \draw [very thin,gray](\i*pi,-0.1)--(\i*pi,0.1) node[below right] at (\i*pi,-0.1) {$\i\pi$};
    %Draw marks on y axis
    \foreach \yticks in {-1,0,1}
    \draw [very thin,gray](-0.1,\yticks)--(0.1,\yticks) node[left] at (0,\yticks) {$\yticks$};
    %Draw the curve
    \draw[red]plot[domain=0:2*pi,samples=90] (\x,{0.5*sin((\x + pi/2) r)});
\end{tikzpicture}

\solution
$y= \cos\left(x-\frac{\pi}{4}\right)$

\begin{tikzpicture}[xscale=1,yscale=1]
    \draw[help lines,xstep=pi/2,ystep=1,color=gray!50,dashed](0,-1.4) grid (2.1*pi,1.4);
    \draw[->,thick] (0,0)--(2.1*pi,0) node[right]{$x$};
    \draw[->,thick] (0,0)--(0,1.5) node[right]{$y$};
    \draw[->,thick] (0,0)--(0,-1.5);
    %Draw marks on x axis
    \foreach \i in {1,2}
    \draw [very thin,gray](\i*pi,-0.1)--(\i*pi,0.1) node[below right] at (\i*pi,-0.1) {$\i\pi$};
    %Draw marks on y axis
    \foreach \yticks in {-1,0,1}
    \draw [very thin,gray](-0.1,\yticks)--(0.1,\yticks) node[left] at (0,\yticks) {$\yticks$};
    %Draw the curve
    \draw[red]plot[domain=0:2*pi,samples=90] (\x,{cos((\x - pi/4) r)});
\end{tikzpicture}

\solution
$y= 3\cos\left(x+\frac{\pi}{3}\right)$

\begin{tikzpicture}[xscale=1,yscale=0.5]
    \draw[help lines,xstep=pi/2,ystep=1,color=gray!50,dashed](0,-3.4) grid (2.1*pi,3.4);
    \draw[->,thick] (0,0)--(2.1*pi,0) node[right]{$x$};
    \draw[->,thick] (0,0)--(0,3.5) node[right]{$y$};
    \draw[->,thick] (0,0)--(0,-3.5);
    %Draw marks on x axis
    \foreach \i in {1,2}
    \draw [very thin,gray](\i*pi,-0.1)--(\i*pi,0.1) node[below right] at (\i*pi,-0.1) {$\i\pi$};
    %Draw marks on y axis
    \foreach \yticks in {-3,-2,...,3}
    \draw [very thin,gray](-0.1,\yticks)--(0.1,\yticks) node[left] at (0,\yticks) {$\yticks$};
    %Draw the curve
    \draw[red]plot[domain=0:2*pi,samples=90] (\x,{3*cos((\x + pi/3) r)});
\end{tikzpicture}

\solution
$y= \sin\left(x-2x\right)$ 

\begin{tikzpicture}[xscale=1,yscale=1]
    \draw[help lines,xstep=pi/2,ystep=1,color=gray!50,dashed](0,-1.4) grid (2.1*pi,1.4);
    \draw[->,thick] (0,0)--(2.1*pi,0) node[right]{$x$};
    \draw[->,thick] (0,0)--(0,1.5) node[right]{$y$};
    \draw[->,thick] (0,0)--(0,-1.5);
    %Draw marks on x axis
    \foreach \i in {1,2}
    \draw [very thin,gray](\i*pi,-0.1)--(\i*pi,0.1) node[below right] at (\i*pi,-0.1) {$\i\pi$};
    %Draw marks on y axis
    \foreach \yticks in {-1,0,1}
    \draw [very thin,gray](-0.1,\yticks)--(0.1,\yticks) node[left] at (0,\yticks) {$\yticks$};
    %Draw the curve
    \draw[red]plot[domain=0:2*pi,samples=90] (\x,{cos((\x - 2*pi) r)});
\end{tikzpicture}

\solution
Since the sine wave has shifted right by $\frac{\pi}{3}$. The equation is \[y=\sin\left(x-\frac{\pi}{3}\right) \]

\solution
Since the sine wave has shifted right by $\frac{2\pi}{3}$. The equation is \[y=\sin\left(x-\frac{2\pi}{3}\right) \]

\solution
Since the sine wave has shifted left by $\frac{\pi}{6}$. The equation is \[y=\sin\left(x-\frac{\pi}{6}\right) \]

\solution
Note: this looks like the sine wave has shifted right by $\frac{\pi}{4}$, but at that point, the sine wave is on its wave down. 

The actual shift is the $x$ intercept on the far left, calculate its location by $\frac{\pi}{4}- \pi = -\frac{3\pi}{4}$ as it's half a period away, that is, if $2\pi$ radians is one period, then $\pi$ radians is a half period

Therefore, the sine wave has shifted left by $\frac{3\pi}{4}$. The equation is \[y=\sin\left(x+\frac{3\pi}{4}\right) \]
\end{solutions}

\begin{solutions}{Page 183}
\solution
$y= \sin \frac{1}{2}\left(x-\frac{\pi}{6}\right)$

\begin{tikzpicture}[xscale=0.5,yscale=1]
    \draw[help lines,xstep=pi/2,ystep=1,color=gray!50,dashed](0,-1.4) grid (4.1*pi,1.4);
    \draw[->,thick] (0,0)--(4.1*pi,0) node[right]{$x$};
    \draw[->,thick] (0,0)--(0,1.5) node[right]{$y$};
    \draw[->,thick] (0,0)--(0,-1.5);
    %Draw marks on x axis
    \foreach \i in {1,2,3,4}
    \draw [very thin,gray](\i*pi,-0.1)--(\i*pi,0.1) node[below right] at (\i*pi,-0.1) {$\i\pi$};
    %Draw marks on y axis
    \foreach \yticks in {-1,0,1}
    \draw [very thin,gray](-0.1,\yticks)--(0.1,\yticks) node[left] at (0,\yticks) {$\yticks$};
    %Draw the curve
    \draw[red]plot[domain=0:4*pi,samples=90] (\x,{sin(0.5*(\x - pi/6) r)});
\end{tikzpicture}

\solution
$y= \sin \left(\frac{1}{3}x-\frac{\pi}{6}\right)$

\begin{tikzpicture}[xscale=0.5,yscale=1]
    \draw[help lines,xstep=pi/2,ystep=1,color=gray!50,dashed](0,-1.4) grid (4.1*pi,1.4);
    \draw[->,thick] (0,0)--(4.1*pi,0) node[right]{$x$};
    \draw[->,thick] (0,0)--(0,1.5) node[right]{$y$};
    \draw[->,thick] (0,0)--(0,-1.5);
    %Draw marks on x axis
    \foreach \i in {1,2,3,4}
    \draw [very thin,gray](\i*pi,-0.1)--(\i*pi,0.1) node[below right] at (\i*pi,-0.1) {$\i\pi$};
    %Draw marks on y axis
    \foreach \yticks in {-1,0,1}
    \draw [very thin,gray](-0.1,\yticks)--(0.1,\yticks) node[left] at (0,\yticks) {$\yticks$};
    %Draw the curve
    \draw[red]plot[domain=0:4*pi,samples=90] (\x,{sin((0.5*\x - pi/6) r)});
\end{tikzpicture}

\solution
$y= \cos 2\left(x+\frac{\pi}{3}\right)$

\begin{tikzpicture}[xscale=0.5,yscale=1]
    \draw[help lines,xstep=pi/2,ystep=1,color=gray!50,dashed](0,-1.4) grid (4.1*pi,1.4);
    \draw[->,thick] (0,0)--(4.1*pi,0) node[right]{$x$};
    \draw[->,thick] (0,0)--(0,1.5) node[right]{$y$};
    \draw[->,thick] (0,0)--(0,-1.5);
    %Draw marks on x axis
    \foreach \i in {1,2,3,4}
    \draw [very thin,gray](\i*pi,-0.1)--(\i*pi,0.1) node[below right] at (\i*pi,-0.1) {$\i\pi$};
    %Draw marks on y axis
    \foreach \yticks in {-1,0,1}
    \draw [very thin,gray](-0.1,\yticks)--(0.1,\yticks) node[left] at (0,\yticks) {$\yticks$};
    %Draw the curve
    \draw[red]plot[domain=0:4*pi,samples=90] (\x,{cos((2*(\x+pi/3))  r)});
\end{tikzpicture}

\solution
Since period is $\pi$, making $k=\frac{2\pi}{\pi}=2$, making the equation \[y=\sin 2(x-\beta) \]

To solve for $\beta$, substitute $x=0,y=-1$, as the sine wave passes point $(0,-1)$, giving
\begin{align*}
    -1 &= \sin 2(0-\beta)\\
    -1 &= \sin -2\beta\\
    -\frac{\pi}{2} &= -2\beta\\
    \frac{\pi}{4} &= \beta\\
    \beta &= \frac{\pi}{4}
\end{align*}

Therefore the equation is \[y=\sin 2\left(x-\frac{\pi}{4}\right)\]

\solution
Since half period is $2\pi$, its full period is $4\pi $making $k=\frac{2\pi}{4\pi}=\frac{1}{2}$, making the equation \[y=\sin \frac{1}{2}(x-\beta) \]

To solve for $\beta$, substitute $x=\frac{5\pi}{6},y=1$, as the sine wave passes point $(\frac{5\pi}{6},1)$, giving
\begin{align*}
    1 &= \sin \frac{1}{2} \left( \frac{5\pi}{6}-\beta \right)\\
    \frac{\pi}{2} &= \frac{1}{2}\left(\frac{5\pi}{6}-\beta\right)\\
    \pi &= \frac{5\pi}{6}-\beta\\
    \frac{\pi}{6} &= -\beta\\
    \beta &= -\frac{\pi}{6}
\end{align*}

Therefore the equation is \[y=\sin \frac{1}{2}\left(x-\frac{\pi}{6}\right)\]
\end{solutions}


\begin{solutions}{Page 186}
\solution
Answer A) equal to $\sin x$ since

\begin{align*}
    \sin (x+2\pi) &= \sin (x+2\pi - 2\pi)\\
                  &= \sin (x)
\end{align*}

\solution

Answer C) equal to $-\sin x$ since

\begin{align*}
    \sin (x+3\pi) &= \sin (x+3\pi - 2\pi)\\
                  &= \sin (x+\pi)\\
                  &= -\sin(x)
\end{align*}

\solution
Answer B) equal to $\cos x$ since

\begin{align*}
    \sin \left(x+\frac{9}{2}\pi\right) &= \sin \left(x+\frac{9}{2}\pi - 4\pi\right)\\
                       &= \sin \left(x+\frac{\pi}{2}\right)\\
                      &= \cos(x)
\end{align*}

\solution
Answer D) equal to $-\cos x$ since

\begin{align*}
    \sin \left(x-\frac{\pi}{2}\right) &= -\cos(x) 
\end{align*}

\solution
Answer B) equal to $\cos x$ since

\begin{align*}
    \sin \left(x-\frac{3}{2}\pi\right) &= \sin \left(-\frac{3}{2}\pi + 2\pi\right)\\
                       &= \sin \left(x+\frac{\pi}{2}\right)\\
                      &= \cos(x)
\end{align*}

\solution
Answer D) equal to $-\cos x$ since

\begin{align*}
    \sin \left(x+\frac{19}{2}\pi\right) &= \sin \left(x+\frac{19}{2}\pi - 10\pi\right)\\
                       &= \sin \left(x-\frac{\pi}{2}\right)\\
                      &= -\cos(x)
\end{align*}

\solution
Answer D) equal to $-\cos x$ since

\begin{align*}
    -\sin \left(x-\frac{19}{2}\pi\right) &= -\sin \left(x+\frac{19}{2}\pi + 10\pi\right)\\
                       &= -\sin \left(x+\frac{\pi}{2}\right)\\
                      &= -\cos(x)
\end{align*}

\solution
Answer B) equal to $\cos x$ since

\begin{align*}
    \sin \left(x+\frac{157}{2}\pi\right) &= \sin \left(x+\frac{157}{2}\pi - 78\pi\right)\\
                       &= \sin \left(x+\frac{\pi}{2}\right)\\
                      &= \cos(x)
\end{align*}

\solution
Answer D) equal to $-\cos x$ since

\begin{align*}
    \sin \left(x-\frac{157}{2}\pi\right) &= \sin \left(x-\frac{157}{2}\pi + 78\pi\right)\\
                       &= \sin \left(x-\frac{\pi}{2}\right)\\
                      &= -\cos(x)
\end{align*}

\solution
Note: there is no question 10 in textbook (printing error?)

\solution %need more work
Using definition that $p$ is called a half-period of function $f$ is $f(x+p)= -f(x)$, for all values of x which $f(x)$ and $f(x+p)$ are defined

Since $\cos (x) $ has period of $2\pi$ radians, therefore its half-period is $\pi$ radians, or we can compare diagrams

\begin{tikzpicture}[xscale=0.5,yscale=1]
    \draw[help lines,xstep=pi/2,ystep=1,color=gray!50,dashed](0,-1.4) grid (4.1*pi,1.4);
    \draw[->,thick] (0,0)--(4.1*pi,0) node[right]{$x$};
    \draw[->,thick] (0,0)--(0,1.5) node[right]{$y$};
    \draw[->,thick] (0,0)--(0,-1.5);
    %Draw marks on x axis
    \foreach \i in {1,2,3,4}
    \draw [very thin,gray](\i*pi,-0.1)--(\i*pi,0.1) node[below right] at (\i*pi,-0.1) {$\i\pi$};
    %Draw marks on y axis
    \foreach \yticks in {-1,0,1}
    \draw [very thin,gray](-0.1,\yticks)--(0.1,\yticks) node[left] at (0,\yticks) {$\yticks$};
    %Draw the curve
    \draw[red]plot[domain=0:4*pi,samples=90] (\x,{cos(\x r)}) node[above] at (4*pi,1) {$y=\cos(x)$};
    \draw[blue]plot[domain=0:4*pi,samples=90] (\x,{cos((\x + pi) r)}) node[below] at (4*pi,-1) {$y=\cos(x+\pi)$};
\end{tikzpicture}

Which can be seen that $\cos(x+\pi)$ is the same as $-\cos (x)$, therefore by definition, $\cos x$ has a half-period of $\pi$ radians

For $y = \tan x$
\begin{align*}
    \tan (x + \pi) &= \frac{\sin (x+\pi)}{\cos (x+\pi)}\\
           &= \frac{-\sin x}{-\cos x}\\
           &= \tan x
\end{align*}
So $\pi$ is period of $y=\tan x$ 

For $y=\cot x$
\begin{align*}
    \cot (x + \pi) &= \frac{\cos (x+\pi)}{\sin (x+\pi)}\\
           &= \frac{-\cos x}{-\sin x}\\
           &= \cot x
\end{align*}

So $\pi$ is period of $y=\cot x$ 

\solution
Using definition that $q$ is called a half-period of function $f$ is $f(x+q)= -f(x)$, for all values of x which $f(x)$ and $f(x+q)$ are defined

\begin{align*}
    f(x+2q) &= f\left(x+q+q \right)\\
            &= -f\left(x+q \right)\\
            &= -\left(-f(x)\right)\\
            &= f(x)
\end{align*}
Therefore if $q$ is half period of some function $f$, then $2q$ is a period of $f$

\solution
\begin{subsolutions}

\subsolution
\begin{align*}
    \cos \left( x+k\pi/2  \right) &= \cos \left( x+ \frac{\pi(4n+1)}{2}\right) \\
              &= \cos \left( x+ 2n\pi + \pi/2  \right)\\
              &= \cos (x+ \pi/2)\\
              &= -\sin x
\end{align*}

\subsolution
\begin{align*}
    \cos \left( x+k\pi/2  \right) &= \cos \left( x+ \frac{\pi(4n+2)}{2}\right) \\
              &= \cos \left( x+ 2n\pi + \pi  \right)\\
              &= \cos (x+ \pi)\\
              &= -\cos x
\end{align*}

\subsolution
\begin{align*}
    \cos \left( x+k\pi/2  \right) &= \cos \left( x+ \frac{\pi(4n+3)}{2}\right) \\
          &= \cos \left( x+ 2n\pi + \frac{3\pi}{2} \right)\\
          &= \cos (x+ \frac{3\pi}{2})\\
          &= \cos (x - \frac{\pi}{2})\\
          &= \sin x
\end{align*}

\subsolution
\begin{align*}
    \cos \left( x+k\pi/2  \right) &= \cos \left( x+ \frac{\pi(4n)}{2}\right) \\
          &= \cos \left( x+ 2n\pi\right)\\
          &= \cos x
\end{align*}

\end{subsolutions}
\solution

\begin{subsolutions}
\subsolution
As the equation has a negative coefficient, we need to shift this by a half-period of $\pi$ radians

\begin{align*}
    y&=-2 \sin x\\
    &= 2 \sin (x + \pi)
\end{align*}
Comparing coefficients, we get $a=2$ and $k=1$ and $\beta=\pi$

\subsolution
As the equation has a negative coefficient, we need to shift this by a half-period of $\pi$ radians

\begin{align*}
    y&=2 \sin (x-\pi/3)\\
    &= 2 \sin (x -\pi/3+ \pi)\\
    &= 2 \sin (x +2\pi/3)
\end{align*}
Comparing coefficients, we get $a=2$ and $k=1$ and $\beta=-2\pi/3$

\subsolution
As the equation has a negative coefficient, we need to shift this by a half-period of $\pi$ radians

\begin{align*}
    y&=-2 \sin (x+\pi/4)\\
    &= 2 \sin (x +\pi/4+ \pi)\\
    &= 2 \sin (x +5\pi/4)
\end{align*}
Comparing coefficients, we get $a=2$ and $k=1$ and $\beta=-5\pi/4$

\subsolution
As the equation is a cosine function, we need to re-express as a sine function

\begin{align*}
    3 \cos x &= 3 \sin(x+\pi/2)\\
\end{align*}
Comparing coefficients, we get $a=3$ and $k=1$ and $\beta=\pi/2$

\subsolution
As the equation is a cosine function, we need to re-express as a sine function

\begin{align*}
    y&=3 \cos (x-\pi/6)\\ 
     &= 3 \sin(x-\pi/6+\pi/2)\\
     &= 3 \sin(x+\pi/3)
\end{align*}
Comparing coefficients, we get $a=3$ and $k=1$ and $\beta=-\pi/3$

\subsolution
As the equation is a cosine function, we need to re-express as a sine function. Then re-express as a postive amplitude

\begin{align*}
    y&=-3 \cos (x-\pi/6)\\ 
     &=-3 \sin(x-\pi/6+\pi/2)\\
     &=3 \sin(x-\pi/6+\pi/2+\pi)\\
     &=3 \sin(x+4\pi/3)
\end{align*}
Comparing coefficients, we get $a=3$ and $k=1$ and $\beta=-4\pi/3$
\end{subsolutions}

\solution
\[\cos(x-\pi/5) \]
\begin{tikzpicture}[xscale=0.5,yscale=1]
    \draw[help lines,xstep=pi/2,ystep=1,color=gray!50,dashed](0,-1.4) grid (4.1*pi,1.4);
    \draw[->,thick] (0,0)--(4.1*pi,0) node[right]{$x$};
    \draw[->,thick] (0,0)--(0,1.5) node[right]{$y$};
    \draw[->,thick] (0,0)--(0,-1.5);
    %Draw marks on x axis
    \foreach \i in {1,2,3,4}
    \draw [very thin,gray](\i*pi,-0.1)--(\i*pi,0.1) node[below right] at (\i*pi,-0.1) {$\i\pi$};
    %Draw marks on y axis
    \foreach \yticks in {-1,0,1}
    \draw [very thin,gray](-0.1,\yticks)--(0.1,\yticks) node[left] at (0,\yticks) {$\yticks$};
    %Draw the curve
    \draw[red]plot[domain=0:4*pi,samples=90] (\x,{cos((\x-pi/5)  r)});
\end{tikzpicture}

\solution
We need to shift to the right by $\frac{\pi}{2}$

We need to shift to the left by $\frac{3\pi}{2}$

\solution

To express $k$ as an odd number, let $k=2n+1$ for any integer $n$
\begin{align*}
    \tan \left( x+k\pi/2  \right) &= \tan \left( x+ \frac{\pi(2n+1)}{2}\right) \\
          &= \tan \left( x+ 2n\pi + \pi/2\right)\\
          &= \tan \left(x + \pi/2\right)\\
          &= \frac{\sin(x+\pi/2)}{\cos(x+\pi/2)}\\
          &= \frac{-\cos x}{\sin x}\\
          &= -\cot x
\end{align*}

To express $k$ as an even number, let $k=2n$ for any integer $n$
\begin{align*}
    \tan \left( x+k\pi/2  \right) &= \tan \left( x+ \frac{\pi(2n)}{2}\right) \\
          &= \tan \left( x+ 2n\pi \right)\\
          &= \tan x\\ 
\end{align*}

\end{solutions}


\begin{solutions}{Page 188}
\solution
a) $y= \tan\left(x -\frac{\pi}{6}\right)$

\begin{tikzpicture}[xscale=1,yscale=0.5]
    \draw[help lines,xstep=pi/2,ystep=1,color=gray!50,dashed](0,-3.4) grid (2.1*pi,3.4);
    \draw[->,thick] (0,0)--(2.1*pi,0) node[right]{$x$};
    \draw[->,thick] (0,0)--(0,3.5) node[right]{$y$};
    \draw[->,thick] (0,0)--(0,-3.5);
    %Draw marks on x axis
    \foreach \i in {1,2}
    \draw [very thin,gray](\i*pi,-0.1)--(\i*pi,0.1) node[below right] at (\i*pi,-0.1) {$\i\pi$};
    %Draw marks on y axis
    \foreach \yticks in {-3,-2,...,3}
    \draw [very thin,gray](-0.1,\yticks)--(0.1,\yticks) node[left] at (0,\yticks) {$\yticks$};
    %Draw asymptotes
    \draw [thin,dashed, blue](2/3*pi,-3.5)--(2/3*pi,3.5) node [below right] at (2/3*pi,0) {$\frac{2\pi}{3}$};
    \draw [thin,dashed, blue](5/3*pi,-3.5)--(5/3*pi,3.5) node [below right] at (5/3*pi,0) {$\frac{2\pi}{3}$};
    %Draw the curve
    \draw[red]plot[domain=0:.56*pi,samples=30] (\x,{tan((\x - pi/6) r)});
    \draw[red]plot[domain=0.77*pi:1.56*pi,samples=60] (\x,{tan((\x - pi/6) r)});
    \draw[red]plot[domain=1.77*pi:2*pi,samples=30] (\x,{tan((\x - pi/6) r)});
\end{tikzpicture}

b) $y= 3\tan\left(x\right)$

\begin{tikzpicture}[xscale=1,yscale=0.5]
    \draw[help lines,xstep=pi/2,ystep=1,color=gray!50,dashed](0,-3.4) grid (2.1*pi,3.4);
    \draw[->,thick] (0,0)--(2.1*pi,0) node[right]{$x$};
    \draw[->,thick] (0,0)--(0,3.5) node[right]{$y$};
    \draw[->,thick] (0,0)--(0,-3.5);
    %Draw marks on x axis
    \foreach \i in {1,2}
    \draw [very thin,gray](\i*pi,-0.1)--(\i*pi,0.1) node[below right] at (\i*pi,-0.1) {$\i\pi$};
    %Draw marks on y axis
    \foreach \yticks in {-3,-2,...,3}
    \draw [very thin,gray](-0.1,\yticks)--(0.1,\yticks) node[left] at (0,\yticks) {$\yticks$};
    %Draw asymptotes
    \draw [thin,dashed, blue](0.5*pi,-3.5)--(0.5*pi,3.5) node [below right] at (0.5*pi,0) {$\frac{\pi}{2}$};
    \draw [thin,dashed, blue](1.5*pi,-3.5)--(1.5*pi,3.5) node [below right] at (1.5*pi,0) {$\frac{3\pi}{2}$};
    %Draw the curve
    \draw[red]plot[domain=0:0.25*pi,samples=30](\x,{3*tan(\x r)});
    \draw[red]plot[domain=0.75*pi:1.25*pi,samples=30](\x,{3*tan(\x r)});
    \draw[red]plot[domain=1.75*pi:2*pi,samples=30] (\x,{3*tan(\x r)});
\end{tikzpicture}

c) $y= \cot\left(x+\frac{\pi}{4}\right)$

\begin{tikzpicture}[xscale=1,yscale=0.5]
    \draw[help lines,xstep=pi/2,ystep=1,color=gray!50,dashed](0,-3.4) grid (2.1*pi,3.4);
    \draw[->,thick] (0,0)--(2.1*pi,0) node[right]{$x$};
    \draw[->,thick] (0,0)--(0,3.5) node[right]{$y$};
    \draw[->,thick] (0,0)--(0,-3.5);
    %Draw marks on x axis
    \foreach \i in {1,2}
    \draw [very thin,gray](\i*pi,-0.1)--(\i*pi,0.1) node[below right] at (\i*pi,-0.1) {$\i\pi$};
    %Draw marks on y axis
    \foreach \yticks in {-3,-2,...,3}
    \draw [very thin,gray](-0.1,\yticks)--(0.1,\yticks) node[left] at (0,\yticks) {$\yticks$};
    %Draw asymptotes
    \draw [thin,dashed, blue](0.75*pi,-3.5)--(0.75*pi,3.5) node [below right] at (0.75*pi,0) {$\frac{3\pi}{4}$};
    \draw [thin,dashed, blue](1.75*pi,-3.5)--(1.75*pi,3.5) node [below right] at (1.75*pi,0) {$\frac{7\pi}{4}$};
    %Draw the curve
    \draw[red]plot[domain=0:0.64*pi,samples=30](\x,{cot((\x+pi/4)  r)});
    \draw[red]plot[domain=0.85*pi:1.64*pi,samples=30](\x,{cot((\x+pi/4) r)});
    \draw[red]plot[domain=1.85*pi:2*pi,samples=30] (\x,{cot((\x+pi/4) r)});
\end{tikzpicture}

\solution
Not really, although there is a trigonometric identity where \[\tan x = \cot\left(\frac{\pi}{2} -x\right)\]

This equation has a negative $x$ inside. Therefore it won't be in form of $y= \cot(x+\phi)$ (note the positive x inside)

\end{solutions}

\begin{solutions}{Page 189}

\solution
\begin{subsolutions}
\subsolution
For $y = y_1+y_2$

\begin{tikzpicture}[xscale=1,yscale=0.5]
    \draw[help lines,xstep=pi/2,ystep=1,color=gray!50,dashed](0,-5.4) grid (2.1*pi,5.4);
    \draw[->,thick] (0,0)--(2.1*pi,0) node[right]{$x$};
    \draw[->,thick] (0,0)--(0,5.5) node[right]{$y$};
    \draw[->,thick] (0,0)--(0,-5.5);
    %Draw marks on x axis
    \foreach \i in {1,2}
    \draw [very thin,gray](\i*pi,-0.1)--(\i*pi,0.1) node[below right] at (\i*pi,-0.1) {$\i\pi$};
    %Draw marks on y axis
    \foreach \yticks in {-4,-3,...,4}
    \draw [very thin,gray](-0.1,\yticks)--(0.1,\yticks) node[left] at (0,\yticks) {$\yticks$};
    %Draw the curve
    \draw[red]plot[domain=0:2*pi,samples=90] (\x,{2*sin(\x r) + sin((\x-pi/4) r))});
\end{tikzpicture}

\subsolution
For $y = y_1+y_2$

\begin{tikzpicture}[xscale=1,yscale=0.5]
    \draw[help lines,xstep=pi/2,ystep=1,color=gray!50,dashed](0,-5.4) grid (2.1*pi,5.4);
    \draw[->,thick] (0,0)--(2.1*pi,0) node[right]{$x$};
    \draw[->,thick] (0,0)--(0,5.5) node[right]{$y$};
    \draw[->,thick] (0,0)--(0,-5.5);
    %Draw marks on x axis
    \foreach \i in {1,2}
    \draw [very thin,gray](\i*pi,-0.1)--(\i*pi,0.1) node[below right] at (\i*pi,-0.1) {$\i\pi$};
    %Draw marks on y axis
    \foreach \yticks in {-4,-3,...,4}
    \draw [very thin,gray](-0.1,\yticks)--(0.1,\yticks) node[left] at (0,\yticks) {$\yticks$};
    %Draw the curve
    \draw[red]plot[domain=0:2*pi,samples=90] (\x,{2*sin(\x r) + 3*sin((2*\x) r))});
\end{tikzpicture}

\subsolution
For $y = y_2+y_3$

\begin{tikzpicture}[xscale=1,yscale=0.5]
    \draw[help lines,xstep=pi/2,ystep=1,color=gray!50,dashed](0,-5.4) grid (2.1*pi,5.4);
    \draw[->,thick] (0,0)--(2.1*pi,0) node[right]{$x$};
    \draw[->,thick] (0,0)--(0,5.5) node[right]{$y$};
    \draw[->,thick] (0,0)--(0,-5.5);
    %Draw marks on x axis
    \foreach \i in {1,2}
    \draw [very thin,gray](\i*pi,-0.1)--(\i*pi,0.1) node[below right] at (\i*pi,-0.1) {$\i\pi$};
    %Draw marks on y axis
    \foreach \yticks in {-4,-3,...,4}
    \draw [very thin,gray](-0.1,\yticks)--(0.1,\yticks) node[left] at (0,\yticks) {$\yticks$};
    %Draw the curve
    \draw[red]plot[domain=0:2*pi,samples=90] (\x,{3*sin((2*\x) r)+3*sin(2*\x r)});
\end{tikzpicture}
\end{subsolutions}

\solution
Graph 1a appear to be a sinusoidal function
\end{solutions}

\begin{solutions}{Page 191}
\solution
For $y=2\sin x + 3 \cos x$, compare with standard form $y=A\sin kx + B \cos kx$, we get values of $A=2$, $B=3$, and $k=1$

To convert to the other standard form $y=a\sin k(x+\phi)$, we calculate its amplitude by
\begin{align*}
    a &= \sqrt{A^2+B^2}\\
      &= \sqrt{2^2 + 3^2}\\
      &= \sqrt{13}
\end{align*}
While its phase $\phi=\alpha/k$ is calculated by
\begin{align*}
    \cos \alpha &= \frac{A}{\sqrt{A^2+B^2}}\\
    \cos \alpha &= \frac{2}{\sqrt{13}}\\
    \alpha &= \arccos(2/\sqrt{13})
    \text{Since } & k=1 \\ 
    \phi &= \arccos(2/\sqrt{13})
\end{align*}
Therefore the equation is $y=\sqrt{2}\sin\left(x+\phi\right)$ where $\phi=\arccos{2\sqrt{13}}$

Since amplitude is $\sqrt{2}$, then its maximum value is $\sqrt{2}$

\solution
Since amplitude is $\sqrt{13}$, then its maximum value is $\sqrt{13}$

\solution
For $y=\sin x + \cos x$, compare with standard form $y=A\sin kx + B \cos kx$, we get values of $A=1$, $B=1$, and $k=1$

To convert to the other standard form $y=a\sin k(x+\phi)$, we calculate its amplitude $a$ by
\begin{align*}
    a &= \sqrt{A^2+B^2}\\
      &= \sqrt{1^2 + 1^2}\\
      &= \sqrt{2}
\end{align*}
While its phase $\phi=\alpha/k$ is calculated by
\begin{align*}
    \cos \alpha &= \frac{A}{\sqrt{A^2+B^2}}\\
    \cos \alpha &= \frac{1}{\sqrt{2}}\\
        \alpha &= \pi/4\\
    \text{Since } & k=1 \\ 
        \phi &= \pi/4\\
\end{align*}
Therefore the equation is $y=\sqrt{2}sin\left(x+\pi/4\right)$

Since amplitude is $\sqrt{2}$, then its maximum value is $\sqrt{2}$

\solution
For $y=\sin x - \cos x$, compare with standard form $y=A\sin kx + B \cos kx$, we get values of $A=1$, $B=1$, and $k=1$

To convert to the other standard form $y=a\sin k(x-\beta)$, we calculate its amplitude by
\begin{align*}
    a &= \sqrt{A^2+B^2}\\
      &= \sqrt{1^2 + 1^2}\\
      &= \sqrt{2}
\end{align*}
While its phase $\beta=\alpha/k$ is calculated by
\begin{align*}
    \cos \alpha &= \frac{A}{\sqrt{A^2+B^2}}\\
    \cos \alpha &= \frac{1}{\sqrt{2}}\\
        \alpha &= \pi/4\\
    \text{Since } & k=1 \\ 
        \beta &= \pi/4
\end{align*}
Therefore the equation is $y=\sqrt{2}\sin\left(x-\pi/4\right)$

Since amplitude is $\sqrt{2}$, then its maximum value is $\sqrt{2}$

\solution
For $y=4\sin x + 3\cos x$, compare with standard form $y=A\sin kx + B \cos kx$, we get values of $A=4$, $B=3$, and $k=1$

To convert to the other standard form $y=a\sin k(x+\phi)$, we calculate its amplitude by
\begin{align*}
    a &= \sqrt{A^2+B^2}\\
      &= \sqrt{4^2 + 3^2}\\
      &= \sqrt{25}\\
      &= 5
\end{align*}
While its phase $\phi=\alpha/k$ is calculated by
\begin{align*}
    \cos \alpha &= \frac{A}{\sqrt{A^2+B^2}}\\
    \cos \alpha &= \frac{4}{5}\\
    \alpha &= \arccos(\frac{4}{5})\\
    \text{Since } & k=1 \\ 
    \phi &= \arccos(\frac{4}{5})\\
\end{align*}
Therefore the equation is $y=5\sin\left(x+\phi\right)$ where $\phi=\arccos(4/5)$

Since amplitude is 5, then its maximum value is 5

\solution
For $y=\sin 2x + 3\cos 2x$, compare with standard form $y=A\sin kx + B \cos kx$, we get values of $A=1$, $B=3$, and $k=2$

To convert to the other standard form $y=a\sin k(x+\phi)$, we calculate its amplitude by
\begin{align*}
    a &= \sqrt{A^2+B^2}\\
      &= \sqrt{1^2 + 3^2}\\
      &= \sqrt{10}
\end{align*}
While its phase $\phi=\alpha/k$ is calculated by
\begin{align*}
    \cos \alpha &= \frac{A}{\sqrt{A^2+B^2}}\\
    \cos \alpha &= \frac{1}{\sqrt{10}}\\
    \alpha &= \arccos(1/\sqrt{10})\\
    \text{Since } & k=2 \\ 
    \phi &= \frac{1}{2}\arccos(\frac{1}{\sqrt{10}})\\
\end{align*}
Therefore the equation is $y=\sqrt{10}\sin\left(x+\phi\right)$ where $\phi = \frac{1}{2}\arccos(\frac{1}{\sqrt{10}})$

\solution
For $y=\sin(x-\pi/4)$, compare with standard form $y=a\sin k(x-\beta)$, we get values of $a=1$, $\beta=\pi/4$, and $k=1$

To convert to the other standard form $y=A\sin kx - B \cos kx$, we need to solve for $A$ and $B$ using the definition $A=a\cos k\beta$ and $B=a\sin k\beta$
\begin{align*}
    A &= 1 \cos (\pi/4)\\
      &= \frac{1}{\sqrt{2}}\\
    B &= 1 \sin (\pi/4)\\
      &= \frac{1}{\sqrt{2}}
\end{align*}
Therefore the equation is $y=\frac{1}{\sqrt{2}} \sin x - \frac{1}{\sqrt{2}i} \cos x $

\solution
For $y=4\sin(x+\pi/6)$, compare with standard form $y=a\sin k(x+\phi)$, we get values of $a=4$, $\phi=\pi/6$, and $k=2$

To convert to the other standard form $y=A\sin kx + B \cos kx$, we need to solve for $A$ and $B$ using the definition $A=a\cos k\beta$ and $B=a\sin k\beta$
\begin{align*}
    A &= 4 \cos (2\pi/6)\\
      &= 4 \times 1/2\\
      &= 2\\
    B &= 4 \sin (2\pi/6)\\
      &= 4 \times \sqrt{3}/2\\
      &= 2\sqrt{3}
\end{align*}
Therefore the equation is $y=  2\sin 2x + 2\sqrt{3} \cos 2x $

\end{solutions}

\begin{solutions}{Page 194}
\solution
\begin{subsolutions}

\subsolution
\begin{align*}
    y&= 2\sin(x+\pi/6) + \cos(x+\pi/6)\\
    y&= 2\left(\sin x \cos \pi/6 + \cos x \sin \pi/6 \right) + \left( \cos x \cos \pi/6 - \sin x \sin \pi/6 \right)\\
    y&= 2\left(\frac{\sqrt{3}}{2}\sin x+ \frac{1}{2}\cos x \right) + \left(\frac{\sqrt{3}}{2} \cos x -\frac{1}{2} \sin x\right)\\
    y&= \sqrt{3}\sin x + \cos x +\frac{\sqrt{3}}{2} \cos x -\frac{1}{2} \sin x\\
    y&= \left(\sqrt{3} -\frac{1}{2} \right)\sin x + \left(1 + \frac{\sqrt{3}}{2}\right) \cos x
\end{align*}

\subsolution
\begin{align*}
    y&= 2\sin2(x+\pi/4) - \cos2(x+\pi/4)\\
    y&= 2\sin(2x+\pi/2) - \cos(2x+\pi/2)\\
    y&= 2\left(\sin 2x \cos \pi/2 + \cos 2x \sin \pi/2 \right) + \left( \cos 2x \cos \pi/2 - \sin 2x \sin \pi/2 \right)\\
    y&= 2\left(\sin 2x \times 0 + \cos 2x \times 1\right) + \left( \cos 2x \times 0 - \sin 2x \times 1 \right)\\
    y&= 2\cos 2x -\sin 2x\\
    y&= -\sin 2x + 2\cos 2x
\end{align*}
\end{subsolutions}

\solution
\begin{subsolutions}
\subsolution
\begin{align*}
    y_1 + y_1 &= 2\sin x + \sin (x-\pi/4)\\
              &= 2\sin x + \left(\sin x  \cos \pi/4 - \cos x \sin \pi/4 \right)\\
              &= 2\sin x + \frac{1}{\sqrt{2}}\sin x - \frac{1}{\sqrt{2}}\cos x\\
              &= \left(2+ \frac{1}{\sqrt{2}} \right)\sin x - \frac{1}{\sqrt{2}} \cos x
\end{align*}

\subsolution
The functions have different frequencies, which would need advanced mathematics topics such as Fourier Series
\end{subsolutions}
\end{solutions}

\begin{solutions}{Page 196}
\solution
\[y=-x+\sin x \]
\begin{tikzpicture}[xscale=0.5,yscale=0.5]
    \draw[help lines,xstep=pi/2,ystep=1,color=gray!50,dashed](-2.1*pi,-5.4) grid (2.1*pi,5.4);
    \draw[->,thick] (-2.1*pi,0)--(2.1*pi,0) node[right]{$x$};
    \draw[->,thick] (0,0)--(0,5.5) node[right]{$y$};
    \draw[->,thick] (0,0)--(0,-5.5);
    %Draw marks on x axis
    \foreach \i in {1,2}
    \draw [very thin,gray](\i*pi,-0.1)--(\i*pi,0.1) node[below right] at (\i*pi,-0.1) {$\i\pi$};
    %Draw marks on y axis
    \foreach \yticks in {-5,-4,...,5}
    \draw [very thin,gray](-0.1,\yticks)--(0.1,\yticks) node[left] at (0,\yticks) {$\yticks$};
    %Draw the curve
    \draw[red]plot[domain=-2*pi:2*pi,samples=90] (\x,{-\x + sin(\x r))});
\end{tikzpicture}

\solution
\[y=x^2+\sin x \]
\begin{tikzpicture}[xscale=2,yscale=1]
    \draw[help lines,xstep=pi/4,ystep=0.5,color=gray!50,dashed](-pi/2,0) grid (pi/2,2.4);
    \draw[->,thick] (0,0)--(0.6*pi,0) node[right]{$x$};
    \draw[->,thick] (0,0)--(-0.6*pi,0);
    \draw[->,thick] (0,0)--(0,2.5) node[right]{$y$};
    %Draw marks on x axis
    \draw [very thin,gray](-pi/2,-0.1)--(-pi/2,0.1) node[below right] at (-pi/2,-0.1) {-$\frac{\pi}{2}$};
    \draw [very thin,gray](pi/2,-0.1)--(pi/2,0.1) node[below right] at (pi/2,-0.1) {$\frac{\pi}{2}$};
    %Draw marks on y axis
    \foreach \yticks in {1,2}
    \draw [very thin,gray](-0.1,\yticks)--(0.1,\yticks) node[left] at (0,\yticks) {$\yticks$};
    %Draw the curve
    \draw[red]plot[domain=-pi/2:pi/2,samples=90] (\x,{(\x)^2 + sin(\x r))});
\end{tikzpicture}

\solution
\[y=x^2+\cos x \]
Note this is an even function since $f(-x)=f(x)$

\begin{tikzpicture}[xscale=2,yscale=1]
    \draw[help lines,xstep=pi/4,ystep=0.5,color=gray!50,dashed](-pi/2,0) grid (pi/2,2.4);
    \draw[->,thick] (0,0)--(0.6*pi,0) node[right]{$x$};
    \draw[->,thick] (0,0)--(-0.6*pi,0);
    \draw[->,thick] (0,0)--(0,2.5) node[right]{$y$};
    %Draw marks on x axis
    \draw [very thin,gray](-pi/2,-0.1)--(-pi/2,0.1) node[below right] at (-pi/2,-0.1) {-$\frac{\pi}{2}$};
    \draw [very thin,gray](pi/2,-0.1)--(pi/2,0.1) node[below right] at (pi/2,-0.1) {$\frac{\pi}{2}$};
    %Draw marks on y axis
    \foreach \yticks in {1,2}
    \draw [very thin,gray](-0.1,\yticks)--(0.1,\yticks) node[left] at (0,\yticks) {$\yticks$};
    %Draw the curve
    \draw[red]plot[domain=-pi/2:pi/2,samples=90] (\x,{(\x)^2 + cos((\x) r))});
\end{tikzpicture}

\solution
\[y=x^3+\sin x \]
\begin{tikzpicture}[xscale=2,yscale=0.5]
    \draw[help lines,xstep=pi/4,ystep=1,color=gray!50,dashed](-pi/2,-5.4) grid (pi/2,5.4);
    \draw[->,thick] (0,0)--(0.55*pi,0) node[right]{$x$};
    \draw[->,thick] (0,0)--(-0.55*pi,0);
    \draw[->,thick] (0,0)--(0,5.5) node[right]{$y$};
    \draw[->,thick] (0,0)--(0,-5.5);
    %Draw marks on x axis
    \draw [very thin,gray](-pi/2,-0.1)--(-pi/2,0.1) node[below right] at (-pi/2,-0.1) {-$\frac{\pi}{2}$};
    \draw [very thin,gray](pi/2,-0.1)--(pi/2,0.1) node[below right] at (pi/2,-0.1) {$\frac{\pi}{2}$};
    %Draw marks on y axis
    \foreach \yticks in {-5,-4,...,5}
    \draw [very thin,gray](-0.1,\yticks)--(0.1,\yticks) node[left] at (0,\yticks) {$\yticks$};
    %Draw the curve
    \draw[red]plot[domain=-pi/2:pi/2,samples=90] (\x,{(\x)^3 + sin(\x r))});
\end{tikzpicture}

\solution
\[y=x^2+\frac{1}{10}\sin x \]
\begin{tikzpicture}[xscale=2,yscale=1]
    \draw[help lines,xstep=pi/2,ystep=0.5,color=gray!50,dashed](-pi/2,0) grid (pi/2,2.4);
    \draw[->,thick] (0,0)--(0.6*pi,0) node[right]{$x$};
    \draw[->,thick] (0,0)--(-0.6*pi,0);
    \draw[->,thick] (0,0)--(0,2.5) node[right]{$y$};
    %Draw marks on x axis
    \draw [very thin,gray](-pi/2,-0.1)--(-pi/2,0.1) node[below right] at (-pi/2,-0.1) {-$\frac{\pi}{2}$};
    \draw [very thin,gray](pi/2,-0.1)--(pi/2,0.1) node[below right] at (pi/2,-0.1) {$\frac{\pi}{2}$};
    %Draw marks on y axis
    \foreach \yticks in {1,2}
    \draw [very thin,gray](-0.1,\yticks)--(0.1,\yticks) node[left] at (0,\yticks) {$\yticks$};
    %Draw the curve
    \draw[red]plot[domain=-pi/2:pi/2,samples=90] (\x,{(\x)^2 + 0.1*sin((10*\x) r))});
\end{tikzpicture}

\solution
\[y=\cos x + \frac{1}{10}\sin 20x \]
\begin{tikzpicture}[xscale=0.5,yscale=0.5]
    \draw[help lines,xstep=pi/2,ystep=1,color=gray!50,dashed](-2.1*pi,-2.4) grid (2.1*pi,2.4);
    \draw[->,thick] (0,0)--(2.1*pi,0) node[right]{$x$};
    \draw[->,thick] (0,0)--(-2.1*pi,0);
    \draw[->,thick] (0,0)--(0,2.4) node[right]{$y$};
    \draw[->,thick] (0,0)--(0,-2.4);
    %Draw marks on x axis
    \foreach \i in {-2,-1,1,2}
    \draw [very thin,gray](\i*pi,-0.1)--(\i*pi,0.1) node[below right] at (\i*pi,-0.1) {$\i\pi$};
    %Draw marks on y axis
    \foreach \yticks in {-2,-1,1,2}
    \draw [very thin,gray](-0.1,\yticks)--(0.1,\yticks) node[left] at (0,\yticks) {$\yticks$};
    %Draw the curve
    \draw[red]plot[domain=-2*pi:2*pi,samples=180] (\x,{cos (\x r) + 0.1*sin((20*\x) r))});
\end{tikzpicture}

\solution
\[y=2\sin x + \frac{1}{10}\sin 20x \]
\begin{tikzpicture}[xscale=0.5,yscale=0.5]
    \draw[help lines,xstep=pi/2,ystep=1,color=gray!50,dashed](-2.1*pi,-2.4) grid (2.1*pi,2.4);
    \draw[->,thick] (0,0)--(2.1*pi,0) node[right]{$x$};
    \draw[->,thick] (0,0)--(-2.1*pi,0);
    \draw[->,thick] (0,0)--(0,2.5) node[right]{$y$};
    \draw[->,thick] (0,0)--(0,-2.5);
    %Draw marks on x axis
    \foreach \i in {-2,-1,1,2}
    \draw [very thin,gray](\i*pi,-0.1)--(\i*pi,0.1) node[below right] at (\i*pi,-0.1) {$\i\pi$};
    %Draw marks on y axis
    \foreach \yticks in {-2,-1,1,2}
    \draw [very thin,gray](-0.1,\yticks)--(0.1,\yticks) node[left] at (0,\yticks) {$\yticks$};
    %Draw the curve
    \draw[red]plot[domain=-2*pi:2*pi,samples=90] (\x,{2*sin(\x r) + 0.1*sin((20*\x) r))});
\end{tikzpicture}
\end{solutions}

\begin{solutions}{Page 197}
\solution
In $y=\sin 2x + \sin 3x$. The period of $\sin 2x$ has form of $m(2\pi/2)$, while the period of $\sin 3x$ has form of $n(2\pi/3)$.

To be a period of both parts
\begin{align*}
    m\times \frac{2\pi}{2} &= n\times \frac{2\pi}{3}\\
                         3m  &= 2n
\end{align*}

Using $m=2$ and $n=3$. the period of each part will be $2\pi$ radians

\solution
In $y=\sin 3x + \sin 6x$. The period of $\sin 3x$ has form of $m(2\pi/3)$, while the period of $\sin 6x$ has form of $n(2\pi/6)$.

To be a period of both parts
\begin{align*}
    m\times \frac{2\pi}{3} &= n\times \frac{2\pi}{6}\\
                         6m  &= 3n\\
                         2m &= 1n
\end{align*}

Using $m=1$ and $n=2$. the period of each part will be $2\pi/3$ radians

\solution
In $y=\sin 4x + \sin 6x$. The period of $\sin 4x$ has form of $m(2\pi/4)$, while the period of $\sin 6x$ has form of $n(2\pi/6)$.

To be a period of both parts
\begin{align*}
    m\times \frac{2\pi}{4} &= n\times \frac{2\pi}{6}\\
                         6m  &= 4n\\
                         3m &= 2n
\end{align*}

Using $m=2$ and $n=3$. the period of each part will be $\pi$ radians

\solution
\solution
In $y=\sin \sqrt{2}x + \sin 3\sqrt{2}x$. The period of $\sin \sqrt{2}x$ has form of $m(2\pi/\sqrt{2})$, while the period of $\sin 3\sqrt{2}x$ has form of $n(2\pi/3\sqrt{2})$.

To be a period of both parts
\begin{align*}
    m\times \frac{2\pi}{\sqrt{2}} &= n\times \frac{2\pi}{3\sqrt{2}}\\
                         3m &= 1n
\end{align*}

Using $m=1$ and $n=3$. the period of each part will be $2\pi/\sqrt{2}= \sqrt{2}\pi$ radians
\end{solutions}

\begin{solutions}{Page 199}
\solution
\[y=\sin x -\frac{1}{2}\sin 2x + \frac{1}{3} \sin 3x\]
\begin{tikzpicture}[xscale=0.5,yscale=1]
    \draw[help lines,xstep=pi/2,ystep=1,color=gray!50,dashed](-2.1*pi,-1.4) grid (2.1*pi,1.4);
    \draw[->,thick] (0,0)--(2.1*pi,0) node[right]{$x$};
    \draw[->,thick] (0,0)--(-2.1*pi,0);
    \draw[->,thick] (0,0)--(0,1.5) node[right]{$y$};
    \draw[->,thick] (0,0)--(0,-1.5);
    %Draw marks on x axis
    \foreach \i in {-2,-1,1,2}
    \draw [very thin,gray](\i*pi,-0.1)--(\i*pi,0.1) node[below right] at (\i*pi,-0.1) {$\i\pi$};
    %Draw marks on y axis
    \foreach \yticks in {-1,1}
    \draw [very thin,gray](-0.1,\yticks)--(0.1,\yticks) node[left] at (0,\yticks) {$\yticks$};
    %Draw the curve
    \draw[red]plot[domain=-2*pi:2*pi,samples=90] (\x,{sin(\x r) - 0.5*sin((2*\x) r) +1/3*sin((3*\x) r))});
\end{tikzpicture}

\solution
\[y=\sin x -\frac{1}{2}\sin 2x + \frac{1}{3} \sin 3x - \frac{1}{4} \sin 4x + \frac{1}{5}\sin 5x
\]
\begin{tikzpicture}[xscale=0.5,yscale=1]
    \draw[help lines,xstep=pi/2,ystep=1,color=gray!50,dashed](-2.1*pi,-1.4) grid (2.1*pi,1.4);
    \draw[->,thick] (0,0)--(2.1*pi,0) node[right]{$x$};
    \draw[->,thick] (0,0)--(-2.1*pi,0);
    \draw[->,thick] (0,0)--(0,1.5) node[right]{$y$};
    \draw[->,thick] (0,0)--(0,-1.5);
    %Draw marks on x axis
    \foreach \i in {-2,-1,1,2}
    \draw [very thin,gray](\i*pi,-0.1)--(\i*pi,0.1) node[below right] at (\i*pi,-0.1) {$\i\pi$};
    %Draw marks on y axis
    \foreach \yticks in {-1,1}
    \draw [very thin,gray](-0.1,\yticks)--(0.1,\yticks) node[left] at (0,\yticks) {$\yticks$};
    %Draw the curve
    \draw[red]plot[domain=-2*pi:2*pi,samples=90] (\x,{sin(\x r) - 0.5*sin((2*\x) r) +1/3*sin((3*\x) r)) - 1/4*sin((4*\x) r) + 1/5*sin((5*\x) r)});
\end{tikzpicture}

\solution
The limit will be a sawtooth wave

\begin{tikzpicture}[xscale=0.5,yscale=1]
    \draw[help lines,xstep=pi/2,ystep=1,color=gray!50,dashed](-2.1*pi,-1.4) grid (2.1*pi,1.4);
    \draw[->,thick] (0,0)--(2.1*pi,0) node[right]{$x$};
    \draw[->,thick] (0,0)--(-2.1*pi,0);
    \draw[->,thick] (0,0)--(0,1.5) node[right]{$y$};
    \draw[->,thick] (0,0)--(0,-1.5);
    %Draw marks on x axis
    \foreach \i in {-2,-1,1,2}
    \draw [very thin,gray](\i*pi,-0.1)--(\i*pi,0.1) node[below right] at (\i*pi,-0.1) {$\i\pi$};
    %Draw marks on y axis
    \foreach \yticks in {-1,1}
    \draw [very thin,gray](-0.1,\yticks)--(0.1,\yticks) node[left] at (0,\yticks) {$\yticks$};
    %Draw the curve
    \draw[red]plot[domain=-2*pi:2*pi,samples=90] (\x,{2*Mod((\x-pi/2)/pi,1)-1});
\end{tikzpicture}

\solution
\[y=\cos x -\frac{1}{9}\cos 3x + \frac{1}{25} \cos 5x - \frac{1}{49} \cos 7x\] 

\begin{tikzpicture}[xscale=0.5,yscale=1]
    \draw[help lines,xstep=pi/2,ystep=1,color=gray!50,dashed](-2.1*pi,-1.4) grid (2.1*pi,1.4);
    \draw[->,thick] (0,0)--(2.1*pi,0) node[right]{$t$};
    \draw[->,thick] (0,0)--(-2.1*pi,0);
    \draw[->,thick] (0,0)--(0,1.5) node[right]{$y$};
    \draw[->,thick] (0,0)--(0,-1.5);
    %Draw marks on x axis
    \foreach \i in {-2,-1,1,2}
    \draw [very thin,gray](\i*pi,-0.1)--(\i*pi,0.1) node[below right] at (\i*pi,-0.1) {$\i\pi$};
    %Draw marks on y axis
    \foreach \yticks in {-1,1}
    \draw [very thin,gray](-0.1,\yticks)--(0.1,\yticks) node[left] at (0,\yticks) {$\yticks$};
    %Draw the curve
    \draw[red]plot[domain=-2*pi:2*pi,samples=90] (\x,{cos (\x r)+ 1/9*cos((3*\x) r)+ 1/25*cos((5*\x) r)+ 1/49*cos((7*\x) r)});
\end{tikzpicture}

Therefore, the limiting function is the triangular function

\begin{tikzpicture}[xscale=0.5,yscale=1]
    \draw[help lines,xstep=pi/2,ystep=1,color=gray!50,dashed](-2.1*pi,-1.4) grid (2.1*pi,1.4);
    \draw[->,thick] (0,0)--(2.1*pi,0) node[right]{$x$};
    \draw[->,thick] (0,0)--(-2.1*pi,0);
    \draw[->,thick] (0,0)--(0,1.5) node[right]{$y$};
    \draw[->,thick] (0,0)--(0,-1.5);
    %Draw marks on x axis
    \foreach \i in {-2,-1,1,2}
    \draw [very thin,gray](\i*pi,-0.1)--(\i*pi,0.1) node[below right] at (\i*pi,-0.1) {$\i\pi$};
    %Draw marks on y axis
    \foreach \yticks in {-1,1}
    \draw [very thin,gray](-0.1,\yticks)--(0.1,\yticks) node[left] at (0,\yticks) {$\yticks$};
    %Draw the curve
    \draw[red]plot[domain=-2*pi:0,samples=90] (\x,{abs((\x+pi)/(pi/2))-1});
    \draw[red]plot[domain=0:2*pi,samples=90] (\x,{abs((\x-pi)/(pi/2))-1});
\end{tikzpicture}

\end{solutions}

\begin{solutions}{Page 200}
\solution
The graph will be 

\begin{tikzpicture}[xscale=1,yscale=1]
    \draw[help lines,xstep=1,ystep=1,color=gray!50,dashed](0,0) grid (2.1*pi,1.4);
    \draw[->,thick] (0,0)--(6.5,0) node[right]{$t$};
    \draw[->,thick] (0,0)--(0,1.5) node[right]{$y$};
    %Draw marks on x axis
    \foreach \i in {1,2,3,4,5,6}
    \draw [very thin,gray](\i,-0.1)--(\i,0.1) node[below] at (\i,-0.1) {$\i$};
    %Draw marks on y axis
    \foreach \yticks in {1}
    \draw [very thin,gray](-0.1,\yticks)--(0.1,\yticks) node[left] at (0,\yticks) {$\yticks$};
    %Draw the curve
    \draw[red]plot[domain=0:2,samples=90] (\x,{-abs(\x-1)+1});
    \draw[red]plot[domain=2:4,samples=90] (\x,{-abs(\x-3)+1});
    \draw[red]plot[domain=4:6,samples=90] (\x,{-abs(\x-5)+1});
\end{tikzpicture}

\solution
The graph will be 

\begin{tikzpicture}[xscale=1,yscale=1]
    \draw[help lines,xstep=1,ystep=1,color=gray!50,dashed](0,-1.4) grid (2.1*pi,1.4);
    \draw[->,thick] (0,0)--(6,0) node[right]{$t$};
    \draw[->,thick] (0,0)--(0,1.5) node[right]{$y$};
    \draw[->,thick] (0,0)--(0,-1.5);
    %Draw marks on x axis
    \foreach \i in {1,2,3,4,5,6}
    \draw [very thin,gray](\i,-0.1)--(\i,0.1) node[below] at (\i,-0.1) {$\i$};
    %Draw marks on y axis
    \foreach \yticks in {-1,1}
    \draw [very thin,gray](-0.1,\yticks)--(0.1,\yticks) node[left] at (0,\yticks) {$\yticks$};
    %Draw the curve
    \draw[red]plot[domain=0:3,samples=90] (\x,{-abs(\x-1)+1});
    \draw[red]plot[domain=3:6,samples=90] (\x,{-abs(\x-5)+1});
\end{tikzpicture}

\solution
The graph will be 

\begin{tikzpicture}[xscale=0.25,yscale=0.25]
    \draw[help lines,xstep=8,ystep=8,color=gray!50,dashed](0,0) grid (32,8);
    \draw[->,thick] (0,0)--(32.5,0) node[right]{$t$};
    \draw[->,thick] (0,0)--(0,8.5) node[right]{$y$};
    %Draw marks on x axis
    \foreach \i in {8,16,24,32}
    \draw [very thin,gray](\i,-0.1)--(\i,0.1) node[below] at (\i,-0.1) {$\i$};
    %Draw marks on y axis
    \draw [very thin,gray](-0.1,8)--(0.1,8) node[left] at (0,8) {$8$};
    %Draw the curve
    \draw[red]plot[domain=0:8,samples=90] (\x,\x);
    \draw[red]plot[domain=8:16,samples=90] (\x,8);
    \draw[red]plot[domain=16:24,samples=90] (\x,-\x+24);
    \draw[red]plot[domain=24:32,samples=90] (\x,0);
\end{tikzpicture}

\solution
The graph will be 

\begin{tikzpicture}[xscale=0.25,yscale=0.125]
    \draw[help lines,xstep=6,ystep=10,color=gray!50,dashed](0,0) grid (24,40);
    \draw[->,thick] (0,0)--(24.5,0) node[right]{$t$};
    \draw[->,thick] (0,0)--(0,40.5) node[right]{$y$};
    %Draw marks on x axis
    \foreach \i in {6,12,18,24}
    \draw [very thin,gray](\i,-0.1)--(\i,0.1) node[below] at (\i,-0.1) {$\i$};
    %Draw marks on y axis
    \foreach \yticks in {4,40}
    \draw [very thin,gray](-0.1,\yticks)--(0.1,\yticks) node[left] at (0,\yticks) {$\yticks$};
    %Draw the curve
    \draw[red](0,4)--(1,7)--(2,13)--(3,22)--(4,31)--(5,37)--(6,40);
    \draw[red](6,40)--(7,37)--(8,31)--(9,22)--(10,13)--(11,7)--(12,4);
    \draw[red](12,4)--(13,7)--(14,13)--(15,22)--(16,31)--(17,37)--(18,40);
    \draw[red](18,40)--(19,37)--(20,31)--(21,22)--(22,13)--(23,7)--(24,4);
\end{tikzpicture}

Tide is rising fastest at 3rd, 4th, 15th, and 16th hour 

Tide is receding fastest at 9th, 10th, 21st, and 22nd hour 

Tide is rising slowest at 1st, 5th, 13th and 17th hour 

Tide is receding slowest at 7th, 11th, 19th and 23rd hour 

\end{solutions}

\begin{solutions}{Page 203}
\solution
Substituing $x$ values
\begin{itemize}
    \item $\sin \pi=0$
    \item $\sin 2\pi=0$
    \item $\sin 3\pi=0$
    \item $\sin 4\pi=0$
\end{itemize}

\solution
For $y=\sin 4\pi x$, the value of $k$ is $k=4\pi$, therefore the period is $\frac{2\pi}{4\pi}$, so the period is $\frac{1}{2}$

\solution
Since period is $\frac{2\pi}{k}$, if period is 3
\begin{align*}
    3 &= \frac{2\pi}{k}\\
    k &= \frac{2\pi}{3}
\end{align*}
Therefore the equation is $y=\sin \frac{2\pi}{3}x$

\solution
Since period is $\frac{2\pi}{k}$, if period is 2
\begin{align*}
    2 &= \frac{2\pi}{k}\\
    k &= \frac{2\pi}{2}
\end{align*}
Therefore the equation is $y=\sin \pi x$

\solution
Since period is $\frac{2\pi}{k}$, if period is $n$
\begin{align*}
    n &= \frac{2\pi}{k}\\
    k &= \frac{2\pi}{n}
\end{align*}
Therefore the equation is $y=\sin \frac{2\pi}{n} x$
\end{solutions}

\begin{solutions}{Page 205}
\solution
The period is estimated to be 12 months by distance between peaks at $t=8$ and $t=20$. Therefore the value for $k$ is $k=2\pi/20$

The amplitude is estimated to be mid-way between peak (15) and trough (9.5), which is $(15-9.5)/2 =2.75 $

Therefore the equation is $y=2.75 \sin \frac{\pi}{10}(x-\pi)$

\solution
If (a) is in northern hemisphere, (b) and (c) would be in northern hemisphere too as the peaks occur at similar month. (d) would be in southern hemispehere as the wave is shifted by half-period

\solution
For all graphs, the 'average' number of daylight hours occur at month 4, month 10, month 16, and month 22
\end{solutions}

\section*{Chapter 9: Inverse Functions and Trigonometric Equations}

\begin{solutions}{Page 213}
\solution
\begin{subsolutions}
    \subsolution $\arcsin 0.5 = \pi/6$
    \subsolution $\arccos 0.5 = \pi/3$
    \subsolution $\arctan 0.5 = \pi/4$
    \subsolution $\arcsin (-\sqrt{3}/2) = -\pi/3$
    \subsolution $\arccos (-\sqrt{3}/2) = 5\pi/6$
    \subsolution $\arctan (-\sqrt{3}) = -\pi/3$
    \subsolution Note: not valid as the domain of $\arcsin x$ is $-1 \leq x \leq 1$
\end{subsolutions}
\solution
\begin{subsolutions}
    \subsolution 
    \begin{align*}
        &\sin(\arcsin 0.5)\\
        &=\sin(\pi/6) \\
        &=-0.5
    \end{align*}
    \subsolution 
    \begin{align*}
        &\cos(\arccos 0.5)\\
        &=\cos(\pi/3)\\
        &= 0.5
    \end{align*}
    \subsolution 
    \begin{align*}
        &\tan(\arctan (-1))\\
        &=\tan(\pi/4)\\
        &= -\pi/4
    \end{align*}
    \subsolution 
    \begin{align*}
        &\arcsin(\sin (\pi/3))\\
        &=\arcsin(-\sqrt{3}/2)\\
        &= \pi/3
    \end{align*}
    \subsolution 
    \begin{align*}
        &\arccos (\cos 11\pi/6) \\
        &=\arccos(-\sqrt{3}/2)\\
        &= \pi/6
    \end{align*}
\end{subsolutions}
\solution
Let $\theta = \arccos b$ which mean $\cos \theta = b$. If we assume angle $\theta$ is acute, then we can draw a right angle triangle

    \begin{tikzpicture}[scale=1,font=\sffamily]
    \coordinate (C) at (0,0);
    \coordinate (A) at (-2,0);
    \coordinate (B) at (0,1.5);
    \draw (C) -- node[right] {$\sqrt{1-b^2}$} (B) -- node[above] {$1$} (A) -- node[below] {$b$} (C);
    \node (theta) at ([shift=({0.4,0.15})]A) {$\theta$};
    \draw (C) rectangle (-0.2,0.2);
    \end{tikzpicture}
    
    Giving $\sin \theta = \sqrt{1-b^2}$ when $0 \leq \theta \leq \pi/2$

For $\pi/2 \leq \theta \pi$, $\sin \theta = \sin (\pi - \theta)$, making the sign positive

For $\pi \leq \theta 3\pi/2$, $\sin \theta = -\sin (\theta-\pi)$, making the sign negative

For $3\pi/2 \leq \theta 2\pi$, $\sin \theta = -\sin (2\theta-\theta)$, making the sign negative

\solution
Let $\theta = \arcsin b$ which mean $\sin \theta = b$. If we assume angle $\theta$ is acute, then we can draw a right angle triangle

    \begin{tikzpicture}[scale=1,font=\sffamily]
    \coordinate (C) at (0,0);
    \coordinate (A) at (-2,0);
    \coordinate (B) at (0,1.5);
    \draw (C) -- node[right] {$b$} (B) -- node[above] {$1$} (A) -- node[below] {$\sqrt{1-b^2}$} (C);
    \node (theta) at ([shift=({0.4,0.15})]A) {$\theta$};
    \draw (C) rectangle (-0.2,0.2);
    \end{tikzpicture}
    
    Giving $\tan \theta = \frac{b}{\sqrt{1-b^2}}$ when $0 \leq \theta \leq \pi/2$

For $\pi/2 \leq \theta \pi$, $\tan \theta = -\tan (\pi - \theta)$, making the sign negative

For $\pi \leq \theta 3\pi/2$, $\tan \theta = \tan (\theta-\pi)$, making the sign positive

For $3\pi/2 \leq \theta 2\pi$, $\tan \theta = -\tan (2\theta-\theta)$, making the sign negative

\solution
Let $\theta = \arctan b$ which mean $\tan \theta = b$. If we assume angle $\theta$ is acute, then we can draw a right angle triangle

    \begin{tikzpicture}[scale=1,font=\sffamily]
    \coordinate (C) at (0,0);
    \coordinate (A) at (-2,0);
    \coordinate (B) at (0,1.5);
    \draw (C) -- node[right] {$b$} (B) -- node[above left] {$\sqrt{1+b^2}$} (A) -- node[below] {$1$} (C);
    \node (theta) at ([shift=({0.4,0.15})]A) {$\theta$};
    \draw (C) rectangle (-0.2,0.2);
    \end{tikzpicture}
    
    Giving $\cos \theta = \frac{1}{\sqrt{1+b^2}}$ when $0 \leq \theta \leq \pi/2$

For $\pi/2 \leq \theta \pi$, $\cos \theta = -\cos (\pi - \theta)$, making the sign negative

For $\pi \leq \theta 3\pi/2$, $\cos \theta = -\cos (\theta-\pi)$, making the sign negative

For $3\pi/2 \leq \theta 2\pi$, $\cos \theta = \cos (2\theta-\theta)$, making the sign positive

\solution
Since angle $\alpha$ is acute, then we can draw a right angle triangle 

    \begin{tikzpicture}[scale=1,font=\sffamily]
    \coordinate (C) at (0,0);
    \coordinate (A) at (-2,0);
    \coordinate (B) at (0,1.5);
    \draw (C) -- node[right] {$a$} (B) -- node[above] {$c$} (A) -- node[below] {$b$} (C);
    \node (alpha) at ([shift=({0.4,0.15})]A) {$\alpha$};
    \node (beta) at ([shift=({-0.4,-0.5})]B) {$\beta$};
    \draw (C) rectangle (-0.2,0.2);
    \end{tikzpicture}

Note that $\sin \alpha = a/b$ and $\cos \beta =a/b$, and $\beta = \pi/2-\alpha$
\begin{align*}
    \sin \alpha &= \cos \beta\\
    \arccos (\sin \alpha) &= \beta\\
    \arccos (\sin \alpha) &= \pi/2 - \alpha
\end{align*}

For $\pi/2 \leq \alpha \leq  \pi$, the expression $\sin\alpha$ becomes $\sin (\pi - \alpha)$, therefore making 
\begin{align*}
    \arccos(\sin \alpha) &= \arccos(\sin (\pi-\alpha))\\
                         &= \pi/2 - (\pi/2-\alpha)\\
                         &= \alpha
\end{align*}

For $\pi \leq \alpha \leq  3\pi/2$, the expression $\sin\alpha$ becomes $-\sin (\alpha - \pi)$, therefore making 
\begin{align*}
    \arccos(\sin \alpha) &= \arccos(-\sin (\alpha-\pi))\\
                         &= \arccos(\sin (-\alpha+\pi))\\
                         &= \pi/2 - (-\alpha + \pi)\\
                         &= \alpha-\pi/2
\end{align*}

For $3\pi/2 \leq \alpha \leq  2\pi$, the expression $\sin\alpha$ becomes $-\sin (2\pi - \alpha)$, therefore making 
\begin{align*}
    \arccos(\sin \alpha) &= \arccos(-\sin (2\pi-\alpha))\\
                        &= \arccos(\sin (-2\pi+\alpha))\\
                        &= \arccos(\sin (\alpha))\\
                         &= \pi/2 - \alpha
\end{align*}

\solution
Note that the angles for parts (a) to (e) are less than $\pi/2$, so no change in angle is needed, unlike (f)
\begin{subsolutions}
    \subsolution $\arcsin (\sin \pi/11) = \pi/11$
    \subsolution $\arcsin (\sin 2\pi/11) = 2\pi/11$
    \subsolution $\arcsin (\sin 3\pi/11) = 3\pi/11$
    \subsolution $\arcsin (\sin 4\pi/11) = 4\pi/11$
    \subsolution $\arcsin (\sin 5\pi/11) = 5\pi/11$
    \subsolution 
    \begin{align*}
        &\arcsin (\sin 6\pi/11) \\
        &= \arcsin (\sin (\pi-6\pi/11))\\
        &= \arcsin (\sin 5\pi/11)\\
        &= 5\pi/11
    \end{align*}
\end{subsolutions}
\solution
Graph of $y=\cos(\arccos x)$

\begin{tikzpicture}[xscale=2,yscale=2]
    \draw[help lines,xstep=1,ystep=1,color=gray!50,dashed](-1.5,-1.5) grid (1.5,1.5);
    \draw[->,thick] (-1.5,0)--(1.5,0) node[right]{$x$};
    \draw[->,thick] (0,-1.5)--(0,1.5) node[right]{$y$};
    %Draw marks on x axis
    \foreach \i in {-1,1}
    \draw [very thin,gray](\i,-0.1)--(\i,0.1) node[below] at (\i,-0.1) {$\i$};
    %Draw marks on y axis
    \foreach \yticks in {1}
    \draw [very thin,gray](-0.1,\yticks)--(0.1,\yticks) node[left] at (0,\yticks) {$\yticks$};
    %Draw the curve
    \draw[red] (-1,-1) -- (1,1);
\end{tikzpicture}
\solution
Graph of $y=\arccos(\cos x)$

\begin{tikzpicture}[xscale=1,yscale=1]
    \draw[help lines,xstep=pi,ystep=pi,color=gray!50,dashed](-2*pi,-pi) grid (2*pi,pi);
    \draw[->,thick] (-2*pi,0)--(2*pi,0) node[right]{$x$};
    \draw[->,thick] (0,-pi)--(0,pi) node[right]{$y$};
    %Draw marks on x axis
    \foreach \i in {-2,-1,1,2}
    \draw [very thin,gray](\i*pi,-0.1)--(\i*pi,0.1) node[below] at (\i*pi,-0.1) {$\i \pi$};
    %Draw marks on y axis
    \foreach \yticks in {-1,1}
    \draw [very thin,gray](-0.1,\yticks)--(0.1,\yticks) node[left] at (0,\yticks) {$\yticks \pi$};
    %Draw the curve
    \draw[red] (-2*pi,0)--(-pi,-pi)--(0,0)--(pi, pi)--(2*pi,0); 
\end{tikzpicture}
\solution
Letting $\alpha=\arcsin 3/5$ and $\beta=\arcsin 5/13$. Then drawing the triangles

    \begin{tikzpicture}[scale=1,font=\sffamily]
    \coordinate (C) at (0,0);
    \coordinate (A) at (-2,0);
    \coordinate (B) at (0,1.5);
    \draw (C) -- node[right] {$3$} (B) -- node[above] {$5$} (A) -- node[below] {$4$} (C);
    \node (alpha) at ([shift=({0.4,0.15})]A) {$\alpha$};
    \draw (C) rectangle (-0.2,0.2);
    \end{tikzpicture}

    \begin{tikzpicture}[scale=1,font=\sffamily]
    \coordinate (C) at (0,0);
    \coordinate (A) at (-2,0);
    \coordinate (B) at (0,1.5);
    \draw (C) -- node[right] {$5$} (B) -- node[above] {$13$} (A) -- node[below] {$12$} (C);
    \node (beta) at ([shift=({0.4,0.15})]A) {$\beta$};
    \draw (C) rectangle (-0.2,0.2);
    \end{tikzpicture}

\begin{align*}
    \sin (\alpha + \beta)&=\sin\alpha \cos\beta +\cos\alpha \sin\beta\\
    &=3/5 \times 12/13 + 4/5 \times  5/13\\
    &= 56/65
\end{align*}
\solution
Letting $\alpha = \arctan a$ and $\beta = \arctan b$. Therefore we have $\tan \alpha = a$ amd $\tan \beta = b$
\begin{align*}
    \tan(\alpha + \beta)&= \frac{\tan \alpha + \tan \beta}{1-\tan \alpha \tan \beta}\\
                        &= \frac{a+b}{1-ab}
\end{align*}

\solution
Let $\tan \alpha = 1/3$ and $\tan \beta = 1/2$ and $\tan \gamma = 1$
\begin{align*}
    \tan(\alpha + \beta)&= \frac{\tan \alpha + \tan \beta}{1-\tan \alpha \tan \beta}\\
              \tan(\alpha + \beta)&= \frac{1/3+1/2}{1-1/3 \times 1/2}\\
              \tan(\alpha + \beta)&= 1
          \end{align*}
But Since both $\tan (\alpha + \beta)=1$ and $\tan \gamma =1$. Therefore $\alpha + \beta = \gamma$

Note: I think the author meant Question 11 instead of 'Problem 8'
\solution
Using plane geometry, draw another row of squares underneath the diagram and draw lines $AH$ and $HC$.

Note that the lengths of $AH$ and $HC$ are the same, making triangle $AHC$ an isosceles triangle, therefore $\angle CAH = \angle ACH$

Since triangle $AHG$ and $HEC$ is congruent with triangle $BCD$, $\angle BAH = \beta$, and $\angle HCE = \beta$

Since triangle $FCA$ is congruent with triangle $ACD$, $\angle FCA = \alpha$

Therefore 
\begin{align*}
    \angle FCA + \angle ACH + \angle HCE &= 90^{\circ}\\
    \alpha + \angle ACH + \beta = 90^{\circ}\\
    \angle ACH = 90^{\circ} -\alpha -\beta
\end{align*}

Since $\angle \gamma = 45^{\circ }$, $\angle CAH = \alpha + \beta$ and $\angle CAH = \angle ACH$

\begin{align*}
    \alpha + \beta &= 90^{\circ} -\alpha -\beta\\
    2(\alpha + \beta) &= 90^{\circ}\\
    \alpha + \beta &= 45^{\circ}\\
    \alpha + \beta &= \gamma
\end{align*}

    \begin{tikzpicture}[scale=1,font=\sffamily]
    \draw[help lines,xstep=2,ystep=2,color=gray!50,dashed](0,-2) grid (6,2);
    \coordinate (A) at (0,0);
    \coordinate (B) at (2,0);
    \coordinate (C) at (6,2);
    \coordinate (G) at (4,0);
    \coordinate (E) at (6,-2);
    \coordinate (F) at (0,2);
    \coordinate (H) at (4,-2);
    \draw (A)--(C)--(G)--cycle;
    \draw (B)--(C);
    \draw (A)--(H)--(C);
    %\draw (C) -- node[right] {$b$} (B) -- node[above] {$\sqrt{1+b^2}$} (A) -- node[below] {$1$} (C);
    \node (A) at ([shift=({1,0.15})]A) {$\alpha$};
    \node (B) at ([shift=({0.8,0.15})]B) {$\beta$};
    \node (G) at ([shift=({0.5,0.15})]G) {$\gamma$};
    \node[below] at (0,0) {$A$};
    \node[below] at (2,0) {$B$};
    \node[right] at (6,2) {$C$};
    \node[right] at (6,0) {$D$};
    \node[right] at (0,2) {$E$};
    \node[below] at (4,0) {$G$};
    \node[below] at (4,-2) {$H$};
    %\draw (C) rectangle (-0.2,0.2);
    \end{tikzpicture}


Note: I think the author meant Question 12 instead of 'Problem 9'
\end{solutions}

\begin{solutions}{Page 220}
\solution
From the graph, $\sin x > 1/2$ when $\pi/6 \leq x \leq 5\pi/6$ and $13\pi/6 \leq x \leq 17\pi/6$
\solution
Using the general equation $\pi k + (-1)^k (\alpha)$ for any integer $k$

For $\sin x = -1/2$, the value for $\alpha$ is $\arcsin(-1/2) = -\pi/6$, so the solution is $\pi k + (-1)^k (-\pi/6)$
\solution
Using the general equation $2n\pi \pm (\alpha)$ for any integer $n$

For $\cos x = \sqrt{2}/2$, the value for $\alpha$ is $\arccos(\sqrt{2}/2) = \pi/4$, so the solution is $2\pi n \pm \pi/4$
\solution
Using the general equation $n\pi + (\alpha)$ for any integer $n$

For $\tan x = 1$, the value for $\alpha$ is $\arctan(1) = \pi/4$, so the solution is $\pi n + \pi/4$
\solution
Using the general equation $\pi k + (-1)^k (\alpha)$ for any integer $k$

For $\sin x = -1$, the value for $\alpha$ is $\arcsin(-1) = -\pi/2$, so the solution is $\pi k + (-1)^k (-\pi/2)$
\end{solutions}

\begin{solutions}{Page 221}
\solution
Using the general equation $\pi k + (-1)^k (\alpha)$ for any integer $k$

For $\sin x = \sin \pi/5$, the value for $\alpha$ is $\pi/5$, so the solution is $\pi k + (-1)^k (\pi/5)$
\solution
Using the general equation $\pi k + (-1)^k (\alpha)$ for any integer $k$

For $\sin x = \sin \pi/2$, the value for $\alpha$ is $\pi/2$, so the solution is $\pi k + (-1)^k (\pi/2)$
\solution
\begin{tikzpicture}[xscale=0.5,yscale=1]
    \draw[help lines,xstep=pi,ystep=1,color=gray!50,dashed](0,-1.4) grid (6.1*pi,1.4);
    \draw[->,thick] (0,0)--(6.1*pi,0) node[right]{$x$};
    \draw[->,thick] (0,0)--(0,1.5) node[right]{$y$};
    \draw[->,thick] (0,0)--(0,-1.5);
    %Draw marks on x axis
    \foreach \i in {1,2,...,6}
    \draw [very thin,gray](\i*pi,-0.1)--(\i*pi,0.1) node[below right] at (\i*pi,-0.1) {$\i\pi$};
    %Draw marks on y axis
    \foreach \yticks in {-1,0,1}
    \draw [very thin,gray](-0.1,\yticks)--(0.1,\yticks) node[left] at (0,\yticks) {$\yticks$};
    %Draw the curve
    \draw[red]plot[domain=0:6*pi, samples=90] (\x,{cos(\x r)});
    \draw[blue] (6*pi,1/2)--(0,1/2) node[left] at (0,1/2){$\cos \alpha$};
    \draw[blue] (pi/3,1/2)--(pi/3,0) node[below] at (pi/3,0){$\alpha$};
    \draw[blue] (5*pi/3,1/2)--(5*pi/3,0) node[below] at (5*pi/3,0){$2\pi - \alpha$};
\end{tikzpicture}

Since period of $\cos x$ is $2\pi$, the solutions are $2\pi n+ \alpha$ and $2\pi n- \alpha$ for any integer n
\solution
Using the general equation $2n\pi \pm (\alpha)$ for any integer $n$

For $\cos x = \cos \pi/5$, the value for $\alpha$ is $\pi/5$, so the solution is $2\pi n \pm \pi/5$

\solution
Since period of $\tan x$ is $\pi$, then if $\alpha$ is a solution, then the general equation for solutions are $\alpha + \pi n$ for any integer n

\begin{tikzpicture}[xscale=1,yscale=0.5]
    \draw[help lines,xstep=pi/2,ystep=1,color=gray!50,dashed](0,-3.4) grid (2.1*pi,3.4);
    \draw[->,thick] (0,0)--(2.1*pi,0) node[right]{$x$};
    \draw[->,thick] (0,0)--(0,3.5) node[right]{$y$};
    \draw[->,thick] (0,0)--(0,-3.5);
%    %Draw marks on x axis
    \foreach \i in {1,2}
    \draw [very thin,gray](\i*pi,-0.1)--(\i*pi,0.1) node[below right] at (\i*pi,-0.1) {$\i\pi$};
%    %Draw marks on y axis
    \foreach \yticks in {-3,-2,...,3}
    \draw [very thin,gray](-0.1,\yticks)--(0.1,\yticks) node[left] at (0,\yticks) {$\yticks$};
    %Draw asymptotes
    \draw [thin,dashed, blue](0.5*pi,-3.5)--(0.5*pi,3.5) node [below right] at (0.5*pi,0) {$\frac{\pi}{2}$};
    \draw [thin,dashed, blue](1.5*pi,-3.5)--(1.5*pi,3.5) node [below right] at (1.5*pi,0) {$\frac{3\pi}{2}$};
    %Draw the curve
    \draw[red]plot[domain=0:0.4*pi,samples=30](\x,{tan(\x r)});
    \draw[red]plot[domain=0.60*pi:1.4*pi,samples=30](\x,{tan(\x r)});
    \draw[red]plot[domain=1.60*pi:2*pi,samples=30] (\x,{tan(\x r)});
    \draw[blue] (0.25*pi,1)--(0.25*pi,0) node[below]at(0.25*pi,0) {$\alpha$};
    \draw[blue] (1.25*pi,1)--(1.25*pi,0) node[below]at(1.25*pi,0) {$\alpha+\pi$};
\end{tikzpicture}

\solution
Using the general equation $n\pi + (\alpha)$ for any integer $n$

For $\tan x = \tan \pi/5$, the value for $\alpha$ is $\pi/5$, so the solution is $\pi n + \pi/5$

\solution
Using the general equation $\pi k + (-1)^k (\alpha)$ for any integer $k$
\begin{align*}
    \sin x &= -\sin \alpha\\
    \sin x &= \sin (-\alpha)\\
    x &= \pi k + (-1)^k (-\alpha)
\end{align*}
\solution
Using the general equation $2n\pi \pm (\alpha)$ for any integer $n$
\begin{align*}
    \cos x &= -\cos \alpha\\
    \cos x &= \cos (\pi-\alpha)\\
    x &= 2n\pi \pm (\pi-\alpha)
\end{align*}
\solution
Using the general equation $\pi k + (-1)^k (\alpha)$ for any integer $k$
\begin{align*}
    \sin x &= \sin \alpha\\
    \sin x &= \sin (\alpha)\\
    x &= \pi k + (-1)^k (\alpha)
\end{align*}
\end{solutions}

\begin{solutions}{Page 225}
\solution
Using the general equation $\pi k + (-1)^k (\alpha)$ for any integer $k$
\begin{align*}
    \sin 2x &= 1\\
    2x &= \pi k + (-1)^k \times \arcsin 1\\
    2x &= \pi k + (-1)^k \times \pi/2\\
    x &= \frac{1}{2}\left(\pi k + (-1)^k \times \pi/2\right)
\end{align*}
\solution
Using the general equation $\pi k + (-1)^k (\alpha)$ for any integer $k$
\begin{align*}
    \sin x/2 &= 1/2\\
    x/2 &= \pi k + (-1)^k \times \arcsin 1/2\\
    x/2 &= \pi k + (-1)^k \times \pi/6\\
    x &= 2\left(\pi k + (-1)^k \times \pi/6\right)
\end{align*}
\solution
\begin{align*}
    \cos x &= \sin 2x\\
    \cos x &= 2\sin x \cos x\\
    0&= 2\sin x \cos x- \cos x\\
    0&= \cos x\left(2\sin x - 1\right)\\
\end{align*}
Therefore there are two set of solutions. Using general solutions $2n\pi \pm (\alpha)$ 
\begin{align*}
    \cos x &= 0\\
    x &= 2n\pi \pm (\arccos 0)\\
    x &= 2n\pi \pm (\pi/2)\\
\end{align*}
and $\pi k + (-1)^k (\alpha)$
\begin{align*}
    2\sin x -1 &= 0\\
    \sin x &= 1/2\\
    x &= \pi k + (-1)^k (\arcsin 1/2)\\
    x &= \pi k + (-1)^k (\pi/6)\\
\end{align*}

Note: I think the question is supposed to be $\cos x = \sin^2 x$?

\solution
\begin{align*}
    \sin x &= \sin 3x\\
    \sin x &= \sin (2x+x)\\
    \sin x &= \sin 2x \cos x + \cos 2x \sin x\\
    \sin x &= 2\sin x\cos x \cos x + (\cos^2 x - \sin^2 x)\sin x\\
    \sin x &= 2\sin x(1-\sin^2 x)+ (1-\sin^2 x -\sin^2 x)\sin x\\
    \sin x &= 2\sin x-2\sin^3 x+ \sin x-2\sin^3 x\\
    0 &= -4\sin^3 x + 2\sin x\\
    0 &= -2\sin x (2\sin^2-1)
\end{align*}

Therefore there are multiple sets of solutions to solve and using the general equation $\pi k + (-1)^k (\alpha)$
\begin{align*}
    \sin x &= 0\\
    x &= \pi k + (-1)^k (\arcsin 0)\\
    x &= \pi k
\end{align*}

The other sets of solutions are 
\begin{align*}
    2\sin^2 x -1 &= 0\\
    \sin^2 x &= 1/2\\
    \sin x &= \pm 1/\sqrt{2}\\
    \text{For } \sin x &= 1/\sqrt{2}\\
    x  &= \pi k + (-1)^k (\arcsin 1/\sqrt{2})\\
    x &= \pi k + (-1)^k \times \pi/4\\
    \text{For } \sin x &= -1/\sqrt{2}\\
    x  &= \pi k + (-1)^k (\arcsin -1/\sqrt{2})\\
    x &= \pi k + (-1)^k \times -\pi/4\\
\end{align*}

Note: I think the questions is supposed to be $\sin x = \sin^3 x$?
\solution
\begin{align*}
    \cos x &= \sin 4x\\
    \cos x &= 2\sin 2x \cos 2x\\
    \cos x &= 2 \times 2\sin x \cos x \times (\cos^2 x - \sin^2 x)\\
    \cos x &= 4\sin x \cos x \times (1- \sin^2 x - \sin^2 x)\\
    0 &= 4\sin x \cos x \times (1- 2\sin^2 x) - \cos x\\
    0 &= \cos x \left(4\sin x (1- 2\sin^2 x) - 1\right)\\
    0 &= \cos x \left(-8 \sin ^3 x + 4\sin x - 1\right)
\end{align*}

Therefore there are multiple sets of solutions to solve and using the general equation $\pi k + (-1)^k (\alpha)$ and $2n\pi \pm (\alpha)$
\begin{align*}
    \cos x &= 0\\
    x &= 2n\pi \pm \arccos 0\\
    x &= 2n\pi \pm \pi/2
\end{align*}

Solving $-8\sin^3 x+ 4\sin x -1=0$ requires using cubic functions, which is advanced level and is beyond this book's level

Note: I think the question is supposed to be $\cos x = \sin 4x$?

\solution
\begin{align*}
    26\sin^2 x + \cos^2 x &= 10\\
    26\sin^2 x + 1- \sin^2 x -10 =0\\
    25\sin^2 x -9 = 0\\
    (5\sin x +3)(5\sin x-3)=0
\end{align*}

Therefore there are multiple sets of solutions to solve and using the general equation $\pi k + (-1)^k (\alpha)$

For $5\sin x+3 = 0$
\begin{align*}
    5\sin x +3 &= 0\\
    \sin x &= -3/5\\
    x &= \pi k + (-1)^k (\arcsin -3/5)
\end{align*}

For $5\sin x-3 = 0$
\begin{align*}
    5\sin x -3 &= 0\\
    \sin x &= 3/5\\
    x &= \pi k + (-1)^k (\arcsin 3/5)
\end{align*}
\solution
\begin{align*}
    \cos^2 x - \cos x &= \sin^2 x \\
    \cos^2 x - \cos x &= 1-\cos^2 x\\
    2\cos^2 x -\cos x -1 &=0\\
    (2\cos x +1)(\cos x -1) &=0
\end{align*}

Therefore there are multiple sets of solutions to solve and using the general equation $2\pi n \pm (\alpha)$

For $2\cos x +1=0$
\begin{align*}
    2\cos x +1&=0\\
    \cos x &= -1/2\\
    x &= 2\pi n \pm \arccos(-1/2)\\
    x &= 2\pi n \pm 2\pi/3\\
\end{align*}

For $2\cos x -1=0$
\begin{align*}
    2\cos x -1&=0\\
    \cos x &= 1/2\\
    x &= 2\pi n \pm \arccos(1/2)\\
    x &= 2\pi n \pm \pi/3\\
\end{align*}
\solution
\begin{align*}
    3\tan^2 x &= 12\\
    \tan^2 x &= 4\\
    \tan^2 x -4&= 0\\
    (\tan x +2)(\tan x -2)= 0
\end{align*}

Therefore there are multiple sets of solutions to solve and using the general equation $\pi k + (\alpha)$

For $\tan x +2 =0$
\begin{align*}
    \tan x +2 &= 0\\
    \tan x &= -2\\
     x &= \pi k + \arctan -2
\end{align*}

For $\tan x -2 =0$
\begin{align*}
    \tan x -2 &= 0\\
    \tan x &= 2\\
     x &= \pi k + \arctan 2
\end{align*}
\solution
\begin{align*}
    \cos 2x &= 2\sin^2 x\\
    \cos^2 x - \sin^2 x &= 2\sin^2 x\\
    1-\sin^2 x -\sin^2 x &= 2\sin^2 x\\
    1 - 4\sin^2 x &=0\\
    (1-2\sin x)(1+2\sin x) &=0
\end{align*}

Therefore there are multiple sets of solutions to solve and using the general equation $\pi k + (-1)^k (\alpha)$

For $1-2\sin x =0$
\begin{align*}
    1-2\sin x &=0\\
    \sin x &= 1/2\\
    x &= \pi k + (-1)^k \arcsin 1/2\\
    x &= \pi k + (-1)^k \pi/6
\end{align*}

For $1+2\sin x =0$
\begin{align*}
    1+2\sin x &=0\\
    \sin x &= -1/2\\
    x &= \pi k + (-1)^k \arcsin -1/2\\
    x &= \pi k + (-1)^k (-\pi/6)
\end{align*}
\solution
\begin{align*}
    \tan^2 x &= \cot x\\
    \tan^3 x &= 1\\
    \tan^3 x -1 &= 0\\
    (\tan x -1)(\tan^2 x + \tan x +1)&= 0
\end{align*}
Therefore there are multiple sets of solutions to solve and using the general equation $\pi k + \alpha$

For $\tan x -1=0$
\begin{align*}
    \tan x - 1 &= 0\\
    \tan x =& 1\\
    x &= \pi k + \arctan 1\\
    x &= \pi k + \pi/4\\
\end{align*}

For $\tan^2 x + \tan x +1 =0$, apply quadratic formula
\begin{align*}
    \tan x &= \frac{-1 \pm \sqrt{1^2 - 4(1)(1)}}{2(1)}\\
    \tan x &= \frac{-1 \pm \sqrt{-3}}{2}
\end{align*}
Which does not have any solutiosn as the number inside the square root is negative
\solution
\begin{align*}
    \frac{5}{\cos^2 x} &= 7 \tan x +3\\
    5\sec^2 x &= 7 \tan x +3\\
    5(\tan^2 x + 1) &= 7 \tan x +3\\
    5\tan^2 x + 5 &= 7 \tan x +3\\
    5\tan^2 x -7\tan x + 2 &= 0
\end{align*}
Apply quadratic formula
\begin{align*}
    \tan x &= \frac{7\pm \sqrt{49-4(5)(2)}}{2(5)}\\
    \tan x &= \frac{7\pm 3}{10}\\
\end{align*}
Therefore there are multiple sets of solutions to solve and using the general equation $\pi k + \alpha$

For $\tan x = \frac{7+3}{10}$
\begin{align*}
    \tan x &= 1\\
    x &= \pi k + \arctan 1\\
    x &= \pi k + \pi/4
\end{align*}
For $\tan x = \frac{7-3}{10}$
\begin{align*}
    \tan x &= 2/5\\
    x &= \pi k + \arctan 2/5
\end{align*}
\solution
\begin{align*}
    \sqrt{3}\tan^2 x + 1 &= (1+\sqrt{3})\tan x\\
    (\sqrt{3}\tan x +1)(\tan x +1) &= 0
\end{align*}
Therefore there are multiple sets of solutions to solve and using the general equation $\pi k + \alpha$

For $\sqrt{3}\tan x+1 =0$
\begin{align*}
    \sqrt{3}\tan x + 1 &= 0\\
    \sqrt{3} \tan x &= -1\\
    \tan x &= -1/\sqrt{3}\\
    x &= \pi n + \arctan -1/\sqrt{3}\\
    x &= \pi n + (-\pi/6)
\end{align*}

For $\tan x + 1 = 0$
\begin{align*}
    \tan x + 1 &= 0\\
    \tan x &= -1\\
    x &= \pi n + \arctan -1\\
    x &= \pi n + (-\pi/4)\\
\end{align*}
\solution

For Solution 1, re-express the equations such that
\begin{align*}
    \text{with n}\\
    x&= \pi/6 + 2\pi n/3\\
    \text{with n+1}\\
     &= \pi/6 +2\pi(n+1)/3\\
     &= \pi/6 + 2\pi n/3 + 2\pi/3\\
     &= 5\pi/6 + 2\pi n/3\\
     \text{with n+2}\\
     &= \pi/6 +2\pi(n+2)/3\\
     &= \pi/6 + 2\pi n/3 + 4\pi/3\\
     &= 9\pi/6 + 2\pi n/3\\
     &= 3\pi/2 + 2\pi n/3
\end{align*}
\end{solutions}
Which corresponds to each line of Solution 2, provided that the number n for Solution 1 is multiples of 3
\begin{solutions}{Page 227}
\solution
If $\alpha = 1.6$ then
\begin{align*}
    x &= \alpha - \tan \alpha\\
      &\approx 35.8325
\end{align*}
Since near $x=\pi/2$, the tangent will be very flat and only reaches the x-axis after travelling a long distance

\solution
If $\alpha = \pi/4$ then
\begin{align*}
    x &= \alpha - \tan \alpha\\
      &\approx -0.2146
\end{align*}
Since the tangent is intersecting x axis to the left

Note, the smaller the $\alpha$, the closer to it intersecting x-axis at $x=0$

If $\alpha = 0.1$ then
\begin{align*}
    x &= \alpha - \tan \alpha\\
      &\approx -0.000334
\end{align*}
\solution
To solve for location of R where $x=0$ and noting $x \approx \tan x$ for small $x$ value
\begin{align*}
    x &= \alpha -\tan \alpha\\
    0 &= \alpha -\tan \alpha\\
    \tan \alpha &= \alpha \\
    \alpha &= 0
\end{align*}
Therefore R will be at origin when we take a tangent at $x=0$
\solution
if $x=\pi/2$, then the tangent will be horizontal, therefore not intersecting with x-axis at all

If x is a bit smaller than $\pi/2$, then the tangent will intersect with x-axis to the left (far away)

If x is a bit bigger than $\pi/2$, then the tangent will intersect with x-axis to the right (far away)
\end{solutions}

\begin{solutions}{Page 229}
\solution
Using $A \approx \frac{h}{\sin h/2}$, where $h=\pi/n$

Using $n=4$, so $h=\pi/4$ gives $A \approx \frac{\pi/4}{\sin \pi/8}$, which is 2.05234

Using $n=8$, so $h=\pi/8$ gives $A \approx \frac{\pi/8}{\sin \pi/16}$, which is 2.01291
\solution
Since area between $0 \leq x \leq \pi$ is approximately 2, threfore the area under curve $y=\sin x$ from $x=0$ to $x=\pi/2$ would be around 1
\solution
\begin{align*}
    &h\sin x_1 + h\sin x_2 + h\sin x_3 +\cdots + h\sin x_m  \\
    &=h\left( \sin \frac{\pi}{n} + \sin \frac{2\pi}{n}+ \cdots + \sin \frac{n\pi}{2\pi} \right)\\
    &= h \frac{\sin \frac{n+1}{2} \frac{\pi}{2n} \sin \frac{n}{2}\frac{\pi}{2n}}{\sin \frac{\pi}{2n}}\\
    &= h \frac{\sin \frac{n+1}{2}\frac{\pi}{2n}}{\sin \frac{\pi}{2n}} \sin \frac{\pi}{4}\\
    &= h \frac{\sin \frac{n+1}{n}\frac{\pi}{4}}{\sin \frac{\pi}{2n}} \frac{1}{\sqrt{2}} \\
    \text{if n is very large}\\
    &\approx h\frac{\sin \pi/4}{\sin \pi/2n}\frac{1}{\sqrt{2}} \\
    &\approx \frac{h}{2\sin h/2}
\end{align*}
letting $n=8$, giving $h = \pi/8$, gives approximate area of $\frac{\pi/8}{2\sin \pi/16} \approx 1.00645$ 

\end{solutions}

\end{document}
