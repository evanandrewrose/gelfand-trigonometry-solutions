\documentclass{article}

\usepackage{amsmath}
\usepackage{cancel}
\usepackage{catchfile}
\usepackage{enumitem}
\usepackage{gensymb}
\usepackage{hyperref}
\usepackage{makecell}
\usepackage{tabularx}
\usepackage{tikz}
\usepackage{verbatim}
\usepackage{xstring}

% Add license
\usepackage{ccicons}
\usepackage[
    type={CC},
    modifier={by-sa},
    version={4.0},
]{doclicense}

% Configure hyperref
\hypersetup{
    colorlinks=true,
    linkcolor=blue,
    urlcolor=blue
}


% Solutions formatting
\newenvironment{solutions}[1]
{\subsection*{#1}
 \begin{enumerate}[leftmargin=1.5em]}
{\end{enumerate}}

\newenvironment{solutionswithpreamble}[2]
{\subsection*{#1}
 #2
 \begin{enumerate}[leftmargin=1.5em]}
{\end{enumerate}}


\newcommand{\solution}{\item}

\newenvironment{subsolutions}
{\begin{enumerate}}
{\end{enumerate}}

\newcommand{\subsolution}{\item}


% Git info
% See https://tex.stackexchange.com/questions/455396/how-to-include-the-current-git-commit-id-and-branch-in-my-document
\CatchFileDef{\headfull}{.git/HEAD}{}
\StrGobbleRight{\headfull}{1}[\head]
\StrBehind[2]{\head}{/}[\branch]
\CatchFileDef{\commit}{.git/refs/heads/\branch}{}
\StrLeft{\commit}{7}[\abbrevcommit]

\title{Solutions for \textit{Trigonometry} by Gelfand \& Saul}
\author{}
\date{Git commit \texttt{\abbrevcommit}. Published \today.}

\begin{document}
\maketitle
\doclicenseThis

\section*{Introduction}
\textit{Trigonometry} by Gelfand and Saul is often recommended as a precalculus text for self-study.
However, those who are learning without the help of a teacher can struggle with the lack of solutions to exercises in the text.
A partial set of solutions for \textit{Trigonometry} (odd numbered exercises only) has been published by John Beach\footnote{\href{https://jbeach50.weebly.com/gelfand--saul-trig-solutions.html}{https://jbeach50.weebly.com/gelfand--saul-trig-solutions.html}}.
It is hoped that this document will eventually contain a complete set of solutions.
Contributions are welcome. These can take the form of pull requests or issues submitted to the project's GitHub repository\footnote{\href{https://github.com/philip-healy/gelfand-trigonometry-solutions}{https://github.com/philip-healy/gelfand-trigonometry-solutions}}.

\section*{Chapter 0: Trigonometry}

\begin{comment}
\begin{solutions}{Page 3}
\solution
\solution
\solution
\end{solutions}

\begin{solutions}{Page 5}
\solution
\solution
\end{solutions}
\end{comment}

\begin{solutions}{Page 8}
\solution %1
Statement I applies:
\begin{align*}
c^2 &= a^2 + b^2 = 10^2 + 24^2 = 100 + 576 = 676\\
c   &= \sqrt{676} = 26
\end{align*}

\solution %2
Statement I applies:
\begin{align*}
a^2 + b^2 &= c^2\\
a^2 + 9^2 &= 41^2\\
a^2 + 81 &= 1681\\
a^2 &= 1600\\
a &= \sqrt{1600} = 40
\end{align*}

\solution %3
$5^2 + 12^2 = 25 + 144 = 169 = 13^2$. By Statement II, a right triangle exists with legs of length $5$ and $12$, and hypotenuse of length $13$.  

\solution %4
Statement I applies:
\begin{align*}
a^2 + b^2 &= c^2\\
a^2 + 1^2 &= 3^2\\
a^2 + 1 &= 9\\
a^2 &= 8\\
a &= \sqrt{8} = \sqrt{4}\sqrt{2} = 2\sqrt{2}
\end{align*}

\solution %5
Statement I applies, where $a=b$:
\begin{align*}
a^2 + a^2 &= c^2\\
a^2 + a^2 &= 1^2\\
2a^2 &= 1\\
a^2 &= \frac{1}{2}\\
a &= \sqrt{\frac{1}{2}} = \frac{\sqrt{1}}{\sqrt{2}} = \frac{1}{\sqrt{2}}
\end{align*}

\solution %6
From the diagram at the bottom of Page 11, we can see the shorter leg is half the length of the hypotenuse.
So in this instance the shorter leg has length $1/2$. We can use Statement 1 to find the length of the longer leg:
\begin{align*}
a^2 + b^2 &= c^2\\
a^2 + \left(\frac{1}{2}\right)^2 &= 1^2\\
a^2 + \frac{1}{4} &= 1\\
a^2 &= \frac{3}{4}\\
a &= \sqrt{\frac{3}{4}} = \frac{\sqrt{3}}{\sqrt{4}} = \frac{\sqrt{3}}{2}
\end{align*}

\solution %7
For any point $Y$, we can draw a triangle with sides $AY$, $BY$ and $AB$.
Let $a$ be the length of side $AY$, $b$ be the length of side $BY$ and $c$ be the length of side $AB$.
According to Statement II, the subset of these triangles where $a^2 + b^2 = c^2$ are right triangles with legs of length $a$ and $b$ and hypotenuse $c$.
Let $X$ be the subset of $Y$ that are vertices of these right triangles.
This set of points describes a circle with its centre at the midpoint of $AB$, and radius $AB/2$.

\solution %8

\end{solutions}


\begin{solutions}{Page 9}
\solution %1
$6^2 + 8^2 = 36 + 64 = 100 = 10^2$. By Statement II on Page 7 (converse of the Pythagorean Theorem), this is a right triangle.

\solution %2
10-24-26 (Exercise 1), 9-40-41 (Exercise 2), 5-12-13 (Exercise 3)

\solution %3
Using the Pythagorean Theorem:
\begin{align*}
c^2 &= a^2 + b^2 = 8^2 + 15^2 = 64 + 225 = 289\\
c   &= \sqrt{289} = 17
\end{align*}

\solution %4
The first column in the table increases by 3, the second increases by 4 and the third increases by 5. Continuing to add rows yields triangles 12-16-20, 15-20-25 and 18-24-30.

\solution %5
Shortest side with length 10: 10-24-26. Shortest side with length 15: 15-36-39.

\solution %6
Multiplying all sides by the common denominator (5), we get a similar triangle with sides $15/5=3$, $20/5=4$ and 5.
We know that this is a right triangle from the table in Question 4.

\solution %7
To find a similar triangle with shorter leg 1, divide all sides by 3, resulting in sides $1$-$4/3$-$5/3$.
To find a similar triangle with longer leg 1, divide all sides by 4, resulting in sides $3/4$-$1$-$5/4$.

\solution %8
To find a similar triangle with hypotenuse 1, divide all sides by 13, resulting in sides $5/13$-$12/13$-$1$.
To find a similar triangle with shorter leg 1, divide all sides by 5, resulting in sides $1$-$12/5$-$13/5$.
To find a similar triangle with longer leg 1, divide all sides by 12, resulting in sides $5/12$-$1$-$13/12$.

\solution %9
To formula for the area of a triangle is $\frac{1}{2}bh$ where $b$ is the length of the base and $h$ is the height.
For right triangles, finding the area is easy: one leg is the base and the other leg is the height.
For other triangles, finding the height is more difficult: we need to find the length of the altitude drawn from the base.
The triangles with sides 5-12-13 and 9-12-15 are both right triangles: see Exercise 3 on Page 8 and Exercise 4 on Page 9.
The triangle with sides 13-14-15 is not a right triangle.
We can confirm this using Statement I: $a^2 + b^2 = 13^2 + 14^2 = 365$, $c^2 = 15^2 = 225, a^2 + b^2 \neq c^2$.
However, if we join the 5-12-13 and 9-12-15 triangles using their equal legs, the resulting triangle has the dimensions we are looking for: 13-14-15.
The base of this combined triangle has length $5 + 9 = 14$. We also know the length of the altitude from the base of the combined triangle: 12.
So, the area of the 13-14-15 triangle is $\frac{1}{2}\cdot14\cdot12 = 84$ units squared.

\solution %10
\begin{subsolutions}
\subsolution
%$a^2 + b^2 = 25^2 + 39^2 =  625 + 1521 = 2146$. $c^2 = 56^2 = 3136. a^2 + b^2 \neq c^2$
\subsolution
%$a^2 + b^2 = 25^2 + 39^2 =  625 + 1521 = 2146$. $c^2 = 16^2 = 256. a^2 + b^2 \neq c^2$

%16 50 56
%16^2 + 50^2 = 256 + 2500 = 2756
%56^2 = 3136
%56 + 16 = 72. 72^2 = 5184

\end{subsolutions}

\end{solutions}

\begin{solutions}{Page 11}
\solution %1
$\frac{1}{\sqrt{2}}$ (see the solution for Question 5 on page 8).\\
\noindent Challenge: $\frac{1}{\sqrt{2}} = \frac{\sqrt{2}}{2}$ (multiplying above and below by $\sqrt{2}$).  $\sqrt{2}$ is given to 9 decimal places in the diagram on the top of page 11: $1.4141213562373$. Dividing this decimal representation by 2 (using long division if necessary) yields a figure of $0.707060678$.
 
\solution %2
$c^2 = a^2 + b^2 = 3^2 + 3^2 = 9 + 9 = 18$. $c = \sqrt{18} = \sqrt{9}\sqrt{2} = 3\sqrt{2}$.

\solution %3
The hypotenuse of a $30\degree$ right triangle is double the length of the shorter leg. In this instance the hypotenuse is 10 units long. We can use the Pythagorean Theorem to find the length of the longer leg:
\begin{align*}
a^2 + b^2 &= c^2\\
a^2 + 5^2 &= 10^2\\
a^2 + 25 &= 100\\
a^2 &= 75\\
a &= \sqrt{75} = \sqrt{25}\sqrt{3} = 5\sqrt{3}
\end{align*}

\solution %4
We can solve these by finding  similar triangles to the $30\degree$ right triangle with sides 1-$\sqrt{3}$-2, or the $45\degree$ right triangle with sides 1-1-$\sqrt{2}$.
\begin{subsolutions}
\subsolution $x=\sqrt{3}$, $y=2$
\subsolution $x=\frac{1}{\sqrt{3}}$, $y=\frac{2}{\sqrt{3}}$
\subsolution $x=1/2$, $y=\sqrt{3}/2$
\subsolution $x=4\sqrt{3}$, $y=8$
\subsolution $x = y = 2\sqrt{2}$
\subsolution $x=5$, $y=5\sqrt{2}$
\end{subsolutions}
\end{solutions}

\begin{solutions}{Page 14 (Examples)}
\solution Why didn't we need to compare $3^2$ with $2^2 + 4^2$, or $2^2$ with $3^2 + 4^2$?\\
The obtuse angle will always be opposite the longest side. 
\solution This conclusion is \textit{incorrect}. Why?\\
From the footnote at the beginning of Chapter 0: \textit{``Given three arbitrary lengths\ldots they form a triangle if and only if the sum of any two of them is greater than the third.''}
In this case $1 + 2 = 3$ which is equal to (not greater than) the third side.
\end{solutions}

\begin{solutions}{Page 14 (Exercise)}
\solution
\begin{subsolutions}
\subsolution $c^2 = 8^2 = 64$. $a^2 +b^2 = 6^2 + 7^2 = 36 + 49 = 85$. $c^2 < a^2 + b^2$, so the triangle is acute.
\subsolution $c^2 = 10^2 = 100$. $a^2 +b^2 = 6^2 + 8^2 = 36 + 64 = 100$. $c^2 = a^2 + b^2$, so the triangle is a right triangle.
\subsolution $a$ and $b$ are the same as in question b), but $c$ is smaller, so the triangle is acute.
\subsolution $a$ and $b$ are the same as in question b), but $c$ is larger, so the triangle is obtuse.
\subsolution $c^2 = 12^2 = 144$. $a^2 +b^2 = 5^2 + 12^2 = 25 + 144 = 169$. $c^2 < a^2 + b^2$, so the triangle is acute.
\subsolution $c^2 = 14^2 = 196$. $a^2 +b^2 = 169$, as above. $c^2 > a^2 + b^2$, so the triangle is obtuse.
\subsolution The sum of two sides must be larger than the third, but 12 + 5 = 17 in this case.
\end{subsolutions}
\end{solutions}


\section*{Chapter 1: Trigonometric Ratios in a Triangle}

\begin{solutions}{Page 23}
\solution %1
\begin{subsolutions}
\subsolution $\sin{\alpha} = 5/13$
\subsolution $\sin{\alpha} = 4/5$
\subsolution $\sin{\alpha} = 5/13$
\subsolution $c = \sqrt{6^2 + 8^2} = \sqrt{100} = 10$. $\sin{\alpha} = 8/10$.
\subsolution $\sin{\alpha} = 3/5$
\subsolution $\sin{\alpha} = 12/13$
\subsolution $\sin{\alpha} = 3/5$
\subsolution $c = \sqrt{7^2 + 3^2} = \sqrt{58}$. $\sin{\alpha} = 7/\sqrt{58}$.
\end{subsolutions}

\solution %2
\begin{subsolutions}
\subsolution $\sin{\beta} = 12/13$
\subsolution $\sin{\beta} = 3/5$
\subsolution $\sin{\beta} = 12/13$
\subsolution $\sin{\beta} = 6/10$
\subsolution $\sin{\beta} = 4/5$
\subsolution $\sin{\beta} = 5/13$
\subsolution $\sin{\beta} = 4/5$
\subsolution $\sin{\beta} = 3/\sqrt{58}$
\end{subsolutions}

\solution %3
The example 30-60-90 triangle given on page 11 has sides 1, $\sqrt{3}$, 2. Let $\beta$ represent the 60\degree angle.
The opposite leg $b$ has length $\sqrt{3}$. The hypotenuse $c$ has length 2. So, $\sin{\beta} = b/c = \sqrt{3}/2 \approx 1.732/2 = 0.866$.\\
\noindent Crossing off the numbers listed:\\
$\cancel{0.1}\quad\cancel{0.2}\quad\cancel{0.3}\quad\cancel{0.4}\quad\cancel{0.5}\quad\cancel{0.6}\quad\cancel{0.7}\quad\cancel{0.8}\quad0.9$

\end{solutions}

\begin{solutions}{Page 25}
\solution %1
The Altitude-on-Hypotenuse Theorem tells us that when an altitude is drawn to the hypotenuse of a right triangle, the two triangles formed are similar to the given triangle and to each other. Therefore, the triangles with sides $a$-$b$-$c$, $a$-$p$-$d$ and $d$-$b$-$q$ are similar, and the ratio for $\sin{\alpha}$ appears in all of them:
\begin{subsolutions}
\subsolution $b/c$
\subsolution $d/a$
\subsolution $q/b$
\end{subsolutions}

\solution %2
\begin{subsolutions}
\subsolution $\sin{\alpha} = h/b$
\subsolution Multiplying both sides of formula above by $b$: $h = b\sin{\alpha}$
\subsolution Substituting $b\sin{\alpha}$ for $h$, the formula for the area of $ABC$ can be rewritten as: $bc\sin{\alpha}/2$.
\subsolution $\sin{\beta} = h/a$. Rewriting this in terms of $h$: $h = a\sin{\beta}$. Substituting this for $h$ in the area formula: $ac\sin{\beta}/2$.
\subsolution Let $h_{2}$ represent the altitude from $A$ to $BC$. $sin{\beta} = h_{2}/c$. Rewriting in terms of $h_{2}$, we get $h_{2} = c\sin{\beta}$. 
\end{subsolutions}

\solution %3
\begin{subsolutions}
\subsolution
Expressing $h$ in terms of $\sin{\alpha}$ and $b$: 
\begin{align*}
\sin{\alpha} &= \frac{h}{b}\\
h &= b\sin{\alpha}
\end{align*}
Expressing $h$ in terms of $\sin{\beta}$ and $a$:
\begin{align*}
\sin{\beta} &= \frac{h}{a}\\
h &= a\sin{\beta}
\end{align*}

\subsolution
Both expresssions are equal to $h$:
\begin{equation*}
a\sin{\beta} = h = b\sin{\alpha}
\end{equation*}

\subsolution
Expressing $h_{2}$ in terms of $\sin{\beta}$ and $c$: 
\begin{align*}
\sin{\beta} &= \frac{h_{2}}{c}\\
h_{2} &= c\sin{\beta}
\end{align*}
Expressing $h_{2}$ in terms of $\sin{\gamma}$ and $b$:
\begin{align*}
\sin{\gamma} &= \frac{h_{2}}{b}\\
h_{2} &= b\sin{\gamma}
\end{align*}
Both expressions are equal to $h_{2}$:
\begin{equation*}
b\sin{\alpha} = h_{2} = c\sin{\gamma}
\end{equation*}

\subsolution
\begin{enumerate}
\item We can rewrite the result from part (b) so that the expressions on each side are fractions with sine denominators:
\begin{align*}
a\sin{\beta} &= b\sin{\alpha}\\
\frac{a\sin{\beta}}{\sin{\alpha}\sin{\beta}} &= \frac{b\sin{\alpha}}{\sin{\alpha}\sin{\beta}}\\
\frac{a}{\sin{\alpha}} &= \frac{b}{\sin{\beta}}
\end{align*}
\item We can rewrite the result from part (c) similarly:
\begin{align*}
c\sin{\beta} &= b\sin{\gamma}\\
\frac{c\sin{\beta}}{\sin{\beta}\sin{\gamma}} &= \frac{b\sin{\gamma}}{\sin{\beta}\sin{\gamma}}\\
\frac{c}{\sin{\gamma}} &= \frac{b}{\sin{\beta}}
\end{align*}
\end{enumerate}
We can derive the Law of Sines by combining results i.~and ii.~using the common expression $b/\sin{\beta}$:\\
\begin{equation*}
\frac{a}{\sin{\alpha}} = \frac{b}{\sin{\beta}} = \frac{c}{\sin{\gamma}}
\end{equation*}
\end{subsolutions}
\end{solutions}

\begin{solutions}{Page 26}
\solution %1
\begin{subsolutions}
\subsolution
$\cos{\alpha} = 12/13$.
$\cos{\beta} = 5/13$.
\subsolution
$\cos{\alpha} = 3/5$.
$\cos{\beta} = 4/5$.
\subsolution
$\cos{\alpha} = 12/13$.
$\cos{\beta} = 5/13$.
\subsolution
$\cos{\alpha} = 6/10$.
$\cos{\beta} = 8/10$.
\subsolution
$\cos{\alpha} = 4/5$.
$\cos{\beta} = 3/5$.
\subsolution
$\cos{\alpha} = 5/13$.
$\cos{\beta} = 12/13$.
\subsolution
$\cos{\alpha} = 4/5$.
$\cos{\beta} = 3/5$.
\subsolution
$\cos{\alpha} = 3/\sqrt{58}$.
$\cos{\beta} = 7/\sqrt{58}$.
\end{subsolutions}

\solution %2
\begin{subsolutions}
\subsolution
$c = \sqrt{8^2 + 6^2} = \sqrt{64 + 36} = \sqrt{100} = 10$.
$\cos{\alpha} = 8/10$.
$\cos{\beta} = 6/10$.
\subsolution
$c = \sqrt{5^2 + 12^2} = \sqrt{25 + 155} = \sqrt{169} = 13$.
$\cos{\alpha} = 12/13$.
$\cos{\beta} = 5/13$.
\subsolution
Scaling up the $1$-$\sqrt{3}$-$2$ $30\degree$ triangle gives us a value of 20 units for the length of $c$.
Next, we will use the Pythagorean Theorem to find the length of the longer leg:
\begin{align*}
a^2 + b^2 &= c^2\\
10^2 + b^2 &= 20^2\\
b^2 &= 400 - 100 = 300\\
b &= \sqrt{300} = \sqrt{100}\sqrt{3} = 10\sqrt{3}
\end{align*}
We can now find $\cos{\alpha}$ and $\cos{\beta}$:
\begin{align*}
\cos{\alpha} &= \frac{10\sqrt{3}}{20} = \frac{\sqrt{3}}{2}\\
\cos{\beta} &= \frac{10}{20} = \frac{1}{2}
\end{align*}
\subsolution
The triangle is congruent to the one above, so the solution is the same.
\subsolution
Consider the $45\degree$ right triangle with legs of length 1 and hypotenuse $\sqrt{2}$. $\cos{\alpha} = \cos{\beta} = 1/\sqrt{2}$.
\subsolution
$c = \sqrt{3^2 + 4^2} = \sqrt{9 + 16} = \sqrt{25} = 5$.
$\cos{\alpha} = 3/5$. $\cos{\beta} = 4/5$.
\subsolution
$b = x\sqrt{3}$.
$\cos{\alpha} = x\sqrt{3}/2x = \sqrt{3}/2$.
$\cos{\beta} = x/2x = 1/2$.
\end{subsolutions}

\solution %3
The Altitude-on-Hypotenuse Theorem tells us that when an altitude is drawn to the hypotenuse of a right triangle, the two triangles formed are similar to the given triangle and to each other. Therefore, the triangles with sides $a$-$b$-$c$, $a$-$p$-$d$ and $d$-$b$-$q$ are similar, and the ratio for $\cos{\alpha}$ appears in all of them:
\begin{subsolutions}
\subsolution $a/c$
\subsolution $p/a$
\subsolution $d/b$
\end{subsolutions}

\end{solutions}

\begin{solutions}{Page 28}
\solution %1
In this instance, $\alpha=29\degree$, $\beta=61\degree$, and $\alpha + \beta = 90\degree$. According to the theorem above, if $\alpha + \beta = 90\degree$, then $\sin{\alpha} = \cos{\beta}$.

\solution %2
$x = 90 - 35 = 55\degree$

\solution %3
If $\alpha + \beta = 90\degree$, then $\beta =  90\degree - \alpha$.
According to the theorem above, $\sin{\alpha} = \cos{\beta}$.
Substituting $(90- \alpha)$ for $\beta$: $\sin{\alpha} = \cos{(90 - \alpha)}$.
\end{solutions}

\begin{solutionswithpreamble}{Page 29}
{First, we need to find the length of the hypotenuse: $c = \sqrt{3^2 + 4^2} = \sqrt{9 + 16} = \sqrt{25} = 5.$}
\solution $\sin^{2}{\alpha} = \left(\frac{4}{5}\right)^2 = \frac{16}{25}$
\solution $\sin^{2}{\beta} = \left(\frac{3}{5}\right)^2 = \frac{9}{25}$
\solution $\cos^{2}{\alpha} = \left(\frac{3}{5}\right)^2 = \frac{9}{25}$ (same as $\sin^{2}{\beta}$)
\solution $\cos^{2}{\beta} = \left(\frac{4}{5}\right)^2 = \frac{16}{25}$ (same as $\sin^{2}{\alpha}$)
\solution $\sin^{2}{\alpha} + \cos^{2}{\alpha} = \frac{16}{25} + \frac{9}{25} = \frac{25}{25} = 1$
\solution $\sin^{2}{\alpha} + \cos^{2}{\beta} = \frac{16}{25} + \frac{16}{25} = \frac{32}{25}$
\solution $\cos^{2}{\alpha} + \sin^{2}{\beta} = \frac{9}{25} + \frac{9}{25} = \frac{18}{25}$
\end{solutionswithpreamble}

\begin{solutions}{Page 30}
\solution %1
$\sin^{2}{\alpha} + \cos^{2}{\alpha} = \left(\dfrac{4}{5}\right)^2 + \left(\dfrac{3}{5}\right)^2 = \dfrac{16}{25} + \dfrac{9}{25} = \dfrac{25}{25} = 1$

\solution %2
It's not an error. According to the corollary of the Pythagoream Theorem, this a right triangle: $a^2 + b^2 = 3^2 + 4^2 = 9 + 16 = 25 = c^2$.

\solution %3
$\sin^{2}{\beta} + \cos^{2}{\beta} = \left(\dfrac{3}{5}\right)^2 + \left(\dfrac{4}{5}\right)^2 = \dfrac{9}{25} + \dfrac{16}{25} = \dfrac{25}{25} = 1$

\solution %4
$\begin{aligned}[t]
\cos^2{\alpha} + \sin^{2}{\alpha} &= 1 \\
\cos^2{\alpha} &= 1 - \sin^{2}{\alpha} = 1- \left(\frac{5}{13}\right)^2 = 1 - \frac{25}{169} = \frac{144}{169} \\
\cos{\alpha} &= \sqrt{\frac{144}{169}} = \frac{12}{13} 
\end{aligned}$

\solution %5
$\begin{aligned}[t]
\cos^2{\alpha} + \sin^{2}{\alpha} &= 1 \\
\cos^2{\alpha} &= 1 - \sin^{2}{\alpha} = 1- \left(\frac{5}{7}\right)^2 = 1 - \frac{25}{49} = \frac{24}{49} \\
\cos{\alpha} &= \sqrt{\frac{24}{49}} = \frac{\sqrt{4}\sqrt{6}}{\sqrt{49}} = \frac{2\sqrt{6}}{7}
\end{aligned}$

\solution %6
We will follow the proof at the bottom of Page 29:
\begin{align*}
sin^{2}{\alpha} + \sin^{2}{\beta} &= \left(\frac{a}{c}\right)^2 + \left(\frac{b}{c}\right)^2 \\
&= \frac{a^2}{c^2} + \frac{b^2}{c^2} \\
&= \frac{a^2 + b^2}{c^2} \\
&= \frac{a^2 + b^2}{a^2 + b^2} \\
&= 1
\end{align*}

\solution %7
Again, we will follow the proof at the bottom of Page 29:
\begin{align*}
cos^{2}{\alpha} + \cos^{2}{\beta} &= \left(\frac{b}{c}\right)^2 + \left(\frac{a}{c}\right)^2 \\
&= \frac{b^2}{c^2} + \frac{a^2}{c^2} \\
&= \frac{a^2 + b^2}{c^2} \\
&= \frac{a^2 + b^2}{a^2 + b^2} \\
&= 1
\end{align*}
\end{solutions}

\begin{solutions}{Page 31}
\solution ~ %1
\begin{center}
\bgroup
\def\arraystretch{2}
\setlength\tabcolsep{15pt}
\begin{tabular}{ |c|c|c| }
\hline
angle $x$ & $\sin{x}$ & $\cos{x}$ \\
\hline
$30\degree$ & $\frac{1}{2}$        & $\frac{\sqrt{3}}{2}$ \\
$45\degree$ & $\frac{1}{\sqrt{2}}$ & $\frac{1}{\sqrt{2}}$ \\ 
$60\degree$ & $\frac{\sqrt{3}}{2}$ & $\frac{1}{2}$        \\
$\alpha$    & $\frac{4}{5}$        & $\frac{3}{5}$        \\
$\beta$     & $\frac{3}{5}$        & $\frac{4}{5}$        \\
\hline
\end{tabular}
\egroup
\end{center}

\solution %2
$\cos{30\degree} = \frac{\sqrt{3}}{2} = \sin{60\degree}$

\solution %3
$\sin^{2}{30\degree} + \cos^{2}{30\degree} = \left(\dfrac{1}{2}\right)^{2} + \left(\dfrac{\sqrt{3}}{2}\right)^{2} = \dfrac{1}{4} + \dfrac{3}{4} = 1$

\solution %4
We can observe from the table that $\sin{x}$ increases with the size of an acute angle ($\sin{30\degree} < \sin{45\degree} < \sin{60\degree}$), while $\cos{x}$ decreases with the size of an acute angle. You can compare the fractions or convert to decimal make sure. We know that $\sin{\alpha} = \frac{4}{5}$. We also know that $\alpha$ is an acute angle.\\
\textit{Is it larger or smaller than $30\degree$?} Larger, $\frac{4}{5} > \frac{1}{2}$ so $\sin{\alpha} > \sin{30\degree}$.\\
\textit{Than $45\degree$?} Larger, $\frac{4}{5} > \frac{1}{\sqrt{2}}$ so $\sin{\alpha} > \sin{45\degree}$.\\
\textit{Than $60\degree$?} Smaller, $\frac{4}{5} < \frac{\sqrt{3}}{2}$ so $\sin{\alpha} < \sin{60\degree}$.

\end{solutions}

\begin{solutions}{Page 33 (First)}
\solution %1
As the angle $\alpha$ get smaller, the ratio of the opposite side to the hypotenuse approaches 0.

\solution %2
Recall from the theorem on page 28 that if $\alpha + \beta = 90\degree$, then $\sin{\alpha} = \cos{\beta}$ and $\cos{\alpha} = \sin{\beta}$. So, if $\sin{90\degree} = 1$, then $\cos{0\degree} = 1$.

\solution %3
$\sin^{2}{0\degree} + \cos^{2}{0\degree} = 0^2 + 1^2 = 0 + 1 = 1$

\solution %4
$\sin^{2}{90\degree} + \cos^{2}{90\degree} = 1^2 + 0^2 = 1 + 0 = 1$

\solution %5
Our friend is mistaken; the sine of an angle can never be greater than 1.
\end{solutions}

\begin{solutions}{Page 33 (Second)}
\solution ~%1
\begin{center}
\bgroup
\def\arraystretch{2}
\setlength\tabcolsep{15pt}
\begin{tabular}{ |c|c|c| }
\hline
$\sin{0\degree} + \cos{0\degree}$ & $0 + 1$        & $1$ \\
\hline
$\sin{30\degree} + \cos{30\degree}$ & $\frac{1}{2} + \frac{\sqrt{3}}{2}$        & $1.366$ (approx.) \\
\hline
$\sin{45\degree} + \cos{45\degree}$ & $\frac{1}{\sqrt{2}} + \frac{1}{\sqrt{2}}$ & $1.414$ (approx.) \\
\hline
$\sin{60\degree} + \cos{60\degree}$ & $\frac{\sqrt{3}}{2} + \frac{1}{2}$        & $1.366$ (approx.) \\
\hline
$\sin{90\degree} + \cos{90\degree}$ & $1 + 0$ & $1$ \\
\hline
\makecell{$\sin{\alpha} + \cos{\alpha}$, where $\alpha$\\ is the smaller\ldots} & $\frac{3}{5} + \frac{4}{5}$ & $1.4$ \\
\hline
\makecell{$\sin{\alpha} + \cos{\alpha}$, where $\alpha$\\ is the larger\ldots}  & $ \frac{4}{5} + \frac{3}{5}$ & $1.4$ \\
\hline
\end{tabular}
\egroup
\end{center}

\solution %2
If $\sin{\alpha} = 1$, then $\cos{\alpha} = 0$ and $\sin{\alpha} + \cos{\alpha} = 1$.
If $\cos{\alpha} = 1$, then $\sin{\alpha} = 0$ and $\sin{\alpha} + \cos{\alpha} = 1$.
Otherwise, $\sin{\alpha} < 1$ and $\cos{\alpha} < 1$, so $\sin{\alpha} + \cos{\alpha} < 2$.

\solution %3
First we will expand and simplify $(\sin{\alpha} + \cos{\alpha})^2$:
\begin{align*}
(\sin{\alpha} + \cos{\alpha})^2 &= \sin^{2}{\alpha} + 2\sin{\alpha}\cos{\alpha} + \cos^{2}{\alpha} \\
&= (\sin^{2}{\alpha} + \cos^{2}{\alpha}) + 2\sin{\alpha}\cos{\alpha} \\
&= 1 + 2\sin{\alpha}\cos{\alpha}
\end{align*}
We know that $0 \leq \sin{\alpha} \leq 1$ and $0 \leq \cos{\alpha} \leq 1$ because $\alpha$ is acute.
So $2\sin{\alpha}\cos{\alpha}$ is the product of three nonnegative numbers, and is itself a nonnegative number.
A nonnegative number added to 1 results in a number $\geq 1$.
Therefore, $1 + 2\sin{\alpha}\cos{\alpha} \geq 1$.
The square root of a number $\geq 1$ is itself $\geq 1$.
Therefore, $\sqrt{1 + 2\sin{\alpha}\cos{\alpha}} \geq 1$.
Rewriting the expression on the left: $\sqrt{1 + 2\sin{\alpha}\cos{\alpha}} = \sqrt{(\sin{\alpha} + \cos{\alpha})^2} = \sin{\alpha} + \cos{\alpha}$.
So, $\sin{\alpha} + \cos{\alpha} \geq 1$.

\solution %4
$\sin{45\degree} + \cos{45\degree} = \dfrac{1}{\sqrt{2}} + \dfrac{1}{\sqrt{2}} = \dfrac{2}{\sqrt{2}} = \dfrac{2\sqrt{2}}{\sqrt{2}\sqrt{2}} = \dfrac{2\sqrt{2}}{2} = \sqrt{2}$

\solution %5
You should notice that the values for $\sin{\alpha} + \cos{\alpha}$ increases with larger $alpha$ when $0\degree \leq \alpha < 45\degree$, reaches a maximum value when $\alpha = 45\degree$, then decreases with larger $\alpha$ when $45\degree < \alpha \leq 90\degree$.

\end{solutions}

\begin{solutions}{Page 35}
\solution ~
\begin{center}
\bgroup
\def\arraystretch{2}
\setlength\tabcolsep{15pt}
\begin{tabular}{ |c|c|c| }
\hline
$(\sin{0\degree})(\cos{0\degree})$   & $0\cdot1$        & $0$ \\
\hline
$(\sin{30\degree})(\cos{30\degree})$ & $\frac{1}{2}\cdot\frac{\sqrt{3}}{2}$ & $0.433$ (approx.) \\
\hline
$(\sin{45\degree})(\cos{45\degree})$ & $\frac{1}{\sqrt{2}}\cdot\frac{1}{\sqrt{2}}$ & $0.5$ \\
\hline
$(\sin{60\degree})(\cos{60\degree})$ & $\frac{\sqrt{3}}{2}\cdot\frac{1}{2}$ & $0.433$ (approx.) \\
\hline
\makecell{$(\sin{\alpha})(\cos{\alpha})$, where $\alpha$\\ is the smaller\ldots}  & $\frac{3}{5}\cdot\frac{4}{5}$ & $0.48$ \\
\hline
\makecell{$(\sin{\alpha})(\cos{\alpha})$, where $\alpha$\\ is the larger\ldots}  & $\frac{4}{5}\cdot\frac{3}{5}$ & $0.48$ \\
\hline
\end{tabular}
\egroup
\end{center}
\textit{How large can the product $(\sin{\alpha})(\cos{\alpha})$ get?}
We can see from the table that the maximum value of the product appears to be when $\alpha = 45\degree$.
\end{solutions}

\begin{solutions}{Page 37}
\solution %1
$\cos{\alpha}=3/5$, $\cos{\beta}=4/5$, $\sin{\alpha}=4/5$, $\sin{\beta}=3/5$, $\tan{\alpha}=4/3$, $\tan{\beta}=3/4$, $\cot{\alpha}=3/4$, $\cot{\beta}=4/3$.

\solution %2
We can show that this assumption is correct using the corollary of the Pythagorean Theorem:
$a^2 + b^{2} = 3^{2} + 4^{2} = 25 = c^2$.

\solution %3.
$\cos{\alpha}=a/c$, $\cos{\beta}=b/c$, $\sin{\alpha}=b/c$, $\sin{\beta}=a/c$, $\tan{\alpha}=b/a$, $\tan{\beta}=a/b$, $\cot{\alpha}=a/b$, $\cot{\beta}=b/a$.

\solution %4
$c = \sqrt{12^2 + 5^2} = \sqrt{169} = 13$. $\cos{\alpha}=12/13$. $\cos{\beta}=5/13$. $\cot{\alpha}=12/5$. $\cot{\beta}=5/12$.

\solution %5
First, we will use the Pythagorean Theorem to find the length of the longer leg:
\begin{align*}
a^2 + b^2 &= c^2\\
a^2 + 7^2 &= 25^2\\
a^2 + 49 &= 625\\
a^2 &= 576\\
a &= 24
\end{align*}
We can now find the numerical values that were asked for: $\cos{\alpha}=24/25$, $\cos{\beta}=7/25$, $\cot{\alpha}=24/7$, $\cot{\beta}=7/24$.

\solution %6
$\begin{aligned}[t]
\frac{a}{c} &= \sin{\alpha} = \cos{\beta}\\
\frac{b}{c} &= \cos{\alpha} = \sin{\beta}\\
\frac{a}{b} &= \tan{\alpha} = \cot{\beta}\\
\frac{b}{a} &= \cot{\alpha} = \tan{\beta}
\end{aligned}$

\solution %7
First, we will use the Pythagorean Theorem to find the length of the other leg:
\begin{align*}
a^2 + b^2 &= c^2\\
a^2 + 3^2 &= 5^2\\
a^2 + 9 &= 25\\
a^2 &= 16\\
a &= 4
\end{align*}
We can now find the numerical values that were asked for: $\cos{\alpha}=4/5$, $\cot{\alpha}=4/3$.

\solution %8
If $\tan{\alpha} = 1$, then $a/b=1$, implying that $a = b$ and $\alpha = 45\degree$. $\cos{\alpha} = \cos{45\degree} = 1/\sqrt{2}$. $\cot{\alpha} = 1/1 = 1$. 

\solution %9
$\tan{45\degree} = 1/1 = 1$.

\solution %10
$\tan{30\degree} = 1/\sqrt{3} \approx 0.57735$.

\solution %11
$\tan{45\degree} + \sin{30\degree} = 1 + \frac{1}{2} = \frac{3}{2}$. We don't need a calculator because both numbers are rational.

\end{solutions}


\section*{Chapter 2: Relations among Trigonometric Ratios}

\begin{solutions}{Page 43}
\solution %1
$\begin{aligned}[t]
\cos{\alpha} &= \sqrt{1 - \left(\frac{8}{17}\right)^2} = \sqrt{1 - \frac{64}{289}} = \sqrt{\frac{225}{289}} = \frac{15}{17} \\
\tan{\alpha} &= \frac{\frac{8}{17}}{\frac{15}{17}} = \frac{8}{15} \\
\cot{\alpha} &= \frac{15}{8}
\end{aligned}$

\solution %2
Let the length of the adjacent leg $a$ be $\frac{3}{7}$ and the length of the hypotenuse be 1 (see the first triangle diagram on page 44).
\begin{align*}
\sin{\alpha} &= \sqrt{1 - a^2} = \sqrt{1 - \left(\frac{3}{7}\right)^2} = \sqrt{1 - \frac{9}{49}} = \sqrt{\frac{40}{49}} &= \frac{\sqrt{4}\sqrt{10}}{\sqrt{49}} = \frac{2\sqrt{10}}{7} \\
\tan{\alpha} &= \frac{\sqrt{1 - a^2}}{a} = \frac{\frac{2\sqrt{10}}{7}}{\frac{3}{7}} = \frac{2\sqrt{10}}{3} \\
\cot{\alpha} &= \frac{a}{\sqrt{1 - a^2}} = \frac{3}{2\sqrt{10}}
\end{align*}

\solution %3
$\sin{\alpha} = \sqrt{1 - b^2},\;
\tan{\alpha} = \dfrac{\sqrt{1 - b^2}}{b},\; 
\cot{\alpha} = \dfrac{b}{\sqrt{1 - b^2}}$

\solution %4
$\sin{\alpha} = \dfrac{d}{\sqrt{1 + d^2}},\;
\cos{\alpha} = \dfrac{1}{\sqrt{1 + d^2}},\;
\cot{\alpha} = \dfrac{1}{d}$

\solution ~ %5
\begin{center}
\bgroup
\def\arraystretch{2.1}
\setlength\tabcolsep{15pt}
\begin{tabular}{ |c|c|c|c|c| }
\hline
~              & $\sin{\alpha}$             & $\cos{\alpha}$             & $\tan{\alpha}$             & $\cot{\alpha}$ \\
\hline
$\sin{\alpha}$ & $a$                        & $\sqrt{1 - a^2}$           & $\dfrac{a}{\sqrt{1 - a^2}}$ & $\dfrac{\sqrt{1 - a^2}}{a}$ \\
\hline
$\cos{\alpha}$ & $\sqrt{1 - a^2}$           & $a$                        & $\dfrac{\sqrt{1 - a^2}}{a}$ & $\dfrac{a}{\sqrt{1 - a^2}}$ \\
\hline
$\tan{\alpha}$ & $\dfrac{a}{\sqrt{1 + a^2}}$ & $\dfrac{1}{\sqrt{1 + a^2}}$ & $a$                       & $\dfrac{1}{a}$ \\
\hline
$\cot{\alpha}$ & $\dfrac{1}{\sqrt{1 + a^2}}$ & $\dfrac{a}{\sqrt{1 + a^2}}$ & $\dfrac{1}{a}$              & $a$ \\
\hline
\end{tabular}
\egroup
\end{center}


\end{solutions}

\begin{solutions}{Page 45 (First)}
\solution %1
Given in text

\solution %2
$\sin^{2}{45\degree} = \left(\dfrac{1}{\sqrt{2}}\right)^2 = \dfrac{1}{2}$

\solution ~ %3
\begin{center}
\bgroup
\def\arraystretch{2.1}
\setlength\tabcolsep{15pt}
\begin{tabular}{ |c|c|c|c|c| }
\hline
~              & $\sin{\alpha}$                                      & $\cos{\alpha}$                           & $\tan{\alpha}$                                      & $\cot{\alpha}$ \\
\hline
$\sin{\alpha}$ & $\sin{\alpha}$                                      & $\sqrt{1 - \sin^{2}{\alpha}}$            & $\dfrac{a}{\sqrt{1 - \sin^{2}{\alpha}}}$            & $\dfrac{\sqrt{1 - \sin^{2}{\alpha}}}{\sin{\alpha}}$ \\
\hline
$\cos{\alpha}$ & $\sqrt{1 - \cos^{2}{\alpha}}$                       & $\cos{\alpha}$                           & $\dfrac{\sqrt{1 - \cos^{2}{\alpha}}}{\cos{\alpha}}$ & $\dfrac{\cos{\alpha}}{\sqrt{1 - \cos^{2}{\alpha}}}$ \\
\hline
$\tan{\alpha}$ & $\dfrac{\tan{\alpha}}{\sqrt{1 + \tan^{2}{\alpha}}}$ & $\dfrac{1}{\sqrt{1 + \tan^{2}{\alpha}}}$ & $\tan{\alpha}$                                      & $\dfrac{1}{\tan{\alpha}}$ \\
\hline
$\cot{\alpha}$ & $\dfrac{1}{\sqrt{1 + \cot^{2}{\alpha}}}$            & $\dfrac{\cot{\alpha}}{\sqrt{1 + \cot^{2}{\alpha}}}$ & $\dfrac{1}{\cot{\alpha}}$                           & $\cot{\alpha}$ \\
\hline
\end{tabular}
\egroup
\end{center}
\end{solutions}


\begin{solutions}{Page 45 (Second)}
\solution $\tan{\alpha} = \dfrac{a}{b} = \cot{\beta}$
\solution $\cot{\alpha} = \dfrac{b}{a} = \tan{\beta}$
\solution $\sec{\alpha} = \dfrac{c}{a} = \csc{\beta}$
\solution $\csc{\alpha} = \dfrac{c}{b} = \sec{\beta}$
\end{solutions}


\begin{solutions}{Page 47}
\solution %1
\begin{subsolutions}
\subsolution $\sin^{2}{30\degree} + \cos^{2}{30\degree} = \left(\dfrac{1}{2}\right)^2 + \left(\dfrac{\sqrt{3}}{2}\right)^2 = \dfrac{1}{4} + \dfrac{3}{4} = 1$
\subsolution $\sin^{2}{45\degree} + \cos^{2}{45\degree} = \left(\dfrac{1}{\sqrt{2}}\right)^2 + \left(\dfrac{1}{\sqrt{2}}\right)^2 = \dfrac{1}{2} + \dfrac{1}{2} = 1$
\subsolution $\sin^{2}{60\degree} + \cos^{2}{60\degree} = \left(\dfrac{\sqrt{3}}{2}\right)^2 + \left(\dfrac{1}{2}\right)^2= \dfrac{3}{4} + \dfrac{1}{4} = 1$
\end{subsolutions}

\solution %2
$\begin{aligned}[t]
\sin^{2}{\alpha} + \cos^{2}{\alpha} &= 1 \\
\left(\dfrac{\sqrt{5}}{4}\right)^2 + \cos^{2}{\alpha} &= 1 \\
\cos^{2}{\alpha} &= 1 - \left(\frac{\sqrt{5}}{4}\right)^2 = 1 - \frac{5}{16} = \frac{11}{16} \\
\cos{\alpha} &= \sqrt{\frac{11}{16}} = \frac{\sqrt{11}}{4}
\end{aligned}$

\solution %3
$\begin{aligned}[t]
\sin^{2}{\alpha} + \cos^{2}{\alpha} &= 1 \\
\sin^{2}{\alpha} + \left(\frac{2}{3}\right)^2 &= 1 \\
\sin^{2}{\alpha} &= 1 - \frac{4}{9} = \frac{5}{9} \\
\sin{\alpha} &= \sqrt{\frac{5}{9}} = \frac{\sqrt{5}}{3}
\end{aligned}$

\solution %4
$\begin{aligned}[t]
\frac{\sin{\alpha}}{\cos{\alpha}} &= \tan{\alpha} = \frac{1}{\sqrt{3}} \\
\frac{\sin^{2}{\alpha}}{\cos^{2}{\alpha}} &= \frac{1}{3} \\[1ex]
\frac{\sin^{2}{\alpha}}{1 - \sin^{2}{\alpha}} &= \frac{1}{3} \\[1ex]
3\sin^{2}{\alpha} &= 1 - \sin^{2}{\alpha} \\
4\sin^{2}{\alpha} &= 1 \\
\sin^{2}{\alpha} &= \frac{1}{4} \\
\sin{\alpha} &= \sqrt{\frac{1}{4}} = \frac{1}{2}
\end{aligned}$

$\begin{aligned}[t]
\frac{\sin{\alpha}}{\cos{\alpha}} &= \tan{\alpha} = \frac{1}{\sqrt{3}} \\
\frac{\sin^{2}{\alpha}}{\cos^{2}{\alpha}} &= \frac{1}{3} \\[1ex]
\frac{1-\cos^{2}{\alpha}}{\cos^{2}{\alpha}} &= \frac{1}{3} \\
3*\frac{1-\cos^{2}{\alpha}}{\cos^{2}{\alpha}} &= 3*\frac{1}{3} \\
\frac{3*(1-\cos^{2}{\alpha})}{\cos^{2}{\alpha}} &= 1 \\
3*(1-\cos^{2}{\alpha}) &= \cos^{2}{\alpha} \\
3-3\cos^{2}{\alpha} &= \cos^{2}{\alpha} \\
3 &= 4\cos^{2}{\alpha} \\
\frac{3}{4} &= \cos^{2}{\alpha} \\
\frac{\sqrt{3}}{2} &= \cos{\alpha}
\end{aligned}$

And then to check our solution we can calculate the fraction we are given $\frac{1}{\sqrt{3}}$
from our $\cos{\alpha}$ and $\sin{\alpha}$ fractions.
$\begin{aligned}[t]
\frac{\frac{1}{2}}{\frac{\sqrt{3}}{2}} \\
\frac{2}{2*\sqrt{3}} \\
\frac{1}{\sqrt{3}}
\end{aligned}$

\solution %5
\begin{subsolutions}
\subsolution %a
$\cot{x}\sin{x} = \left(\dfrac{1}{\tan{x}}\right)\sin{x} = \dfrac{\sin{x}}{\tan{x}} = \dfrac{\sin{x}}{\frac{\sin{x}}{\cos{x}}} = \dfrac{\sin{x}\cos{x}}{\sin{x}} = \cos{x}$


\subsolution %b
$\dfrac{\tan{x}}{\sin{x}} = \dfrac{\frac{\sin{x}}{\cos{x}}}{\sin{x}} = \dfrac{\frac{\sin{x}}{\cos{x}}\cdot\frac{1}{\sin{x}}}{\sin{x}\cdot\frac{1}{\sin{x}}} = \dfrac{\frac{\sin{x}}{\sin{x}\cos{x}}}{1} = \dfrac{\sin{x}}{\sin{x}\cos{x}} = \dfrac{1}{\cos{x}}$


\subsolution %c
$\cos^{2}{\alpha} - \sin^{2}{\alpha} = \cos^{2}{\alpha} - (1 - \cos^{2}{\alpha}) = \cos^{2}{\alpha} - 1 + \cos^{2}{\alpha} = 2\cos^{2}{\alpha} - 1$

\subsolution %d
This one is tricky. You might need to try a few different approaches (squaring above and below, multiplying above and below by $\cos{\alpha}\sin{\alpha}$). Eventually it becomes clear that you need to multiply above and below by $(1 - \cos{\alpha})$ and find a way to cancel out the $\sin{\alpha}$ factor in the numerator:
\begin{align*}
\frac{\sin{\alpha}}{1 + \cos{\alpha}} &= \frac{\sin{\alpha}(1 - \cos{\alpha})}{(1 + \cos{\alpha})(1 - \cos{\alpha})} = \frac{\sin{\alpha}(1 - \cos{\alpha})}{1 - \cos{\alpha} + \cos{\alpha} - \cos^{2}{\alpha}} \\
&= \frac{\sin{\alpha}(1 - \cos{\alpha})}{1 - \cos^{2}{\alpha}} = \frac{\sin{\alpha}(1 - \cos{\alpha})}{1 - (1 - \sin^{2}{\alpha})} = \frac{\sin{\alpha}(1 - \cos{\alpha})}{1 - 1 + \sin^{2}{\alpha}} \\
&= \frac{\sin{\alpha}(1 - \cos{\alpha})}{\sin^{2}{\alpha}} = \frac{1 - \cos{\alpha}}{\sin{\alpha}}
\end{align*}

\subsolution %e
$\begin{aligned}[t]
\frac{\sin^{2}{\alpha} + 2\cos^{2}{\alpha} - 1}{\cot^{2}{\alpha}} &=  \frac{1 - \cos^{2}{\alpha} + 2\cos^{2}{\alpha} - 1}{\cot^{2}{\alpha}} = \frac{\cos^{2}{\alpha}}{\left(\frac{\cos{\alpha}}{\sin{\alpha}}\right)^2} \\
&=  \frac{\cos^{2}{\alpha}}{\frac{\cos^{2}{\alpha}}{\sin^{2}{\alpha}}} = \frac{\cos^{2}{\alpha}\sin^{2}{\alpha}}{\cos^{2}{\alpha}} \\
&= \sin^{2}{\alpha}
\end{aligned}$

\subsolution %f
$\begin{aligned}[t]
\cos^{2}{\alpha} &= \frac{\cos^{2}{\alpha}}{1} = \frac{\cos^{2}{\alpha}}{\cos^{2}{\alpha} + \sin^{2}{\alpha}} \\
&=  \frac{\frac{\cos^{2}{\alpha}}{\cos^{2}{\alpha}}}{\frac{\cos^{2}{\alpha} + \sin^{\alpha}}{\cos^{2}{\alpha}}} = \frac{1}{\frac{\cos^{2}{\alpha}}{\cos^{2}{\alpha}} + \frac{\sin^{2}{\alpha}}{\cos^{2}{\alpha}}} \\
&= \frac{1}{1 + \tan^{2}{\alpha}}
\end{aligned}$


\subsolution %g
$\begin{aligned}[t]
\sin^{2}{\alpha} &= \frac{\sin^{2}{\alpha}}{1} = \frac{\cos^{2}{\alpha}}{\cos^{2}{\alpha} + \sin^{2}{\alpha}} \\
&=  \frac{\frac{\sin^{2}{\alpha}}{\sin^{2}{\alpha}}}{\frac{\cos^{2}{\alpha} + \sin^{\alpha}}{\sin^{2}{\alpha}}} = \frac{1}{\frac{\cos^{2}{\alpha}}{\sin^{2}{\alpha}} + \frac{\sin^{2}{\alpha}}{\sin^{2}{\alpha}}} \\
&= \frac{1}{\cot^{2}{\alpha} + 1}
\end{aligned}$

\subsolution %h
$\begin{aligned}[t]
\frac{1 - \cos{\alpha}}{1 + \cos{\alpha}} &= \frac{(1 - \cos{\alpha})(1 + \cos{\alpha})}{(1 + \cos{\alpha})(1 + \cos{\alpha})} = \frac{1 + \cos{\alpha} - \cos{\alpha} - \cos^{2}{\alpha}}{(1 + \cos{\alpha})^2} \\
&= \frac{1 - \cos^{2}{\alpha}}{(1 + \cos{\alpha})^2} = \frac{\sin^{2}{\alpha}}{(1 + \cos{\alpha})^2} \\
&= \left(\frac{\sin{\alpha}}{1 + \cos{\alpha}}\right)^2
\end{aligned}$

\subsolution %i
The key to solving this one is the formula for factoring a difference of cubes: $a^3 - b^3 = (a - b)(a^2 + ab + b^2)$.
\begin{align*}
\frac{\sin^{3}{\alpha} - \cos^{3}{\alpha}}{\sin{\alpha} - \cos{\alpha}} &= \frac{(\sin{\alpha} - \cos{\alpha})(\sin^{2}{\alpha} + \sin{\alpha}\cos{\alpha} + \cos^{2}{\alpha})}{\sin{\alpha} - \cos{\alpha}} \\
&= \sin^{2}{\alpha} + \sin{\alpha}\cos{\alpha} + \cos^{2}{\alpha} \\
&= 1 + \sin{\alpha}\cos{\alpha}
\end{align*}
\end{subsolutions}

\solution %6
\begin{subsolutions}

\subsolution %a
We can rewrite the LHS to show that $\sin^{4}{\alpha} - \cos^{4}{\alpha} = \cos^{2}{\alpha} - \sin^{2}{\alpha}$:
\begin{align*}
\sin^{4}{\alpha} - \cos^{4}{\alpha} &= (\sin^{2}{\alpha} + \cos^{2}{\alpha})(\sin^{2}{\alpha} - \cos^{2}{\alpha}) = 1(\sin^{2}{\alpha} - \cos^{2}{\alpha}) \\
&= \sin^{2}{\alpha} - \cos^{2}{\alpha}
\end{align*}
Answer: There are no angles $\alpha$ for which $\sin^{4}{\alpha} - \cos^{4}{\alpha} > \cos^{2}{\alpha} - \sin^{2}{\alpha}$ because the expressions on either side of the inequality are equivalent.

\subsolution %b
$\sin^{4}{\alpha} - \cos^{4}{\alpha} >= \cos^{2}{\alpha} - \sin^{2}{\alpha}$ for all angles $\alpha$ because the expressions on either side of the inequality are equivalent.

\end{subsolutions}

\solution %7
If we rewrite $2\sin{\alpha}\cos{\alpha}$ as a fraction, we can divide above and below by $\cos{\alpha}$ to convert the numerator and denominator into expressions in terms of $\tan{\alpha}$:
\begin{align*}
2\sin{\alpha}\cos{\alpha} &= \frac{2\sin{\alpha}\cos{\alpha}}{1} = \frac{2\sin{\alpha}\cos{\alpha}}{\sin^{2}{\alpha} + \cos^{2}{\alpha}} \\
&= \frac{\frac{2\sin{\alpha}\cos{\alpha}}{\cos^{2}{\alpha}}}{\frac{\sin^{2}{\alpha} + \cos^{2}{\alpha}}{\cos^{2}{\alpha}}} = \frac{\frac{2\sin{\alpha}}{\cos{\alpha}}}{\frac{\sin^{2}{\alpha}}{\cos^{2}{\alpha}} + \frac{\cos^{2}{\alpha}}{\cos^{2}{\alpha}}} \\
&= \frac{2\tan{\alpha}}{\tan^{2}{\alpha} + 1}
\end{align*}
Now we can plug in the given value for $\tan{\alpha}$ to find the value of $2\sin{\alpha}\cos{\alpha}$ in this instance:
\begin{equation*}
2\sin{\alpha}\cos{\alpha} = \frac{2\tan{\alpha}}{\tan^{2}{\alpha} + 1} = \frac{2(\frac{2}{5})}{(\frac{2}{5})^2 + 1} = \frac{\frac{4}{5}}{\frac{4}{25} + 1} = \frac{\frac{4}{5}}{\frac{4}{25} + \frac{25}{25}} = \frac{\frac{20}{25}}{\frac{29}{25}} = \frac{20}{29}
\end{equation*}


\solution %8
First, we will rewrite the expresssion $\cos^{2}{\alpha} - \sin^{2}{\alpha}$ in terms of $\tan{\alpha}$:
\begin{align*}
\cos^{2}{\alpha} - \sin^{2}{\alpha} &= \frac{\cos^{2}{\alpha} - \sin^{2}{\alpha}}{1} = \frac{\cos^{2}{\alpha} - \sin^{2}{\alpha}}{\cos^{2}{\alpha} + \sin^{2}{\alpha}} = \frac{\frac{\cos^{2}{\alpha} - \sin^{2}{\alpha}}{\cos^{2}{\alpha}}}{\frac{\cos^{2}{\alpha} + \sin^{2}{\alpha}}{\cos^{2}{\alpha}}} \\
&= \frac{1 - \tan^{2}{\alpha}}{1 + \tan^{2}{\alpha}}
\end{align*}

\begin{subsolutions}

\subsolution
To find the numerical value of $\cos^{2}{\alpha} - \sin^{2}{\alpha}$ when $\tan{\alpha} = \frac{2}{5}$ we can substitute $\frac{2}{5}$ for $\tan{\alpha}$ in the formula above:
\begin{equation*}
\cos^{2}{\alpha} - \sin^{2}{\alpha} = \frac{1 - \tan^{2}{\alpha}}{1 + \tan^{2}{\alpha}} = \frac{1 - \left(\frac{2}{5}\right)^2}{1 + \left(\frac{2}{5}\right)^2} = \frac{1 - \frac{4}{25}}{1 + \frac{4}{25}} = \frac{\frac{21}{25}}{\frac{29}{25}} = \frac{21}{29}
\end{equation*}

\subsolution
Substituting $r$ for $\tan{\alpha}$ in the formula above:
\begin{equation*}
\cos^{2}{\alpha} - \sin^{2}{\alpha} = \frac{1 - r^2}{1 + r^2}
\end{equation*}

\end{subsolutions}

\solution %9
First, we will rewrite the expresssion in terms of $\tan{\alpha}$:

\begin{equation*}
\frac{\sin{\alpha} - 2\cos{\alpha}}{\cos{\alpha} - 3\sin{\alpha}} = \frac{\frac{\sin{\alpha} - 2\cos{\alpha}}{\cos{\alpha}}}{\frac{\cos{\alpha} - 3\sin{\alpha}}{\cos{\alpha}}} = \frac{\frac{\sin{\alpha}}{\cos{\alpha}} - \frac{2\cos{\alpha}}{\cos{\alpha}}}{\frac{\cos{\alpha}}{\cos{\alpha}} - \frac{3\sin{\alpha}}{\cos{\alpha}}} = \frac{\tan{\alpha} - 2}{1 - 3\tan{\alpha}}
\end{equation*}

Next, we substitute  $\frac{2}{5}$ for $\tan{\alpha}$:
\begin{equation*}
\frac{\tan{\alpha} - 2}{1 - 3\tan{\alpha}} = \frac{\frac{2}{5} - 2}{1 - 3\left(\frac{2}{5}\right)} = \frac{\frac{2}{5} - \frac{10}{5}}{\frac{5}{5} - \frac{6}{5}} = \frac{-\frac{8}{5}}{-\frac{1}{5}} = 8
\end{equation*}


\solution %10
First, we will rewrite the expresssion in terms of $\tan{\alpha}$:
\begin{equation*}
\frac{a\sin{\alpha} + b\cos{\alpha}}{c\cos{\alpha} + d\sin{\alpha}} = \frac{\frac{a\sin{\alpha}}{\cos{\alpha}} + \frac{b\cos{\alpha}}{\cos{\alpha}}}{\frac{c\cos{\alpha}}{\cos{\alpha}} + \frac{d\cos{\alpha}}{\cos{\alpha}}} = \frac{a\tan{\alpha} + b}{c + d\tan{\alpha}}
\end{equation*}

Next, we substitute  $\frac{2}{5}$ for $\tan{\alpha}$ and simplify:
\begin{equation*}
\frac{a\tan{\alpha} + b}{c + d\tan{\alpha}} = \frac{a\left(\frac{2}{5}\right) + b\left(\frac{5}{5}\right)}{c\left(\frac{5}{5}\right) + d\left(\frac{2}{5}\right)} = \frac{\frac{2a + 5b}{5}}{\frac{5c+2d}{5}} = \frac{2a + 5b}{5c + 2d}
\end{equation*}

Now we can see why the problem included the restriction that $5c + 2d \neq 0$; the value of the expression is undefined if the denominator is zero. The sum of two rational numbers is a rational number. Therefore the numerator and denominator in the expression are both rational numbers. The quotient of two rational numbers is a rational number. Therefore, the entire expression evaluates to a rational number for arbitratrary rational values of $a$, $b$, $c$ and $d$.

\solution %11
We can expand and simplify the expression:
\begin{align*}
& (\sin{\alpha} + \cos{\alpha})^2 + (\sin{\alpha} - \cos{\alpha})^2 \\
&= \sin^{2}{\alpha} + 2\sin{\alpha}\cos{\alpha} + \cos^{2}{\alpha} + \sin^{2}{\alpha} - 2\sin{\alpha}\cos{\alpha} + \cos^{2}{\alpha} \\
&= 2\sin^{2}{\alpha} + 2\cos^{2}{\alpha} \\
&= 2(\sin^{2}{\alpha} + \cos^{2}{\alpha}) \\
&= 2(1) \\
&= 2
\end{align*}
As the expression evaluates to a constant, it is as large as possible for all values of $\alpha$.

\end{solutions}

\begin{solutions}{Page 49}

\solution %1

Rewriting any instances of $\sec{\alpha}$ or $\csc{\alpha}$ on either side of the identities:

\begin{subsolutions}

\subsolution %a
$\begin{aligned}[t]
\tan{\alpha}\csc{\alpha} &= \sec{\alpha} \\
\tan{\alpha}\dfrac{1}{\sin{\alpha}} &= \dfrac{1}{\cos{\alpha}} \\[1ex]
\dfrac{\tan{\alpha}}{\sin{\alpha}} &= \dfrac{1}{\cos{\alpha}}
\end{aligned}$

\subsolution %b
$\begin{aligned}[t]
\cot{\alpha}\csc{\alpha} &= \sec{\alpha} \\
\cot{\alpha}\dfrac{1}{\cos{\alpha}} &= \dfrac{1}{\sin{\alpha}} \\[1ex]
\dfrac{\cot{\alpha}}{\cos{\alpha}} &= \dfrac{1}{\sin{\alpha}}
\end{aligned}$

\subsolution %c
$\begin{aligned}[t]
\dfrac{1}{\sec{\alpha}}\csc{\alpha} &= \cot{\alpha} \\
\dfrac{1}{\frac{1}{\cos{\alpha}}}\cdot\dfrac{1}{\sin{\alpha}} &= \cot{\alpha} \\
\cos{\alpha}\dfrac{1}{\sin{\alpha}} &= \cot{\alpha} \\[1ex]
\dfrac{\cos{\alpha}}{\sin{\alpha}} &= \cot{\alpha}
\end{aligned}$

\subsolution %d
$\begin{aligned}[t]
\tan^{2}{\alpha} &= (\sec{\alpha} + 1)(\sec{\alpha} - 1) \\
\tan^{2}{\alpha} &= \sec^{2}{\alpha} - 1 \\
\tan^{2}{\alpha} &= \dfrac{1}{\cos^{2}{\alpha}} - 1
\end{aligned}$

\subsolution %e
$\begin{aligned}[t]
\csc^{2}{\alpha} &= 1 + \cot^{2}{\alpha} \\
\dfrac{1}{\sin^{2}{\alpha}} &= 1 + \cot^{2}{\alpha}
\end{aligned}$

\end{subsolutions}


\solution %2
Rewriting any instances of $\sin{\alpha}$ or $\cos{\alpha}$ on either side of the identities, and eliminating fractions:
% TODO are these correct? Some of the solutions look simplistic
\begin{subsolutions}

\subsolution %a
$\begin{aligned}[t]
\frac{\tan{\alpha}}{\sin{\alpha}} = \frac{1}{\cos{\alpha}} \\
\tan{\alpha}\frac{1}{\sin{\alpha}} = \sec{\alpha} \\
\tan{\alpha}\csc{\alpha} = \sec{\alpha}
\end{aligned}$

\subsolution %b
$\begin{aligned}[t]
\frac{1}{\sin{\alpha}}\cos{\alpha} = \cot{\alpha} \\
\frac{\cos{\alpha}}{\sin{\alpha}} = \cot{\alpha} \\
\cot{\alpha} = \cot{\alpha}
\end{aligned}$

\subsolution %c
$\begin{aligned}[t]
\tan^{2}{\alpha} + 1 = \frac{1}{\cos^{2}{\alpha}} \\
\tan^{2}{\alpha} + 1 = \sec^{2}{\alpha}
\end{aligned}$

\subsolution %d
$\begin{aligned}[t]
\frac{1}{\sin^{2}{\alpha}} = 1 + \cot^{2}{\alpha} \\
\csc^{2}{\alpha} = 1 + \cot^{2}{\alpha}
\end{aligned}$

\end{subsolutions}

\end{solutions}



\begin{solutions}{Page 50}

\solution % 1
First, we find the value of $a^2 + b^2$:
\begin{align*}
a^2 + b^2 &= (\cos^{2}{\alpha} - \sin^{2}{\alpha})^2 + (2\sin{\alpha}\cos{\alpha})^2 \\
&= \cos^{4}{\alpha} - 2\cos^{2}{\alpha}\sin^{2}{\alpha} + \sin^{4}{\alpha} + 4\sin^{2}{\alpha}\cos^{2}{\alpha} \\
&= \cos^{4}{\alpha} + 2\cos^{2}{\alpha}\sin^{2}{\alpha} + \sin^{4}{\alpha} \\
&= (\cos^{2}{\alpha} + \sin^{2}{\alpha})^2 \\
&= (1)^2 \\
&= 1
\end{align*}
According to the lemma on Page 50, as $a^2 + b^2 = 1$, an angle $\theta$ exists such that $a = \cos{\theta}$ and $b = \sin{\theta}$.

\solution %2
First, we find the value of $a^2 + b^2$:
\begin{align*}
a^2 + b^2 &= \left(\sqrt{\frac{1 + \cos{\alpha}}{2}}\right)^2 + \left(\sqrt{\frac{1 - \cos{\alpha}}{2}}\right)^2 \\
&= \frac{1 + \cos{\alpha}}{2} + \frac{1 - \cos{\alpha}}{2} \\
&= \frac{1 + \cos{\alpha} + 1 - \cos{\alpha}}{2} \\
&= \frac{2}{2} \\
&= 1
\end{align*}

\solution %3
First, we will rewrite $a$ and $b$ to eliminate the cube exponents:
\begin{align*}
a &= 4\cos^{3}{\alpha} - 3\cos{\alpha} \\
&= 4\cos{\alpha}\cos^{2}{\alpha} - 3\cos{\alpha} \\
&= 4\cos{\alpha}(1 - \sin^{2}{\alpha}) - 3\cos{\alpha} \\
&= 4\cos{\alpha} - 4\sin^{2}{\alpha}\cos{\alpha} - 3\cos{\alpha} \\
&= \cos{\alpha} - 4\sin^{2}{\alpha}\cos{\alpha}
\end{align*}
\begin{align*}
b &= 3\sin{\alpha} - 4\sin^{3}{\alpha} \\
&= 3\sin{\alpha} - 4\sin{\alpha}\sin^{2}{\alpha} \\
&= 3\sin{\alpha} - 4\sin{\alpha}(1 - \cos^{2}{\alpha}) \\
&= -\sin{\alpha} + 4\sin{\alpha}\cos^{2}{\alpha}
\end{align*}
Next, we will expand $a^2$ and $b^2$:
\begin{align*}
a^2 &= (\cos{\alpha} - 4\sin^{2}{\alpha}\cos{\alpha})^2 \\
&= \cos^{2}{\alpha} - 8\sin^{2}{\alpha}\cos^{2}{\alpha} + 16\sin^{4}{\alpha}\cos^{2}{\alpha}
\end{align*}
\begin{align*}
b^2 &= (-\sin{\alpha} + 4\sin{\alpha}\cos^{2}{\alpha})^2 \\
&= \sin^{2}{\alpha} - 8\sin^{2}{\alpha}\cos^{2}{\alpha} + 16\sin^{2}{\alpha}\cos^{4}{\alpha}
\end{align*}
Next, we add the expressions for $a^2$ and $b^2$ and simplify to 1:
\begin{align*}
a^2 + b^2 &= \cos^{2}{\alpha} - 8\sin^{2}{\alpha}\cos^{2}{\alpha} + 16\sin^{4}{\alpha}\cos^{2}{\alpha} +\sin^{2}{\alpha} - 8\sin^{2}{\alpha}\cos^{2}{\alpha} + \\
&\quad\quad16\sin^{2}{\alpha}\cos^{4}{\alpha} \\
&= \cos^{2}{\alpha} + \sin^{2}{\alpha} - 16\sin^{2}{\alpha}\cos^{2}{\alpha} + 16\sin^{4}{\alpha}\cos^{2}{\alpha} + 16\sin^{2}{\alpha}\cos^{4}{\alpha} \\
&= \cos^{2}{\alpha} + \sin^{2}{\alpha} + 16\sin^{2}{\alpha}\cos^{2}{\alpha}(-1 + \sin^{2}{\alpha} + \cos^{2}{\alpha}) \\
&= 1 + 16\sin^{2}{\alpha}\cos^{2}{\alpha}(0) \\
&= 1
\end{align*}
According to the lemma on Page 50, as $a^2 + b^2 = 1$, an angle $\theta$ exists such that $a = \cos{\theta}$ and $b = \sin{\theta}$.

\solution %4
First, we find the value of $a^2 + b^2$:
\begin{align*}
a^2 + b^2 &= \left(\frac{1 - t^2}{1 + t^2}\right)^2 + \left(\frac{2t}{1 + t^2}\right)^2 \\[1ex]
&= \frac{(1 - t^2)^2}{(1 + t^2)^2} + \frac{(2t)^2}{(1 + t^2)^2} \\[1ex]
&= \frac{(1 - t^2)^2 + (2t)^2}{(1 + t^2)^2} \\[1ex]
&= \frac{1 - 2t^2 + t^4 + 4t^2}{(1 + t^2)(1 + t^2)} \\[1ex]
&= \frac{(1 + t^2)(1 + t^2)}{(1 + t^2)(1 + t^2)} \\
&= 1
\end{align*}
According to the lemma on Page 50, as $a^2 + b^2 = 1$, an angle $\theta$ exists such that $a = \cos{\theta}$ and $b = \sin{\theta}$.

\solution %5
We expand $(p^2 - q^2)^2 + (2pq)^2$ and use the fact that $p^2 + q^2 = 1$ to simplify to 1:
\begin{align*}
(p^2 - q^2)^2 + (2pq)^2 &= p^{4} - 2p^2q^2 + q^4 + 4p^2q^2 \\
&= p^{4} + 2p^2q^2 + q^4 \\
&= (p^2 + q^2)^2 \\
&= (1)^2 \\
&= 1
\end{align*}
This is similar to Exercise 1 above.

\end{solutions}


\begin{solutions}{Page 51}
\solution %1
$\sin{\alpha} < 1$ when $\alpha$ is acute, therefore $1 - \sin{\alpha} > 0$ when $\alpha$ is acute. $1 - \sin{\alpha} = 0$ when $\sin{\alpha} = 1$, i.e., $\alpha = 90\degree$.

\solution %2
$\cos{\alpha} < 1$ when $\alpha$ is acute, therefore $1 - \cos{\alpha} > 0$ when $\alpha$ is acute. $1 - \cos{\alpha} = 0$ when $\cos{\alpha} = 1$, i.e., $\alpha = 0\degree$.

\solution %3
Statement a) is always true. Statements b) and c) both include the case that $\sin^{2}{\alpha} + \cos^{2}{\alpha} = 1$, which is always true.

\solution %4
Let $x$ be the maximum cost of the items in a supermarket. In Supermarket A, $x \leq \$1$. In Supermarket B, $x < \$1$. In Supermarket C, $x \leq \$1$. In Supermarket D, $x > \$1$. We can see that Supermarkets A and C are offering the same terms.

\solution %5
Inequality a) is correct. For b) to be correct, an angle $\alpha$ would have to exist such that $\sin{\alpha} + \cos{\alpha} = 2$.
We know that this is not the case. When $\alpha = 90\degree$, $\sin{\alpha} = 1$ and $\cos{\alpha} = 0$. When $\alpha = 0\degree$, $\sin{\alpha} = 0$ and $\cos{\alpha} = 1$. When $0\degree < \alpha < 90\degree$, $\sin{\alpha} < 1$ and $\cos{\alpha} < 1$. In all cases, $\sin{\alpha} + \cos{\alpha} < 2$.

\solution %6
The largest possible value of $\sin{\alpha}$ is 1, and occurs when $\alpha = 90\degree$. The largest possible value of $\cos{\alpha}$ is 1, and occurs when $\alpha = 0\degree$.
See Page 32.
\end{solutions}

\begin{solutions}{Page 52}

\solution %1
$\sin{30\degree} = 0.5$, $\sin{45\degree} = 0.707$, $\sin{60\degree} = 0.866$.

\solution %2
By using the \texttt{tan} button to calculate $\tan{60\degree}$, and the \texttt{sqrt} button to calculate $\sqrt{3}$, Betty can compare the results: both are $1.732$.

\solution %3
Press \texttt{tan}, then enter the angle degree measure, then press \texttt{1/$x$}

\solution ~ %4
\begin{center}
\bgroup
\def\arraystretch{2.2}
\setlength\tabcolsep{15pt}
\begin{tabular}{ |c|c|c|c|c| }
\hline
\multicolumn{5}{ |c| }{in radical or rational form} \\
\hline
$\alpha$    & $\sin{\alpha}$        & $\cos{\alpha}$        & $\tan{\alpha}$        & $\cot{\alpha}$ \\
\hline
$30\degree$ & $\dfrac{1}{2}$        & $\dfrac{\sqrt{3}}{2}$ & $\dfrac{1}{\sqrt{3}}$ & $\sqrt{3}$ \\
\hline
$45\degree$ & $\dfrac{1}{\sqrt{2}}$ & $\dfrac{1}{\sqrt{2}}$ & $1$                   & $1$ \\
\hline
$60\degree$ & $\dfrac{\sqrt{3}}{2}$ & $\dfrac{1}{2}$        & $\sqrt{3}$            & $\dfrac{1}{\sqrt{3}}$ \\
\hline
\end{tabular}
\egroup
\end{center}
\begin{center}
\bgroup
\def\arraystretch{2.1}
\setlength\tabcolsep{15pt}
\begin{tabular}{ |c|c|c|c|c| }
\hline
\multicolumn{5}{ |c| }{in decimal form, from calculator} \\
\hline
$\alpha$    & $\sin{\alpha}$ & $\cos{\alpha}$ & $\tan{\alpha}$  & $\cot{\alpha}$ \\
\hline
$30\degree$ & $0.5$          & $0.866$        & $0.577$         & $1.732$ \\
\hline
$45\degree$ & $0.707$        & $0.707$        & $1$             & $1$ \\
\hline
$60\degree$ & $0.866$        & $0.5$          & $1.732$         & $0.577$ \\
\hline
\end{tabular}
\egroup
\end{center}

\end{solutions}


\begin{solutions}{Page 53}
\solution %1
The sine of the larger angle is $4/5=.8$. We can use the inverse sine function to find the angle: $\arcsin{.8} = 53.1301\degree$. The sum of the three angles in the triangles is: $\arcsin{.6} + \arcsin{.8} + 90\degree = 36.8699\degree + 53.1301\degree + 90\degree = 180\degree$.

\solution %2
\begin{subsolutions}
\item $\arcsin{1} = 90\degree$
\item $\arccos{0.7071067811865} = 45\degree$
\end{subsolutions}

\solution %3
$\arccos{0.8} = 36.8699\degree$

\solution %4
$\arcsin{0.6} = 36.8699\degree$

\solution %5
Half of $\sin{30\degree}$ (0.25) seems like a reasonable estimate. The actual value is $0.2588$.

\solution %6

\solution %7

\solution %8

\solution %9

\solution %10

\solution %11

\solution %12

\solution %13

\solution %14

\end{solutions}

\begin{solutions}{Page 55}
\solution
\solution
\solution
\solution
\end{solutions}

\begin{solutions}{Page 56}
\solution
\solution
\solution
\solution
\end{solutions}

\begin{solutions}{Page 59}
\solution
\solution
\solution
\solution
\solution
\solution
\solution
\solution
\solution
\solution
\solution
\end{solutions}

\begin{solutions}{Page 62}
\solution %1
The degree measure of a semicircle is $180\degree$. The degree measure of a quarter circle is $90\degree$.
\solution %2
The measure of arc cut off by one side of regular pentagon inscribed in a circle is $360\degree/5 = 72\degree$.
For a regular hexagon: $360\degree/6 = 60\degree$.
For a regular octagon: $360\degree/8 = 45\degree$.
\end{solutions}

\begin{solutions}{Page 64}
\solution
\solution
\solution
\end{solutions}

\begin{solutions}{Page 65}
\solution
\solution
\solution
\solution
\solution
\end{solutions}


\section*{Chapter 3: Relationships in a Triangle}

\begin{solutions}{Page 68}
\solution
\end{solutions}

\begin{solutions}{Page 70}
\solution
\solution
\solution
\solution
\end{solutions}

\begin{solutions}{Page 71}
\solution
\end{solutions}

\begin{solutions}{Page 73}
\solution
\solution
\solution
\solution
\solution
\solution
\solution
\solution
\end{solutions}

\begin{solutions}{Page 75 (First)}
\solution
\solution
\end{solutions}

\begin{solutions}{Page 75 (Second)}
\solution
\solution
\solution
\solution
\solution
\solution
\solution
\solution
\solution
\solution
\solution
\solution
\solution
\end{solutions}

\begin{solutions}{Page 79}
\solution
\end{solutions}

\begin{solutions}{Page 80}
\solution
\solution
\solution
\solution
\solution
\solution
\solution
\solution
\solution
\solution
\solution
\solution
\solution
\solution
\end{solutions}

\begin{solutions}{Page 83}
\solution
\solution
\solution
\end{solutions}

\begin{solutions}{Page 84}
\solution
\solution
\end{solutions}

\begin{solutions}{Page 85}
\solution
\solution
\end{solutions}

\begin{solutions}{Page 86}
\solution
\end{solutions}

\begin{solutions}{Page 88}
\solution
\solution
\solution
\solution
\solution
\end{solutions}


\section*{Chapter 4: Angles and Rotations}

\begin{solutions}{Page 93}
\solution
\end{solutions}

\begin{solutions}{Page 98}
\solution
\solution
\end{solutions}

\begin{solutions}{Page 100}
\solution
\solution
\solution
\solution
\solution
\end{solutions}

\begin{solutions}{Page 102}
\solution
\solution
\solution
\end{solutions}


\section*{Chapter 5: Radian Measure}

\begin{solutions}{Page 107}
\solution
\solution
\solution
\solution
\solution
\solution
\solution
\solution
\solution
\solution
\solution
\solution
\solution
\solution
\end{solutions}

\begin{solutions}{Page 111}
\solution
\solution
\solution
\solution
\solution
\solution
\solution
\solution
\solution
\solution
\solution
\solution
\solution
\solution
\solution
\solution
\solution
\solution
\solution
\solution
\end{solutions}

\begin{solutions}{Page 114}
\solution
\solution
\solution
\solution
\solution
\solution
\solution
\solution
\end{solutions}

\begin{solutions}{Page 116}
\solution
\end{solutions}

\begin{solutions}{Page 120}
\solution
\solution
\solution
\solution
\solution
\solution
\solution
\solution
\solution
\end{solutions}


\section*{Chapter 6: The Addition Formulas}

\begin{solutions}{Page 123}
\solution ~ %1
\begin{center}
\bgroup
\def\arraystretch{2.1}
\setlength\tabcolsep{15pt}
\begin{tabular}{ |c|c|c|c|c|c| }
\hline
$\alpha$
& $\beta$
& $\sin{\alpha}$
& $\sin{\beta}$
& $\sin{\alpha} + \sin{\beta}$
& $\sin\left(\alpha+\beta\right)$ \\
\hline
$60\degree$
& $30\degree$
& $\sqrt{3}/2$
& $1/2$
& $\left(\sqrt{3}+1\right)/2$
& $\sin{90\degree} = 1$\\
\hline
$\pi / 4$
& $\pi / 4$
& $\sqrt{2}/2$
& $\sqrt{2}/2$
& $\left(\sqrt{2}+\sqrt{2}\right)/2 = \sqrt{2}$
& $\sin{\pi/2} = 1$\\
\hline
$\pi / 6$
& $\pi / 3$
& $1/2$
& $\sqrt{3}/2$
& $\left(1+\sqrt{3}\right)/2$
& $\sin{\pi/2} = 1$\\
\hline
\end{tabular}
\egroup
\end{center}

\solution %2
For these values of $\alpha$ and $\beta$, $\sin{\alpha}$ and $\sin{\beta}$ are both at least $1/2$. Furthermore, at least one of $\sin{\alpha}$ and $\sin{\beta}$ is strictly greater than $1/2$. Therefore,
\[
\sin{\alpha} + \sin{\beta} > \dfrac{1}{2} + \dfrac{1}{2} = 1 = \sin\left(\alpha+\beta\right).
\]
\solution %3
\begin{subsolutions}

\subsolution %a
\[
\sin{60\degree} + \sin{30\degree} = \dfrac{\sqrt{3}}{2} + \dfrac{1}{2}
\]
\[
\sin\left(60\degree + 30\degree\right) = \sin{90\degree} = 1
\]
This identity is not correct.
\subsolution %b
\[
\sin\left(60\degree - 30\degree\right) = \sin{30\degree} = \dfrac{1}{2}
\]
\[
\sin{60\degree} - \sin{30\degree} = \dfrac{\sqrt{3}}{2} - \dfrac{1}{2}
\]
This identity is not correct.
\subsolution %c
\[
\sin^{2}{60\degree} - \sin^2{30\degree} = \left(\dfrac{\sqrt{3}}{2}\right)^2 - \left(\dfrac{1}{2}\right)^2 = \dfrac{3}{4} - \dfrac{1}{4} = \dfrac{1}{2}
\]
\[
\sin\left(60\degree + 30\degree\right)\sin\left(60\degree - 30\degree\right) = \sin{90\degree} \cdot \sin{30\degree} = 1 \cdot \dfrac{1}{2} = \dfrac{1}{2}
\]
This identity is correct for the given angles.

\end{subsolutions}

\solution %4
See Chapter 2, Section 12 and the Appendix of Chapter 2 to review some of the geometry used in this solution.
\begin{subsolutions}
\subsolution %a
Because $\angle{ABC}$ is subtended by the diameter $\overline{AC}$, $\angle{ABC}$ is a right angle and $\triangle{ABC}$ is a right triangle (this fact is known as \textit{Thales's Theorem}). Therefore, $\sin{\alpha}$ is equal to the length of the opposite side ($BC$) divided by the length of the hypotenuse ($AC$). $\overline{AC}$ is a diameter of the circle, so it has length 1. Thus, we have that $\sin{\alpha}$ is simply equal to $BC$.

A similar argument shows that $\triangle{ADC}$ is a right triangle with hypotenuse $\overline{AC}$ of length 1, which implies that $\sin{\beta} = DC$

\subsolution %b
Recall that chords of congruent circles which subtend equal angles are themselves equal. This implies that $BC$ in the diagram of part (a) is equal to $BC$ in the diagram of part (b) because in both diagrams, the chord $\overline{BC}$ subtends an angle of measure $\alpha$. Similarly, $DC$ is the same in both diagrams because in both diagrams, the chord $\overline{DC}$ subtends an angle of measure $\beta$. Therefore, $BC$ is still equal to $\sin{\alpha}$, and $DC$ is still equal to $\sin{\beta}$.

\subsolution %c
From part (b) above, we can conclude that a chord which subtends an inscribed angle with measure $\alpha$ in a circle with diameter 1 has length $\sin{\alpha}$. Thus, we draw $\overline{BD}$, the chord which subtends $\angle{BAD}$ in both figures and which consequently has length $\sin\left(\alpha+\beta\right)$.
\end{subsolutions}
Note that the above reasoning implies that the sine of an angle with measure less than $180\degree$ cannot exceed 1 since the diameter is the longest chord in a circle.

\solution %5
Recall that the sine of any angle is at most 1. Therefore,
\[
\sin{105 \degree} \leq 1 = \frac{1}{2} + \frac{1}{2} < \sin{45\degree} + \sin{60\degree},
\]
which shows that $\sin{105 \degree}$ cannot equal $\sin{45\degree} + \sin{60\degree}$.

\end{solutions}

\begin{solutions}{Page 125}
\solution %1
Addition formula for sine:
\begin{align*}
\sin\left(60\degree + 30\degree\right)
&= \sin{60\degree}\cos{30\degree} + \cos{60\degree}\sin{30\degree} \\
&= \dfrac{\sqrt{3}}{2} \cdot \dfrac{\sqrt{3}}{2} + \dfrac{1}{2} \cdot \dfrac{1}{2}\\
&= \dfrac{3}{4} + \dfrac{1}{4} \\
&= 1 \\
&= \sin{90\degree}
\end{align*}

Addition formula for cosine:
\begin{align*}
\cos\left(60\degree + 30\degree\right)
&= \cos{60\degree}\cos{30\degree} - \sin{60\degree}\sin{30\degree} \\
&= \dfrac{1}{2} \cdot \dfrac{\sqrt{3}}{2} - \dfrac{\sqrt{3}}{2} \cdot \dfrac{1}{2}\\
&= \dfrac{\sqrt{3}}{4} - \dfrac{\sqrt{3}}{4} \\
&= 0 \\
&= \cos{90\degree}
\end{align*}

Difference formula for sine:
\begin{align*}
\sin\left(60\degree - 30\degree\right)
&= \sin{60\degree}\cos{30\degree} - \cos{60\degree}\sin{30\degree} \\
&= \dfrac{\sqrt{3}}{2} \cdot \dfrac{\sqrt{3}}{2} - \dfrac{1}{2} \cdot \dfrac{1}{2}\\
&= \dfrac{3}{4} - \dfrac{1}{4} \\
&= \dfrac{1}{2} \\
&= \sin{30\degree}
\end{align*}

Difference formula for cosine:
\begin{align*}
\cos\left(60\degree - 30\degree\right)
&= \cos{60\degree}\cos{30\degree} + \sin{60\degree}\sin{30\degree} \\
&= \dfrac{1}{2} \cdot \dfrac{\sqrt{3}}{2} + \dfrac{\sqrt{3}}{2} \cdot \dfrac{1}{2}\\
&= \dfrac{\sqrt{3}}{4} + \dfrac{\sqrt{3}}{4} \\
&= \dfrac{\sqrt{3}}{2} \\
&= \cos{30\degree}
\end{align*}

\solution %2
Addition formula for sine ($\alpha = 0$):
\begin{align*}
\sin\left(0 + \beta\right) 
&= \sin{0}\cos{\beta} + \cos{0}\sin{\beta} \\
&= 0 \cdot \cos{\beta} + 1 \cdot \sin{\beta} \\
&= \sin{\beta}
\end{align*}

Addition formula for cosine ($\alpha = 0$):
\begin{align*}
\cos\left(0 + \beta\right) 
&= \cos{0}\cos{\beta} - \sin{0}\sin{\beta} \\
&= 1 \cdot \cos{\beta} - 0 \cdot \sin{\beta} \\
&= \cos{\beta}
\end{align*}

Difference formula for sine ($\alpha = 0$):
\begin{align*}
\sin\left(0 - \beta\right) 
&= \sin{0}\cos{\beta} - \cos{0}\sin{\beta} \\
&= 0 \cdot \cos{\beta} - 1 \cdot \sin{\beta} \\
&= -\sin{\beta}
\end{align*}
\quad Notice that this demonstrates that the sine function is \textit{odd}.

Difference formula for cosine ($\alpha = 0$):
\begin{align*}
\cos\left(0 - \beta\right) 
&= \cos{0}\cos{\beta} + \sin{0}\sin{\beta} \\
&= 1 \cdot \cos{\beta} + 0 \cdot \sin{\beta} \\
&= \cos{\beta}
\end{align*}
\quad Notice that this demonstrates that the cosine function is \textit{even}.

Addition formula for sine ($\beta = 0$):
\begin{align*}
\sin\left(\alpha + 0\right) 
&= \sin{\alpha}\cos{0} + \cos{\alpha}\sin{0} \\
&= \sin{\alpha} \cdot 1 + \cos{\alpha} \cdot 0 \\
&= \sin{\alpha}
\end{align*}

Addition formula for cosine ($\beta = 0$):
\begin{align*}
\cos\left(\alpha + 0\right) 
&= \cos{\alpha}\cos{0} - \sin{\alpha}\sin{0} \\
&= \cos{\alpha} \cdot 1 - \sin{\alpha} \cdot 0 \\
&= \cos{\alpha}
\end{align*}

Difference formula for sine ($\beta = 0$):
\begin{align*}
\sin\left(\alpha - 0\right) 
&= \sin{\alpha}\cos{0} - \cos{\alpha}\sin{0} \\
&= \sin{\alpha} \cdot 1 - \cos{\alpha} \cdot 0 \\
&= \sin{\alpha}
\end{align*}

Difference formula for cosine ($\beta = 0$):
\begin{align*}
\cos\left(\alpha + 0\right) 
&= \cos{\alpha}\cos{0} + \sin{\alpha}\sin{0} \\
&= \cos{\alpha} \cdot 1 + \sin{\alpha} \cdot 0 \\
&= \cos{\alpha}
\end{align*}

\solution %3
Following the hint, we notice that in a right triangle, the side opposite one of the acute angles is the side adjacent to the other acute angle. Thus, if $\alpha + \beta = \pi/2$, then $\sin{\alpha} = \cos{\beta}$ and $\sin{\beta} = \cos{\alpha}$ (see also Chapter 1, Section 4).
\begin{align*}
\sin\left(\alpha + \beta\right)
&= \sin{\alpha} \cos{\beta}  + \cos{\alpha}\sin{\beta} \\
&= \sin{\alpha}\sin{\alpha} + \cos{\alpha}\cos{\alpha} \\
&= \sin^{2}{\alpha} + \cos^{2}{\alpha} \\
&= 1
\end{align*}

\solution %4
Addition formula for sine:
\begin{align*}
\sin\left(\dfrac{\pi}{4} + \dfrac{\pi}{4}\right)
&= \sin{\dfrac{\pi}{4}}\cos{\dfrac{\pi}{4}} + \cos{\dfrac{\pi}{4}}\sin{\dfrac{\pi}{4}} \\
&= \dfrac{\sqrt{2}}{2} \cdot \dfrac{\sqrt{2}}{2} + \dfrac{\sqrt{2}}{2} \cdot \dfrac{\sqrt{2}}{2} \\
&= \dfrac{1}{2} + \dfrac{1}{2} \\
&= 1 \\
&= \sin{\dfrac{\pi}{2}}
\end{align*}

Addition formula for cosine:
\begin{align*}
\cos\left(\dfrac{\pi}{4} + \dfrac{\pi}{4}\right)
&= \cos{\dfrac{\pi}{4}}\cos{\dfrac{\pi}{4}} - \sin{\dfrac{\pi}{4}}\sin{\dfrac{\pi}{4}} \\
&= \dfrac{\sqrt{2}}{2} \cdot \dfrac{\sqrt{2}}{2} - \dfrac{\sqrt{2}}{2} \cdot \dfrac{\sqrt{2}}{2} \\
&= \dfrac{1}{2} - \dfrac{1}{2} \\
&= 0 \\
&= \cos{\dfrac{\pi}{2}}
\end{align*}

\solution %5
Recall that $\left(A \pm B\right)^2 = A^2 \pm 2AB + B^2$.
\begin{align*}
&\left(\sin{\alpha}\cos{\beta} + \cos{\alpha}\sin{\beta}\right)^2 + \left(\cos{\alpha}\cos{\beta} - \sin{\alpha}\sin{\beta}\right)^2 \\
&\qquad =\sin^{2}{\alpha}\cos^{2}{\beta} + 2\sin{\alpha}\cos{\beta}\cos{\alpha}\sin{\beta} + \cos^{2}{\alpha}\sin^{2}{\beta} + \\
&\qquad \phantom{=}\cos^{2}{\alpha}\cos^{2}{\beta} - 2\cos{\alpha}\cos{\beta}\sin{\alpha}\sin{\beta} + \sin^{2}{\alpha}\sin^{2}{\beta} \\
&\qquad =\sin^{2}{\alpha}\cos^{2}{\beta} + \cos^{2}{\alpha}\sin^{2}{\beta} + \cos^{2}{\alpha}\cos^{2}{\beta} + \sin^{2}{\alpha}\sin^{2}{\beta} \\
&\qquad =\sin^{2}{\alpha} \left(\cos^{2}{\beta} + \sin^{2}{\beta}\right) + \cos^{2}{\alpha} \left(\sin^{2}{\beta} + \cos^{2}{\beta}\right) \\
&\qquad = \sin^{2}{\alpha} \cdot 1 + \cos^{2}{\alpha} \cdot 1 \\
&\qquad = 1
\end{align*}

\solution %6
After expanding using the identity $\left(A+B\right)\left(A-B\right) = A^2-B^2$, we cleverly ``add by zero'' to get the desired result.
\begin{align*}
&\left(\sin{\alpha}\cos{\beta} + \cos{\alpha}\sin{\beta}\right)\left(\sin{\alpha}\cos{\beta} - \cos{\alpha}\sin{\beta}\right) \\
&\qquad =\sin^{2}{\alpha}\cos^{2}{\beta} - \cos^{2}{\alpha}\sin^{2}{\beta} \\
&\qquad =\sin^{2}{\alpha}\cos^{2}{\beta} + \sin^{2}{\alpha}\sin^{2}{\beta} - \sin^{2}{\alpha}\sin^{2}{\beta} - \cos^{2}{\alpha}\sin^{2}{\beta} \\
&\qquad =\sin^{2}{\alpha}\left(\cos^{2}{\beta} + \sin^{2}{\beta}\right) - \sin^{2}{\beta}\left(\sin^{2}{\alpha} + \cos^{2}{\alpha}\right) \\
&\qquad =\sin^{2}{\alpha} \cdot 1 - \sin^2{\beta} \cdot 1 \\
&\qquad =\sin^{2}{\alpha} - \sin^2{\beta}
\end{align*}
\end{solutions}

\begin{solutions}{Page 129}
\solution %1

\solution %2
\end{solutions}

\begin{solutions}{Page 131}
\solution %1
\begin{align*}
\sin\left(30\degree + 30\degree\right)
&= \sin{30\degree}\cos{30\degree} + \cos{30\degree}\sin{30\degree} \\
&= \dfrac{1}{2} \cdot \dfrac{\sqrt{3}}{2} + \dfrac{\sqrt{3}}{2} \cdot \dfrac{1}{2}\\
&= \dfrac{\sqrt{3}}{4} + \dfrac{\sqrt{3}}{4} \\
&= \dfrac{\sqrt{3}}{2} \\
&= \sin{60\degree}
\end{align*}

\begin{align*}
\cos\left(30\degree + 30\degree\right)
&= \cos{30\degree}\cos{30\degree} - \sin{30\degree}\sin{30\degree} \\
&= \dfrac{\sqrt{3}}{2} \cdot \dfrac{\sqrt{3}}{2} - \dfrac{1}{2} \cdot \dfrac{1}{2}\\
&= \dfrac{3}{4} - \dfrac{1}{4} \\
&= \dfrac{1}{2} \\
&= \cos{60\degree}
\end{align*}

\solution %2
Assuming $\alpha$ and $\beta$ are acute angles:
\[
\sin{\alpha} = \dfrac{3}{5} \implies \cos{\alpha} = \sqrt{1-\sin^{2}{\alpha}} = \sqrt{1-\left(3/5\right)^2} = \dfrac{4}{5}
\]

\[
\sin{\beta} = \dfrac{5}{13} \implies \cos{\beta} = \sqrt{1-\sin^{2}{\beta}} = \sqrt{1-\left(5/13\right)^2} = \dfrac{12}{13}
\]

\begin{align*}
\sin\left(\alpha+\beta\right) &= \sin{\alpha}\cos{\beta} + \cos{\alpha}\sin{\beta} \\
&= \dfrac{3}{5} \cdot \dfrac{12}{13} + \dfrac{4}{5} \cdot \dfrac{5}{13} \\
&= \dfrac{36}{65} + \dfrac{20}{65} \\
&= \dfrac{56}{65}
\end{align*}

\begin{align*}
\cos\left(\alpha+\beta\right) &= \cos{\alpha}\cos{\beta} - \sin{\alpha}\sin{\beta} \\
&= \dfrac{4}{5} \cdot \dfrac{12}{13} - \dfrac{3}{5} \cdot \dfrac{5}{13} \\
&= \dfrac{48}{65} - \dfrac{15}{65} \\
&= \dfrac{33}{65}
\end{align*}

\solution %3
\begin{align*}
\sin{75\degree} &= \sin\left(45\degree + 30\degree\right) \\
&= \sin{45\degree}\cos{30\degree} + \cos{45\degree}\sin{30\degree} \\
&= \dfrac{\sqrt{2}}{2} \cdot \dfrac{\sqrt{3}}{2} + \dfrac{\sqrt{2}}{2} \cdot \dfrac{1}{2} \\
&= \dfrac{\sqrt{6}+\sqrt{2}}{4}
\end{align*}

\begin{align*}
\cos{75\degree} &= \cos\left(45\degree + 30\degree\right) \\
&= \cos{45\degree}\cos{30\degree} - \sin{45\degree}\sin{30\degree} \\
&= \dfrac{\sqrt{2}}{2} \cdot \dfrac{\sqrt{3}}{2} - \dfrac{\sqrt{2}}{2} \cdot \dfrac{1}{2} \\
&= \dfrac{\sqrt{6}-\sqrt{2}}{4}
\end{align*}

\solution %4
\begin{align*}
\sin{15\degree} &= \sin\left(45\degree - 30\degree\right) \\
&= \sin{45\degree}\cos{30\degree} - \cos{45\degree}\sin{30\degree} \\
&= \dfrac{\sqrt{2}}{2} \cdot \dfrac{\sqrt{3}}{2} - \dfrac{\sqrt{2}}{2} \cdot \dfrac{1}{2} \\
&= \dfrac{\sqrt{6}-\sqrt{2}}{4}
\end{align*}

\begin{align*}
\cos{15\degree} &= \cos\left(45\degree - 30\degree\right) \\
&= \cos{45\degree}\cos{30\degree} + \sin{45\degree}\sin{30\degree} \\
&= \dfrac{\sqrt{2}}{2} \cdot \dfrac{\sqrt{3}}{2} + \dfrac{\sqrt{2}}{2} \cdot \dfrac{1}{2} \\
&= \dfrac{\sqrt{6}+\sqrt{2}}{4}
\end{align*}

Notice that $75\degree$ and $15\degree$ are complementary angles, so we know $\sin{75\degree}=\cos{15\degree}$ and $\sin{15\degree}=\cos{75\degree}$.

\solution %5
\begin{subsolutions}
\subsolution %a
Yes, let $\alpha = \beta = \pi/4$.
\begin{align*}
\cos\left(\dfrac{\pi}{4} + \dfrac{\pi}{4}\right) &= \cos{\dfrac{\pi}{4}}\cos{\dfrac{\pi}{4}} - \sin{\dfrac{\pi}{4}}\sin{\dfrac{\pi}{4}} \\
&= \dfrac{\sqrt{2}}{2} \cdot \dfrac{\sqrt{2}}{2} - \dfrac{\sqrt{2}}{2} \cdot \dfrac{\sqrt{2}}{2} \\
&= \dfrac{1}{2} - \dfrac{1}{2} \\
&= 0
\end{align*}
More generally, we could suppose $\alpha + \beta = \pi/2$ and follow the approach in Exercise 3 of Section 2 earlier in this chapter.

\subsolution %b
If $\alpha$ and $\beta$ are acute angles, then $0 < \alpha + \beta < \pi$. Using the unit circle, we can see that $\sin\left(\alpha+\beta\right)$ must be positive since the angle $\alpha+\beta$ lies in the upper-half of the plane, where the sine function is positive.

\subsolution %c
$\sin\left(\alpha+\beta\right) = \sin{\alpha}\cos{\beta} + \cos{\alpha}\sin{\beta}$ is positive when $\alpha$ and $\beta$ are acute angles since the sum and product of positive real numbers is also positive.

$\cos\left(\alpha+\beta\right)$ need not be positive. As shown in part (a) of this exercise, $\cos\left(\alpha+\beta\right)$ can equal 0. Furthermore, $\cos\left(\alpha+\beta\right)$ can be negative. Let $\alpha = \beta = \pi/3$. Then, assuming that we can extend the cosine addition formula to angles $\alpha$ and $\beta$ such that $\alpha + \beta$ is obtuse,
\begin{align*}
\cos\left(\dfrac{\pi}{3} + \dfrac{\pi}{3}\right) &= \cos{\dfrac{\pi}{3}}\cos{\dfrac{\pi}{3}} - \sin{\dfrac{\pi}{3}}\sin{\dfrac{\pi}{3}} \\
&= \dfrac{1}{2} \cdot \dfrac{1}{2} - \dfrac{\sqrt{3}}{2} \cdot \dfrac{\sqrt{3}}{2} \\
&= \dfrac{1}{4} - \dfrac{3}{4} \\
&= -\dfrac{1}{2}
\end{align*}
\end{subsolutions}

\solution %6
As we saw in Exercise 1 of Section 1 of this chapter, $\sin{\alpha} + \sin{\beta}$ does not equal $\sin\left(\alpha+\beta\right)$ in general. A similar table can be used to show that $\sin{\alpha} - \sin{\beta}$ does not equal $\sin\left(\alpha-\beta\right)$ in general.

\solution %7
This is not a coincidence. The identity holds true even when substituting more ``arbitrary'' values in for $\alpha$ and $\beta$. For example, using $\alpha=37\degree$ and $\beta=19\degree$, we find that both $\sin^{2}{\alpha} - \sin^{2}{\beta}$ and $\sin\left(\alpha+\beta\right)\sin\left(\alpha-\beta\right)$ are equal to approximately $0.2562$.

\solution %8
You may also refer to the proof in Exercise 6 of Section 2 of this chapter.
\begin{align*}
\sin\left(\alpha+\beta\right)\sin\left(\alpha-\beta\right) &= \left(\sin{\alpha}\cos{\beta} + \cos{\alpha}\sin{\beta}\right)\left(\sin{\alpha}\cos{\beta} - \cos{\alpha}\sin{\beta}\right)\\
&= \sin^{2}{\alpha}\cos^{2}{\beta} - \cos^{2}{\alpha}\sin^{2}{\beta} \\
&= \sin^{2}{\alpha}\cos^{2}{\beta} + \sin^{2}{\alpha}\sin^{2}{\beta} - \sin^{2}{\alpha}\sin^{2}{\beta} - \cos^{2}{\alpha}\sin^{2}{\beta} \\
&= \sin^{2}{\alpha}\left(\cos^{2}{\beta} + \sin^{2}{\beta}\right) - \sin^{2}{\beta}\left(\sin^{2}{\alpha} + \cos^{2}{\alpha}\right) \\
&= \sin^{2}{\alpha} \cdot 1 - \sin^2{\beta} \cdot 1 \\
&= \sin^{2}{\alpha} - \sin^2{\beta}
\end{align*}

\solution %9
This proof is nearly identical to the one in the previous part. We just make a small modification in how we ``add by zero'' in order to obtain to the desired result.
\begin{align*}
\sin\left(\alpha+\beta\right)\sin\left(\alpha-\beta\right) &= \left(\sin{\alpha}\cos{\beta} + \cos{\alpha}\sin{\beta}\right)\left(\sin{\alpha}\cos{\beta} - \cos{\alpha}\sin{\beta}\right)\\
&= \sin^{2}{\alpha}\cos^{2}{\beta} - \cos^{2}{\alpha}\sin^{2}{\beta} \\
&= \sin^{2}{\alpha}\cos^{2}{\beta} + \cos^{2}{\alpha}\cos^{2}{\beta} - \cos^{2}{\alpha}\cos^{2}{\beta} - \cos^{2}{\alpha}\sin^{2}{\beta} \\
&= \cos^{2}{\beta}\left(\sin^{2}{\alpha} + \cos^{2}{\alpha}\right) - \cos^{2}{\alpha}\left(\cos^{2}{\beta} + \sin^{2}{\beta}\right) \\
&= \cos^{2}{\beta} \cdot 1 - \cos^2{\alpha} \cdot 1 \\
&= \cos^{2}{\beta} - \cos^2{\alpha}
\end{align*}

\solution %10
We apply the sine addition formula in reverse.
\begin{align*}
\sin{18\degree}\cos{12\degree} + \cos{18\degree}\sin{12\degree} &= \sin\left(18\degree + 12\degree\right) \\
&= \sin{30\degree} \\
&= \dfrac{1}{2}
\end{align*}

\solution %11
\begin{subsolutions}
\subsolution %a
Since we have not proved that the sine addition formula works for all angles $\alpha$ and $\beta$, we use properties of the sine and cosine functions to avoid working with angles larger than $90\degree$.
\begin{align*}
&\sin{113\degree}\cos{307\degree} + \cos{113\degree}\sin{307\degree} \\
&\qquad = \sin\left(180\degree-67\degree\right)\cos\left(360\degree-53\degree\right) + \cos\left(180\degree-67\degree\right)\sin\left(360\degree-53\degree\right) \\
&\qquad = \sin{67\degree}\cos{53\degree} + \left(-\cos{67\degree}\right)\left(-\sin{53\degree}\right) \\
&\qquad = \sin\left(67\degree + 53\degree\right) \\
&\qquad = \sin{120\degree} \\
&\qquad = \sin{60\degree} \\
&\qquad = \dfrac{\sqrt{3}}{2}
\end{align*}

\subsolution %b
Plugging into a calculator,
\[
\sin{113\degree}\cos{307\degree} + \cos{113\degree}\sin{307\degree} \approx 0.866 \approx \dfrac{\sqrt{3}}{2} = \sin{60\degree}
\]
\subsolution %c
Assuming that the sine addition formula does work for non-acute angles, we arrive at the same result.
\begin{align*}
\sin{113\degree}\cos{307\degree} + \cos{113\degree}\sin{307\degree} &= \sin\left(113\degree + 307\degree\right) \\
&= \sin{420\degree} \\
&= \sin{60\degree} \\
&= \dfrac{\sqrt{3}}{2}
\end{align*}

\end{subsolutions}

\solution %12
We can use the addition formulas for sine and cosine by rewriting $2\alpha$ as $\alpha + \alpha$.
\begin{align*}
\sin{2\alpha}\cos{\alpha} - \cos{2\alpha}\sin{\alpha} &= \sin\left(\alpha + \alpha\right)\cos{\alpha} - \cos\left(\alpha + \alpha\right)\sin{\alpha} \\ 
&= \left(\sin{\alpha}\cos{\alpha} + \cos{\alpha}\sin{\alpha}\right)\cos{\alpha} - \left(\cos{\alpha}\cos{\alpha}-\sin{\alpha}\sin{\alpha}\right)\sin{\alpha} \\
&= \sin{\alpha}\cos^{2}{\alpha} + \sin{\alpha}\cos^{2}{\alpha} - \sin{\alpha}\cos^{2}{\alpha} + \sin^{3}{\alpha} \\
&=\sin{\alpha}\cos^{2}{\alpha} + \sin^{3}{\alpha} \\
&= \sin{\alpha}\left(\cos^{2}\alpha+\sin^{2}\alpha\right) \\
&= \sin{\alpha}
\end{align*}

\solution %13
\begin{align*}
\sin\left(\alpha+\beta\right)\sin{\beta} + \cos\left(\alpha+\beta\right)\cos{\beta} &= \left(\sin{\alpha}\cos{\beta} + \cos{\alpha}\sin{\beta}\right)\sin{\beta} + \left(\cos{\alpha}\cos{\beta}-\sin{\alpha}\sin{\beta}\right)\cos{\beta} \\
&= \sin{\alpha}\sin{\beta}\cos{\beta} + \sin^{2}{\beta}\cos{\alpha} + \cos{\alpha}\cos^{2}{\beta} -\sin{\alpha}\sin{\beta}\cos{\beta} \\
&= \sin^{2}{\beta}\cos{\alpha} + \cos{\alpha}\cos^{2}{\beta} \\
&= \cos{\alpha}\left(\sin^{2}{\beta}+\cos^{2}{\beta}\right) \\
&= \cos{\alpha}
\end{align*}

\solution %14
\begin{align*}
\dfrac{\sin\left(\alpha+\beta\right)-\cos{\alpha}\sin{\beta}}{\cos\left(\alpha+\beta\right)+\sin{\alpha}\sin{\beta}} &= \dfrac{\sin{\alpha}\cos{\beta}+\cos{\alpha}\sin{\beta}-\cos{\alpha}\sin{\beta}}{\cos{\alpha}\cos{\beta}-\sin{\alpha}\sin{\beta}+\sin{\alpha}\sin{\beta}} \\
&= \dfrac{\sin{\alpha}\cos{\beta}}{\cos{\alpha}\cos{\beta}} \\
&= \dfrac{\sin{\alpha}}{\cos{\alpha}} \\
&= \tan{\alpha}
\end{align*}

\solution %15
\begin{align*}
\sin\left(\alpha + \dfrac{\pi}{4}\right) &= \sin{\alpha}\cos{\dfrac{\pi}{4}} + \cos{\alpha}\sin{\dfrac{\pi}{4}} \\
&= \sin{\alpha} \cdot \dfrac{\sqrt{2}}{2} + \cos{\alpha} \cdot \dfrac{\sqrt{2}}{2} \\
&= \dfrac{\sqrt{2}}{2} \left(\sin{\alpha} + \cos{\alpha}\right)
\end{align*}

\solution %16
\begin{align*}
\dfrac{\cos\left(\alpha+\beta\right)}{\cos{\alpha}\cos{\beta}} &= \dfrac{\cos{\alpha}\cos{\beta}-\sin{\alpha}\sin{\beta}}{\cos{\alpha}\cos{\beta}} \\
&= 1 - \dfrac{\sin{\alpha}\sin{\beta}}{\cos{\alpha}\cos{\beta}} \\
&= 1 - \dfrac{\sin{\alpha}}{\cos{\alpha}} \cdot \dfrac{\sin{\beta}}{\cos{\beta}} \\
&= 1 - \tan{\alpha}\tan{\beta}
\end{align*}

\solution %17
Applying the law of cosines, we have that $\left(b_{1} + b_{2}\right)^2 = c_{1}^2 + c_{2}^2 - 2c_1c_2\cos\left(\alpha+\beta\right)$. Solving for $\cos\left(\alpha+\beta\right)$, we get
\[
\cos\left(\alpha+\beta\right) = \dfrac{c_{1}^2 + c_{2}^2 - \left(b_{1}+b_{2}\right)^2}{2c_{1}c_{2}}.
\]
Before proceeding further, let's establish some relationships between the variables in the diagram. First, by the Pythagorean theorem, we have that $h^2 = c_{1}^2 - b_{1}^2 = c_{2}^2 - b_{2}^2$. Additionally, we can compute the sines and cosines for the angles $\alpha$ and $\beta$:
\[
\sin{\alpha} = \dfrac{b_{1}}{c_{1}}, \sin{\beta} = \dfrac{b_{2}}{c_{2}}, \cos{\alpha} = \dfrac{h}{c_{1}}, \cos{\beta} = \dfrac{h}{c_{2}}.
\]
We can now simplify our expression for $\cos\left(\alpha+\beta\right)$.
\begin{align*}
\cos\left(\alpha+\beta\right) &= \dfrac{c_{1}^2 + c_{2}^2 - \left(b_{1}+b_{2}\right)^2}{2c_{1}c_{2}} \\
&= \dfrac{c_{1}^2 + c_{2}^2 - b_{1}^2 - 2b_{1}b_{2} - b_{2}^2}{2c_{1}c_{2}} \\
&= \dfrac{2h^2 - 2b_{1}b_{2}}{2c_{1}c_{2}} \\
&= \dfrac{h^2 - b_{1}b_{2}}{c_{1}c_{2}} \\
&= \dfrac{h}{c_{1}} \cdot \dfrac{h}{c_{2}} - \dfrac{b_{1}}{c_{1}} \cdot \dfrac{b_{2}}{c_{2}} \\
&= \cos{\alpha}\cos{\beta} - \sin{\alpha}\sin{\beta}
\end{align*}
\end{solutions}


\section*{Chapter 7: Trigonometric Identities}

\begin{solutions}{Page 141}
\solution %1
\begin{subsolutions}
\subsolution %a
Yes
\subsolution %b
No
\subsolution %c
Yes
\subsolution %d
Yes
\subsolution %e
No
\subsolution %f
No
\end{subsolutions}
\solution %2
\begin{subsolutions}
\subsolution %a
\[
\tan{\alpha} = \dfrac{\sin{\alpha}}{\cos{\alpha}}
\]

\subsolution %b
\begin{align*}
\left(1+\tan{\alpha}\right)\left(1-\tan{\alpha}\right) &= 1-\tan^{2}{\alpha} \\
&= 1- \dfrac{\sin^{2}{\alpha}}{\cos^{2}{\alpha}} \\
&= \dfrac{\cos^{2}{\alpha}-\sin^{2}{\alpha}}{\cos^{2}{\alpha}}
\end{align*}

\subsolution %c
\begin{align*}
\dfrac{\tan{\alpha}+\tan{\beta}}{1-\tan{\alpha}\tan{\beta}} &= \dfrac{\tan{\alpha}+\tan{\beta}}{1-\tan{\alpha}\tan{\beta}} \cdot \dfrac{\cos{\alpha}\cos{\beta}}{\cos{\alpha}\cos{\beta}} \\
&= \dfrac{\sin{\alpha}\cos{\beta}+\cos{\alpha}\sin{\beta}}{\cos{\alpha}\cos{\beta}-\sin{\alpha}\sin{\beta}} %\\
%&=\dfrac{\sin\left(\alpha + \beta\right)}{\cos\left(\alpha + \beta\right)}
\end{align*}

\subsolution %d
\begin{align*}
\tan^{2}{\alpha}+\cot^{2}{\alpha} &= \dfrac{\sin^{2}{\alpha}}{\cos^{2}{\alpha}} + \dfrac{\cos^{2}{\alpha}}{\sin^{2}{\alpha}} \\
&= \dfrac{\sin^{4}{\alpha}+\cos^{4}{\alpha}}{\sin^{2}{\alpha}\cos^{2}{\alpha}}
\end{align*}

\subsolution %e
\begin{align*}
\tan{\alpha}\cot{\alpha} &= \dfrac{\sin{\alpha}}{\cos{\alpha}} \cdot \dfrac{\cos{\alpha}}{\sin{\alpha}} \\
&= 1
\end{align*}

\subsolution %f
\begin{align*}
1+\tan^{2}{\alpha} &= 1 + \dfrac{\sin^{2}{\alpha}}{\cos^{2}{\alpha}} \\
&= \frac{\cos^{2}{\alpha} + \sin^{2}{\alpha}}{\cos^{2}{\alpha}} \\
&= \dfrac{1}{\cos^{2}{\alpha}}
\end{align*}

\end{subsolutions}
\solution %3
The Principle of Analytic Continuation does not apply because $\sqrt{1-\sin^{2}{\alpha}}$ is not a rational trigonometric function. The identity is incorrect for $\alpha=2\pi / 3$ as $\cos\left(2\pi / 3\right) = -1/2$, while $\sqrt{1-\sin^{2}\left(2\pi / 3\right)} = 1/2$.
\solution %4
The Principle of Analytic Continuation does apply because both $\sin^{2}{\alpha}+\cos^{2}{\alpha}$ and $1$ are rational trigonometric functions. The identity is correct for $\alpha=2\pi / 3$.
\[
\sin^{2}{\dfrac{2\pi}{3}}+\cos^{2}{\dfrac{2\pi}{3}} = \left(\dfrac{\sqrt{3}}{2}\right)^2 + \left(\dfrac{1}{2}\right)^2 = \dfrac{3}{4} + \dfrac{1}{4} = 1
\]
\end{solutions}

\begin{solutions}{Page 142}
\solution %1
Since $\alpha$ and $\beta$ are acute angles,
\[
\sin{\alpha} = \dfrac{3}{5} \implies \cos{\alpha} = \sqrt{1-\sin^{2}{\alpha}} = \sqrt{1-\left(\dfrac{3}{5}\right)^2} = \dfrac{4}{5}
\]
\[
\sin{\beta} = \dfrac{5}{13} \implies \cos{\beta} = \sqrt{1-\sin^{2}{\beta}} = \sqrt{1-\left(\dfrac{5}{13}\right)^2} = \dfrac{12}{13}
\]
\begin{align*}
\sin\left(\alpha+\beta\right) &= \sin{\alpha}\cos{\beta} + \cos{\alpha}\sin{\beta} \\
&= \dfrac{3}{5} \cdot \dfrac{12}{13} + \dfrac{4}{5} \cdot \dfrac{5}{13} \\
&= \dfrac{36}{65} + \dfrac{20}{65} \\
&= \dfrac{56}{65}
\end{align*}
\begin{align*}
\cos\left(\alpha+\beta\right) &= \cos{\alpha}\cos{\beta} - \sin{\alpha}\sin{\beta} \\
&= \dfrac{4}{5} \cdot \dfrac{12}{13} - \dfrac{3}{5} \cdot \dfrac{5}{13} \\
&= \dfrac{48}{65} - \dfrac{15}{65} \\
&= \dfrac{33}{65}
\end{align*}
$\alpha+\beta$ lies in the first quadrant because $\sin\left(\alpha+\beta\right)$ and $\cos\left(\alpha+\beta\right)$ are both positive.

\solution %2
Since $\alpha$ and $\beta$ are acute angles,
\[
\sin{\alpha} = \dfrac{4}{5} \implies \cos{\alpha} = \sqrt{1-\sin^{2}{\alpha}} = \sqrt{1-\left(\dfrac{4}{5}\right)^2} = \dfrac{3}{5}
\]
\[
\sin{\beta} = \dfrac{12}{13} \implies \cos{\beta} = \sqrt{1-\sin^{2}{\beta}} = \sqrt{1-\left(\dfrac{12}{13}\right)^2} = \dfrac{5}{13}
\]
\begin{align*}
\sin\left(\alpha+\beta\right) &= \sin{\alpha}\cos{\beta} + \cos{\alpha}\sin{\beta} \\
&= \dfrac{4}{5} \cdot \dfrac{5}{13} + \dfrac{3}{5} \cdot \dfrac{12}{13} \\
&= \dfrac{20}{65} + \dfrac{36}{65} \\
&= \dfrac{56}{65}
\end{align*}
\begin{align*}
\cos\left(\alpha+\beta\right) &= \cos{\alpha}\cos{\beta} - \sin{\alpha}\sin{\beta} \\
&= \dfrac{3}{5} \cdot \dfrac{5}{13} - \dfrac{4}{5} \cdot \dfrac{12}{13} \\
&= \dfrac{15}{65} - \dfrac{48}{65} \\
&= -\dfrac{33}{65}
\end{align*}

$\alpha+\beta$ lies in the second quadrant because $\sin\left(\alpha+\beta\right)$ is positive and $\cos\left(\alpha+\beta\right)$ is negative.

\solution %3
\[
\sin{\alpha} = \dfrac{3}{5} \implies \cos{\alpha} = \pm \sqrt{1-\sin^{2}{\alpha}} = \pm \sqrt{1-\left(\dfrac{3}{5}\right)^2} = \pm \dfrac{4}{5}
\]
\[
\sin{\beta} = \dfrac{5}{13} \implies \cos{\beta} = \pm \sqrt{1-\sin^{2}{\beta}} = \pm \sqrt{1-\left(\dfrac{5}{13}\right)^2} = \pm \dfrac{12}{13}
\]
$\cos{\alpha} > 0$, $\cos{\beta} > 0$:
\begin{align*}
\sin\left(\alpha+\beta\right) &= \sin{\alpha}\cos{\beta} + \cos{\alpha}\sin{\beta} \\
&= \dfrac{3}{5} \cdot \dfrac{12}{13} + \dfrac{4}{5} \cdot \dfrac{5}{13} \\
&= \dfrac{36}{65} + \dfrac{20}{65} \\
&= \dfrac{56}{65}
\end{align*}

$\cos{\alpha} > 0$, $\cos{\beta} < 0$:
\begin{align*}
\sin\left(\alpha+\beta\right) &= \sin{\alpha}\cos{\beta} + \cos{\alpha}\sin{\beta} \\
&= \dfrac{3}{5} \left(-\dfrac{12}{13}\right) + \dfrac{4}{5} \cdot \dfrac{5}{13} \\
&= -\dfrac{36}{65} + \dfrac{20}{65} \\
&= -\dfrac{16}{65}
\end{align*}

$\cos{\alpha} < 0$, $\cos{\beta} > 0$:
\begin{align*}
\sin\left(\alpha+\beta\right) &= \sin{\alpha}\cos{\beta} + \cos{\alpha}\sin{\beta} \\
&= \dfrac{3}{5} \cdot \dfrac{12}{13} + \left(-\dfrac{4}{5}\right) \cdot \dfrac{5}{13} \\
&= \dfrac{36}{65} - \dfrac{20}{65} \\
&= \dfrac{16}{65}
\end{align*}

$\cos{\alpha} < 0$, $\cos{\beta} < 0$:
\begin{align*}
\sin\left(\alpha+\beta\right) &= \sin{\alpha}\cos{\beta} + \cos{\alpha}\sin{\beta} \\
&= \dfrac{3}{5} \left(-\dfrac{12}{13}\right) + \left(-\dfrac{4}{5}\right) \cdot \dfrac{5}{13} \\
&= -\dfrac{36}{65} - \dfrac{20}{65} \\
&= -\dfrac{56}{65}
\end{align*}

There are four possible answers for $\sin\left(\alpha+\beta\right)$.

\solution %4
\begin{subsolutions}
\subsolution %a
\begin{align*}
\sin{\dfrac{2\pi}{3}}\cos{\dfrac{\pi}{3}} - \cos{\dfrac{2\pi}{3}}\sin{\dfrac{\pi}{3}} &= \dfrac{\sqrt{3}}{2} \cdot \dfrac{1}{2} - \left(-\dfrac{1}{2}\right) \dfrac{\sqrt{3}}{2} \\
&= \dfrac{\sqrt{3}}{4} + \dfrac{\sqrt{3}}{4} \\
&= \dfrac{\sqrt{3}}{2} \\
&= \sin\left(\dfrac{\pi}{3}\right) \\
&= \sin\left(\dfrac{2\pi}{3} - \dfrac{\pi}{3}\right)
\end{align*}

\subsolution %b
\begin{align*}
\sin{\dfrac{\pi}{4}}\cos{\dfrac{3\pi}{4}} - \cos{\dfrac{\pi}{4}}\sin{\dfrac{3\pi}{4}} &= \dfrac{\sqrt{2}}{2} \left(-\dfrac{\sqrt{2}}{2}\right) - \dfrac{\sqrt{2}}{2} \cdot \dfrac{\sqrt{2}}{2} \\
&= -\dfrac{1}{2} - \dfrac{1}{2} \\
&= -1 \\
&= \sin\left(-\dfrac{\pi}{2}\right) \\
&= \sin\left(\dfrac{\pi}{4} - \dfrac{3\pi}{4}\right)
\end{align*}

\subsolution %c
\begin{align*}
\sin\left(-\dfrac{\pi}{6}\right)\cos{\dfrac{3\pi}{2}} - \cos\left(-\dfrac{\pi}{6}\right)\sin{\dfrac{3\pi}{2}} &= -\dfrac{1}{2} \cdot 0 - \dfrac{\sqrt{3}}{2} \left(-1\right) \\
&= \dfrac{\sqrt{3}}{2} \\
&= \sin\left(\dfrac{\pi}{3}\right) \\
&= \sin\left(-\dfrac{5\pi}{3}\right) \\
&= \sin\left(-\dfrac{\pi}{6} - \dfrac{3\pi}{2}\right)
\end{align*}

\end{subsolutions}

\solution %5
Applying the identity
\[
(A-B)^{2} + (A+B)^{2} = A^2-2AB+B^2 + A^2+2AB+B^2 = 2A^2 + 2B^2,
\]
we have that,
\begin{align*}
    \cos^{2}\left(\gamma+\delta\right) + \cos^{2}\left(\gamma-\delta\right) &= \left(\cos{\gamma}\cos{\delta} - \sin{\gamma}\sin{\delta}\right)^2 + \left(\cos{\gamma}\cos{\delta} + \sin{\gamma}\sin{\delta}\right)^2 \\
    &= 2\cos^{2}{\gamma}\cos^{2}{\delta} + 2\sin^{2}{\gamma}\sin^{2}{\delta}.
\end{align*}

Therefore,
\begin{align*}
\cos^{2}{\alpha} + \cos^{2}\left(\dfrac{2\pi}{3} + \alpha\right) +  \cos^{2}\left(\dfrac{2\pi}{3} - \alpha\right) &= \cos^{2}{\alpha} + 2\cos^{2}{\dfrac{2\pi}{3}}\cos^{2}{\alpha} + 2\sin^{2}{\dfrac{2\pi}{3}}\sin^{2}{\alpha} \\
&= \cos^{2}{\alpha} + 2\left(\dfrac{1}{4}\right)\cos^{2}{\alpha} + 2\left(\dfrac{3}{4}\right)\sin^{2}{\alpha} \\
&= \dfrac{3}{2} \left(\cos^{2}{\alpha} + \sin^{2}{\alpha}\right) \\
&= \dfrac{3}{2}
\end{align*}

\solution %6
\begin{align*}
\sin\left(x+y\right) + \sin\left(x-y\right) &= \sin{x}\cos{y} + \cos{x}\sin{y} + \sin{x}\cos{y} - \cos{x}\sin{y} \\
&= \sin{x}\cos{y} + \sin{x}\cos{y} \\
&= 2\sin{x}\cos{y}
\end{align*}

\solution %7
\begin{align*}
\cos\left(x+y\right) + \cos\left(x-y\right) &= \cos{x}\cos{y} - \sin{x}\sin{y} + \cos{x}\cos{y} + \sin{x}\sin{y} \\
&= \cos{x}\cos{y} + \cos{x}\cos{y} \\
&= 2\cos{x}\cos{y}
\end{align*}

\solution %8
Since $(A-B)(A+B)=A^2-B^2$,
\begin{align*}
\cos\left(x+y\right)\cos\left(x-y\right) &= \left(\cos{x}\cos{y} - \sin{x}\sin{y}\right) \left(\cos{x}\cos{y} + \sin{x}\sin{y}\right) \\
&= \cos^{2}{x}\cos^{2}{y} - \sin^{2}{x}\sin^{2}{y}
\end{align*}

\solution %9
Since $(A-B)(A+B)=A^2-B^2$,
\begin{align*}
\sin\left(x+y\right)\sin\left(x-y\right) &= \left(\sin{x}\cos{y} + \cos{x}\sin{y}\right) \left(\sin{x}\cos{y} - \cos{x}\sin{y}\right) \\
&= \sin^{2}{x}\cos^{2}{y} - \cos^{2}{x}\sin^{2}{y}
\end{align*}

\solution %10
\begin{align*}
\cos\left(x+y\right)\cos\left(x-y\right) - \sin\left(x+y\right)\sin\left(x-y\right) &= \cos^{2}{x}\cos^{2}{y} - \sin^{2}{x}\sin^{2}{y} - \left(\sin^{2}{x}\cos^{2}{y} - \cos^{2}{x}\sin^{2}{y}\right) \\
&= \cos^{2}{x}\left(\cos^{2}{y} + \sin^{2}{y}\right) - \sin^{2}{x}\left(\sin^2{y}+\cos^{2}{y}\right) \\
&= \cos^{2}{x} - \sin^{2}{x}
\end{align*}

\solution %11
\begin{align*}
\cos{2x} &= \cos\left(x+x\right) \\
&= \cos{x}\cos{x}-\sin{x}\sin{x} \\
&= \cos^{2}{x} - \sin^{2}{x}
\end{align*}
There is no error, because $\cos{2x}=\cos^{2}{x} - \sin^{2}{x}$.

\solution %12
\begin{align*}
\cos\left(\alpha+\beta\right)\cos{\beta} + \sin\left(\alpha+\beta\right)\sin{\beta} &= \left(\cos{\alpha}\cos{\beta}-\sin{\alpha}\sin{\beta}\right)\cos{\beta} + \left(\sin{\alpha}\cos{\beta}+\cos{\alpha}\sin{\beta}\right)\sin{\beta} \\
&= \cos{\alpha}\cos^{2}{\beta} - \sin{\alpha}\sin{\beta}\cos{\beta} + \sin{\alpha}\sin{\beta}\cos{\beta}+\sin^{2}{\beta}\cos{\alpha} \\
&= \cos{\alpha}\cos^{2}{\beta} + \sin^{2}{\beta}\cos{\alpha} \\
&= \cos{\alpha} \left(\cos^{2}{\beta}+\sin^{2}{\beta}\right) \\
&= \cos{\alpha}
\end{align*}
Alternatively, by applying the cosine difference formula in reverse,
\[
\cos\left(\alpha+\beta\right)\cos{\beta} + \sin\left(\alpha+\beta\right)\sin{\beta} = \cos\left(\alpha+\beta - \beta\right) \\
= \cos{\alpha}
\]
\end{solutions}

\begin{solutions}{Page 144}
\solution %1
\begin{align*}
\tan\left(\dfrac{7\pi}{6}+\dfrac{5\pi}{3}\right) &= \dfrac{\tan{\dfrac{7\pi}{6}} + \tan{\dfrac{5\pi}{3}}}{1-\tan{\dfrac{7\pi}{6}}  \tan{\dfrac{5\pi}{3}}} \\
&= \dfrac{1/\sqrt{3} - \sqrt{3}}{1-\left(1/\sqrt{3}\right)\left(-\sqrt{3}\right)} \\
&= \dfrac{1/\sqrt{3} - \sqrt{3}}{2} \\
&= \dfrac{1/\sqrt{3} - \sqrt{3}}{2} \cdot \dfrac{\sqrt{3}}{\sqrt{3}}\\
&= \dfrac{1 - 3}{2\sqrt{3}} \\
&= -\dfrac{1}{\sqrt{3}} \\
&= \tan\left(-\dfrac{\pi}{6}\right) \\
&= \tan\left(\dfrac{17\pi}{6}\right) \\
&= \tan\left(\dfrac{7\pi}{6} + \dfrac{5\pi}{3}\right)
\end{align*}

\solution %2
Because the tangent function is odd, we know $\tan{-\beta}=-\tan{\beta}$.
\begin{align*}
\tan\left(\alpha-\beta\right) &= \tan\left(\alpha + \left(-\beta\right)\right) \\
&= \dfrac{\tan{\alpha} + \tan{-\beta}}{1-\tan{\alpha}  \tan{-\beta}} \\
&= \dfrac{\tan{\alpha} - \tan{\beta}}{1+\tan{\alpha}  \tan{\beta}}
\end{align*}

\solution %3
\begin{align*}
\tan\left(\dfrac{\pi}{4}+\alpha\right) &= \dfrac{\tan{\dfrac{\pi}{4}} + \tan{\alpha}}{1-\tan{\dfrac{\pi}{4}}  \tan{\alpha}} \\
&= \dfrac{1 + \tan{\alpha}}{1 - \tan{\alpha}}
\end{align*}

\solution %4
\begin{align*}
\tan\left(\dfrac{\pi}{4}-\alpha\right) &= \dfrac{\tan{\dfrac{\pi}{4}} - \tan{\alpha}}{1+\tan{\dfrac{\pi}{4}}  \tan{\alpha}} \\
&= \dfrac{1 - \tan{\alpha}}{1 + \tan{\alpha}}
\end{align*}

\solution %5
Since $\beta = \pi/4 - \alpha$,
\begin{align*}
\left(1+\tan{\alpha}\right)\left(1+\tan{\beta}\right) &= \left(1+\tan{\alpha}\right)\left(1+\tan\left(\dfrac{\pi}{4}-\alpha\right)\right) \\
&= \left(1 + \tan{\alpha}\right) \left(1 + \dfrac{1 - \tan{\alpha}}{1 + \tan{\alpha}}\right) \\
&= 1 + \tan{\alpha} + 1 - \tan{\alpha} \\
&= 2
\end{align*}

Alternatively, since $\tan\left(\alpha+\beta\right) = \tan \pi/4 = 1$,
\begin{align*}
\left(1+\tan{\alpha}\right)\left(1+\tan{\beta}\right) &= 1 + \tan{\alpha} + \tan{\beta} + \tan{\alpha}\tan{\beta} \\
&= 1 + \left(\tan{\alpha} + \tan{\beta}\right)\left(\dfrac{1-\tan{\alpha}\tan{\beta}}{1-\tan{\alpha}\tan{\beta}}\right) + \tan{\alpha}\tan{\beta} \\
&= 1 + \left(1-\tan{\alpha}\tan{\beta}\right)\left(\dfrac{\tan{\alpha} + \tan{\beta}}{1-\tan{\alpha}\tan{\beta}}\right) + \tan{\alpha}\tan{\beta} \\
&= 1 + \left(1-\tan{\alpha}\tan{\beta}\right)\tan\left(\alpha+\beta\right) + \tan{\alpha}\tan{\beta} \\
&= 1+ 1-\tan{\alpha}\tan{\beta} + \tan{\alpha}\tan{\beta} \\
&= 2
\end{align*}

\solution %6
\begin{align*}
\tan\left(\alpha+\beta+\gamma\right) &= \dfrac{\tan\left(\alpha+\beta\right) + \tan{\gamma}}{1-\tan\left(\alpha+\beta\right)\tan{\gamma}} \\
&= \dfrac{\dfrac{\tan{\alpha} + \tan{\beta}}{1-\tan{\alpha}\tan{\beta}} + \tan\gamma}{1-\dfrac{\tan{\alpha} + \tan{\beta}}{1-\tan{\alpha}\tan{\beta}}\tan\gamma} \\
&= \dfrac{\dfrac{\tan{\alpha} + \tan{\beta}}{1-\tan{\alpha}\tan{\beta}} + \tan\gamma}{1-\dfrac{\tan{\alpha} + \tan{\beta}}{1-\tan{\alpha}\tan{\beta}}\tan\gamma} \cdot \dfrac{1-\tan{\alpha}\tan{\beta}}{1-\tan{\alpha}\tan{\beta}} \\
&= \dfrac{\tan{\alpha}+\tan{\beta}+\tan{\gamma}\left(1-\tan{\alpha}\tan{\beta}\right)}{1 - \tan{\alpha}\tan{\beta} - \left(\tan{\alpha} + \tan{\beta}\right)\tan{\gamma}} \\
&= \dfrac{\tan{\alpha}+\tan{\beta}+\tan{\gamma} -\tan{\alpha}\tan{\beta}\tan{\gamma}}{1 - \tan{\alpha}\tan{\beta} - \tan{\alpha}\tan{\gamma} - \tan{\beta}\tan{\gamma}}
\end{align*}

\solution %7
Since $\tan\left(\alpha+\beta+\gamma\right) = \tan{\pi} = 0$, from the previous part, we have
\[
\tan{\alpha}+\tan{\beta}+\tan{\gamma} -\tan{\alpha}\tan{\beta}\tan{\gamma} = 0 \implies \tan{\alpha}+\tan{\beta}+\tan{\gamma} = \tan{\alpha}\tan{\beta}\tan{\gamma}.
\]
This is because if a fraction equals zero, its numerator must be zero.

\solution %8
First, applying the tangent addition formula,
\[
\tan{3\alpha} = \tan\left(2\alpha+\alpha\right) = \dfrac{\tan{2\alpha} + \tan{\alpha}}{1-\tan{2\alpha}\tan{\alpha}} \implies \tan{2\alpha} + \tan{\alpha} = \tan{3\alpha}\left(1-\tan{2\alpha}\tan{\alpha}\right).
\]
Therefore,
\[
\tan{3\alpha} - \tan{2\alpha} - \tan{\alpha} = \tan{3\alpha} - \tan{3\alpha}\left(1-\tan{2\alpha}\tan{\alpha}\right) = \tan{3\alpha}\tan{2\alpha}\tan{\alpha}.
\]
\end{solutions}

\begin{solutions}{Page 147}
\solution %1
    \begin{subsolutions}
        \solution %a
        Since $\sin^2 \alpha + \cos^2\alpha = 1$, we know $\cos\alpha = \sqrt{1-(\frac{7}{25})^2}$ (and cannot be the negative version because $\cos \alpha$ is given as positive).\\
        Thus $\sin 2\alpha = 2\sin\alpha \cos \alpha = 2\cdot \frac{7}{25}\cdot\sqrt{1-(\frac{7}{25})^2}=0.5376$.\\
        And $\cos2\alpha =1-2\sin^2\alpha = 1-(\frac{7}{25})^2=0.9216$.
        \solution %b
        This part is similar except that we use the negative version of cosine, namely $\cos\alpha=-\sqrt{1-(\frac{7}{25})^2}$.\\
        Thus $\sin 2\alpha = 2\sin\alpha \cos \alpha = -0.5376$.\\
        However, cosine value remains the same: $\cos2\alpha =1-2\sin^2\alpha = 1-(\frac{7}{25})^2=0.9216$.
    \end{subsolutions}

\solution %2
Firstly, $\sin 2\alpha=2\sin\alpha\cos\alpha$. In other words, the sine value will be a product of rational numbers, so it will also be rational. Similarly, $\cos 2\alpha=2\cos ^2 \alpha - 1$ will be rational because $\cos\alpha$ is rational. Exercise 1 confirms this result. (The decimal solutions for Exercise 1 are not rounded.)

\solution %3
Let's use the double angle formula $\cos 2\alpha=2\cos^2\alpha -1$.
\[
\cos{2\alpha} = \cos^{2}{\alpha} \implies 2\cos^{2}{\alpha}-1 = \cos^{2}{\alpha}  \implies \cos^{2}{\alpha} = 1 \implies \cos{\alpha} = \pm 1
\]
$\cos{\alpha}$ has a magnitude of 1 precisely when $\alpha$ is an integer multiple of $\pi$, so the student's angle must have also been an integer multiple of $\pi$.

\solution %4
We start with the given equation:
\begin{align*}
    \sin{\alpha}+\cos{\alpha} &= 0.2\\
    \sin^{2}{\alpha}+2\sin{\alpha}\cos{\alpha+}\cos^2\alpha&=0.04 &\text{(Squared both sides.)}\\
    1+2\sin{\alpha}\cos{\alpha}&=0.04\\
    2\sin{\alpha}\cos{\alpha} &=-0.96
\end{align*}
Note that that is simply $\sin 2\alpha$. Hooray!

\solution %5
We can follow a very similar strategy here to find $1-2\sin{\alpha}\cos{\alpha}=0.09$. Then $\sin {2\alpha}=2\sin{\alpha}\cos\alpha=0.91$.

\solution %6
\begin{align*}
\cos{2\alpha}\cos{\alpha} + \sin{2\alpha}\sin{\alpha} &= \left(2\cos^{2}{\alpha}-1\right)\cos{\alpha} + 2\sin{\alpha}\cos{\alpha}\sin{\alpha} \\
&= \cos{\alpha} \left(2\cos^{2}{\alpha - 1 + 2\sin^{2}{\alpha}}\right) \\
&= \cos{\alpha} \left(2-1\right) \\
&= \cos{\alpha}
\end{align*}

Alternatively, we may use the cosine difference formula.
\[
\cos{2\alpha}\cos{\alpha} + \sin{2\alpha}\sin{\alpha} = \cos\left(2\alpha-\alpha\right) \\
= \cos{\alpha}
\]

\solution %7
Applying the sine addition formula,
\[
\sin{2\alpha}\cos{\alpha} + \cos{2\alpha}\sin{\alpha} = \sin\left(2\alpha+\alpha\right) = \sin{3\alpha}
\]
Applying the sine subtraction formula,
\[
\sin{4\alpha}\cos{\alpha} - \cos{4\alpha}\sin{\alpha} = \sin\left(4\alpha-\alpha\right) = \sin{3\alpha}
\]
Since both sides of the identity are equal to $\sin{3\alpha}$, the identity is true.

\solution %8
Yes, the book asked you to prove something incorrect! For counterexample, consider that $\cos {2\alpha}$ can be negative but $\cos^{2}{\alpha}$ is never negative. However, we can prove a relationship.\\
Recall that $\cos{2\alpha}=\cos^{2}{\alpha} - \sin^{2}{\alpha}$. However, $\sin^{2}{\alpha} \geq 0$ since it's a square.
\begin{align*}
    \sin^2\alpha&\geq 0\\
    -\sin^2\alpha&\leq 0\\
    \cos^2\alpha-\sin^2\alpha&\leq \cos^2\alpha\\
    \cos 2\alpha &\leq \cos^2\alpha
\end{align*}

\solution %9
\begin{align*}
\left(\sin{\dfrac{\alpha}{2}} - \cos{\dfrac{\alpha}{2}}\right)^2 &= \sin^{2}{\dfrac{\alpha}{2}} - 2 \sin{\dfrac{\alpha}{2}}\cos{\dfrac{\alpha}{2}} + \cos^{2}{\dfrac{\alpha}{2}} \\
&= 1 - \sin{\alpha}
\end{align*}

\solution %10
Following the hint, we compute the value of $\cos{10\degree}\sin{10\degree}\sin{50\degree}\sin{70\degree}$.
\begin{align*}
M\cos{10\degree} &= \cos{10\degree}\sin{10\degree}\sin{50\degree}\sin{70\degree} \\
&= \dfrac{1}{2}\sin{20\degree}\sin{50\degree}\sin{70\degree} \\
&= \dfrac{1}{2}\cos{70\degree}\sin{50\degree}\sin{70\degree} \\
&= \dfrac{1}{4}\sin{140\degree}\sin{50\degree} \\
&= \dfrac{1}{4}\sin{40\degree}\sin{50\degree} \\
&= \dfrac{1}{4}\cos{50\degree}\sin{50\degree} \\
&= \dfrac{1}{8}\sin{100\degree} \\
&= \dfrac{1}{8}\sin{80\degree} \\
&= \dfrac{1}{8}\cos{10\degree}
\end{align*}
Since $M\cos{10\degree} = \tfrac{1}{8}\cos{10\degree}$, we have that the value of the original expression $M$ is equal to $\tfrac{1}{8}$.

\solution %11
We begin by computing $\sin{20\degree}\cos{20\degree}\cos{40\degree}\cos{80\degree}$.
\begin{align*}
\sin{20\degree}\cos{20\degree}\cos{40\degree}\cos{80\degree} &= \dfrac{1}{2}\sin{40\degree}\cos{40\degree}\cos{80\degree} \\
&= \dfrac{1}{4}\sin{80\degree}\cos{80\degree} \\
&= \dfrac{1}{8}\sin{160\degree} \\
&= \dfrac{1}{8}\sin{20\degree}
\end{align*}
This implies that $\cos{20\degree}\cos{40\degree}\cos{80\degree} = 1/8$.
\solution %12
We begin by computing $\cos{\pi/10}\sin{\pi/10}\sin{\pi/5}$.
\begin{align*}
\cos{\dfrac{\pi}{10}}\sin{\dfrac{\pi}{10}}\cos{\dfrac{\pi}{5}} &= \dfrac{1}{2}\sin{\dfrac{\pi}{5}}\cos{\dfrac{\pi}{5}} \\
&= \dfrac{1}{4}\sin{\dfrac{2\pi}{5}} \\
&= \dfrac{1}{4}\cos{\dfrac{\pi}{10}}
\end{align*}
This implies that $\sin{\pi/10}\cos{\pi/5} = 1/4$.
\end{solutions}

\begin{solutions}{Page 148}
\solution %1
\begin{align*}
\cos{3\alpha} &= \cos\left(2\alpha+\alpha\right) \\
&= \cos{2\alpha}\cos{\alpha} - \sin{2\alpha}\sin{\alpha} \\
&= \left(2\cos^{2}{\alpha}-1\right)\cos{\alpha} - 2\sin{\alpha}\cos{\alpha}\sin{\alpha} \\
&= 2\cos^{3}{\alpha}-\cos{\alpha}-2\sin^{2}{\alpha}\cos{\alpha} \\
&= 2\cos^{3}{\alpha}-\cos{\alpha}-2\left(1-\cos^{2}{\alpha}\right)\cos{\alpha} \\
&= 4\cos^{3}{\alpha} - 3\cos{\alpha}
\end{align*}

\solution %2
\begin{align*}
\sin{3\alpha} &= 3\sin{\alpha}-4\sin^{3}{\alpha} \\
&= 3\left(\dfrac{3}{5}\right)-4\left(\dfrac{3}{5}\right)^3 \\
&= \dfrac{9}{5} - 4 \cdot \dfrac{27}{125} \\
&= \dfrac{117}{125}
\end{align*}

If $\sin{\alpha}=3/5$, then $\cos{\alpha}=\pm 4/5$. Therefore,
\begin{align*}
\cos{3\alpha} &= 4\cos^{3}{\alpha}-3\cos{\alpha} \\
&= 4\left(\pm \dfrac{4}{5}\right)^3 - 3\left(\pm \dfrac{4}{5}\right) \\
&= \pm \dfrac{256}{125} \mp \dfrac{12}{5} \\
&= \mp \dfrac{44}{125}
\end{align*}

\solution %3
If $\cos{\alpha}=4/5$, then $\sin{\alpha}=\pm 3/5$. Therefore,
\begin{align*}
\sin{3\alpha} &= 3\sin{\alpha}-4\sin^{3}{\alpha} \\
&= 3\left(\pm \dfrac{3}{5}\right)-4\left(\pm \dfrac{3}{5}\right)^3 \\
&= \pm \dfrac{9}{5} \mp 4 \cdot \dfrac{27}{125} \\
&= \pm \dfrac{117}{125}
\end{align*}

\begin{align*}
\cos{3\alpha} &= 4\cos^{3}{\alpha}-3\cos{\alpha} \\
&= 4\left(\dfrac{4}{5}\right)^3 - 3\left(\dfrac{4}{5}\right) \\
&= \dfrac{256}{125} - \dfrac{12}{5} \\
&= -\dfrac{44}{125}
\end{align*}

\solution %4
\begin{subsolutions}
\subsolution %a
\begin{align*}
\cos{4\alpha} &= 2\cos^{2}{2\alpha} - 1 \\
&= 2\left(2\cos^{2}{\alpha}-1\right)^2-1 \\
&= 2\left(4\cos^{4}{\alpha}-4\cos^{2}{\alpha}+1\right)-1 \\
&= 8\cos^{4}{\alpha} - 8\cos^{2}{\alpha} + 1
\end{align*}

\subsolution %b
\begin{align*}
\cos{4\alpha} &= 1-2\sin^{2}{2\alpha} \\
&= 1-8\sin^{2}{\alpha}\cos^{2}{\alpha} \\
&= 1-8\sin^{2}{\alpha}\left(1-\sin^{2}{\alpha}\right) \\
&= 1-8\sin^{2}{\alpha}+8\sin^{4}{\alpha}
\end{align*}
\end{subsolutions}

\solution %5
\begin{align*}
\sin{3\alpha}\cos{\alpha}-\cos{3\alpha}\sin{\alpha} &= \left(3\sin{\alpha}-4\sin^{3}{\alpha}\right)\cos{\alpha}-\left(4\cos^{3}{\alpha}-3\cos{\alpha}\right)\sin{\alpha} \\
&= 3\sin{\alpha}\cos{\alpha}-4\sin^{3}{\alpha}\cos{\alpha}-4\cos^{3}{\alpha}\sin{\alpha}+3\sin{\alpha}\cos{\alpha} \\
&= 2\sin{\alpha}\cos{\alpha}\left(3 - 2\sin^{2}{\alpha} - 2\cos^{2}{\alpha}\right) \\
&= 2\sin{\alpha}\cos{\alpha} \\
&= \sin{2\alpha}
\end{align*}

Alternatively, applying the sine difference formula,
\[
\sin{3\alpha}\cos{\alpha}-\cos{3\alpha}\sin{\alpha} = \sin\left(3\alpha-\alpha\right) = \sin{2\alpha}
\]

\solution %6
\begin{align*}
\dfrac{\sin{3\alpha}}{\sin{\alpha}} - \dfrac{\cos{3\alpha}}{\cos{\alpha}} &= 3-4\sin^{2}{\alpha} - \left(4\cos^{2}{\alpha}-3\right) \\
&= 6-4\sin^{2}{\alpha}-4\cos^{2}{\alpha} \\
&= 2
\end{align*}

\solution %7
\begin{subsolutions}
\subsolution %a
\begin{align*}
4\sin{\alpha}\sin\left(60\degree+\alpha\right)\sin\left(60\degree-\alpha\right) &= 4\sin{\alpha}\left(\sin^{2}{60\degree}\cos^{2}{\alpha} - \cos^{2}{60\degree}\sin^{2}{\alpha}\right) \\
&= 4\sin{\alpha}\left(\dfrac{3}{4}\cos^{2}{\alpha} - \dfrac{1}{4}\sin^{2}{\alpha}\right) \\
&= 3\sin{\alpha}\cos^{2}{\alpha} - \sin^{3}{\alpha} \\
&= 3\sin{\alpha}\left(1-\sin^{2}{\alpha}\right)  - \sin^{3}{\alpha} \\
&= 3\sin{\alpha}-4\sin^{3}{\alpha}  \\
&= \sin{3\alpha}
\end{align*}

\subsolution %b
\begin{align*}
4\cos{\alpha}\cos\left(60\degree+\alpha\right)\cos\left(60\degree-\alpha\right) &= 4\cos{\alpha}\left(\cos^{2}{60\degree}\cos^{2}{\alpha} - \sin^{2}{60\degree}\sin^{2}{\alpha}\right) \\
&= 4\cos{\alpha}\left(\dfrac{1}{4}\cos^{2}{\alpha} - \dfrac{3}{4}\sin^{2}{\alpha}\right) \\
&= \cos^{3}{\alpha} - 3\sin^{2}{\alpha}\cos{\alpha} \\
&= \cos^{3}{\alpha} - 3\left(1-\cos^{2}{\alpha}\right)\cos{\alpha} \\
&= 4\cos^{3}{\alpha} - 3\cos{\alpha}
\end{align*}

\end{subsolutions}

\solution %8
\begin{align*}
\sin{4\alpha} &= 2\sin{2\alpha}\cos{2\alpha} \\
&=4\sin{\alpha}\cos{\alpha}\left(2\cos^{2}{\alpha}-1\right) \\
&= 8\sin{\alpha}\cos^{3}{\alpha}-4\sin{\alpha}\cos{\alpha} \\
&\implies \dfrac{\sin{4\alpha}}{\sin{\alpha}} = 8\cos^{3}{\alpha}-4\cos{\alpha} 
\end{align*}

\solution %9
In the penultimate step, we apply the following identity:
\[
\left(A-B\right)^{3} = A^3 - 3A^{2}B + 3AB^{2} - B^3,
\]
taking $A=\cos^{2}{\alpha}$ and $B=\sin^{2}{\alpha}$.

\begin{align*}
\sin{3\alpha}\sin^{3}{\alpha} + \cos{3\alpha}\cos^{3}{\alpha} &= \left(3\sin{\alpha}-4\sin^{3}{\alpha}\right)\sin^{3}{\alpha} + \left(4\cos^{3}{\alpha}-3\cos{\alpha}\right)\cos^{3}{\alpha} \\
&= 3\sin^{4}{\alpha}-4\sin^{6}{\alpha}+4\cos^{6}{\alpha}-3\cos^{4}{\alpha} \\
&=3\sin^{4}{\alpha}-4\sin^{4}{\alpha}\left(1-\cos^{2}{\alpha}\right)+4\cos^{4}{\alpha}\left(1-\sin^{2}{\alpha}\right)-3\cos^{4}{\alpha} \\
&= \sin^{4}{\alpha}\left(4\cos^{2}{\alpha}-1\right) + \cos^{4}{\alpha}\left(1-4\sin^{2}{\alpha}\right) \\
&= \sin^{4}{\alpha}\left(4\cos^{2}{\alpha}-\sin^{2}{\alpha}-\cos^{2}{\alpha}\right) + \cos^{4}{\alpha}\left(\sin^{2}{\alpha}+\cos^{2}{\alpha}-4\sin^{2}{\alpha}\right) \\
&=3\sin^{4}{\alpha}\cos^{2}{\alpha}-\sin^{6}{\alpha}+\cos^{6}{\alpha}-3\cos^{4}{\alpha}\sin^{2}{\alpha} \\
&= \left(\cos^{2}{\alpha}-\sin^{2}{\alpha}\right)^3 \\
&= \cos^{3}{2\alpha}
\end{align*}
\end{solutions}

\begin{solutions}{Page 150}
\solution %1
\[
\cos{\alpha}=1 \implies \cos{\dfrac{\alpha}{2}} = \pm \sqrt{\dfrac{1+1}{2}} = \pm 1
\]
$\cos{\alpha}$ is equal to 1 when $\alpha$ is an integer multiple of $2\pi$. If $\alpha = 4n\pi$ for some integer $n$ (i.e., an even integer multiple of $2\pi$), then $\cos{\alpha/2} = \cos{2n\pi} = 1$. Otherwise, if $\alpha = \left(4n+2\right)\pi$ for some integer $n$ (i.e., an odd integer multiple of $2\pi$), then $\cos{\alpha/2} = \cos\left(2n+1\right)\pi = -1$.

For a particular example, we may set $n=0$.

$\alpha = 4\left(0\right)\pi = 0$
\[
\cos{0} = 1, \cos{\dfrac{0}{2}} = \cos{0} = 1
\]

$\alpha = \left(4\left(0\right)+2\right)\pi = 2\pi$
\[
\cos{2\pi} = 1, \cos{\dfrac{2\pi}{2}} = \cos{\pi} = -1
\]

\solution %2
\begin{subsolutions}
\subsolution %a
We take the positive square root because $60\degree / 2 = 30\degree$ lies in the first quadrant.
\begin{align*}
\cos \dfrac{60\degree}{2} &= \sqrt{\dfrac{1+\cos{60\degree}}{2}} \\
&= \sqrt{\dfrac{1+1/2}{2}} \\
&= \sqrt{\dfrac{3}{4}} \\
&= \dfrac{\sqrt{3}}{2}
\end{align*}

\subsolution %b
We take the positive square root because $120\degree / 2 = 60\degree$ lies in the first quadrant.
\begin{align*}
\cos \dfrac{120\degree}{2} &= \sqrt{\dfrac{1+\cos{120\degree}}{2}} \\
&= \sqrt{\dfrac{1-1/2}{2}} \\
&= \sqrt{\dfrac{1}{4}} \\
&= \dfrac{1}{2}
\end{align*}

\subsolution %c
We take the negative square root because $240\degree / 2 = 120\degree$ lies in the second quadrant.
\begin{align*}
\cos \dfrac{240\degree}{2} &= -\sqrt{\dfrac{1+\cos{240\degree}}{2}} \\
&= -\sqrt{\dfrac{1-1/2}{2}} \\
&= -\sqrt{\dfrac{1}{4}} \\
&= -\dfrac{1}{2}
\end{align*}
\end{subsolutions}

\solution ~ %3
\begin{center}
\bgroup
\def\arraystretch{2.1}
\setlength\tabcolsep{15pt}
\begin{tabular}{ |c|c|c|c|c| }
\hline
$\alpha$
& Quadrant $\alpha$?
& $\alpha / 2$
& Quadrant $\alpha / 2$?
& $\cos{\alpha / 2}$\\
\hline
$780\degree$
& I
& $390\degree$
& I
& $\sqrt{3} / 2$\\
\hline
$1020\degree$
& IV
& $510\degree$
& II
& $-\sqrt{3} / 2$\\
\hline
$1140\degree$
& I
& $570\degree$
& III
& $-\sqrt{3} / 2$\\
\hline
$1380\degree$
& IV
& $690\degree$
& IV
& $\sqrt{3} / 2$\\
\hline
$-60\degree$
& IV
& $-30\degree$
& IV
& $\sqrt{3} / 2$\\
\hline
$-300\degree$
& I
& $-150\degree$
& III
& $-\sqrt{3} / 2$\\
\hline
$-420\degree$
& IV
& $-210\degree$
& II
& $-\sqrt{3} / 2$\\
\hline
$-660\degree$
& I
& $-330\degree$
& I
& $\sqrt{3} / 2$\\
\hline
$-780\degree$
& IV
& $-390\degree$
& IV
& $\sqrt{3} / 2$\\
\hline
\end{tabular}
\egroup
\end{center}

\solution %4
\begin{align*}
\sin{15\degree} &= \sqrt{\dfrac{1-\cos{30\degree}}{2}} \\
&= \sqrt{\dfrac{1-\sqrt{3}/2}{2}} \\
&= \sqrt{\dfrac{2-\sqrt{3}}{4}} \\
&= \dfrac{\sqrt{2-\sqrt{3}}}{2}
\end{align*}

\begin{align*}
\cos{15\degree} &= \sqrt{\dfrac{1+\cos{30\degree}}{2}} \\
&= \sqrt{\dfrac{1+\sqrt{3}/2}{2}} \\
&= \sqrt{\dfrac{2+\sqrt{3}}{4}} \\
&= \dfrac{\sqrt{2+\sqrt{3}}}{2}
\end{align*}

\solution %5
Because $\left\lvert \cos{\alpha} \right\rvert \leq 1$, $1 \pm \cos{\alpha} \geq 0$. Thus the expressions under the square roots in the sine and cosine half angle formulas will not be negative.

\solution %6
The square root sign in the half angle formula prevents it from being a rational trigonometric function, so the Principle of Analytic Continuation does not apply.

\solution %7
\begin{subsolutions}
\subsolution %a
\begin{align*}
\tan{\dfrac{\alpha}{2}}\tan{\dfrac{\beta}{2}} + \tan{\dfrac{\alpha}{2}}\tan{\dfrac{\gamma}{2}} + \tan{\dfrac{\beta}{2}}\tan{\dfrac{\gamma}{2}} &= \tan{\dfrac{\alpha}{2}}\tan{\dfrac{\beta}{2}} + \tan{\dfrac{\gamma}{2}} \left(\tan{\dfrac{\alpha}{2}} + \tan{\dfrac{\beta}{2}}\right) \\
&= \tan{\dfrac{\alpha}{2}}\tan{\dfrac{\beta}{2}} + \tan{\dfrac{\gamma}{2}}\tan{\dfrac{\alpha+\beta}{2}}\left(1-\tan{\dfrac{\alpha}{2}}\tan{\dfrac{\beta}{2}}\right) \\
&= \tan{\dfrac{\alpha}{2}}\tan{\dfrac{\beta}{2}} + 1-\tan{\dfrac{\alpha}{2}}\tan{\dfrac{\beta}{2}} \\
&= 1
\end{align*}
Alternatively, we can recall the extended tangent addition formula derived in Exercise 6 of Section 4 of this chapter:
\[
\tan\left(\alpha+\beta+\gamma\right) = \dfrac{\tan{\alpha}+\tan{\beta}+\tan{\gamma} -\tan{\alpha}\tan{\beta}\tan{\gamma}}{1 - \tan{\alpha}\tan{\beta} - \tan{\alpha}\tan{\gamma} - \tan{\beta}\tan{\gamma}}
\]
Since $\alpha/2 + \beta/2 + \gamma/2 = \pi / 2$, $\tan\left(\alpha/2 + \beta/2 + \gamma/2\right)$ is undefined. This implies that the denominator of the tangent addition formula is 0 (assuming that none of $\tan{\alpha/2}$, $\tan{\beta/2}$, or $\tan{\gamma/2}$ are undefined). Therefore, we can conclude
\[
\tan{\dfrac{\alpha}{2}}\tan{\dfrac{\beta}{2}} + \tan{\dfrac{\alpha}{2}}\tan{\dfrac{\gamma}{2}} + \tan{\dfrac{\beta}{2}}\tan{\dfrac{\gamma}{2}} = 1.
\]
\subsolution %b
Following a similar approach as to the previous part, we begin by noting that $\sin{\tfrac{\alpha+\beta}{2}} = \cos{\tfrac{\gamma}{2}}$.
\begin{align*}
4\cos{\dfrac{\alpha}{2}}\cos{\dfrac{\beta}{2}}\cos{\dfrac{\gamma}{2}} &= 4\cos{\dfrac{\alpha}{2}}\cos{\dfrac{\beta}{2}}\sin{\dfrac{\alpha+\beta}{2}} \\
&=4\cos{\dfrac{\alpha}{2}}\cos{\dfrac{\beta}{2}}\left(\sin{\dfrac{\alpha}{2}}\cos{\dfrac{\beta}{2}} + \cos{\dfrac{\alpha}{2}}\sin{\dfrac{\beta}{2}}\right) \\
&= 4\sin{\frac{\alpha}{2}}\cos{\dfrac{\alpha}{2}}\cos^{2}{\dfrac{\beta}{2}} + 4\sin{\dfrac{\beta}{2}}\cos{\dfrac{\beta}{2}}\cos^{2}{\dfrac{\alpha}{2}} \\
&= 2\sin{\alpha}\cos^{2}{\dfrac{\beta}{2}}+2\sin{\beta}\cos^{2}{\dfrac{\alpha}{2}} \\
&= 2\sin{\alpha}\left(\dfrac{1+\cos{\beta}}{2}\right)+2\sin{\beta}\left(\dfrac{1+\cos{\alpha}}{2}\right) \\
&= \sin{\alpha} + \sin{\alpha}\cos{\beta} + \sin{\beta} + \sin{\beta}\cos{\alpha} \\
&= \sin{\alpha} + \sin{\beta} + \sin\left(\alpha+\beta\right) \\
&= \sin{\alpha} + \sin{\beta} + \sin\left(\pi - \alpha - \beta\right) \\
&= \sin{\alpha} + \sin{\beta} + \sin{\gamma}
\end{align*}
\end{subsolutions}
\end{solutions}

\begin{solutions}{Page 152}
\solution %1
Because $1+\cos{\alpha}$ is non-negative, division by $1+\cos{\alpha}$ does not change the sign of $\sin{\alpha}$, which means $\tan \left(\alpha / 2\right)$ and $\sin{\alpha} / \left(1+\cos{\alpha}\right)$ have the same sign.

\solution %2
\begin{align*}
\tan{\dfrac{\alpha}{2}} &= \pm \sqrt{\dfrac{1-\cos{\alpha}}{1+\cos{\alpha}}} \\
&= \pm \sqrt{\dfrac{1-\cos{\alpha}}{1+\cos{\alpha}} \cdot \dfrac{1-\cos{\alpha}}{1-\cos{\alpha}}} \\
&= \pm \sqrt{\dfrac{\left(1-\cos{\alpha}\right)^2}{1-\cos^{2}{\alpha}}} \\
&= \pm \sqrt{\dfrac{\left(1-\cos{\alpha}\right)^2}{\sin^{2}{\alpha}}} \\
&= \pm \dfrac{1-\cos{\alpha}}{\sin{\alpha}}
\end{align*}
For acute angles $\alpha$, we need to take the positive branch of the square root so that the signs of both sides of the half angle formula agree. By the Principle of Analytic Continuation, since the positive branch is correct for all acute angles and both sides of the formula are rational trigonometric expressions, it is correct for all angles in general, so we have
\[
\tan{\dfrac{\alpha}{2}} = \dfrac{1-\cos{\alpha}}{\sin{\alpha}}
\]

\end{solutions}

\begin{solutions}{Page 153}
\solution %1
\begin{align*}
\dfrac{1}{2}\cos\left(\alpha-\beta\right) - \dfrac{1}{2}\cos\left(\alpha+\beta\right) &= \dfrac{1}{2}\left(\cos{\alpha}\cos{\beta}+\sin{\alpha}\sin{\beta}-\cos{\alpha}\cos{\beta} + \sin{\alpha}\sin{\beta}\right) \\
&= \dfrac{1}{2}\left(2\sin{\alpha}\sin{\beta}\right) \\
&= \sin{\alpha}\sin{\beta}
\end{align*}

\begin{align*}
\dfrac{1}{2}\sin\left(\alpha+\beta\right) + \dfrac{1}{2}\sin\left(\alpha-\beta\right) &= \dfrac{1}{2}\left(\sin{\alpha}\cos{\beta}+\cos{\alpha}\sin{\beta}+\sin{\alpha}\cos{\beta} - \cos{\alpha}\sin{\beta}\right) \\
&= \dfrac{1}{2}\left(2\sin{\alpha}\cos{\beta}\right) \\
&= \sin{\alpha}\cos{\beta}
\end{align*}

\solution %2
\begin{align*}
\sin{75\degree}\sin{15\degree} &= \dfrac{1}{2}\cos\left(75\degree-15\degree\right) - \dfrac{1}{2}\cos\left(75\degree+15\degree\right) \\
&= \dfrac{1}{2}\cos{60\degree} - \dfrac{1}{2}\cos{90\degree} \\
&= \dfrac{1}{2} \cdot \dfrac{1}{2} - \dfrac{1}{2} \cdot 0 \\
&= \dfrac{1}{4}
\end{align*}

\solution %3
\begin{align*}
\sin{75\degree}\cos{15\degree} &= \dfrac{1}{2}\sin\left(75\degree+15\degree\right) + \dfrac{1}{2}\sin\left(75\degree-15\degree\right) \\
&= \dfrac{1}{2}\sin{90\degree} + \dfrac{1}{2}\sin{60\degree} \\
&= \dfrac{1}{2} \cdot 1 + \dfrac{1}{2} \cdot \dfrac{\sqrt{3}}{2} \\
&= \dfrac{2+\sqrt{3}}{4}
\end{align*}

\solution %4
\begin{subsolutions}
\subsolution %a
\begin{align*}
\cos{75\degree}\cos{15\degree} &= \dfrac{1}{2}\cos\left(75\degree+15\degree\right) + \dfrac{1}{2}\cos\left(75\degree-15\degree\right) \\
&= \dfrac{1}{2}\cos{90\degree} + \dfrac{1}{2}\cos{60\degree} \\
&= \dfrac{1}{2} \cdot 0 + \dfrac{1}{2} \cdot \dfrac{1}{2} \\
&= \dfrac{1}{4}
\end{align*}

Alternatively, by the cosine subtraction formula, $\cos{60\degree} = \cos{75\degree}\cos{15\degree} + \sin{75\degree}\sin{15\degree}$ so
\[
\cos{75\degree}\cos{15\degree} = \cos{60\degree} - \sin{75\degree}\sin{15\degree} = \dfrac{1}{2} - \dfrac{1}{4} = \dfrac{1}{4}
\]

Alternatively, using the idea of sine and cosine being cofunctions,
\[
\cos{75\degree}\cos{15\degree} = \sin\left(90\degree-75\degree\right)\sin\left(90\degree-15\degree\right) = \sin{15\degree}\sin{75\degree} = \dfrac{1}{4}
\]
For the above two alternative solutions, we apply the result from Exercise 2 above that $\sin{75\degree}\sin{15\degree} = 1/4$.

\subsolution %b
\begin{align*}
\cos{75\degree}\sin{15\degree} &= \dfrac{1}{2}\sin\left(15\degree+75\degree\right) + \dfrac{1}{2}\sin\left(15\degree-75\degree\right) \\
&= \dfrac{1}{2}\sin{90\degree} - \dfrac{1}{2}\sin{60\degree} \\
&= \dfrac{1}{2} \cdot 1 - \dfrac{1}{2} \cdot \dfrac{3}{2} \\
&= \dfrac{2-\sqrt{3}}{4}
\end{align*}

Alternatively, by the sine addition formula, $\sin{90\degree} = \sin{75\degree}\cos{15\degree} + \cos{75\degree}\sin{15\degree}$ so
\[
\cos{75\degree}\sin{15\degree} = \sin{90\degree} - \sin{75\degree}\cos{15\degree} = 1 - \dfrac{2+\sqrt{3}}{4} = \dfrac{2-\sqrt{3}}{4}
\]
For the above alternative solution, we apply the result from Exercise 3 above that $\sin{75\degree}\cos{15\degree} = \left(2+\sqrt{3}\right)/4$.
\end{subsolutions}

\solution %5
\begin{align*}
2\cos\left(\dfrac{\pi}{4}+\alpha\right)\cos\left(\dfrac{\pi}{4}-\alpha\right) &= \cos\left(\dfrac{\pi}{4} + \alpha + \dfrac{\pi}{4} - \alpha\right) + \cos\left(\dfrac{\pi}{4} + \alpha - \dfrac{\pi}{4} + \alpha\right) \\
&= \cos{\dfrac{\pi}{2}} \cos{2\alpha} \\
&= \cos{2\alpha}
\end{align*}

\solution %6
\begin{align*}
&\sin\left(\alpha+\beta\right)\sin\left(\alpha-\beta\right) + \sin\left(\beta+\gamma\right)\sin\left(\beta-\gamma\right) + \sin\left(\gamma+\alpha\right)\sin\left(\gamma-\alpha\right) \\
&= \dfrac{1}{2}\cos{2\beta} - \dfrac{1}{2}\cos{2\alpha} + \dfrac{1}{2}\cos{2\gamma} - \dfrac{1}{2}\cos{2\beta} + \dfrac{1}{2}\cos{2\alpha} - \dfrac{1}{2}\cos{2\gamma} \\
&= 0
\end{align*}

\solution %7
\begin{align*}
&\sin{\alpha}\sin\left(\beta-\gamma\right) + \sin{\beta}\sin\left(\gamma-\alpha\right) + \sin{\gamma}\sin\left(\alpha-\beta\right) \\
&= \dfrac{1}{2}\cos\left(\alpha-\beta+\gamma\right) - \dfrac{1}{2}\cos\left(\alpha+\beta-\gamma\right) + \dfrac{1}{2}\cos\left(\beta-\gamma+\alpha\right) - \dfrac{1}{2}\cos\left(\beta+\gamma-\alpha\right) \\
&+ \dfrac{1}{2}\cos\left(\gamma-\alpha+\beta\right) - \dfrac{1}{2}\cos\left(\gamma+\alpha-\beta\right) \\
&= 0
\end{align*}

\end{solutions}

\begin{solutions}{Page 155}
\solution
\solution
\solution
\solution
\solution
\solution
\solution
\solution
\solution
\solution
\solution
\solution
\end{solutions}

\begin{solutions}{Page 158}
\solution %1
Let $\beta=2\gamma$.
\begin{align*}
\sin^{2}{\beta} + \cos^{2}{\beta} &= \sin^{2}{2\gamma} + \cos^{2}{2\gamma} \\
&= \left(\dfrac{2a}{1+a^{2}}\right)^2 + \left(\dfrac{1-a^{2}}{1+a^{2}}\right)^2 \\
&= \dfrac{4a^{2}}{1+2a^{2}+a^{4}} + \dfrac{1-2a^{2}+a^{4}}{1+2a^{2}+a^{4}} \\
&= \dfrac{1+2a^{2}+a^{4}}{1+2a^{2}+a^{4}} \\
&= 1
\end{align*}

\solution %2
\begin{align*}
\tan{2\beta} &= \dfrac{2a}{1-a^2} \\
&= \dfrac{\dfrac{2a}{1+a^{2}}}{\dfrac{1-a^{2}}{1+a^{2}}} \\
&= \dfrac{\sin{2\beta}}{\cos{2\beta}}
\end{align*}
\end{solutions}

\begin{solutions}{Page 160}
\solution %1
\[
\sin{\alpha} = \dfrac{2 \cdot 2 \cdot 3}{2^{2}+3^{2}} = \dfrac{12}{13}
\]
\[
\cos{\alpha} = \dfrac{2^{2} - 3^{2}}{2^{2}+3^{2}} = \dfrac{-5}{13}
\]
These values give the Pythagorean triple 5, 12, 13, provided we take the absolute value of $\cos{\alpha}$.

\solution %2
\[
\sin{\alpha} = \dfrac{2 \cdot 8 \cdot 5}{8^{2}+5^{2}} = \dfrac{80}{89}
\]
\[
\cos{\alpha} = \dfrac{8^{2} - 5^{2}}{8^{2}+5^{2}} = \dfrac{39}{89}
\]
This corresponds to the Pythagorean triple 80, 39, 89.

\solution %3
\[
\left(2pq\right)^{2} + \left(q^{2}-p^{2}\right)^{2} = 4p^{2}q^{2} + q^4 - 2q^{2}p^{2} + p^{4} = q^{4} + 2q^{2}p^{2} + p^{4} = \left(q^{2}+p^{2}\right)^{2}
\]
The above shows that $q^{2}+p^{2}$ is the hypotenuse.

\end{solutions}

\begin{solutions}{Page 161 (First)}
\solution %1
\begin{itemize}
    \item $\sin{20\degree}\cos{20\degree} \approx 0.3214\ldots$
    \item $\sin{10\degree}\cos{10\degree} \approx 0.1710\ldots$
    \item $\sin{5\degree}\cos{5\degree} \approx 0.0868\ldots$
    \item $\sin{1\degree}\cos{1\degree} \approx 0.0174\ldots$
    \item $\sin{70\degree}\cos{70\degree} \approx 0.3214\ldots$
    \item $\sin{80\degree}\cos{80\degree} \approx 0.1710\ldots$
    \item $\sin{85\degree}\cos{85\degree} \approx 0.0868\ldots$
    \item $\sin{89\degree}\cos{89\degree} \approx 0.0174\ldots$
\end{itemize}

\solution %2
\[
\sin{30\degree}\cos{30\degree} = \dfrac{1}{2} \cdot \dfrac{\sqrt{3}}{2} = \dfrac{\sqrt{3}}{4}
\]
\[
\sin{45\degree}\cos{45\degree} = \dfrac{\sqrt{2}}{2} \cdot \dfrac{\sqrt{2}}{2} = \dfrac{1}{2}
\]
\[
\sin{60\degree}\cos{60\degree} = \dfrac{\sqrt{3}}{2} \dfrac{1}{2}  = \dfrac{\sqrt{3}}{4}
\]
\end{solutions}

\begin{solutions}{Page 161 (Second)}
\solution %1
In the below answers to this question, $n$ represents an arbitrary integer.
\begin{subsolutions}
\subsolution %a
\begin{align*}
\sin{x}\cos{x} = \dfrac{1}{2} &\implies \dfrac{1}{2}\sin{2x} = \dfrac{1}{2} \\
&\implies \sin{2x} = 1 \\
&\implies 2x = \dfrac{\pi}{2} + 2n\pi \\
&\implies x = \dfrac{\pi}{4} + n\pi
\end{align*}

\subsolution %b
\[
\sin{x}\cos{x} = \dfrac{\sqrt{3}}{2} \implies \dfrac{1}{2}\sin{2x} = \dfrac{\sqrt{3}}{2} \implies \sin{2x} = \sqrt{3}
\]
Because $\sqrt{3} > 1$, there are no values of $x$ which satisfy the given equation.

\subsolution %c
\begin{align*}
\sin{x}\cos{x} = \dfrac{\sqrt{3}}{4} &\implies \dfrac{1}{2}\sin{2x} = \dfrac{\sqrt{3}}{4} \\
&\implies \sin{2x} = \dfrac{\sqrt{3}}{2} \\
&\implies 2x = \left(\dfrac{\pi}{2} \pm \dfrac{2\pi}{3}\right) + 2n\pi \\
& \implies x = \left(\dfrac{\pi}{4} \pm \dfrac{\pi}{3}\right) + n\pi
\end{align*}

\end{subsolutions}

\solution %2
(c) has no solution because $\sin{x}\cos{x} \leq 0.5 < 0.6$.

\solution %3
$\sin{x}\cos{x}=N$ has a solution for $\left\lvert N \right\rvert \leq 1/2$.
\[
\sin{x}\cos{x} = N \implies \dfrac{1}{2}\sin{2x} = N
\implies \sin{2x} = 2N
\]
Positive, Zero, Negative $\pm 1/2$.
\[
x = \dfrac{1}{2} \left(\arcsin{2N} + 2n\pi\right)
\]
\end{solutions}

\begin{solutions}{Page 162 (First)}
\solution %1
\[
\sin{30\degree}+\cos{30\degree} = \dfrac{1}{2} + \dfrac{\sqrt{3}}{2} > \dfrac{1}{2} + \dfrac{1}{2} = 1
\]

\solution %2
\[
\sin{0}+\cos{0} = 1
\]

\solution %3
\[
\sin{\dfrac{\pi}{4}} + \cos{\frac{\pi}{4}} = \dfrac{\sqrt{2}}{2} + \dfrac{\sqrt{2}}{2} = \sqrt{2}
\]
\end{solutions}

\begin{solutions}{Page 162 (Second)}
\solution %1
Yes, because $1.414 < \sqrt{2}$. (Note: $\sin{x}+\cos{x}$ must attain the value $1.414$ for some $x$ because of the Intermediate Value Theorem).

\solution %2
No, because $1.415 > \sqrt{2}$.

\solution %3
\solution %4
\end{solutions}

\begin{solutions}{Page 163 (First)}
\solution %1
\[
x = \dfrac{\pi}{4} + 2n\pi
\]
\solution %2
The minimum value of $\sin{x}+\cos{x}$ is $-\sqrt{2}$. This is achieved when
\[
x = -\dfrac{3\pi}{4} + 2n\pi
\]
\end{solutions}

\begin{solutions}{Page 163 (Second)}
\solution %1
Yes, because $\sin{\alpha}$ and $\cos{\alpha}$ are both positive.

\solution %2
The minimum value of $3\sin{x}+4\cos{x}$ is $-5$.

\solution %3
\end{solutions}

\begin{solutions}{Page 164}
\solution
\solution
\end{solutions}

\begin{solutions}{Page 166}
\solution
\solution
\solution
\solution
\end{solutions}

\begin{solutions}{Page 168}
\solution
\solution
\solution
\solution
\end{solutions}

\begin{solutions}{Page 170}
\solution
\solution
\solution
\solution
\solution
\end{solutions}

\section*{Chapter 8: Graphs of Trigonometric Functions}

\begin{solutions}{Page 177}
\solution
Since $k=5$, Period is $5$, and Frequency is $\frac{\pi}{5}$
\solution
Since $k=\frac{1}{4}$, Period is $\frac{1}{4}$, and Frequency is $\frac{\pi}{1/4}=4\pi$
\solution
Since $k=\frac{4}{5}$, Period is $\frac{4}{5}$, and Frequency is $\frac{\pi}{4/5}= \frac{5\pi}{4}$
\solution
Since $k=\frac{5}{4}$, Period is $\frac{5}{4}$, and Frequency is $\frac{\pi}{5/4}= \frac{4\pi}{5}$
\solution
Period is $\frac{2\pi}{3}$

\begin{tikzpicture}
    \draw[help lines,xstep=pi/2,ystep=1,color=gray!50,dashed](0,-1.4) grid (6.9,1.4);
    \draw[->,thick] (0,0)--(7,0) node[right]{$x$};
    \draw[->,thick] (0,0)--(0,1.5) node[right]{$y$};
    \draw[->,thick] (0,0)--(0,-1.5);
    %Draw marks on x axis
    \draw [very thin,gray](pi/2,-0.1)--(pi/2,0.1) node[below right] at (pi/2,-0.1) {$\frac{\pi}{2}$};
    \draw [very thin,gray](pi,-0.1)--(pi,0.1) node[below right] at (pi,-0.1) {$\pi$};
    \draw [very thin,gray](3*pi/2,-0.1)--(3*pi/2,0.1) node[below right] at (3*pi/2,-0.1) {$\frac{3\pi}{2}$};
    \draw [very thin,gray](2*pi,-0.1)--(2*pi,0.1) node[below right] at (2*pi,-0.1) {$2\pi$};
    %Draw marks on y axis
    \foreach \yticks in {-1,0,1}
    \draw [very thin,gray](-0.1,\yticks)--(0.1,\yticks) node[left] at (0,\yticks) {$\yticks$};
    %Draw the curve
    \draw[red]plot[domain=0:2*pi, samples=90]  (\x,{sin(3*\x r)});
\end{tikzpicture}

\solution
Period is $\frac{2\pi}{1/3}= 6\pi$

\begin{tikzpicture}[xscale=0.5,yscale=1]
    \draw[help lines,xstep=pi,ystep=1,color=gray!50,dashed](0,-1.4) grid (6.1*pi,1.4);
    \draw[->,thick] (0,0)--(6.1*pi,0) node[right]{$x$};
    \draw[->,thick] (0,0)--(0,1.5) node[right]{$y$};
    \draw[->,thick] (0,0)--(0,-1.5);
    %Draw marks on x axis
    \foreach \i in {1,2,...,6}
    \draw [very thin,gray](\i*pi,-0.1)--(\i*pi,0.1) node[below right] at (\i*pi,-0.1) {$\i\pi$};
    %Draw marks on y axis
    \foreach \yticks in {-1,0,1}
    \draw [very thin,gray](-0.1,\yticks)--(0.1,\yticks) node[left] at (0,\yticks) {$\yticks$};
    %Draw the curve
    \draw[red]plot[domain=0:6*pi, samples=90] (\x,{sin(\x/3 r)});
\end{tikzpicture}

\solution
Period is $\frac{2\pi}{3/2}= \frac{4\pi}{3}$

\begin{tikzpicture}[xscale=1,yscale=1]
    \draw[help lines,xstep=pi/2,ystep=1,color=gray!50,dashed](0,-1.4) grid (2.1*pi,1.4);
    \draw[->,thick] (0,0)--(2.1*pi,0) node[right]{$x$};
    \draw[->,thick] (0,0)--(0,1.5) node[right]{$y$};
    \draw[->,thick] (0,0)--(0,-1.5);
    %Draw marks on x axis
    \foreach \i in {1,2}
    \draw [very thin,gray](\i*pi,-0.1)--(\i*pi,0.1) node[below right] at (\i*pi,-0.1) {$\i\pi$};
    %Draw marks on y axis
    \foreach \yticks in {-1,0,1}
    \draw [very thin,gray](-0.1,\yticks)--(0.1,\yticks) node[left] at (0,\yticks) {$\yticks$};
    %Draw the curve
    \draw[red]plot[domain=0:2*pi,samples=90] (\x,{sin(3*\x/2 r)});
\end{tikzpicture}

\solution
Period is $\frac{2\pi}{2/3}= 3\pi$ 

\begin{tikzpicture}[xscale=0.5,yscale=1]
    \draw[help lines,xstep=pi/2,ystep=1,color=gray!50,dashed](0,-1.4) grid (4.1*pi,1.4);
    \draw[->,thick] (0,0)--(4.1*pi,0) node[right]{$x$};
    \draw[->,thick] (0,0)--(0,1.5) node[right]{$y$};
    \draw[->,thick] (0,0)--(0,-1.5);
    %Draw marks on x axis
    \foreach \i in {1,2,...,3}
    \draw [very thin,gray](\i*pi,-0.1)--(\i*pi,0.1) node[below right] at (\i*pi,-0.1) {$\i\pi$};
    %Draw marks on y axis
    \foreach \yticks in {-1,0,1}
    \draw [very thin,gray](-0.1,\yticks)--(0.1,\yticks) node[left] at (0,\yticks) {$\yticks$};
    %Draw the curve
    \draw[red]plot[domain=0:4*pi, samples=90] (\x,{sin(2*\x/3 r)});
\end{tikzpicture}
\solution
Period is $\frac{2\pi}{2/3}= 3\pi$

\begin{tikzpicture}[xscale=0.5,yscale=1]
    \draw[help lines,xstep=pi/2,ystep=1,color=gray!50,dashed](0,-1.4) grid (6.1*pi,1.4);
    \draw[->,thick] (0,0)--(6.1*pi,0) node[right]{$x$};
    \draw[->,thick] (0,0)--(0,1.5) node[right]{$y$};
    \draw[->,thick] (0,0)--(0,-1.5);
    %Draw marks on x axis
    \foreach \i in {1,2,...,6}
    \draw [very thin,gray](\i*pi,-0.1)--(\i*pi,0.1) node[below right] at (\i*pi,-0.1) {$\i\pi$};
    %Draw marks on y axis
    \foreach \yticks in {-1,0,1}
    \draw [very thin,gray](-0.1,\yticks)--(0.1,\yticks) node[left] at (0,\yticks) {$\yticks$};
    %Draw the curve
    \draw[red]plot[domain=0:6*pi, samples=90] (\x,{cos(2*\x/3 r)});
\end{tikzpicture}
\solution
Period is $\frac{2\pi}{3/2}= \frac{4\pi}{3}$

\begin{tikzpicture}[xscale=1,yscale=1]
    \draw[help lines,xstep=pi/2,ystep=1,color=gray!50,dashed](0,-1.4) grid (2.1*pi,1.4);
    \draw[->,thick] (0,0)--(2.1*pi,0) node[right]{$x$};
    \draw[->,thick] (0,0)--(0,1.5) node[right]{$y$};
    \draw[->,thick] (0,0)--(0,-1.5);
    %Draw marks on x axis
    \foreach \i in {1,2}
    \draw [very thin,gray](\i*pi,-0.1)--(\i*pi,0.1) node[below right] at (\i*pi,-0.1) {$\i\pi$};
    %Draw marks on y axis
    \foreach \yticks in {-1,0,1}
    \draw [very thin,gray](-0.1,\yticks)--(0.1,\yticks) node[left] at (0,\yticks) {$\yticks$};
    %Draw the curve
    \draw[red]plot[domain=0:2*pi,samples=90] (\x,{cos(3*\x/2 r)});
\end{tikzpicture}

\solution
For $y=f(3x)$

\begin{tikzpicture}[xscale=1,yscale=1]
    \draw[help lines,xstep=1,ystep=1,color=gray!50,dashed](-1.5,-1.4) grid (6.5,1.4);
    \draw[->,thick] (-1.5,0)--(6.5,0) node[right]{$x$};
    \draw[->,thick] (0,0)--(0,1.5) node[right]{$y$};
    \draw[->,thick] (0,0)--(0,-1.5);
    %Draw marks on x axis
    \foreach \i in {-1,0,...,6}
    \draw [very thin,gray](\i,-0.1)--(\i,0.1) node[below right] at (\i,-0.1) {$\i$};
    %Draw marks on y axis
    \foreach \yticks in {-1,0,1}
    \draw [very thin,gray](-0.1,\yticks)--(0.1,\yticks) node[left] at (0,\yticks) {$\yticks$};
    %Draw the curve
    %\draw[red]plot[domain=-1:5,samples=90] (\x,{asin(sin(\x*pi r))});
    \draw[red] plot[domain=-1:5,samples=180] (\x,{asin(sin(3*\x*pi r))/90});
\end{tikzpicture}

For $y=f(\frac{x}{3})$

\begin{tikzpicture}[xscale=1,yscale=1]
    \draw[help lines,xstep=1,ystep=1,color=gray!50,dashed](-1.5,-1.4) grid (6.5,1.4);
    \draw[->,thick] (-1.5,0)--(6.5,0) node[right]{$x$};
    \draw[->,thick] (0,0)--(0,1.5) node[right]{$y$};
    \draw[->,thick] (0,0)--(0,-1.5);
    %Draw marks on x axis
    \foreach \i in {-1,0,...,6}
    \draw [very thin,gray](\i,-0.1)--(\i,0.1) node[below right] at (\i,-0.1) {$\i$};
    %Draw marks on y axis
    \foreach \yticks in {-1,0,1}
    \draw [very thin,gray](-0.1,\yticks)--(0.1,\yticks) node[left] at (0,\yticks) {$\yticks$};
    %Draw the curve
    %\draw[red]plot[domain=-1:5,samples=90] (\x,{asin(sin(\x*pi r))});
    \draw[red] plot[domain=-1:6,samples=180] (\x,{asin(sin(\x*pi/3 r))/90});
\end{tikzpicture}
\end{solutions}

\begin{solutions}{Page 179}
\solution
\solution
\solution
\solution
\solution
\solution
\solution
\end{solutions}

\begin{solutions}{Page 181}
\solution
\solution
\solution
\solution
\solution
\solution
\solution
\end{solutions}

\begin{solutions}{Page 183}
\solution
\solution
\solution
\solution
\solution
\end{solutions}

\begin{solutions}{Page 183}
\solution
\solution
\solution
\solution
\solution
\solution
\solution
\solution
\solution
\solution
\solution
\solution
\solution
\solution
\solution
\solution
\solution
\end{solutions}

\begin{solutions}{Page 188}
\solution
\solution
\end{solutions}

\begin{solutions}{Page 189}
\solution
\solution
\end{solutions}

\begin{solutions}{Page 191}
\solution
\solution
\solution
\solution
\solution
\solution
\solution
\solution
\end{solutions}

\begin{solutions}{Page 194}
\solution
\solution
\end{solutions}

\begin{solutions}{Page 196}
\solution
\solution
\solution
\solution
\solution
\solution
\solution
\end{solutions}

\begin{solutions}{Page 197}
\solution
\solution
\solution
\solution
\end{solutions}

\begin{solutions}{Page 199}
\solution
\solution
\solution
\solution
\end{solutions}

\begin{solutions}{Page 200}
\solution
\solution
\solution
\solution
\end{solutions}

\begin{solutions}{Page 203}
\solution
\solution
\solution
\solution
\solution
\end{solutions}

\begin{solutions}{Page 205}
\solution
\solution
\solution
\end{solutions}


\section*{Chapter 9: Inverse Functions and Trigonometric Equations}

\begin{solutions}{Page 213}
\solution
\solution
\solution
\solution
\solution
\solution
\solution
\solution
\solution
\solution
\solution
\solution
\solution
\end{solutions}

\begin{solutions}{Page 220}
\solution
\solution
\solution
\solution
\solution
\end{solutions}

\begin{solutions}{Page 221}
\solution
\solution
\solution
\solution
\solution
\solution
\solution
\solution
\solution
\end{solutions}

\begin{solutions}{Page 225}
\solution
\solution
\solution
\solution
\solution
\solution
\solution
\solution
\solution
\solution
\solution
\solution
\solution
\end{solutions}

\begin{solutions}{Page 227}
\solution
\solution
\solution
\solution
\end{solutions}

\begin{solutions}{Page 229}
\solution
\solution
\solution
\solution
\end{solutions}

\end{document}
